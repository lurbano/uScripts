

\documentclass[9pt]{article}
\usepackage[a5paper]{geometry}

\newcommand{\attrib}[1]{%
	\nopagebreak{\vspace{2ex}\raggedleft #1\par}}

\usepackage{titling}
\setlength{\droptitle}{-5em}
\pretitle{\LARGE\bfseries\raggedleft}
\posttitle{\par\vskip 0.5em}

\preauthor{\large\bfseries\raggedleft}
\postauthor{\par\vskip 0.5em}

\predate{\large\bfseries\raggedleft}
\postdate{\par\vskip 0.5em}

\title{Hellspark}
\author{Janet Kagan}
\date{1988}

\begin{document}
	
	\maketitle
	

This one’s for
Eileen Enquist,
Lincoln Park Volunteer Hose Company No. 2,
Bob Lippman, Warren LeMay, Tom Cleary, and
Danny ???,
David G. Hartwell, and Rick Sternbach
—all of whom came to the rescue—
THANK YOU ALL! and for
Susan and Gardner, fellow alumni of the hottest writers’ workshop in the history of sf;
Chris, who had the “unique perspective;”
and Ricky, as always—with love
Sometimes I think if it wasn’t for the words, Corporal,
I should be very given to talking. There’s things
To be said which would surprise us if we ever said them.
—Christopher Fry, A Sleep of PrisonersA Note on Orthography:
I have chosen to follow GalLing’ usage: italicizing the term layli-layli calulan and capitalizing it only
when it begins a sentence. This should serve as a constant reminder that what appears to be a name is
rather a designation. Layli-layli calulan’s name is unknown to any but a very few of her most trusted
intimates. And her world of origin bears the designation (not the name) Y, meaning (very roughly) both
“sound of strength” and “source of strength.”
Again following GalLing’ usage, I do not capitalize the Jenji title “swift—” except where it begins a
sentence.
—MLL, ed.

\section{Prologue: Lassti}
SOUTH OF BASE camp, a daisy-clipper skimmed through the flashwood, buffeting the
undergrowth into a brilliant display of light. Its beauty was lost on swift-Kalat twis Jalakat. The dazzle
was merely one more distraction that might prevent him from finding some trace of Oloitokitok, the
survey team’s physicist—he had been missing for two days now.

Swift-Kalat, a small slender man with a ruddy complexion and, normally, an easygoing temperament,
punched the daisy-clipper’s comtab as if it were to blame for Oloitokitok’s disappearance. The weighty
silver bracelets that on his homeworld of Jenje would have chimed his status here clashed and jangled.

The sound only served to remind him that such expertise was useless in the situation he faced, and he
jammed the bracelets almost to his elbows to silence them. When he addressed himself to base camp, his
voice was clipped with exhaustion and anger.
“Swift-Kalat and Megeve,” he began, identifying himself and his companion, “we have completed the
search of sector four.” He paused to choose his words with care. In his own language, he would have
had no hesitation; his own language would have included in any statement the warning that he was neither
suited to this task nor physically reliable because of his weariness. In GalLing’, he was unable to speak
with such accuracy. He found himself limited to saying: “We’ve seen nothing we are able to interpret as
an indication of Oloitokitok’s presence.” His eyes flicked to the right, seeking a denial from Timosie
Megeve, the Maldeneantine who piloted, but it was as futile as asking the loan of a Bluesippan’s knife.
He received only a glare of anger and frustration.
“Nothing in sector four. Acknowledged.” The answering voice was low and weary, despite its careful
control: it was that of layli-layli calulan, the team’s physician—and Oloitokitok’s wife. She went on,
“Dyxte says there’s another storm, a bad one, coming up fast in your area. Return to base and get some
rest.”
The small screen on the pilot’s side lit to show the projected path of the storm. Frowning at it,
Timosie Megeve opened his mouth as if to voice an objection, but before he could even begin, layli-layli
added, “Doctor’s orders.”
“Acknowledged,” said swift-Kalat wearily. He thumbed the comtab off and closed his eyes.
“She’s right, I suppose,” said Megeve. “We’ve been searching for nearly twenty hours.” He ran a
cream-colored hand through a tangle of gray curls, dropped it to his thigh, and stared at it unseeing.
“We’re both so tired we’d likely miss a drab-death’s-eye if somebody dropped it into our laps.—And if
we miss something we should spot, we’re worse than useless.”
What Megeve spoke was true, swift-Kalat knew, but he also knew that rest would not come easily:
even Oloitokitok’s disappearance could not drive the sprookjes from his mind.
Megeve shifted forward, glared at the instrument panel, then thrust out a hand to tap a nail against an
indicator. He said something in his own language that was clearly a curse and tapped it again before
returning to GalLing’. “One equipment failure after another,” he said, still growling. “This wouldn’t have
happened if that transceiver hadn’t failed on us.”
“This wouldn’t have happened if Tinling Alfvaen had been here,” swift-Kalat countered, surprised tofind that the statement approached the proper degree of reliability even in GalLing’.
“Who?—Oh, your serendipitist friend.” With a second disgusted snort, Megeve gave up on the
indicator and guided the daisy-clipper forward, following the snaky curve of the river back to base camp.
“Maybe, maybe not. A serendipitist isn’t all-seeing, you know.”
Swift-Kalat made no response, but the thought worried him further.
Allowing three months for the letter he’d sent with the last supply ship to reach Alfvaen and another
for Alfvaen to act on it, the polyglot he’d requested was at least four months overdue. Perhaps he had
misjudged—not Alfvaen—but Alfvaen’s culture, which was so alien to him. Perhaps her custom
prevented her from assisting him. Swift-Kalat had worked with people of differing cultures long enough
to be aware that one culture’s truth was not necessarily another’s.
He called from his memory the image of her smiling face with its exotic pale skin, sharp features, eyes
a striking green. She was beautiful to him, but it was her eyes that held him always, even in memory: the
fierceness of her eyes when she believed in something or someone. She would have believed the message
he’d sent because he had sent it. Even custom could not have prevented her from acting on it, as custom
would not have prevented him from aiding her were their situations reversed.
If not custom, then what had delayed her?
He formed the truth for himself: his real fear was for Alfvaen’s safety. Perhaps the disease she had
contracted on Inumaru was more severe than she, or he, knew.
The shock of the discovery jerked him back to reality. To his added surprise, he found that Megeve
had turned the daisy-clipper to a new heading.
“What is it? Can you see something?” Swift-Kalat looked out, forcing himself to alertness.
He saw only a small stream, still swollen from the noon storm. A lush growth of drunken dabblers
bobbed and weaved in the rumbling water; at every surge their dead-black leaves came alight with veins
of eye-burning amber, the precise shade and glare of an antique sodium light. Beside them, smug erics
danced, churning and whirring—each pale white leaf edged, each silver stem spined, with a harsh glitter
of actinic blue. There was no sign of Oloitokitok. Blinking to clear his eyes, swift-Kalat turned to Megeve
for explanation.
Megeve listened to a faint roll of thunder and said, “I make it twenty minutes before that storm hits
here. That means we have enough time to reach your blind and change the tapes. There’s always a
chance they may show some act of the sprookjes that the captain can credit as intelligent.”
It was a faint hope and both of them knew it, but swift-Kalat accepted it gratefully, and Megeve went
on, “I know you’re concerned about the sprookjes. So was—is—Oloitokitok.”
Despite the immediate correction, Megeve’s use of the past tense chilled swift-Kalat. GalLing’ was
an artificial language and it did not have the same accountability as Jenji, but swift-Kalat still reacted
sharply when someone misspoke in such a matter.
It affected Megeve almost as strongly. He and Oloitokitok had been close companions since the
beginning of the survey. He took a deep breath and went on, “Oloitokitok wants to prove their sentience
as much as you—and he’s bought them a reprieve. Kejesli won’t send his status report while a member
of the team is missing. I only wish it hadn’t happened this way.”
Megeve turned the daisy-clipper across country, threading it through the flashwood, where the
turbulence of its wash whipped the Shante damasks from pure white to ripples of silver and stirred the
blue-monks mistily alight. To their right, a row of smoldering pines went from black to the dull red glow
of embers that had earned them their name. As the craft rose to avoid a deadly Eilo’s-kiss, swift-Kalat
pointed to a vast, gaunt stand of lightning rods, black and limbless spikes that rose to astonishing heights.
“About thirty meters to the right of that,” he said.
Megeve brought the daisy-clipper to a hovering stop in the small patch of flashgrass swift-Kalat
indicated and asked, “Shall I go in closer, or will that disturb your wildlife?”
“You wait here,” responded swift-Kalat, “I’ll be quick.” He folded back the transparent membrane,
but was stopped by Megeve, who said, “Remember? We’re back to Extraordinary Precautions.”
Swift-Kalat had indeed forgotten. To lose a team member this late in a preliminary survey implied a
danger that had not been catalogued. Until Oloitokitok was found, the team was to take the sameprecautions they had their first few months on Lassti.
The first and foremost of those precautions was to seal his 2nd skin. He popped his epaulets to draw
out his hood and gloves, laying them across his knees. Once the epaulets were closed, he shook the
hood open and coiled his glossy black braid into it; pulling it tight over his head, he ran a finger about his
neck to seal it. Even where there was no need for life support canisters, the habit remained; gloves came
second because they were clumsy enough to make sealing the hood difficult.
As Megeve double-checked the seams for him, swift-Kalat found himself wondering how much good
Oloitokitok’s 2nd skin might be doing him. Even a carefully sealed 2nd skin was no proof against electric
shock—and shock was Lassti’s major hazard.
“Sealed,” Megeve pronounced.
Swift-Kalat thanked him and slid the few feet to the ground, buffeted at a slight slant by the
daisy-clipper’s ground effect. Around his ankles, flashgrass whipped violently to and fro. Like so many of
Lassti’s plants, it tapped energy from motion piezoelectrically, discharging any excess as alternating
flickers of vivid green and white light. Swift-Kalat paused a moment to tune his hood, shielding his eyes
from the ever-increasing dazzle the oncoming storm winds raised within the flashwood, and then plunged
into its riot of light.
He pushed through a stand of solemnly chiding tick-ticks, thinking as he did so that it was too bad the
2nd skins MGE supplied its employees weren’t sophisticated enough to damp his other senses to this
world as well. Squat hilarities cackled, competing noisily with the tick-ticks for the attention of a swarm
of vikries, Lassti’s version of the bumblebee.
Some hundred yards in, he reached the clearing where he had erected his blind. Here,
flames-of-Veschke and penny-Jannisett unfurled their deep red and copper leaves. Both species used the
more conventional method of photosynthesis, and against the storm-brought brilliance of the background,
they looked almost black—and deeply restful. He breathed a sigh of relief at the quiet.
And then stopped in his tracks. The clearing should not have been so still, even in the absence of
thunder or roar of rain.
The first time the survey team had stepped into this clearing, those small, golden-furred creatures had
shrieked out. Oloitokitok had shrieked back at them, startling everyone as much as the creatures
themselves had. Laughing, but defiant, Oloitokitok had explained that in his tongue they seemed to be
saying, “I don’t believe it! Not for a minute!”
“I couldn’t let it pass without comment,” he had added. “I had to tell them to believe it.”
On each subsequent visit swift-Kalat had paid to the blind, no matter what precautions he had taken,
the flock of golden scoffers—for so they’d become in the surveyors’ common tongue—had shrieked out
their incredulity at his presence.
Now, there was no flash and beat of wings, no scornful shrilling. The only sound was the distant
chiding and cackling of plants.
In the uncanny stillness, a sudden whiplike crack against his ankle made swift-Kalat start. He looked
down to find he had brushed against a small blue-striped zap-me. The zap-me fed on electricity and
obtained it by startling small animals that used a charge for defense. Swift-Kalat did not respond in the
desired manner: he gave no shocks. As he watched, the plant patiently reset its whip-tendril to await a
creature that would.
Something gold lay at the base of the zap-me; swift-Kalat knelt for a closer look.
It was a golden scoffer. Its bright fur was unmarked, but it was dead. Three more were scattered a
few feet beyond. All dead,
A flicker of motion partially hidden behind his blind caught his eye. For one brief moment, hope rose
to sting his eyes. Here? Oloitokitok here? But before he could shout a query, he saw a flash of scarlet,
and a different hope stifled any sound from his throat.
A sprookje!
Swift-Kalat forgot the golden scoffers, forgot the oncoming storm. A crested sprookje! Afraid to
disturb it by rising, he moved only his head, craning awkwardly for a better look.
It was humanoid, but neither parody nor deformation of human. It was instead exotically beautiful:tall, slender, and deceptively fragile. Like its fellows at base camp, it was covered with short feathers,
subtly patterned in shades of brown. (After dark or in dim light of an overcast, swift-Kalat knew, the
feathers would emit a ghostly light.)
This sprookje, however, was a type that the survey team had not seen since their first contact with
the species nearly three years ago. It was superbly crested in scarlet, and its long, smooth neck rose from
a swirling yoke of red and blue feathers.
It knelt on both knees, over something shiny that was hidden from swift-Kalat’s view by the
art-nouveau tracings of an arabesque vine. Its head dipped rapidly—once, twice, three times—but
swift-Kalat was unable to see what it was doing.
At last the sprookje stood and turned to face him. Enormous golden eyes stared at swift-Kalat from
the sharp-featured, scarlet face. It opened its beaklike mouth as if to speak, but made no sound. Its
tongue glowed an ominous red. Then, feathers ruffling, it backed slowly away and vanished into the
flashwood.
Swift-Kalat realized that he had been holding his breath. He exhaled with a sigh and rose, just as a
rattle of thunder recalled the need for haste.
Cautiously, he pushed through the heavy underbrush to see what had so interested the sprookje. A
large object with the sheen of plastic lay beside his blind, reflecting bloody red the flames-of-Veschke it
lay among. Scattered around it were a dozen more dead golden scoffers. For a long moment, his mind
fought identification of the object.
He closed his eyes. The golden scoffers were scavengers. When swift-Kalat opened his eyes once
again, he saw that Oloitokitok was dead. In death, Oloitokitok had silenced the scoffers once and for all.
Megeve and swift-Kalat found Oloitokitok’s daisy-clipper on the far side of the stand of lightning
rods. They lifted the remains of his body into it and Megeve switched the hovercraft to follow mode.
This bitter parody of a funeral cortege—the only rites Oloitokitok would have until the cause of his death
had been ascertained—arrived at base camp on the edge of the breaking storm.
Torrents of rain dimmed even the field of flashgrass. It distorted into unrecognizability the tiny crowd
of surveyors who huddled grimly at the main gate. Only layli-layli calulan seemed sharp-edged, in
focus, as she came forward to take charge of the body.
No one could find the words to speak to her. A moment later, the crowd disbanded in total silence.
Swift-Kalat sat in the grounded daisy-clipper and watched them all go.
Wearily, he gathered up his specimen bag and fought through the thick red mud of the compound to
his cabin.
He taped a record of his sprookje-sighting while it was still fresh in his mind; then, unable to sleep, he
took the dead golden scoffers from his specimen bag and spent the next few hours dissecting one. His
exhaustion had at last caught up with him. He put the second small corpse to one side and played back
his report: the voice that issued from the recorder sounded chilled and shaky.
The thunderstorm passed and the rain settled down to a steady drizzle. He fastened the cabin door
open—he wanted company but he was too tired to seek it out—and his sprookje entered. (At least, he
assumed this one was “his”; like Gaian cats, each of the sprookjes in camp seemed to favor a particular
person.) It was not the company he had hoped for but, unlike most of the other surveyors, swift-Kalat
didn’t mind the sprookje. His inability to communicate with it was troublesome; its presence was not.
It shook rainwater from its feathers with a controlled shiver.
Swift-Kalat rubbed his eyes. “Don’t drip on the floor,” he said. As always, he spoke to the sprookje
as if it might understand.
The creature rubbed its own silver-blue eyes and blinked at him. “Don’t drip on the floor,” it said, its
Adam’s apple bobbing; and swift-Kalat was again disturbed to hear the shakiness in his own voice, this
time captured by the sprookje.
It parroted everything he said with the same accuracy and retention as his recorder, and only the
beaklike shape of its mouth made its mimicry imperfect.
Swift-Kalat sighed.
The sprookje did likewise. Then it looked down at the table and saw the golden scoffer. It leanedover and opened its mouth.
“Hey! Don’t do that!” said swift-Kalat sharply.
The sprookje echoed both his words and his tone and went on as it had intended. Swift-Kalat caught
a quick glimpse of the sprookje’s “sample tooth”—the single retractable needlelike organ that was
ordinarily concealed within its beak—as the sprookje nipped the golden scoffer.
It was an irrational response, he knew, but the sharp thrust of the beak, the bite, always seemed
aggressive. The first time they had seen crested sprookjes, van Zoveel had stepped forward to attempt to
communicate with them. He had been examined and bitten. And everyone assumed he was being
attacked. The resultant commotion had driven the sprookjes away.
Now he reacted not only to that, but to the thought of the dead golden scoffers as well. Eating
Oloitokitok’s flesh had poisoned them, as eating from the humans’ garbage dump had poisoned the
scavengers near base camp. With considerable relief, swift-Kalat remembered that he’d been bitten by
the sprookje when it first arrived at base camp, with no ill effects to either of them.
The sprookje lifted its brown cheek-feathers slightly, as if in surprise; then it walked away, out the
door and out into Lassti’s brilliant dusk. Swift-Kalat was too tired to follow, too tired to wonder at the
sprookje’s behavior. He sank into his chair, closed his eyes, and lay his head on his arms just for a
moment...
When he awoke, it was to the sound of thunder and the spray of rain streaming through the open
door. Stiffly, he crossed the room and drew the opaque membrane closed. Reflected patterns of dull
yellow light made him turn to the computer in the corner—he donned his spectacles and reluctantly called
up the message the computer was holding for him.
The image of Ruurd van Zoveel, the survey team’s polyglot, sprang into view. Van Zoveel was a
large, solidly built man with a smokewood face, shaggy dark blond hair, and shaggier sideburns. Even
seated before his computer console to tape a message, he was in constant motion. His gaudily
beribboned tunic rippled with his agitation.
He spoke Jenji without a trace of accent, however, and he spoke it with a high degree of reliability.
Swift-Kalat closed his eyes and found comfort in the sounds of his native tongue that he did not find in
the content of the message.
“Layli-layli calulan has finished the autopsy: she concludes that Oloitokitok died of heart failure due
to severe electric shock. I saw what I took to be burns on his chest and shoulder. His locator had been
fused; Megeve says in that condition even Oloitokitok’s death wouldn’t have set it off. The captain
concludes that Oloitokitok startled a shocker—”
At that, swift-Kalat found himself frowning. Several of the indigenous predators used an electric
charge to stun or kill their prey but the idea that a live-wire or a blitzen would mistake Oloitokitok for
prey seemed out of keeping with what he knew of the habits of the creatures the team had dubbed
shockers. A charger, perhaps? Unlikely in that area of the flashwood...
“—or perhaps was simply struck by lightning. The captain has therefore lifted Extraordinary
Precautions.”
Something in van Zoveel’s voice made swift-Kalat open his eyes. Van Zoveel brushed his sideburns
anxiously, then he swallowed hard and finished, “There will be no final rites—at least, no public ones.
Layli-layli calulan’s culture restricts death rites to the surviving family, in private.”
Apparently, layli-layli had broken one of the taboos of van Zoveel’s culture. For the sake of
understanding, swift-Kalat would have to look into the matter.
With visible effort, van Zoveel composed himself. After a moment, he said, “That’s not what I called
about. I must speak to you, swift-Kalat, as soon as you’ve rested. Captain Kejesli wants my formal
decision on the sprookjes.”
To swift-Kalat’s surprise, van Zoveel abruptly switched to GalLing’ and went on, “I apologize for
switching languages on you, swift-Kalat, but I can’t say this in Jenji without creating an untruth: I believe
that Kejesli wants the sprookjes found nonsentient. No! No, it’s not even that—I think he wants this
survey over and done with, and it doesn’t matter to him how sloppily he goes about it. It doesn’t matter
to him what the sprookjes are as long as we decide now.”Again, he made the effort to compose himself. Then he added, “Please understand that this statement
has no reliability whatsoever; it’s only the feeling I have. But I must discuss my report with you.”
The tape beeped end-of-message and then there was nothing to see but cabin wall. Swift-Kalat took
off his spectacles and continued to stare at it. Even in GalLing’, and even with van Zoveel’s careful
disclaimer, the words were chilling. They could do the sprookjes no good. He did not want to face van
Zoveel for fear of the further harm the man might do with his words.
To postpone the unpleasant duty as long as he could, he ate, telling himself throughout that van
Zoveel’s words in GalLing’ could have no adverse effect on the sprookjes’ situation. That, in fact, he
knew that reliability was an aid to understanding only; that it was only superstition that it had an effect on
reality. Having reassured himself somewhat, he showered as well, rebraiding his hair while it was still
damp. There was no point in waiting for it to dry: it would only get wet again as he crossed the
compound to van Zoveel’s cabin.
The thunderstorm had not let up. He stood on the sheltered step of his cabin for a long time, reluctant
to venture into the storm. Thunder rattled, numbing his ears; a sheet of lightning whited out even the red
mud of the compound.
For a long moment, he was deaf and blind. He blinked furiously to clear his eyes, shielding them with
a raised hand from ensuing flashes as the lightning repeatedly struck the stand of lightning rods that grew
only two kilometers from camp.
When at last he found his vision returning, swift-Kalat could no longer distinguish between the dazzle
of the Lassti flash wood and that in his own optic nerves. He drew an angry breath and plunged into the
pouring rain. All around him sparks flew.
Chapter One
Sheveschke, on the Rim of The Goblet.
WIND ROSE TO sweep the great bay known as The Goblet, where the Sheveschkem fleet
gathered to honor Veschke, patron saint of thieves and traders, and to be blessed by her priests. The
hissing light of torches along the wharf shaped and shadowed a hundred small craft, all alive with
whispered sounds as if they shared the festival excitement. Ironwood hulls groaned and ropes creaked to
the pulse of the waves; pennants and ribbons snapped counterpoint in the wind. They spoke of a
thousand more ships beyond the acrid blaze of torchlight.
The same wind brought the wood-smoke of the festival fires, the tang of keshri bark, and the warm,
rich smell of great cauldrons of stew.
It was the sailing wind of Sheveschke, and it whipped through Tocohl Susumo’s red-gold hair and
sent her moss cloak streaming about her. Her 2nd skin glistened over her tanned flesh like rubbed-in oil,
reflecting the sparks riding the wind.
She was tall and spare, and she acknowledged her kinship to the captains of these tiny craft with a
nod that, on another world, would have been a bow. Momentarily caught by torchlight, her eyes flared
gold.
Beyond the bay, a thousand extra stars bejeweled the clear, cold skies of Sheveschke, their light
splintered and spattered by the rowdy waves of Shatterglass Sea. A thousand extra stars—the Hellspark
traders come to pay their own respects to Veschke, to have their ships blessed, side by side with the tiny
skiffs and the sleek schooners of Sheveschke.
Tocohl Susumo looked up at the sky, into constellations old and new. (Where are you, Maggy?) she
subvocalized. (Here,) came the response, and a tiny arrow appeared against the night sky, projected on
Tocohl’s spectacles, to indicate a new star at the tail-tip of the smallest Lunatic Cat.
Tocohl smiled her satisfaction, then leaned against the ironwood railing and said, (Now play back the
message from Nevelen Darragh.)
(Your adrenaline level has dropped two points in the last five minutes. Playing Nevelen Darragh’s
message would only raise it again,) said Maggy; and Tocohl imagined a plump and prim Trethowanattempting to speak Jannisetti without using any taboo words.
(Cheeky,) Tocohl said, (don’t argue with me.)
(I can’t argue.)
Not half, you can’t, thought Tocohl, amused; then she subvocalized again, (Play back the tape.)
This time Maggy made no objection.
There was no image, only the voice of a stranger. Her words were crisp, formal, and legally binding:
“Tocohl Susumo is hereby notified of case pending judgment and enjoined from her proposed run to
dOrnano to answer the charge of Tinling Alfvaen.” A single bell-like note sounded. The crisp voice said,
in signature, “Byworld Judge Nevelen Darragh,” and then there was silence, except for the night sounds
of the bay.
Tocohl drew her cloak tightly to her, not for warmth—that was amply provided by the 2nd skin—but
for gesture, as a cat lays back its ears in preparation for a fight.
(Your adrenaline is up to—)
(Shut up a minute and let me think.) Tocohl breathed deeply and, reminded of the Festival of Ste.
Veschke by spicy odors, decided that she did not need the Methven ritual for calm.
She was more puzzled than angry. A byworld judge dealt with cases where two cultures met and
clashed—tourists who got themselves in trouble through ignorance of local customs, for example—or
cases where no world claimed jurisdiction, in deep space or on worlds without a charter,
Tocohl shifted to Jannisetti and said, (As far as I know, I haven’t stepped on any cultural toes lately,)
turning the Sheveschkem cliche into a Jannisetti obscenity.
(Is that funny?) Maggy asked.
(I thought so; how did you know?)
(You smiled.)
In Jannisetti, a smile was limited to the face, so Maggy was apparently reading the implants at
Tocohl’s ear and throat rather than feedback from the 2nd skin. Tocohl touched the spot just before her
ear and smiled again, but she could feel neither the transceiver nor play of muscle. She frowned slightly
without meaning to.
Maggy said, (Then why are you worried?)
Tocohl grunted. (It could be about that “farm equipment” we sold on Solomon’s Seal; two of the
people I dealt with were third-generation Siveyn, and Tinling Alfvaen is as Siveyn as names come.—To
be honest, Maggy, it could be about a lot of things, but that would worry me.)
(I don’t understand. The manifest said “farm equipment” and that’s what we delivered.)
(Maggy, this is a little difficult to explain: they expected arms.)
(Then why would they request farm equipment?)
(To make the shipment seem legal.) To forestall the inevitable question, Tocohl said firmly, (Yes,
Maggy, the shipment we made was entirely legal, but we didn’t deliver what the customer wanted.)
(I don’t understand. If the shipment was legal—)
(What kind of charge could they bring? Price-gouging, as much as I hate to say it. They paid a lot
more for farm equipment than they intended to. And serves them right.)
Maggy made no response. This was apparently beyond her and it was clear she felt it better for
Tocohl’s adrenaline level that she not inquire further.
Probably just as well, thought Tocohl, though it led her to wonder just what files Maggy might be
checking in that silence. To distract her was a hopeless task, Tocohl knew, so she merely said, (How do
we find Nevelen Darragh? Skip the map.) The projection vanished as quickly as it had come. (Give me
verbal directions for the quickest route to Veschke Plaza.)
(That would take you through an area the Sheveschkemen consider highly dangerous after dark.)
(Fine,) said Tocohl. (Perhaps I’ll have a chance to work off some of that extra adrenaline you’re so
concerned about.)
There was a pause, almost of resignation, then Maggy said, (Turn right and follow the Rim of The
Goblet.)
Tocohl set off as directed. The silver filigree of her cloak streamed behind her and the lightness of herstride gave no evidence of her unsettled thoughts.
Here and there, she eased her way through crowds of merrymakers overspilling from waterfront
taverns onto the wharf. Her captain’s baldric brought her a spate of invitations which she reluctantly
turned down or set aside for another time. Twice, laughing, she pulled stray hands from the pouch slung
at her hip. “Clumsy doesn’t honor Veschke,” she chided the would-be thieves.
Twenty minutes later, Maggy turned her away from the Rim and into the narrow, dimly lit streets of
the Old Quarter.
Tocohl did not slow her pace. One of the minor pleasures of having first-class equipment, Tocohl
thought, was that she needn’t worry about stubbing her toes on cobblestones. She might trip and crack
her head, for her hood lay softly cowled about her neck, but if her toe struck stone the 2nd skin would
spread the impact to absorb it and spare her the bruises.
She reached an unlit square, and Maggy said abruptly, (Trouble.)
Tocohl stopped. In the starlight, she could see only the constricted alleyways and the cramped stone
houses and shops typical of Sheveschke.
Across the square, a solitary figure—a fisher, to judge from his rough-woven clothing and the
pronged knife thrust into his belt—lounged against a stone doorpost. He straightened and whistled shrilly
but made no move toward her.
(What trouble, Maggy?) she asked.
(Three people fighting in the alley.) Maggy pointed to the pitch-black opening to the right of the
whistler.
(Push my vision two points,) said Tocohl, and the scene brightened and sharpened. Around the
edges of the spectacles, Tocohl’s peripheral vision darkened in contrast. It was as if she looked down a
tunnel of light, the end of which was whatever object she focused on.
Three dim figures clashed in the alleyway. Two were Sheveschkemen and, like the whistler, wore
fishers’ garb. The third was undoubtably an off-worlder; over the sheen of her 2nd skin she was dressed
in a combination of styles from several different planets—what Hellsparks called worlds’ motley. Not
Hellspark, for she wore no baldric. Tourist, then.
She fought well, outnumbered as she was, but her movements were slow and broad. Drunk, thought
Tocohl, her timing’s off—and that’s the standard surveyor’s 2nd skin, not much help in a brawl. She’s
going to lose this fight.
Tocohl didn’t much like the odds. (I’m going to pull rank, Maggy: watch my back.) Unclasping her
moss cloak, she let it drift gently to the ground.
Few people in the Extremities would argue with a Hellspark captain on whose good will their
interstellar trade depended, but Tocohl took the elementary precaution nonetheless. The deceptively
simple action exposed all of the sensors in her 2nd skin but those still covered by her captain’s baldric,
and Maggy could work around those easily enough.
She started across the cobbled square heading for the alleyway.
But the whistler stepped forward to meet her. His knife flashed upward in a swift, glittering arc.
Tocohl had no time to be surprised: she shrugged gracefully and the blade missed its mark. Before he
could recover sufficiently to thrust at her a second time, she slammed her edged hand into his wrist and
the knife jarred away, clanging on the cobbles.
The Sheveschkemen called a warning to his companions and backed away from the mouth of the
alley, scrambling after the knife. Tocohl had no intention of letting him rearm. She followed—with two
long strides and a lightning kick that took him squarely in the chest just as he bent for the knife.
Her 2nd skin absorbed the impact. Tocohl felt only a mild twanging sensation from foot to thigh but
the whistler slammed against the brick wall, cracked his head, and crumpled forward, unconscious.
Tocohl’s back tingled. (Roll!) said Maggy, and a sandbag blow struck across her shoulders. But for
Maggy’s warning, Tocohl would have been thrown off balance. Instead, she somersaulted and twisted,
came up back to the wall to face a second assailant.
This one too held a knife, but he stared at his weapon dumbly. With Maggy to see it coming, the
force that would have enabled the knife to pierce her had been transferred instead along the warp andwoof of the 2nd skin; and because she had rolled forward at the crucial moment, it was unlikely she’d
even have a bruise from the attempted stabbing.
There was one further advantage: his disbelief gave Tocohl the few seconds necessary to regain her
breath and charge. The Sheveschkemen’s nerve broke. He gave a sharp squeak of panic, dropped his
knife, and fled.
Tocohl wasted no time following him; she rounded the corner into the alleyway—and stopped short.
The third Sheveschkemen was gone, and so was the off-worlder.
(Overlay infrared,) Tocohl snapped, and a line of ghostly red footprints appeared, drag marks trailing
them. The prints steamed away even as she watched, and she followed at a run.
Deep into the alleyway, the prints brightened and led to a narrow door. Even with her vision pushed
for available light, Tocohl might have missed it—it was flush with the alley wall—but in infrared, the
door’s outline was unmistakable and the misty heat patterns told the rest. The Sheveschkemen had
dropped the off-worlder, fumbled for the latch, then dragged her inside...
Once again sounding prim, Maggy began, (Breaking and entering—)
Tocohl cut her warning short, (It’s festival. Read up on it.)
(If you’re going in,) said Maggy, changing tactics, (put on your gloves so I can protect your hands.)
Tocohl gave each hand a sharp snap downward. Her neat cuffs unfolded and met just beyond the
tips of her fingers. She gave Maggy a moment to individuate the 2nd skin between fingers, then reached
for the latch and swung the door inward.
Maggy adjusted the spectacles so smoothly that Tocohl was not blinded by the unexpected glare of
electric lights.
The fisher, a woman almost as tall as Tocohl and twice as massive, wrapped twine tightly, viciously,
about the off-worlder. She looked up at the noise, grunted, and threw a shiny object—
Tocohl swiftly drew the door to, and the object struck it with a thud, splintering wood where
Tocohl’s head had been the moment before, then crashed to the floor and rolled away. It was a heavy
copper sap that fishers used to kill their netted catch.
Still using the door for partial cover, Tocohl kept her eyes on the Sheveschkemen.
Then the fisher’s eyes flicked once to the left. Warned by the movement, Tocohl leapt left even
before the Sheveschkemen.
The fisher’s knife lay beside a skein of netting twine. Tocohl swept it from the ironwood table
seconds before the fisher’s full weight struck her. Tocohl staggered back, but stayed between the fisher
and her knife, and blocked two punches in rapid succession.
Then she saw an opening, whipped the edge of her hand across the fisher’s temple. Maggy was good
to her promise: the 2nd skin stiffened and Tocohl felt bone crunch beneath the blow.
The Sheveschkemen fell, first to her knees, then onto her face. Tocohl stepped aside and, without
taking her eyes from the fisher, knelt for the knife.
Cautiously she rose and stood looking down at the fisher’s prone body. After a long moment, she let
out a sharp breath. (How’s my adrenaline level now?) she asked.
(Still high, but dropping,) Maggy answered, impervious to sarcasm.
Tocohl grinned in relief and turned her attention to the off-worlder. The small woman was still
unconscious and breathing with difficulty. Tocohl first removed the crude gag and blindfold, then set to
work on the rough twine with the fisher’s knife.
Over her 2nd skin, the off-worlder wore a kilt of charcoal gray, black boots, and a fringe bodice of
blue and silver. Silver threads laced through her jet-black hair, which hung in double braids over either
ear. Taken singly, the styles might have identified her world of origin, but together, they gave Tocohl no
clue.
Nor did her face. Her features were angular but gentle, and her skin was shockingly pale in contrast
to her hair, except for the burns on her cheeks caused by the force with which the fisher had gagged her.
Her breathing gradually became normal.
Tocohl sliced through the last of the twine, and the woman slumped forward. Tocohl caught and
eased her gently to the floor. As she did so, the braids fell away from the off-worlder’s ear and exposedtwo bright bits of cloisonne: earpips.
(Definitely a surveyor,) Tocohl said. (Surveyor-grade 2nd skins are fairly common but earpips aren’t.
On holiday, I suppose, though this is an odd place for it.) She bent for a closer look at the earpips.
The first identified the woman’s profession as serendipitist, which caused Tocohl to raise an
eyebrow. To those who believed in espabilities, and Tocohl did, a serendipitist was one who brought
luck to herself and those around her. This is serendipity? thought Tocohl; if so, it certainly takes a peculiar
form.
The second pip was a medic alert. (Maggy, what does this mean?) She raised the emblem slightly to
give Maggy a clear view.
(The wearer suffers from Cana’s disease—)
(In layman’s language, please,) said Tocohl, to forestall a spate of medical jargon that would be of no
practical use.
(—A parasitical infestation that acts like a super-yeast,) Maggy continued. (It converts sugar into
alcohol. Cana’s disease can be controlled in the human but not cured. Under stress, the victim appears to
be—is—drunk.)
(Contagious?) said Tocohl.
(If it were, I would have said so. The parasite undergoes alternation of generation and is only
transmissible through a blood-sucking mammal native to Inumaru, in the system of which it is a symbiont.)
(Sorry,) said Tocohl, reacting more to tone than content. (Is there anything I should do for her?)
(She’ll have her own medication for that, and I’ve sent for a doctor.)
The woman stirred and, without warning, struggled violently from Tocohl’s arms. “Laiven!” she
gasped, “laiven la’ista!
Siveyn, thought Tocohl, and responded in the same language. “Gently. The wild beasts”—she used
the literal meaning of la’ista—“have had their claws pulled.” Tocohl offered the fisher’s knife, hilt-first, as
proof.
The Siveyn blinked pale green eyes at her, and touched the knife lightly but did not take it. Then she
relaxed with a long shuddering intake of breath.
Torchlight flickered through the darkness beyond the splintered door. Tocohl came to her feet,
stepped across the Siveyn, ready for more trouble.
(Police,) said Maggy. (When the lookout called for help, I did too.)
Tocohl relaxed, made a reassuring motion to the Siveyn. (I didn’t ask for police,) she said, (or a
doctor, come to think of it.)
(You didn’t say not to call them. Was I wrong?)
(No, you did just fine.) Tocohl walked to the door and waved broadly to the little knot of uniformed
Sheveschkemen who filled the mouth of the alley. “In here,” she called in Sheveschkem.
She glanced at the splintered door and said, (Thanks, Maggy. If you hadn’t adjusted my spectacles, I
wouldn’t have seen that coming. You saved me quite a headache!)
(You’re welcome,) said Maggy primly. (Next time, though, pull up your hood too.)
The police doctor stooped to the fallen fisher. “She’ll live,” he said; and, without a further word, he
crossed to examine the Siveyn.
The stumpy lieutenant in charge of the local authorities grunted sourly. A brusque wave of his hand
brought two officers to guard the fisher. Then he turned to Tocohl. “Captain, may I have a word with you
in private?”
“Of course,” said Tocohl, and the two of them stepped into the alleyway. Tocohl leaned against the
rough stone, beneath a freshly kindled torch.
“Lieutenant t’Ashem,” he said, offering his hand.
His voice held no accent, but his stance did: Tocohl judged him a northerner still unused to southern
kinesics, despite long residence. Instead of grasping the hand, she touched palms, northern-fashion, as
she gave her own name. His eyes widened slightly, but more than enough to confirm her judgment, and
she knew she had added one more minor embellishment to the Hellsparks’ reputation.
He went on quietly, “May I hope, Captain, that you won’t hold the wasters against us?”“Wasters? Ah!” said Tocohl, “that explains the electric lights. I had wondered about the absence of
Veschke’s candles.” The wasters—the Inheritors of God, to give them the name they used for
themselves—were a fairly recent but widespread religious sect. Arrogant and troubling to most
authorities, because, put simply, whatever they did was right. A more severe case of “God is on my side”
than most of the usual religions. “—No, Lieutenant, they cause you more than enough trouble on their
own. Why should I add to it?”
The lieutenant looked relieved and went on, “Then you won’t mind telling me what happened?”
“Not at all. See for yourself.” Tocohl removed her spectacles and handed them to the
Sheveschkemen. (Maggy,) she said as he donned them, (run back the visual from where you warned me
of trouble to the lieutenant’s arrival.)
Obviously, the lieutenant had had previous experience with a first-person replay. He placed one hand
firmly against the stone wall for physical reference; and only twice did he react involuntarily, the first time
as Tocohl somersaulted, the second as the fisher threw her sap.
The younger of the two guards joined them. Tocohl thumbed her earlobe for silence, and he waited
patiently beside her.
After a moment, the lieutenant removed the spectacles to frown at her. “Are you in need of medical
attention, captain?”
“What?—Oh, the knife blow. No.” Tocohl turned to show him an unmarked back. “I have a
fondness for first-class equipment, even in 2nd skins. This is a stripped assault version.” She didn’t
bother to mention that Maggy had made a considerable improvement even on that; what she had said
was quite sufficient to take him aback.
“Very expensive!” he said.
“Very good trade,” she corrected him with a grin, “—and that is, after all, my business!” It actually
drew an answering smile from him. She reached out to collect her spectacles, saying, “Now you know as
much as I do. Shall I make you a copy of the tape?”
“You needn’t bother,” said the lieutenant, “I know the third one.” His smile turned grim. “I won’t
have any trouble finding him.” Remembering the presence of the younger man, he suddenly said, “Well?”
The guard said, “They jumped her just outside the Shavam Inn and dragged her here. She assumes they
meant to rob her.”
The lieutenant looked swiftly at Tocohl and said, “She’s not Hellspark, then?”
“No,” said Tocohl, evenly, “she’s from Sivy. Robbery seems unlikely, even from the Inheritors of
God. Given that I found her bound and gagged, kidnapping would seem their goal. But why her? I guess
we’ll have to ask the fisher what it was all about.”
The lieutenant grunted. “That’s like asking a Bluesippan for his knife. The Inheritors of God don’t
explain themselves to heretics.” He scowled and shrugged.
The Sheveschkem shrug, southern or northern, took one hand only, and—even as performed by the
stumpy lieutenant—it was an eloquent mime of a man who tests the weight of an object with a bounce
and, having found it unsatisfactory, discards it over his shoulder in disgust.
“Still,” Tocohl prompted.
“Captain, we’ve had any number of incidents from the wasters in the past few months. Most of them
unaccountable. Something stirred them up, and the usual sensible restraints don’t apply. It’ll be a pleasure
to put a few of them away. Take it from me, this is just another bunch of wasters doing what they feel like
at the moment. They break the law to prove the law does not apply to them. Random violence is almost
a sacred act to a waster.” He scowled more deeply.
“At any rate,” he finished, “you and your friend needn’t miss any more of the festival on their
account.”
He stalked to the doorway, pausing on the threshold to pick up the copper sap. “A souvenir,” he
said, and tucked it into the loop on Tocohl’s baldric designed for just such a purpose. “Use it on the next
waster you meet. God may not be on your side, but I certainly will.”
Taking it for a sour joke, Tocohl smiled, but her smile vanished as she entered the lighted room. The
second guard was giving a finishing touch—a boot to the ribs—to the unconscious fisher. The Siveyn,moved to the fisher’s pallet and still being probed by the Sheveschkem doctor, watched and shivered
visibly.
“That’s enough,” snapped Tocohl, and the guard looked up, startled and angry. Seeing her captain’s
baldric, however, he backed away from the fisher and began to make apologetic noises. They were
noises only, none of his anger at Tocohl’s intervention had gone.
Tocohl turned. “I’m sorry, Lieutenant, but that’s taboo to the Siveyn. And I can’t say I much like it
either. If you can’t control yourselves any better than the wasters can, could you at least wait until I get
her out of here?” Between the insult and her baldric, that ought to put a stop to any further beatings.
The lieutenant took in the severity of her disapproval and gestured brusquely again. The guard
muttered and retreated, but not before he had spat once on the fisher and said, bitterly, “Hull-ripping
waster.”
“I agree with his sentiments,” the lieutenant said, sighing, “but not with his expression of them. His
actions aren’t taboo here, but they do make more work for the doctor.”
He stopped abruptly. “Wait a minute,” he said. “Taboo to a Siveyn? The Siveyn fight duels over
anything—they’d fight about a theft at festival!”
“They duel, yes, but a duel is rigidly codified behavior. No Siveyn would dream of striking someone
without first exchanging the proper ritual insults with him or her. Anything else is la’ista, the behavior of
wild beasts; and that’s the attitude that puts your officer there socially on a par with the waster.”
He still looked puzzled. “Lieutenant,” Tocohl went on, “she can challenge people all day long on
Sheveschke, but she won’t fight a duel unless she runs into another Siveyn. A challenge is one thing—but
you simply don’t attack unless you get the proper ritual responses.”
She could see he still wasn’t understanding. “If you went for Veschke’s fire, and the priest didn’t say,
‘For Veschke’s fire, one must shed blood,’ would you continue with the ritual?”
“No, of course not. It wouldn’t be properly done. It would be worthless... Ah, you mean fighting is
somehow worthless to her without the proper responses!”
“That’s it. Nothing more than wild beasts. And it can’t be done on that level. Besides, this Siveyn is
more cosmopolitan than most; she’s worked with a survey team and, judging from the fact that she hasn’t
challenged anybody yet, that’s made her very tolerant.”
The Sheveschkem doctor looked up as they approached and addressed Tocohl. “No concussion.
She’ll have a headache, but she’ll owe it more to her celebrating than to the wasters.”
(Maggy, find me a real doctor.)
(Does Geremy Kantyka qualify?)
The name gave her a start; Geremy was one of the few who’d heard the story of the “farm
equipment” for Solomon’s Seal. (Geremy’s in town? He’ll do nicely, yes.) Aloud, she said to the
Sheveschkem doctor, “Thank you.” Then she added, including the lieutenant in the query, “Is there
anything else, or may we go?”
“Unless your friend wants a judge,” said the lieutenant, “we’d prefer to treat this as a local matter.”
Tocohl bent to the Siveyn. Extending her right hand, she laid her left palm up, fingers lightly curled, in
the crook of her elbow and repeated the lieutenant’s offer in Siveyn.
The small woman’s green eyes focused with difficulty. She glanced obliquely at the guard who’d
kicked the fisher and said, “I’d rather leave.” Then her eyes fell on Tocohl’s outstretched hands. “You
s-stopped them?” Only the slight hesitation in speech betrayed her drunkenness.
“Yes,” said Tocohl. “I apologize for the appearance of la’ista—my own as well as the officer’s.
Sheveschkem ritual is not Siveyn ritual, but Sheveschkem ritual was satisfied.”
The Siveyn took a deep breath. “I see,” she said and rose, bracing herself on Tocohl’s proffered
arm. “As the Hellspark s-say”—like most Siveyn, she pronounced it Hell-spark—” ‘When on
s-Sheveschke, be a s-Sheveschkemen.’ Your apology is unnecessary, and you have the thanks of Tinling
Alfvaen.”
Tocohl frowned. (Maggy, Tinling Alfvaen!) Tocohl missed a sentence or two as Maggy responded,
for her ear alone, in the crisp voice of Nevelen Darragh, (“... to answer to the charge of Tinling
Alfvaen...”), then in her own voice went on, (That is the name of the only surveyor of the twelve whocontracted Cana’s disease on Inumaru who was of Siveyn origin.)
(You might have told me.)
(You were busy. I didn’t want to interrupt.)
(Anything else I should know?)
(She was also the only one of the twelve to lose her job with MGE after that survey.)
When Tocohl snapped her attention back, Tinling Alfvaen was saying, scornfully, “—And
Multi-Galactic thinks I’ve lost my serendipity!” She gave her head an impatient shake that sent her braids
flying. “If I’d lost my serendipity, I’d never have been rescued by the only other Siveyn on Sheveschke!”
“I can’t speak to other circumstances, but I’m not Siveyn.”
“Oh?” Alfvaen paused at the threshold to face Tocohl; she blinked her pale eyes in an effort to clear
them and frowned slightly. “Oh!” she said, after a moment, “You’re Hellspark, then.”
“Yes. Susumo Tocohl, and pleased to meet you, Tinling Alfvaen.”
Alfvaen released her arm and made the Siveyn formal greeting. “That’s the same thing,” she said
warmly.
(She didn’t recognize my name.)
(You didn’t recognize hers, at first,) said Maggy reasonably.
Alfvaen wobbled and Tocohl caught her again. (She’s getting drunker the longer she stands here,)
said Tocohl. (That might explain her lack of recognition.)
Tinling Alfvaen raised a hand level with her throat, palm out, fingers splayed. It was one of the few
gestures that GalLing’, the universal pidgin, recognized as necessary.
“No,” said Tocohl, “you haven’t caused offense. Do you have medication with you?”
The Siveyn looked startled. “Yes-s,” she said and began to pat the pockets of her kilt, her hands
clumsy with haste.
She drew out a small box and gouged at it with her nail—then, exasperation in her sharp features, she
handed it to Tocohl. “Would you please...?”
Tocohl opened the box, and Alfvaen took a pill and gulped it. “I’ll be fine in a minute,” she said.
“How did you know?”
“Your earpip,” said Tocohl. “Which direction are you headed?”
Alfvaen inhaled deeply. “I was on my way to Veschke Plaza, to meet Judge Darragh at the main
festival fire.”
Tocohl smiled wryly. “That’s where I’m going. I’ll accompany you, if I may.”
“Certainly!—Are you a judge, too?”
The Siveyn’s innocence was mystifying. “No,” Tocohl said, “a high percentage of the byworld judges
may be Hellspark, but a high percentage of Hellsparks are not judges.”
Alfvaen frowned and, for a moment, Tocoh thought that the Siveyn had at last recognized the name.
But when she said nothing about it, Tocoh concluded that she had only been reacting to the Hellspark
tradition of alternating the pronunciation of their world’s name: first Hell’s-park, then Hell-spark.
Like most, Alfvaen came to the conclusion she had misheard and let the matter pass, saying instead,
“I s-see. Most of the judges I’ve met have been Hellspark; I guess I do expect the reverse to be true as
well.—You’re a trader, then, or is that also a s-stereotype?”
Tocohl tucked a thumb beneath the black and gold leather of her captain’s baldric and drew it slightly
forward. “I’m a trader, here for the festival. My ship was blessed this morning. And you?”
“I came on an errand for a friend.” Tinling Alfvaen seemed steadier, stood straighter now. She took
several more deep breaths, and gestured a readiness to be on her way. As she followed Tocohl through
the alley to the square, she added, “And if it hadn’t been for you and Judge Darragh, I wouldn’t have
made it this far.”
That only added to Tocohl’s mystification. She stopped to pick up her cloak in passing. From the
scent of it, she knew it had been trampled. Bruised, it was always aromatic but this time it was pungent.
Probably by the guard with the demonstrably heavy feet, she thought, snorting with disgust that owed
more to the guard than the condition of the cloak.
Alfvaen said, “Your cloak was damaged? Perhaps you’d allow me to replace it.”“You couldn’t. There’s only one like this on Sheveschke; customs insists. Don’t worry, it’ll grow
back.” With a critical eye, Tocohl spread it in the torchlight. “In fact, it’s due for a trimming.”
“Grow back? Trimming?”
“It’s a moss cloak. Not moss, to tell the truth, but an epiphyte, a real plant. If I don’t trim it regularly,
one day it will burst into spectacular bloom, seed, and die.” She swirled it across her shoulders, clasped
it, then pointed the direction Maggy indicated. “That way—and go on with your tale. I didn’t mean to
interrupt.”
Alfvaen continued as they walked, “I s-short-hopped my way here, taking whatever transport I could
find when I could find it. While I was on Jannisett, waiting for someone headed this way”—she grinned
with embarrassment—“would you believe somebody s-stole my boots and I was arrested for indecent
exposure?”
Tocohl laughed. “I believe it. A Jannisetti friend of mine once invited me to her private club, where all
the members went barefoot and thought themselves very wicked!”
“Yes,” said Alfvaen with a smile, then more seriously, “but if Judge Darragh hadn’t happened along,
I’d still be in jail.”
They came to a broad avenue, lined with torches and bustling with people. The air was smoky and
pungent; pottery shards crunched beneath their feet at each step.
They pushed through a knot of people, past a woman in the uniform of the local police, and Alfvaen
shivered. “Perhaps you could explain something?” she said, over the noise. “I did read the standard
tourist guide before I got here—and the captain of my last survey was Sheveschkem, so I was chamfered
for Sheveschke, as well.” It was a chamfer’s job to teach one the basics of someone else’s culture, to
avoid any embarrassing or potentially fatal incidents. “He must not have done a very good job: I honestly
thought theft was legal during the festival.”
“In a way. If you’re caught, you have to return what you’ve taken. But there’s no punishment, aside
from the razzing for clumsiness your friends hand you for the next six months.—Of course, taking more
than someone can afford to lose is considered bad form.”
“Then why did the police—” Unable to express her distaste, Alfvaen finished with a gesture.
“You’re confused by a mistranslation,” Tocohl said. “Veschke protects those who steal by verbal
artistry or legerdemain. Skill is all. Anyone who uses brute force—violence or the threat of violence—is
no thief by Sheveschkem standards.”
“So those three weren’t under Veschke’s protection? I see, dastagh”—now that she had sobered,
she came remarkably close to duplicating the Sheveschkem word—“means something like ‘thug’?”
“No, the woman who attacked you was beaten for being an Inheritor of God. Among other things,
they believe that their god gave them dominance over all the other species, and that they’re entitled to use
them, even wipe them out, as they choose. As a philosophy, it’s enough to give an ecologist high-gold
fever. Dastagh is the current derogatory word for a member of the sect; it means ‘waster’.”
The avenue opened onto a great hexagonal plaza, edged with torches and ablaze with the light of a
dozen ritual fires, each attended by a glory-robed priest and her acolytes. Alfvaen stopped short and
gave a wordless exclamation of delight.
(Wait here,) said Maggy, (Geremy’s coming.)
Tocohl was content to wait and, like Alfvaen, drink in the scene. Although she often attended the
Festival of Ste. Veschke, the solemn joy around the fires in Veschke Plaza still elated her.
Despite the crowd’s chatter and the crunch of broken pottery, here it was always quiet enough to
speak in a normal tone of voice, so the traders, both Sheveschkem and Hellspark, gathered to exchange
tales and songs.
A ripple of Apsanti water-music drifted through the smoky air and the laughter, to be picked up by
someone around another fire and tossed back as dolphin song. A black-haired priest threw a double
handful of keshri bark into the central fire and the air grew pungent.
A handful of Sheveschkem youngsters watched Tocohl and Alfvaen for a moment. After much
giggling and gesturing, the smallest of them was urged forward to, shyly, offer Alfvaen a circlet of braided
fair-sea-blues. Alfvaen glanced at Tocohl, who responded, “If you’ll wear it and if you have some smalloff-world token you can give in return, you’ll make it a festival they’ll talk about for the rest of their lives.”
Alfvaen lowered her head to accept the gift, and catching the child’s arm before he could dart away,
she said, “All I have is a brass coin from Jannisett. That’s not very—”
“It’ll do fine.”
Alfvaen looked at Tocohl dubiously, then dipped into an overpocket for the coin. Tocohl stepped an
inch closer to the child, familiar distance here in the south, and said in that language, “She offers you the
Jannisetti truth-coin. The people of that world believe that while one holds this under the tongue, one
cannot lie.”
The child looked from Tocohl to Alfvaen, his eyes very bright and very wide. “Is it true?” he asked.
Tocohl shrugged, Sheveschkem fashion. “At any rate,” she smiled, “one will learn that even truth can
be bitter in the mouth.”
“Oh!” said the child. He took the coin, kissed Alfvaen’s hand, and dashed back to his friends, who
huddled excitedly about to see what he’d been given.
“What did you tell him?” asked Alfvaen. Tocohl translated. When she’d finished, Alfvaen said, “But
won’t they be disappointed when they learn there is no such thing?”
Tocohl grinned. “Being conned by a trader at festival is more an honor than a disappointment.—And
don’t be surprised if, the next time you’re here for festival, someone tries the line on you. The
Sheveschkemen never let a good con go to waste.”
The oldest of the three children waved an arm at Tocohl and called, “In Veschke’s honor,
Hellspark!”
Tocohl smiled and bowed to the child. Then she translated for Tinling Alfvaen, adding, “That is the
polite way of saying she doesn’t believe a word of it, but, since this is festival, she’ll let it pass.”
A thin, wiry man with woeful eyes pushed through the edges of the crowd. He grabbed Tocohl and
swung her around in an enormous hug. “Geremy!” She thumped him joyfully on the shoulders, then
shoved him out at arm’s length for a better look.
He was, as always, a walking work of art. The stylized waves of a darkened sea surged rhythmically
around his 2nd skin to break and spray at the unchanging bulk of his equipment pouch; a handful of
sparks blew past, trailing their reflections in the dark waters. The design was locally generated by a
microprocessor in the suit itself.
“Very nice,” said Tocohl, turning him around to follow the course of the sparks as they blew beneath
his baldric and reappeared on the other side. “Very nice indeed.”
(I could do that with your 2nd skin, if you like,) Maggy said.
(I’d like, but Geremy wouldn’t. I promise, I’ll explain later.) Aloud Tocohl said to Geremy, “Is that
really a Ribeiro?”
“It is, and when Ribeiro took the commission, she said she’d been thinking about the subject for a
long time.” He folded his arms (along them stylized waves crashed soundlessly) and eyed her with
suspicion. “Maggy said you needed a doctor, but you look disgustingly healthy to me.”
“For the Siveyn here.” Tocohl drew Geremy around the two large merrymakers who hid Alfvaen
from his view, but before she could begin a formal introduction, Geremy said, “Alfvaen? What
happened?”
“She took a very nasty beating,” Tocohl said.
Geremy backed off a pace and looked with hurt astonishment at Tocohl. “You?” he said, once more
in Hellspark. “Listen, Tocohl, about that judgment—”
“She knows no more about it than I do,” said Tocohl, then caught the import of his first reaction.
“Geremy, don’t be stupid. I haven’t changed that much since the last time we worked together!” She
gestured at Alfvaen: “Please, look her over.”
Chastened, Geremy shifted back to Siveyn to offer his professional services.
“Your pardon, Geremy, but I’ve already been s-seen to by a doctor,” said Alfvaen.
“I know. Maggy told me he was a quack—honestly, Tocohl, I don’t know where she picks up these
words!”
(Any good dictionary has them,) Maggy said.Tocohl laughed and repeated that for Geremy’s benefit. Then she added, “I’d feel more comfortable
if Geremy assured me of your health, Alfvaen—then we’ll see to finding Judge Darragh.”
While Geremy went professional, Tocohl excused herself to approach the festival fire. All the curious
events of the past few hours vanished from her mind, pushed away by heat and flame and the sound of
shattering pottery...
The priest’s glory robe was orange velvet—the highest of her sect—and she wore the firecrown of
her office with surpassing dignity. Tocohl dropped to one knee before her, spread her arms wide, and
spoke the ritual words: “I come for fire.”
“For Veschke’s fire, one must shed blood,” responded the priest.
“As it must be, let it be.”
The priest sketched Veschke’s sign in the smoky air above her head. “Rise then, and choose.”
An acolyte held a tray of pins before Tocohl. Each bore a different emblem at its head: the pin of
remembrance, the pin of dreams-come-true, the pin of smooth tongues...
On impulse, Tocohl chose the pin of high-change: its emblem was a face in flame. She dropped a
coin in its place. The remaining pins jangled suddenly. Tocohl’s hand shot out to steady the tray and she
looked into the acolyte’s startled eyes and gave a reassuring sign.
The youngster was unaccustomed to the Hellspark penchant for risk—a glance at the priest’s face
confirmed this. The priest drew the girl away to speak quietly to her.
And Tocohl stood alone before the fire. As she held her right hand high, the 2nd skin fell back into a
cuff. She lifted the pin of high-change—it flashed as if of its own accord—and a great drop of blood
welled from her fingertip. She shook the drop onto the broad circle of cast iron in the center of the fire,
where it spat a moment, then was gone.
“Veschke’s fire,” she said softly, “taste my blood that you might hunger for it, that you might seek it
out and devour it. Burn me to the bone and lift my living ashes into the sailing wind to light the way for
those who come behind. As Veschke’s sparks fly with the wind, let me follow.”
Chapter Two
First Judgment
TOCOHL MADE VESCHKE’S sign, turned, and walked away from the fire, her hands and face
still burning from the blistering heat. Only then did she realize that Maggy had recited the ritual words with
her.
(So,) said Tocohl, (we share the pin of high-change.) She used the Hellspark tight-we, the pronoun
reserved for two or more acting as one.
(Did I do wrong?)
(No. We share our fortune, as usual.) Tocohl laced the pin of high-change into a tuft of her cloak.
A second acolyte gestured her to the cauldron of stew, where she turned away a bowl, having eaten
earlier, and accepted a ritual cup. The stew was thick and savory, and she finished quickly, then dashed
the red clay cup to the ground. It shattered with a satisfying crash. By the end of festival week, the
cobblestones of the town would be grouted with the rough red dust of a hundred thousand such cups and
bowls. Like the other captains, she’d carry the dust aboard her ship and count it Veschke’s blessing.
Though luck had little to do with it, she thought. The soles of her 2nd skin were still covered with
it—Maggy had been reading up on her subject indeed, or she would have cleaned them.
(Well done, Maggy,) she said, pleased.
(Thank you,) came the reply, then: (Geremy and Alfvaen are twenty paces from your right elbow.
Thirty if you walk around the cooking fire.)
Tocohl turned her head to line her sight with her right elbow. As the crowd eddied, she saw Geremy
and Alfvaen and a third Hellspark beyond one of the small cooking fires. She strode to join them.
“Well?” she demanded of Geremy.
“She’s fine,” he responded, “aside from a case of Cana’s disease: that leaves her—”“Slightly tipsy at the worst possible times; I saw. Though in this crowd nobody will notice.” Tocohl
pointed, “Pass the flagon and we’ll all catch up.”
The woman holding the flagon offered it with a smile.
She was old, thought Tocohl, with admiration. She had a face worn into comfortableness, seamed
and tanned; her hair was fine and white. There was a mischievous look about her brilliant blue eyes to
which Tocohl took an immediate liking. She smiled back and accepted the flagon, to find the woman had
exceptional taste in dOrnano wine as well.
Alfvaen lifted her hands in the Siveyn formal gesture, fringe trickling from her arms, and said with
affection, “Tocohl, this is Judge Darragh Nevelen.—Judge, this is Susumo Tocohl, the woman I was
telling you about.”
(Geremy,) Tocohl said, for Maggy alone. Her glance swept from Geremy to Darragh and back
again. (Alfvaen had nothing to do with those charges. It was Geremy! I’m going to have him for
breakfast—)
Maggy interjected, (Cannibalism—)
(Right after I’m done with the judge here,) Tocohl went on, overriding Maggy’s attempt to warn her
of the illegalities of cannibalism.
“Your pardon, Alfvaen,” said Tocohl aloud. “Do you understand the language of Dusty Sunday?”
“No,” said Alfvaen, and Tocohl continued, “May I speak it in your presence without giving offense?”
Puzzled, Alfvaen nevertheless granted her permission, and Tocohl shifted her stance to the language.
So did Nevelen Darragh—the woman was good, thought Tocohl.
Judge Darragh slid her spectacles into her hair. Tocohl did not follow suit. On Dusty Sunday, wearing
one’s spectacles in conversation was a deliberate insult. It said plainly that one would rather be listening
to someone else, watching someone else. Nevelen Darragh flushed a vivid scarlet.
When the red had reached the very tips of Darragh’s ears, Tocohl added coolly, “Madame, I expect
an explanation; I do not, of course, expect it to be adequate.”
Nevelen Darragh stared hard at Tocohl for a long moment—then, with a burst of laughter, she
bowed her appreciation.
Geremy said, “I told you she was good, Nevelen.”
“The incident on Solomon’s Seal told me that, Geremy. But she’s better than you know.” Judge
Darragh laughed again. “You haven’t been on Dusty Sunday recently, I take it?”
“Not for ten years,” Geremy said mournfully.
“Then I’d better tell you that what your friend just did was the exact emotional equivalent of ‘In
Veschke’s honor’. She smiled again at Tocohl: “I’m pleased to hear there are no hard feelings.”
Glancing sidelong with mock menace, Tocohl said, “I haven’t had my shot at Geremy yet.” She took
a long pull from the flagon of wine and contemplated him, measuring him until he squirmed with
discomfort. “Perhaps some other time,” she said at last, “when he’s least expecting it.”
More woeful of face than ever, Geremy said, “I’m sorry, Tocohl. Maggy wasn’t letting anybody
through to you. I was the one who suggested that a judgment might override her orders.”
“I’ll bet you told her it was business,” said Tocohl dryly.
Geremy looked abashed. “I didn’t talk to her; Garbo did. And you’re right, the message said
business.”
“If you’d put your money into your equipment, instead of on your back”—Tocohl’s finger traced the
path of the sparks briefly along his arm—“your computer wouldn’t be so damn dumb and it’d do more
than deliver messages verbatim.”
She raised the flagon again, then passed it to Geremy, who hesitated before taking it. “Oh,
Geremy... In Veschke’s honor, then.” At that his eyes brightened within his sad-clown features, and he
accepted the wine to drink deeply his relief. “All right,” Tocohl went on, “let’s talk business and be done
with it so I can get back to celebrating.”
“Your business is with Alfvaen,” said Nevelen Darragh.
Tocohl crossed her ankles, seated herself beside Tinling Alfvaen, and said in Siveyn, “Were you
aware, Alfvaen, that you’d called for a judgment against me?”Alfvaen, startled, splayed a hand at her throat. “I have no quarrel with you, Tocohl.”
“Nor I with you,” said Tocohl. “Someone”—two quick thrusts of an elbow indicated a choice of
Geremy Kantyka or Darragh—“owes us both an explanation.”
It was Judge Darragh who spoke. “Tinling Alfvaen came to the Festival of Ste. Veschke to find a
glossi. As Geremy explained, the judgment was a way to contact you, nothing more.”
Maggy made a pinging noise. Tocohl held up a hand and said, (Yes, Maggy?)
(The judgment has been cancelled. There’s a formal statement. Do you want to hear it?)
(Not necessary,) said Tocohl; aloud, she said to Alfvaen, “All right. I’m here. What is it you want of
me?”
“You’re a polyglot?” Alfvaen asked.
“Glossi,” Geremy corrected, “—from an old, old word meaning ‘speaker of tongues.’ There’s some
evidence that an espability is involved, and if it is, Tocohl’s got it.”
“Your pardon, Tocohl,” said Alfvaen. “You are a glossi?”
“Yes, although Geremy exaggerates. As far as I know, I have a good ear and a good eye, not an
espability.”
Alfvaen looked at her intently, then said, “You were tricked into coming to meet me. I apologize and
I will fulfill any ritual you think just.”
Tocohl gave a reassuring smile. “It was only a theft at festival, as the Sheveschkemen say, and
Geremy’s theft at that.” From the corner of her eye, she saw Geremy flinch quite satisfactorily. To
Alfvaen, she added, “You gave no offense, I take none.”
“Then please hear me out.” Alfvaen leaned forward.
“I’m listening,” said Tocohl, surprised by the small woman’s sudden urgency.
“When I was with Multi-Galactic Enterprises,” Alfvaen began, “I spent a good many years working
with swift-Kalat twis Jalakat of Jenje—perhaps you’ve heard of him?”
“Yes—considered by some to be the best survey ethologist in the business, considered by most to
be ‘crazy as a Hellspark.’ Go on.”
Tinling Alfvaen did. “Swift-Kalat is three years into the survey of a world named Lassti. He has
reason to believe that the planet has a sentient life-form and should be declared off-limits to exploitation
and colonization. The problem is that the survey team’s polyglot—I don’t think I’d call him a glossi,
Geremy—hasn’t been able to make sense of the language.”
“After three years?” said Tocohl. “That is odd. So MGE wants to hire a glossi?”
“No,” said Alfvaen, “swift-Kalat does. He’ll pay your fee.”
Maggy made a querying noise.
“Let me think a moment,” Tocohl said, and explained to Maggy, (It’s not illegal for him to hire
outside talent, especially not with a byworld judge involved, but MGE certainly won’t like it!)
(Do we care?) Maggy asked, using tight-we.
(No, not about MGE’s likes and dislikes. But MGE has a good deal of power on some of the worlds
we trade on, and they could make our lives considerably more difficult. Suppose we do prove
sentience—then MGE has wasted three or more years of a survey team’s time without any return; and
that they’d like even less!)
(So the system works against proving sentience?)
(In a way, yes. You can’t prove sentience without proving a species has a language, but the MGE
polyglots are damn good, usually, and regulation is strict. For the most part, I’d say it’s honest—though
you could probably quote me chapter and verse on honest mistakes that have destroyed cultures.)
(Should I?)
(Skip it. I can think of a couple of nasty examples myself. Maybe we should take this job.)
(Maybe?) said Maggy.
Tocohl smiled. (You’re getting awfully good at holding up your end of the conversation!)
(Thank you,) said Maggy, primly.
(Maybe,) Tocohl repeated. (Swift-Kalat is the survey team’s ethologist. That, and his “swift” status,
give him a lot of credence, but I’d be happier if the polyglot had asked for a glossi.)(I have forty-three files that quote swift-Kalat as the highest authority on ethology. Would you like a
random sampling?)
(No, I concede his expertise. Let me find out more.) Tocohl said aloud, “Stepping into that kind of
situation is asking for trouble, whatever the outcome.”
Tinling Alfvaen said, earnestly, “Tocohl, swift-Kalat is Jenji. The Jenji don’t lie.” That was
conventional wisdom on many worlds, but to Alfvaen the belief seemed to go beyond convention to a
personal conviction. “I’ve known Jaef for a long time—”
Tocohl raised an eyebrow and said, “That you’re entitled to use his soft-name is proof of that.” (And
proof of a strong bond between the two,) she added for Maggy’s benefit.
“—And if he says the species is sentient, I believe him,” Alfvaen finished, “but you must help him
prove it.” She reached into a pocket of her kilt and drew out a folded piece of gold paper. Without a
further word, she offered it to Tocohl.
Tocohl took the paper, unfolded it. The startling boldness of Jenji script seemed to leap from the
page: three lines and the signature, swift-Kalat twis Jalakat of Jenje.
“Geremy,” Tocohl said quietly as she refolded the paper and returned it, “are you free to take a
cargo of winterspice and tapes to dOrnano for me?”
Geremy turned. “What do you say, Nevelen—can you spare me for a few weeks?”
So Geremy was acting as the judge’s aide, Tocohl thought. That explained much. She would have
withdrawn her request, but Darragh spoke first: “I haven’t been to dOrnano for years, Geremy. I’ll go
with you.”
Geremy turned again to Tocohl. “After festival?” he said.
“Of course,” Tocohl replied, and because it was Geremy, their dickering was pro forma. In only a
few moments they had snapped fingers to close the deal.
Alfvaen’s face lit as she realized the import of this exchange. “You’ll go!” she said and looked down
at the paper in her hand. “He told me this would convince the kind of person he needed. I don’t
understand why, but I’m glad.”
Tocohl said, “To say ‘I know’ in Jenji, you must specify how you know. You have a choice of
degree—firsthand experience, inference, hearsay, to name just a few of the options—and each tells your
listener how reliable you think your information and why. That, in turn, reflects on your reliability. The
language is also backed up by strong cultural penalties for using the wrong degree, and a religious belief
that you may, by lying, inadvertently create a truth that would do no one any good.”
Tocohl touched the edge of the paper in Alfvaen’s hand and went on, “He tells me here that he has in
his hands an artifact, and from this artifact he deduces the presence of sentient life—anyone might have
written that in any language. In Jenji what swift-Kalat wrote is very complex and very precise. The
degree of his surety is so high that if I were MGE, I’d pack up the survey team and go home.”
“In three sentences?” Alfvaen unfolded the paper and stared at it in wonder.
“Four,” Tocohl said. “He signed his name—and that puts his status on the line. If he’s wrong about
this, he’ll be forced to drop his ‘swift’ status. And that’s the social equivalent of your going into your
hometown and admitting to la’ista.”
Alfvaen’s eyes widened still farther. “He’s that sure?”
“He’s that sure,” Tocohl said, but before she could say anything further, a group of Sheveschkemen
passed and their cheerful singing momentarily brought conversation to a halt. The song was a lengthy and
awe-inspiring detailing of Ste. Veschke’s sexual adventures.
One gorgeously drunken woman in the green leather baldric of a trading captain leaned down, her
slim hand on Geremy’s shoulder. “You look too solemn,” she told him, “Veschke made a merry blaze
even when she burned!” She pointed to the flagon in his lap; “Drink to Veschke!” she commanded over
the singing.
Geremy raised the flagon, clinked it against the captain’s. She said, grinning hugely, “Veschke was a
Hellspark!”
Geremy shouted his laughter. “To Veschke, then!” he said, and took a long drink. Then the
Sheveschkemen passed on, still singing as they went.Tocohl turned back.
Alfvaen had drawn up her knee and wrapped herself disconsolately about it. Her green eyes were
dark and sad.
“Alfvaen,” said Tocohl, “is something wrong?”
The Siveyn answered slowly. “I didn’t know how strongly he felt about the situation. I should have. I
should have! Tocohl, you’ve never even met him and you know more about him than I do!”
“No,” said Tocohl, “you believed him. Fancy words and fine phrasing are necessary only to convince
strangers.”
“Yes, but—” Alfvaen raised her small hands and grasped air. “I have to learn the words as well.”
She fell silent, but her hands remained clenched.
After a moment, Darragh said, “There’s another matter of language, Tocohl. Before you accept this
job, you should know what happened on Jannisett.”
Keeping a watchful eye on Alfvaen, Tocohl said,
“She told me. She was arrested and held, until you straightened the matter out.”
“She was expertly framed,” Nevelen Darragh said, “—and the man who framed her was a crayden.”
Tocohl stiffened in surprise. She said nothing, but Nevelen Darragh’s expression was all she needed
to confirm that she’d heard correctly.
“I see,” said Tocohl, and she laid her hand on Alfvaen’s shoulder, partly to comfort, partly to draw
Alfvaen from her preoccupation. “What are your plans, now that your message is delivered?”
Alfvaen stared at her, uncomprehending. Then she blinked as if come into sudden sunlight.
“A moment ago, I had none,” she said. “MGE dropped my contract after Inumaru.—I don’t blame
them much. Even I find it hard to believe that catching Cana’s disease could be serendipitous.”
She reached across and caught Tocohl’s wrist. “Take me with you,” she said. “Teach me the words I
need.”
“I was hoping you’d say that,” said Tocohl. “MGE may have doubts about your serendipity, but I
don’t. You know how to deal with members of a survey team,” she paused, then added, “and you might
be safer on Lassti than anywhere else.”
“Safer?” Alfvaen asked. “I don’t have the words to understand that, either.”
“You need just one: the Jannisetti word crayden. It is an exact translation of the Sheveschkem
dastagh—waster. As the Jannisetti say themselves, ‘Once a thing happens twice, you must think about it
three times.’” Tocohl stood. “Geremy, I’m sorry to break up your evening, but could we get that cargo
transferred now? Under the circumstances, I’d prefer to leave immediately.”
Despite its value, the cargo was small and compact. Even Nevelen Darragh pitched in to help, and
the transfer of tapes and winterspice went quickly. As a courtesy, Tocohl registered her new destination
with Sheveschkem traffic control, giving the coordinates Alfvaen had received from swift-Kalat.
“I’ll bet traffic control loved that!” said Geremy. “I take it the captain of the survey team is
Sheveschkem?” he asked Alfvaen. The naming of a new world was often the captain’s privilege.
“Yes,” said Alfvaen. “What’s so funny?”
“Lassti means ‘Flashfever,’” Geremy explained. “It’s a local disease—as common on Sheveschke as
a cold—characterized by bizarre visual effects.”
“It’s like being slugged in the side of the head and seeing sparks,” Tocohl put in. “I know. I had it
once.”
Geremy went on, “The very religious call it ‘the Fist of Veschke’ and would say Tocohl had been
punished for her many sins.”
Tocohl waggled a handful of fingers at Geremy insolently, then got on with the business of rearranging
the interior panels to create a cabin for Alfvaen.
“—That must be some planet!” Tocohl finished.
Maggy pinged for attention. (Two passengers for Flashfever,) she said.
(Passengers?) said Tocohl in surprise. Only rare circumstances would take people to a world still
under survey. Visiting was inadvisable, not illegal, and one stayed only as long as one’s transport stayed.
Transport stayed at the discretion of the survey captain.(Let’s have a look at them,) Tocohl instructed. (Put them on the screen.)
Maggy complied. A man and a woman, Sheveschkemen, appeared on the small screen. Both wore
severely cut green jumpsuits, lacking any adornment. The expressions they wore were equally severe.
The woman spoke in GalLing’. “Captain Susumo? We wish to book passage to Flashfever for the
duration of your stay.”
That should have been icing on the cake. Tocohl’s expenses were already covered by swift-Kalat.
Two passengers would double her profit—and yet Tocohl did not immediately reply. Something about
the two disturbed her. Then she suddenly had it: here were two Sheveschkemen prepared to leave their
planet before the end of festival and neither wore a pin of any sort. Neither had gone for Veschke’s Fire!
(Not worth the risk,) thought Tocohl and only realized she’d subvocalized when Maggy said, (What
risk?)
(The Inheritors of God don’t participate in “pagan” rituals. In fact, I imagine they’d find Veschke
particularly hateful. She burned for her refusal to give Sheveschke’s coordinates to exploitive
second-wave colonizers,) Tocohl explained briefly. (I won’t risk having Inheritors aboard while Alfvaen
is with us. She’s been attacked twice. The first time on Jannisett, when she was framed. The second we
interrupted. And one could deduce that the attacks were escalating.) She said aloud, “You’ll have to find
other transport. I’m taking no passengers this trip.”
“What about the Siveyn?” demanded the woman.
Tocohl glanced coolly around. Only Geremy was within range of the visual pickup. Beside the
stow-webbing, Alfvaen looked at the screen with surprise and started forward. Nevelen Darragh
stopped her with a swift hand on her shoulder.
Geremy, the best of accomplices, shrugged one hand at Tocohl and looked puzzled. Tocohl turned
back to the Sheveschkemen. “What about what Siveyn?” she asked, with an innocence of expression
she’d been practicing since the age of two.
Unlike Geremy, the woman was no actor. Realizing her error, she inhaled sharply and turned from
the screen in an effort to hide her self-reproach.
Her companion elbowed her aside and began, in a conciliatory fashion, “A friend of ours at traffic
control said you’d registered for Flashfever, Captain, and he said something about a Siveyn, so we
naturally thought he meant you were taking passengers...”
(Ping!)
(Yes, Maggy.)
(You said nothing to traffic control about Tinling Alfvaen.)
(I know. Later, Maggy.)
The Sheveschkemen finished, “... Perhaps we were given the wrong ship, then.” The woman had
disappeared from view, and the man glanced off-screen, paused, then said, “May I ask your destination,
Captain?”
“My destination is Flashfever,” said Tocohl. There was no point in lying—the woman was probably
double-checking now.
“Then it’s only a matter of the fee,” said the Sheveschkemen. “I’m sure we can arrange something
that will satisfy you.”
“No passengers,” repeated Tocohl.
The Sheveschkem woman returned, angry. “I see no reason for you to deny us passage,” she said.
“We will ask for a judgment!”
Nevelen Darragh stepped into visual range. “Ask then,” she said, “I am Byworld Judge Nevelen
Darragh, and I will consider the problem. I must, however, point out that it will probably be a waste of
your money: in all but the most exceptional circumstances, the captain has the ultimate say in what occurs
on her vessel, whether it be fishing ketch or starship.”
Nothing had prepared Tocohl for the professional Darragh. The transformation reminded her of the
first time she had seen a Bluesippan dress dagger drawn, the sudden startling realization that the dagger
was fully practical. The judge was layered steel, glittering and razor-sharp. Tocohl was impressed.
The two Sheveschkemen were equally impressed and more than a little unnerved. They made hastyprivate consultation. When they turned the sound on once again, the man said, “It is, after all, the
captain’s privilege. If she wishes to lose income...” He shrugged and went on, “Will she agree to carry a
letter for us?”
“Of course,” said Tocohl instantly. There she had no choice. Automated message capsules were
expensive, so the only reasonably priced interstellar communication was through traders. Mail was
always accepted.
And the refusal of mail could cause a judgment that would most certainly detain Tocohl and might
well go against her.
Perhaps that was what the Sheveschkemen had in mind. At any rate, he seemed disappointed at her
agreement, but said, “We’ll send it up on shuttle.”
“Make it soon. We leave within the hour.”
The image vanished.
Nevelen Darragh said, “I wouldn’t wait, Tocohl.”
“I wasn’t planning to,” Tocohl said with a smile. “Thanks, though.”
Nevelen clapped Alfvaen on the shoulder, turned to Geremy and said, “Come on, we’re wasting
valuable festival time!”
They walked to the hatch, where Geremy wrapped Tocohl in a farewell bear hug. “Say hello to Bayd
and Si for me, will you?” Tocohl said into his shoulder. “Tell them I’m sorry I missed them and I’ll see
them next year for sure.” She tucked a tape into his equipment pouch and patted it as she stepped back.
“Tape for them. See they get it.”
“I will.”
Nevelen Darragh looked on, then fixed her piercing blue eyes on Tocohl one last time and Tocohl
again sensed the steel behind them. “One question, Tocohl.”
“Question, yes. Answer? Ask and we’ll see.”
Darragh laughed but her eyes did not change. “Why did you take farm equipment to Solomon’s
Seal?”
At that, Tocohl laughed. “Because that was what they needed.”
“That’s the answer I expected. It’s been a pleasure meeting you.” She gave a Hellspark two-finger
salute and hustled Geremy through the port.
The inner hatch closed, and Tocohl led Tinling Alfvaen forward to ship’s control.
Chapter Three
THE SHIP’S CONTROL was a spacious room bright with telltales and—because it also served
as captain’s quarters—tapestries, a hammock, a jumble of paintings, and Tocohl’s small but treasured
collection of hardbooks.
(Hop to, Maggy,) said Tocohl, (let’s program that jump—)
(I have a message from Geremy.)
Settling herself at the control console, Tocohl said, (Tell me.)
Maggy complied in Geremy’s own voice: (Tocohl, I’m not the one who told Darragh about
Solomon’s Seal. I swear it.)
(Interesting,) Tocohl commented.
When she said nothing more, Maggy said, (Geremy said good-bye to me, too.)
(Shouldn’t he have?)
(The judge didn’t.)
(The judge was never introduced to you. I’m sorry, Maggy, I must be getting forgetful in my old age.)
(You are only 103. If you are forgetful, it has nothing to do with old age.)
(That was just an expression, Maggy.)
Alfvaen took the seat Tocohl indicated, then glanced at Tocohl curiously and said, “I have no wish to
intrude, nor to violate a Hellspark taboo, but you seem to be listening to something. Judge Darragh andGeremy often gave the same impression.”
(This time I won’t forget, and you can practice your Siveyn, Maggy.) Tocohl tapped the spot just
before her ear and said aloud, “No mystery and no taboo: I have an implanted transceiver.” She made
the formal Siveyn gesture and said, “Alfvaen, may I introduce Lord Lynn Margaret—lord is a title,
something like swift-, but its use is not obligatory in this case. Maggy, Tinling Alfvaen.”
Maggy, as prim in Siveyn as she was in Hellspark, said, “I’m very pleased to make your
acquaintance, Tinling Alfvaen.”
Her brow furrowed, Alfvaen half rose to survey the room. Puzzled, she said, “And I yours,
Maggy—but where are you?”
Maggy chuckled. “All around you,” she said.
Tocohl sat down to the controls. Her fingers danced over the keyboard, then paused as she said, “I
like that chuckle—where did you get that?”
“It’s yours; I changed the pitch to match my voice range. Did I use it correctly?”
“Perfectly,” said Tocohl. Her fingers danced a second time. “What made you decide to use it?”
“You smiled,” said Maggy, once again.
Alfvaen’s features went from total bewilderment to sudden comprehension. “You’re the ship’s
computer?—But you sound human!”
“I’m not,” said Maggy, “I’m only a top-class extrapolative computer with a larger memory bank than
most.”
“You needn’t say only,” Tocohl commented, checking a bank of indicators. “It doesn’t seem
applicable to you.”
“All right,” said Maggy. “I’m a top-class extrapolative computer with a larger memory bank than
most.”
“That’s better.” Tocohl glanced over her shoulder to address Alfvaen: “If you find talking to a
disembodied voice bothers you, Maggy can always activate a small mobile.”
Alfvaen thought for a moment. “That’s not necessary; I’ll get used to it.”
“Will you please correct me?” Maggy asked.
Tocohl gave two final taps to the keyboard, straightened, and turned to face Alfvaen. “Her Siveyn
consists of a basic grammar and an enormous vocabulary to plug into it. She’s had no practical
experience in conversation and she wants you to correct her usage. She learns like a kid does, except
that she only needs to be told once. Anything you tell her she stores for later use; her entire program
undergoes constant revision.
“—I think it helps to tell her when she does something right, too,” Tocohl grinned, “like that chuckle.”
Alfvaen looked around her again. “I’ll be glad to help, Maggy.”
Tocohl said, “Treat her as if she were a friend looking over your shoulder. Believe me, unless you tell
her not to, she is always looking over your shoulder!”
“I will not violate Siveyn taboos,” Maggy said. There was a moment’s pause—obviously supplied for
esthetic reasons—then she added, “Alfvaen, I have a large selection of Siveyn literature. I can read or
display it to you anywhere in the ship. All you need do is ask.”
“How will I know if I’m interrupting your duties?”
“That’s no problem. I can do several things at once.” As if to demonstrate the truth of that, Maggy
pinged and announced, “Jump programmed, Tocohl. Ready when you are.”
“Ready,” said Alfvaen. Tocohl turned back to her console and said, “Then let’s go, Maggy.”
Ordinarily, Tocohl would have done the programming herself, using Maggy only as a double-check,
but that was merely a matter of keeping her hand in. Since they were in something of a hurry, she let
Maggy do it and set them on their way. Then she went through the programming herself for the practice.
Interstellar flight was mostly a matter of long days of waiting, punctuated by an occasional flurry of
programming the next hop. Flashfever, by Tocohl’s estimate, was a hop, three steps, a hop and a skip
away. First hop accomplished, Tocohl worked out the first of the three steps, then, satisfied to find that
Maggy had opted for the same route she would have, she swung her chair.
Alfvaen, she saw, had temporarily chosen to address Maggy as if she were hidden in the blankscreen on the far wall. Not a bad choice, that, since it contained one of the sensor banks that Maggy
used to watch the control room.
She was saying, “I’m sure I wasn’t drunk enough that my ears were playing tricks on me. It sounded
as if Tocohl said, first, ‘Hell Spark,’ and then, ‘Hell’s Park,’ when she talked about her people.”
“She did,” said Maggy.
“But which is it?”
“That’s a state secret,” said Tocohl.
“That’s a joke, Alfvaen,” Maggy said, her prim tone making it sound much like a child’s confidence,
“Tocohl told me.”
Tocohl grinned. “So it is, but you’re entitled to use the joke too, Maggy.”
“All right, but Alfvaen wants to know. She wants to get it right.”
“I accept your reasoning,” Tocohl said. “Alfvaen, the correct pronunciation is to alternate the two
pronunciations—to use first one, then the other, even in the same sentence.”
“How odd. Why?”
“For the same reason anybody does anything in any language: because.”
“That’s not enough reason,” Maggy said, sounding primly offended.
“I know, Maggy; but that’s all the reason there is in most cases. In the case of Hellspark, well, since
that was originally an artificially created language, there’s a bit more reason. The alternation I think was
intended to remind you of the need to be flexible in language. If so, it’s failed in a way. I can no more use
two Hell’s-parks followed by a Hell-spark than some people can learn to alternate them every time. So I
don’t think it achieves the desired result, but it’s retained as a joke all by itself—even without the ‘state
secret’ line.”
Alfvaen added thoughtfully, “Every Siveyn I ever met pronounces it hell-spark. I suppose that’s
because, too.”
“Hell-spark means something in Siveyn, while hell’s-park is only nonsense syllables. One tries to
make any new word fit the parameters one is accustomed to. When I speak Siveyn, I pronounce it
hell-spark, too.”
“You didn’t that time,” said Maggy.
Tocohl considered this. “Had I been speaking Hellspark to you, Maggy, while I was talking to
Alfvaen?”
“Yes.”
“That would account for it, then. A holdover from language to language. Tell me if you catch me
doing that again. It’s bad practice.” This last was directed at Alfvaen as well.
Speaking very slowly and very carefully, Alfvaen said, “But I’d like to try doing it anyway, even if it’s
bad practice for a Hell-spark. It seems common courtesy to pronounce ‘Hell’s-park’ the way a
Hell-spark would.” Her green eyes lit with pleasure. “That’s not easy!”
“No, it’s not,” Tocohl agreed. She rose, crossed the room to hang her cloak near the best source of
light, and said, “If you can hear it and, better still, do it, then I think I have a good pupil. That is, if you’re
still interested in a crash course in Jenji, Alfvaen?”
Alfvaen came to attention instantly, so eager in manner that her words were unnecessary. She said
them anyway, “Oh, yes! Please!” then looked momentarily worried. “I haven’t much to pay you, not after
passage.”
“Passage is for acting as my liaison with the survey team. As for payment for language lessons... if
you’re helping Maggy with her Siveyn, I’ll consider it even.”
(She can pay you by teaching me?) Maggy inquired privately.
(Anything you learn is to my advantage. And it has never seemed right to me to charge for such a
basic tool as language.)
(I think I understand.)
Alfvaen began, “Does Maggy—is that all right with you, Maggy?”
“Of course,” said Maggy, this time aloud. “If I learn Siveyn, I can help Tocohl.”
It was so much like a small child’s absolute assurance that Tocohl couldn’t help but smile. “That’ssettled, then,” she said. “We might as well get started. On your feet, Alfvaen.”
Alfvaen looked at her with surprise.
“Up,” Tocohl said. “You were expecting the basics, weren’t you? Well, they aren’t hello, good-bye,
please, thank you, and Where’s the bathroom?”
“They aren’t?” Alfvaen came immediately to her feet. “Jenji is that different from Siveyn?”
“Not in the sense you mean,” Tocohl said, “but Hellspark language lessons always start with the
proxemics and kinesics of a new language. The earliest of the old Hellspark proverbs is ‘The dance is
sweeter than the song.’
“Let me give you a practical demonstration.” Tocohl glanced down, indicated a broad yellow stripe
that halved the tapestry beneath their feet. “Stand with your toes touching that. If at all possible, I want
you to remain with your toes touching that, and I want you to tell me what you’re feeling while I talk to
you.”
Alfvaen, despite her puzzled look, arranged herself carefully. Tocohl took a step forward and greeted
her formally in Siveyn. Alfvaen responded instantly in kind.
“Look at your toes,” Tocohl said.
“Still on the line, but I...”
“Bear with me. How did I greet you?”
Alfvaen gave this some thought. “I’m not sure I understand your question, Tocohl. You greeted me
as if you were Siveyn, you know that.”
“Aggressively? As if I were a long-lost friend?”
“Neither. As if you were... Tocohl. Just as you are.”
Tocohl pursed her lips slightly. “All right. Keep your toes on the line. I’m going to do it again.” This
time the language she chose was Jannisetti, and it required a step backward on Tocohl’s part to greet
Alfvaen formally.
Alfvaen had clearly learned her hello, please, and thank you in Jannisetti, for she responded in good
kind to the greeting. Her accent was impeccable, but she stepped a full two inches across the line.
“Toes,” said Tocohl. Alfvaen looked down, her eyes widening in astonishment.
“Why did you step forward?”
“I don’t know,” she said, stepping back to stare at the line as if it had somehow moved from under
her.
“Try again,” Tocohl said. Alfvaen fixed a corner of her eye on the line and readied herself visibly.
Again Tocohl greeted her formally in Jannisetti, and again Alfvaen moved forward. This time, however,
she caught herself in midstep.
With great deliberation, she set her foot back, glared at Tocohl, and responded to her greeting in
clipped tones. Then suddenly her anger was gone, lost in the interest she gave to her feet.
“First lesson,” Tocohl said. “Why were you angry?”
“You backed away from me, as if I were diseased.” She was still staring at her feet.
“No,” said Tocohl, “I did not. I greeted you in exactly the same way in both Siveyn and Jannisetti.’
“But you didn’t, Tocohl. In Jannisetti, you—” She closed her mouth abruptly. She stared up at
Tocohl “On Jannisetti, they all backed away from me!”
“Are you that offensive?” Tocohl grinned at her. “I didn’t think so.”
“You thought so in Jannisetti! You backed away! Why, Tocohl?”
Shifting back to Siveyn, Tocohl said, “I’ll show you the emotional equivalent of what you did to the
Jannisetti in Siveyn. Try toeing that line through this... !” she challenged. Once more, she greeted Alfvaen
in her own language. While the words were formal, her movements were not—instead of the requisite
one step forward, Tocohl took two.
And Alfvaen instantly backed away from her.
Tocohl waited patiently where she was, making no further move that could be interpreted as
aggressive.
After a long moment, Alfvaen again looked down at her toes, taking in the distance she had moved
from her mark. She said, “You came at me!”“And why do you suppose the Jannisetti all stepped back?”
Alfvaen stared at her feet in an embarrassed fashion. “Oh, Tocohl,” she said at last, “do you mean
that every time I said hello—and thought I said it in a friendly way in their own tongue—I was... jumping
at them the way you jumped at me?”
“I’m afraid so, yes.”
“But why didn’t someone tell me?”
“Because it’s one of the hardest things in the world to tell. You interpret both spacing and gesture on
a subconscious level as you’ve been trained to interpret them by your culture. In fact, there are at least
six different sets of proxemics and kinesics on Sivy alone, and you’d be hard put to get one of the others
correctly. The fact that, to the ear, you all speak the same tongue, makes it all the more liable to
misinterpretation.”
Alfvaen sat down, put her chin in her hand. After a long time, she said, “I’ve seen it, and I didn’t
know what it was. And if it’s that difficult between two people who’ve known each other all their lives...
Tocohl, perhaps I’ve misinterpreted Jaef altogether.”
“I doubt it, or you’d be calling him ‘swift-Kalat’ like the rest of us. He gave you his soft-name,
Alfvaen; that’s a very good indication of how he feels about you.” For Maggy’s benefit, Tocohl added,
“And I’d say you felt the same way, even though you had no soft-name to give in return.”
Tocohl knelt to look her straight in the eye. “You have a good ear, and you can catch on quickly to
the visual aspects.” She grinned. “And you have a better motivation than most to learn. I’d bet money
you can get all the basics on a conscious level by the time we get to Flashfever. If you’re willing, that
is...?”
“Willing?” And once again, Alfvaen was on her feet. In three steps, she’d set her toes once more
against the broad stripe in the carpet. “All right,” she said, “show me how it works in Jenji. I swear I
won’t move an inch.”
Tocohl laughed. “You probably will—but by next week you won’t.”
In fact, it was the rigidly codified rules of dueling that gave Alfvaen her greatest asset in learning the
proxemics and kinesics of Jenji. Tocohl could explain certain uses of space between two speakers in
terms of the very precise movements of the duel, codifying them in Alfvaen’s mind.
All in all, Tocohl was pleased with her pupil. Even now, as she fairly crackled with anticipation,
Alfvaei spoke in Jenji and carefully maintained the proper polite distance. Tocohl knew it was no easy
task for her—on Alfvaen’s world, physical closeness implied intimacy.
“I mean no denigration of your teaching ability Tocohl,” she said, “I am only afraid that I will forget
my lessons-s. If I’s-stop thinking about it, I will s-step back.” The emotional stress had brought her
slurring back.
Tocohl’s hands moved swiftly as she manually brought the Margaret Lord Lynn into geostationary
orbit above the survey camp; Maggy flashed confirmation. Tocohl said, without looking up, “Take a pill
and don’t worry. I’ll let you know when you can stop thinking about it.”
(Something is bothering her,) Maggy said privately (What’s wrong?)
(Nothing we need worry about, Maggy,) said Tocohl in the same mode, (I’ll explain it later.) Aloud,
she said, “There we are. Good to know my brain hasn’t atrophied. Now see if you can raise Captain
Kejesli.”
She glanced briefly at the serendipitist and wondered why she’d ever thought those quick green eyes
pale. “Your pardon if I speak Sheveschkem?”
“What distance do I s-stand for that?”
In Jenji, it was not a joke, but Tocohl grinned back at her. “Stick to Jenji,” she said. “The usual
survey team is so diverse that no two members speak the same language. Don’t confuse the issue.”
Maggy pinged for attention; “I have Captain Rav Kejesli,” she said. Tocohl pointed to an area of the
screen.
A face appeared in the indicated spot, dwarfed by the full-screen display of the stormy atmosphere
of Flashfever. Tocohl shifted her attention to take in Rav Kejesli. He was a stocky man with gray eyes
and a worried expression. His long dark hair was elaborately beaded and clicked with each movement ofhis head. A festival pin glittered in his vest lapel—a pin of remembrance, in the northern style.
Tocohl made the northern gesture of greeting and introduced herself.
“Yes, yes,” Kejesli responded. He returned the gesture automatically but he spoke in GalLing’, his
voice impatient. “You came because of Tinling Alfvaen?”
“No. I came at the request of swift-Kalat twis Jalakat of Jenje.”
“What is it you want?”
“Your permission to land, and proper coordinates for a skiff.”
“Permission denied,” said Kejesli.
The words chilled her, even as Alfvaen gripped her arm convulsively. Permission to land on a planet
this late in survey should have been a formality, Tocohl knew. She gripped Alfvaen’s hand, answering
convulsion with firmness, and waited, frowning slightly, for the bad news.
“Quarantined? Are you quarantined? Whatzh—what has-s happened?” Alfvaen demanded of him,
the shock of his refusal making her slur violently despite every effort to speak plainly.
Kejesli jerked his head violently, starting a stormy rattle. “No, Alfvaen! Nothing like that! We’re
taking normal precautions. Everything is all right!” He closed his eyes and rubbed his hands across them
as if suddenly he were very tired.
He said at last, “—No, everything is not all right. We lost Oloitokitok. But, Alfvaen, swift-Kalat is
fine. There is nothing to worry about.”
“Then why won’t you let us-ss land?” demanded Alfvaen.
“It is my prerogative.” His hand came up, the Sheveschkem shrug.
There was a single sharp movement to her side and Tocohl turned. Alfvaen stood, rigid, her right arm
shoulder high, her forearm parallel to her chest. The fringe shivered with the tension of her body. “This is
your choice, then! Look on me, child of fools!”
She had spoken in Siveyn, but it was clear from Kejesli’s horrified expression that he knew what had
happened. Tocohl addressed Kejesli brusquely in his own language, as if translating Alfvaen’s words:
“Your whim prevents her from fulfilling her obligation to swift-Kalat. She will challenge you if you do not
reconsider your action. She’s angry enough to do it, too. And if she goes to a full challenge and you don’t
give her satisfaction, you’ll never be able to work with a Siveyn again. And she’d have judgment on her
side; there would be no recourse. Is it worth that much?”
Kejesli jerked his eyes away from Alfvaen. “Veschke’s sparks, no!” he said. “How—?”
“How do you get out of it?” Tocohl finished for him. “—Reverse your decision.” Kejesli frowned and
Tocohl switched back to GalLing’. “It’s not your prerogative, Captain,” she said. “We have mail.”
For Kejesli to refuse the delivery of mail for anything short of the planet-wide quarantine he had just
denied was unthinkable, and he accepted the excuse Tocohl provided him gratefully. “In that case,” he
said, “you have permission to land.”
Again in Sheveschkem, Tocohl said, “Say to her, This is my choice: that you and I clasp hands and
drink together.” Kejesli did so, and although the words came out slightly differently in GalLing’, the effect
was good enough. Alfvaen slowly lowered her arm and turned her back to the screen.
“Well,” Tocohl interpreted in Sheveschkem, “you’re not good enough to drink with, but she’ll forgo
the fight.”
Relief washed his features. Without taking his eyes from Alfvaen, Kejesli went on, “If you’ll link with
our computer, Captain Susumo, we’ll transmit the coordinates you need. I suggest that you wait out the
storm. Lightning is hazardous in a skiff—or any other small craft, for that matter.”
“Then give me coordinates for a class 13 trader, if you will.”
He complied. When she had acknowledged receipt, he said, “I’ll send a daisy-clipper to meet you as
soon as the storm passes.” With one last worried look at Alfvaen, Kejesli broke contact.
Tocohl programmed her landing. While Maggy checked her figures, Tocohl responded to the survey
computer’s customs queries. The only item of interest was her moss cloak, and since no other moss
cloak was present, customs okayed it. It always pleased Tocohl that her cloak was a one-to-a-world
item.
Then she leaned back and waited quietly.At long last, Alfvaen turned back. “My apologies,” she said formally, in her own tongue, “I thank you
for your assistance. I had no cause to challenge. I should know Kejesli better by now. He does what he
thinks MGE expects of him and nothing more.”
“I’m afraid nothing is that simple. Someone on this world must have passed the word to have you
delayed,” said Tocohl, “and Captain Kejesli is as good a suspect as anyone else.”
At Alfvaen’s shocked look of denial, Tocohl said, “Just bear it in mind.” Her hands danced and the
image of Flashfever swelled. “Now, I promised Kejesli mail and mail he shall have,” she grinned. “Go
write a letter.—And just in case Kejesli tries to restrict us further, specify hand delivery to swift-Kalat.”
Alfvaen went to write a letter. Tocohl’s hands danced again on the console. (Now, Maggy,) she said,
(in answer to your question: Alfvaen finds swift-Kalat sexually attractive—judging from the way Kejesli
spoke, that’s no secret. She wants to learn his language in order to be more attractive to him. She’s now
afraid that she’ll do it badly and ruin her chances of a relationship, or of learning that he doesn’t return her
feeling.)
(Oh,) said Maggy. (—So Alfvaen will tell him she loves him and fight a duel with her closest friend
and win and be cruelly wounded?)
(Wait, wait!—Veschke’s sparks, Maggy, what have you been reading?!)
Maggy’s recital of what she had been displaying for Alfvaen lasted through planetfall. (Maggy,) said
Tocohl, firmly, (we’re going to have to have a long talk about fiction. I think you still misunderstand its
purposes: fiction is a lie for entertainment, it’s a lie the listener willingly accepts for the sake of something
else.
(Alfvaen reads formula fiction. Each book, as I’m sure you’ve noticed, follows a set pattern, and the
delight of the reader is in the variations on the theme—while the theme fulfills certain basic emotional
needs. Alfvaen’s a romantic: she wants to see duels fought and won at great cost for great passion.)
She broke off as Alfvaen returned, her letter prepared. Clipped to her belt was one of Maggy’s
hand-helds, striped diagonally with gold and purple for easy identification. Alfvaen touched it lightly and
explained, “Maggy s-said to ask you if I might carry this’s-so’s-she could talk to me.”
Tocohl gestured her permission. “That’s for voice transmission only—remember, Maggy does listen
unless you tell her otherwise.” She looked slightly away from the Siveyn. “Maggy? Why don’t you
activate an arachne and poke around on your own as well?”
“You didn’t tell me to,” said Maggy.
“I’m telling you to, now. Check that construction in your Siveyn grammar: it indicates a nonobligatory
suggestion or request.”
Planetfall accomplished, Tocohl gathered her cloak about her and led the way to the cargo hold to
await local transport. Maggy pinged for attention almost immediately, and relayed a message: “Move ass,
Hellspark!” said a deep, cheerful voice. “This lull won’t last forever, and I’ve an allergy to lightning!” The
words were GalLing’ but the delivery was pure Jannisetti.
Tocohl glanced quickly at Alfvaen’s feet. Yes, the Siveyn was wearing boots. That left only Tocohl
indecent by Jannisetti standards. (Maggy, I need boots—red ones,) she added quickly, knowing that
Maggy would ask. From the soles of her feet to the top of her calves, her 2nd skin turned a dark red,
with stitching in all the appropriate places and a darkening of shadows to suggest thickness. (Thank you.
Now let’s “move ass” like the lady says.)
Maggy popped the hatch. Tocohl whistled her wonder and thrust her head through for a better view.
Truly, this world had been struck by the fist of Veschke!
The broad grassland below was alive with light. As the spray-laden wind rippled through it, it
flickered and flashed in response. Beyond, some two kilometers, the grasslands gave way to
woods—and the woods themselves winked jewel-bright lights. The air was so pungent with ozone it
stung her nostrils; lightning flashed, brief and spectacular, into a far-off group of stiff black structures.
A daisy-clipper edged in, cutting off her view of the shimmering world and substituting that of a broad
brandy-dark face. Still in wonder at Flashfever itself, Tocohl had enough to spare for the remarkable
piloting that brought the pilot virtually nose-to-nose.
The Jannisetti woman stared back at Tocohl and then, suddenly, grinned hugely. “Good,” she said,her satisfaction plain, “you pass. Wait until you see it by night—it’s a Port of Delights and a firework
display all rolled into one! Now, pull your eyes back in your head and let’s get the hell out of here.”
Tocohl stepped lightly from Maggy’s hatch into the daisy-clipper and held out a hand to assist
Alfvaen. Without taking her eyes from the landscape, Tocohl made a circle of her arms. The arachne
squatted on its long, spindly legs and leapt. (Close the hatch, Maggy,) she said, settling the arachne’s fat
round body on her lap and adjusting the legs so she could see past them.
“Buntecreih,” said the Jannisetti, turning the daisy-clipper around and settling into a low fast skim
toward base camp, “but everybody calls me Buntec.—The arachne won’t last long here; you probably
should have left it on board your ship.
“You’ve heard of electric eels? We have electric mice, tigers, buzzards, you name it. Corner any
wildlife around here and you’re in for a shock, literally.” Buntec’s voice turned abruptly grim. “We just
lost a man that way.”—Tocohl touched her forehead in acknowledgment, and Buntec went on, forcing
herself to a lighter tone, “Half the wildlife, plant or animal, on this flashy planet uses electricity for defense
or offense—and one good zzzzzzaaap! from an Eilo’s-kiss will fuse your arachne solid. Either that, or a
tape-belcher will get it.”
“Tape-belcher?” said Alfvaen, and Buntec laughed. “That’s right. The first time we saw one was
when it swooped”—she demonstrated expressively with the hovercraft, and Tocohl clutched at the
arachne to keep it on her lap—“down and scarfed up a tape recorder. Thought about it a minute, gave a
horrendous belch, and barfed it right back up again. And if you think this sounds disgusting, wait until you
see one!”
(Maggy, you’re not to wander around until we get you full descriptions of these things. Maybe losing
a mobile doesn’t hurt you, but until somebody invents a cheap superconductor replacing it takes credit
we could better spend other ways.)
(For more memory, you mean?)
(You’re getting greedy, aren’t you?) Tocohl grinned.
(Yes,) said Maggy. There was a pause, then she added, (Was that the right response?)
(Very. On the nose. Now cut the chatter and let me find out what’s going on here.)
Buntec was commiserating with Alfvaen in no uncertain terms. “Yeah, I heard they dropped you after
Inumaru. SOP for the s.o.b.s. Kejesli was on that one, too, wasn’t he? And he didn’t make any
objections?”
Buntec set the craft down with an abruptness that Tocohl expected to be followed by a hard jolt—it
wasn’t—and answered her own question. “Naw, he wouldn’t. Too worried about his own hide. I never
saw such a rattlebrain!” She lifted her chunky hands from the controls, cracked her knuckles, and twisted
around to face Alfvaen. “And I’ll bet you thought that noise was just the doohickeys in his hair! No, I tell
you, it’s three loose thoughts in an otherwise empty container.”
“Maybe you’re right,” said Alfvaen. “He didn’t want us to land, and I don’t think he’ll let us stay.”
“I fixed him.” Buntec tapped her nose with self-satisfaction. “I figured from the rattling he did when
he told me to pick you up that you were the last person he wanted to see, ever.” Alfvaen flinched, but
Buntec went on, “He may not want to see you but the rest of us do.” She emphasized her point with a
finger-tap, this time to Alfvaen’s nose. “So I jogged his brain a bit on that count... and I made a few calls
on my way out to get you. Half the survey team is waiting for you in the common room—let Old
Rattlebrain try to throw you off planet with us around!”
She turned to take in Tocohl as well and added, “And wait’ll the other half finds out about you,
Hellspark! He’d better let you stay: we’re all sick of looking at each other. With what we’ve just been
through, we need the diversion.” For a brief moment, her face darkened as she added, “Another two
weeks of nothing but sprookjes and I’ll tip darts and hunt Vyrnwy.”
Tocohl raised an eyebrow at this last, but Buntec only spread a flattened hand and said, “Better a
little harmless excitement, I say.—And I say you’ll stay if I have to peel Kejesli and roll him through a
field of zap-mes.”
A sudden gust of wind brought a torrent of rain. “Shit,” snapped Buntec, “me and my big mouth.
Now we’ll have to run for it. Follow me!” and she was off, with Alfvaen at her heels. Tocohl dropped tothe ground, the arachne under her arm, and stopped, transfixed. Water sheeted on her spectacles—and
Maggy compensated for the remaining distortion—as she stared up into the flash-filled sky, her ears filled
with the roar of the rain.
“Hey! Hellspark!” Buntec roared over it. “I said move ass, I meant move ass. This is only the
leading edge. From here on it gets worse!” A chunky hand grasped Tocohl’s, and together they raced
through the field of flashgrass to the thick red mud of the compound.
Chapter Four
SWIFT-KALAT CLAMPED HIS jaw shut, unable to respond to Ruurd van Zoveel’s polite
overtures in GalLing’—they served only to renew his memory of what van Zoveel had so misspoken.
Without a word, he took the towel van Zoveel proffered and focused his attention on drying himself from
his dash through the storm. Again, he told himself that GalLing’, being an artificial language coined for
trade, had none of the reliability of Jenji. Again, he found it difficult to believe.
It wasn’t until van Zoveel addressed him in Jenji that he was able to answer at all. Hearing the Jenji
forms calmed him slowly. He chimed his bracelets in polite response, mildly surprised when van Zoveel
did not follow suit. Of course, he thought, Zoveelians wear no status bracelets, but it disturbed him
nonetheless. Even the youngest child makes the arm motion...
Their conversation continued in Jenji. The sound of it was enormously welcome but swift-Kalat found
himself more and more discomforted. Something in Ruurd van Zoveel’s manner disturbed him
enormously; it never bothered him when he spoke to van Zoveel over the comunit but, here, in his
presence... If only the man would sit down! swift-Kalat thought. For all his courtesy, van Zoveel seemed
always to back away, and swift-Kalat felt obliged to follow.
Instead of sitting, however, van Zoveel paced nervously, his beribboned tunic fluttering. He offered a
glass of dOrnano wine, as if the occasion were one for celebration; and swift-Kalat accepted, knowing it
was not, but grateful because the acceptance took van Zoveel to the far side of the room.
Van Zoveel’s furniture was plush and as gaudy as his clothing. Swift-Kalat chose a plump red and
blue pillow near a low table and sat, piling smaller red and yellow pillows to support his elbow, as he’d
seen van Zoveel do. It was far from comfortable, but it was better than following van Zoveel around the
room.
Van Zoveel returned with the wine and handed him a sheaf of hard-copy as well. “That’s my report,”
he said. “That is what I will have to give the captain. I thought perhaps you should read it.”
“I need only read your conclusion.” Swift-Kalat sat up to take the report. He leafed through to the
final page. It read as he’d expected: “The sprookjes have no language as far as I am able to
determine.” He slapped the report closed and dropped it onto the table with more force than he’d
intended.
Van Zoveel, pouring the wine, jumped; wine splashed. He finished the pouring carefully and wiped
away the droplets. “I’m sorry, swift-Kalat,” he said, not looking up, “I am unable to say otherwise.”
This time the absence of van Zoveel’s status bracelets—or at least the movement that would have set
them ringing—struck swift-Kalat more forcefully.
“Something on this world is sentient.” Swift-Kalat snapped his forearm sharply; his own bracelets
rang emphasis of his words.
“Something has your reliability in its favor. I explained that to Captain Kejesli but the captain hasn’t
the ear to hear the distinction.—And I am unable to match your certainty. I am unable to say otherwise,”
he repeated.
“I made a formal application for a second polyglot, but Kejesli denied my request. My record is too
good, he said—too good!—and he did not wish to go the additional expense of sending an automated
message capsule.” He spat, startling swift-Kalat (who had only read of and never seen the Zoveelian
expression of utter disgust) with his vehemence, and finished. “There is nothing further I can do.”
A peal of thunder rattled the wine glasses. Swift-Kalat put out a hand to steady his but did not drink.“I thank you for your trouble,” he said. “I did not know you had gone so far—”
“Ruurd?” From the comunit, Buntec’s deep voice broke in.
Van Zoveel excused himself and activated the screen. “Could this wait, Buntec?” he said. “I have
company.”
“No, it can’t. You gotta come sweet-talk the captain in his native croak,” Buntec said. “You
remember Tinling Alfvaen? She’s here—”
Swift-Kalat came instantly to his feet. Unable to restrain himself, he clapped his hands sharply above
his head, bracelets clashing triumph. He strode to join van Zoveel.
Buntec acknowledged him with a wave. “She’s here with a Hellspark,” she said, repeating the words
that had been lost to swift-Kalat’s joy. Then she went on, her indignation rising in proportion to their
enthusiasm, “Old Rattlebrain tried to keep ‘em from landing. Now he wants ‘em off planet just as soon
as they’ve delivered their mail. But we need ‘em—we need something after the trouble we’ve been
through!—and native croak always makes a difference, Ruurd. You know that!”
Van Zoveel began a polite refusal, but swift-Kalat said, “We’ll come.” He turned to van Zoveel and
said, in Jenji, “Would you accept the assistance of a Hellspark polyglot?”
“Yes, of course!—Of course, we’ll come!” There was no need to translate for Buntec. The screen
was already dark.
Hellsparks made Rav Kejesli uncomfortable.
As a young man on Sheveschke, Kejesli had haunted the streets at festival time looking for the
traders to the thousand worlds. He’d found them no different from anyone else he knew. Oh, they
dressed differently, that was certainly true, but they spoke Sheveschkem, they acted like
Sheveschkemen. They were a disappointment.
It had taken Kejesli fifty years to make his first jump away from Sheveschke—in search of real
differences—and there were the Hellsparks again. Only this time, they were not like Sheveschkemen;
they were like Jannisetti, Apsanti, Bluesippans, or like the Yns, the Zoveelians, the Maldeneantine. They
were more alien than he could have imagined—or could accept.
He shuddered. What would this one be like, surrounded by a survey team composed of such
variety?
Bad enough dealing with so many aliens. He accepted that as part of the job: the Comity insisted that
as many cultures as possible be represented on a survey team—to widen the scope of its knowledge and
to broaden the range of its available working data. Besides, a planet Sheveschkemen loathed—this one,
for example—might well be attractive to natives of some other world.
But to throw a Hellspark in on top of it all? How would she choose which culture to be?
Perhaps this Tocohl Susumo would simply be Hellspark, whatever that might be. Kejesli was not
sure he wanted to know.
In any event, he was not about to allow her to interfere with his career. MGE would not approve of
an outsider meddling in one of its surveys.
He poured himself a second cup of winter-flame from the warming pot, then hesitated. For a
moment, he thought to join one of the conversations scattered about the common room but he had
already overheard one such and its topic was Tinling Alfvaen. That was not one he had a desire to
discuss. He returned to his seat in the far corner of the room.
A tooth-jarring clap of thunder signaled that the storm had broken in earnest. His hand jumped,
winter-flame slopped red and gold across the tabletop. Involuntarily following the sound, he glanced at
the ceiling. A wave of vertigo made the base of his neck prickle. Forcing his glance down, he wiped
away the sudden sweat—then used the same cloth to mop the spilled winter-flame, trying to concentrate
on the action alone. Buntec and Alfvaen and this Hellspark had not yet come. The thought that they too
might meet the same fate as Oloitokitok...
The more he tried to tell himself that other survey captains had lost team members, the more he felt
responsible for Oloitokitok’s death. This was his third survey, and the first time he had lost a surveyor...
unless one counted the twelve that had contracted Cana’s disease. No, he wouldn’t count them—they
lived and Oloitokitok was dead.A shout of laughter jarred Kejesli from his thoughts. He looked up in time to see Buntec, Alfvaen,
and Tocohl Susumo burst through the door, spattering water about them. The membrane slapped wetly
behind them, and the Hellspark laughed again. Her evident joy in Flashfever’s weather made him
suddenly angry.
After greeting the startled Vielvoye cheerfully, she placed an arachne on the ground beside her, dried
her spectacles and replaced them, and reached up to twist water from her hair. The arachne unfolded a
set of improbable stiltlike legs and immediately began to explore, but Kejesli could not take his attention
from the Hellspark. Their brief conversation by screen had not prepared him for the intensity of her
presence.
She strode to the center of the room, her silver cloak trailing rivulets of water. There she stopped. In
a single turn that focused the attention of every surveyor present on her, she seemed to him to take in
everything, and to pronounce judgment. He waited, terrified of the verdict.
Om im Chadeayne, the team’s geologist, was suddenly on his feet. “Hellspark!” he said. “Hellspark,
what news?” He crossed to her in a few quick strides and stood before her, his hands on his hips, his
head cocked expectantly upward. Om im was tall for a Bluesippan, but he came only to this woman’s
elbow.
Tocohl Susumo held out a palm. “News for news,” she said.
“Hah!” said Om im, touching a finger to his brow. “Yes, payment there will be. Always payment for a
Hellspark. But first, a cup of winter-flame.” He snapped his fingers at Vielvoye, who was nearest the
warming pot, and Vielvoye scurried to bring a fresh cup.
The Hellspark looked at the cup, and then at Om im, warily. “—And the payment?” she said. Om im
clapped his hands, drew them expressively down to indicate the space she occupied. “Your presence,
Ish shan, is more than sufficient pay for a cup of winter-flame.”
The woman bowed low, sweeping the ground with the edge of her cloak. “Tocohl Susumo is my
name,” she said.
Om im returned the bow with equal extravagance. “Om im Chadeayne of Bluesip,” he said, taking
Kejesli by complete surprise. He had thought them old friends from Om im’s initial reaction.
The crowd continued to converge on her, as excited as children with a new toy. Everyone wanted a
look. Not everyone, he corrected—Buntec was talking earnestly into a comunit, and she was probably
passing the word, something she did well. Now only he and John the Smith had not joined the crowd.
John the Smith, Kejesli recalled, was from one of the Navel Worlds, close to the main centers of
civilization. Those worlds no longer needed the independent traders, not the way the people of the
Extremities did. Obviously, John considered himself too sophisticated to court Hellsparks. Kejesli was
mildly annoyed at the thought.
Another burst of thunder combined with nearby movement caught Kejesli’s eye, and he turned to
find the arachne poised beside him like a hunting farrun that had found its quarry. He stared back at it,
surprised that it did not leave when its inspection was completed. A moment later the Hellspark stood
before him, and the arachne was once again on its way.
“With your permission, Captain?” She gestured at the chair facing him. Her gray cloak, still glinting
silver droplets, cascaded softly about her as she sat. She pushed back a tangle of red hair made darker
by Flashfever’s downpour.
The tangle caught momentarily. Only when she had tugged it free did he see the cause of the snag: a
pin of high-change was thrust through her cloak!
His first thought was that she must be mad—only the desperate would choose to take that risk—but
for all his sudden stare he could find nothing desperate in those gold eyes, and nothing mad either.
Instead he found something disconcertingly familiar. He had seen those gold eyes somewhere—
He found himself fingering the pin of remembrance in his vest lapel. He had worn it not for Veschke
but for remembrance of Oloitokitok.
The Hellspark’s gold eyes followed his fingers. He knew she could tell from the pin’s design that it
was four years old, that being the last time he had attended the Festival of Ste. Veschke. She smiled,
indicating the pin. She was Sheveschkem at that moment. “Don’t worry,” she said, “I’ve tracked inenough of Veschke’s blessing from this year’s festival to cover us both.” Thrusting out a foot to show him
that it was covered with red mud, she went on, “I assure you only half of that is local.”
Surprised to find that it did reassure him, he looked at her face again—and realized why she had
seemed so familiar. He had seen those gold eyes a thousand times in his youth, smiling triumphantly from
an icon that depicted Veschke’s burning...
He suddenly wished for John the Smith’s sophistication—or his ignorance.
Where else but on a survey where his ship had not been blessed, where else but on a world he had
given the ill-omened name of Flashfever, could all these things coincide? The death of Oloitokitok,
Alfvaen (deny it as he would, he was responsible for the twelve of Inumaru as well), and this woman with
the pin of high-change. Veschke was renowned for her sense of humor.
He fought the imagery: all he had to do was send a report to MGE and he could leave this world. He
made a conscious effort and his hand dropped from the pin of remembrance.
Tocohl Susumo smiled at him again. She raised her cup, made Veschke’s sign with her left hand, and
said. “To Veschke!”
“To Veschke!” he repeated, without intending to, and drank with her.
By the time Tocohl rejoined Alfvaen, the crowd had doubled in size; Buntec beamed at this result of
her handiwork. Amid a cheerful pandemonium of greetings in a dozen different tongues, Tocohl spoke
quietly to Alfvaen in Siveyn, “We have a local day’s grace. Speak to your friends—perhaps they’ll put
some pressure on Captain Kejesli for us.” Alfvaen set to the task, drawing aside first one member of the
survey team and then another.
Om im poured Tocohl another cup of the scarlet and gold drink, then, as if he were the aide of a
prince, he presented the surveyors to her one by one.
(Maggy, keep a file of faces and names.)
(I always do,) Maggy responded as Tocohl greeted each surveyor in his or her native tongue with
due respect to ritual. To Dyxte ti-Amax, she bowed; to Vielvoye ha-Somol, she respectfully tipped a
nonexistent hat; both were Tobians but ha and ti spoke different languages. Hitoshi Dan, she greeted
with a soft version of a whistle that had originally developed to be heard for several miles. And to
Timosie Megeve, the Maldeneantine, she raised her left hand, crossing it with her right. Before he could
reply, Alfvaen suddenly reappeared at Tocohl’s side.
Pointing to the doorway, Alfvaen said anxiously, “There’s swift-Kalat.”
Tocohl laid a reassuring hand on her shoulder. In Siveyn, to avoid offending Buntec, Tocohl said,
“Toes. Don’t move: let him come to you. And stop worrying—he’ll appreciate your attempt even if your
execution isn’t perfect.” Unobtrusively, she took the added measure of placing a set of her own toes
where Alfvaen would stamp them if she backed away from swift-Kalat. It was an old Hellspark
technique for helping a child remember her proxemics.
“Swift-Kalat,” Om im announced, smiling up at Alfvaen, “I can hear him chiming this way.” His smile
faded before her obvious anxiety. After a second’s consideration of the problem, he reached for
Alfvaen’s elbow, with the clear intent of escorting her, as shy as she might be, to swift-Kalat’s side.
Tocohl, blocking his hand with her own, said softly, “No.” He gave her a curious look but drew back his
hand and patiently folded his arms to wait with them.
Of the two approaching men, Tocohl thought, the smaller would be swift-Kalat: his skin was a rich
glowing red, almost the color of Dusty Sunday glass; bracelets gleamed the entire length of his forearm,
jangling cheerfully. Tocohl had never seen a Jenji with quite so many. (Up to his elbows in silver,) she
said.
(What?)
(Jenji expression for very, very smart,) she explained. (Now I see why.)
The other man, dressed in a tunic flamboyant enough to coin a Jannisetti phrase, was unmistakably
Zoveelian.
The crowd parted just enough to let the newcomers through. Quietly, in GalLing’, swift-Kalat said,
“Alfvaen, I’m so glad you’ve come. I’m so glad you’re safe.” Then he strengthened his words with Jenjin
emphasis, snapping his forearm down so sharply that his bracelets clashed and rang as he moved closer.Alfvaen had learned her lessons well: as he passed the point Alfvaen’s culture considered the proper
distance for general talk and closed in to the comfortable position for his own, Alfvaen tensed slightly but
did not step back. Right down to her toes, she greeted him in perfect Jenji. “I am so glad to see you,”
she said, snapping her bare arm down for emphasis of her own.
There was no chime of bracelets, but swift-Kalat more than amply compensated for the lack. His
sharp intake of breath told both Alfvaen and Tocohl that Alfvaen’s attempt was a complete success.
Swift-Kalat’s eyes and smile widened in delight.
Alfvaen smiled back shyly and, with this encouragement, went on to make proper introductions. She
assumed, Tocohl saw, that Ruurd van Zoveel spoke Jenji as well as she. The polyglot spoke excellent
Jenji, but that was all; he was clearly ignorant of both proxemics and kinesics. Tocohl automatically
switched to Zoveelian to reply to his greeting and then returned to GalLing’ out of courtesy to Om im.
“We have a day,” she said.
Swift-Kalat looked at Alfvaen in distress, and van Zoveel exclaimed, “A day! What can you do in a
day?”
Tocohl smiled. “Change Captain Kejesli’s mind,” she said.
“It can be done, Ish shan.” Om im craned toward the door and said in his own tongue, “If Buntec
was willing to call Edge-of-Dark, her feelings run high on the subject.”
Tocohl followed his look to the latest arrival and raised an eyebrow in surprise. No worlds’ motley
for this woman! Her 2nd skin was an unavoidable exception and that was transparent to minimize its
intrusion. Everything else about her was pure Vyrnwyn high-born, from the feathered crown interwoven
in her black hair to the tips of her fingers and toes, polished dark green to match her victoria ribbon.
That made sense of Buntec’s threat to tip darts and hunt Vyrnwy. Buntec might have been able to
deal with bare feet—but the outright perversion of polished toenails would have tried the most
cosmopolitan Jannisetti.
Tocohl said, “Now that’s what I call getting off on the wrong foot.”
The joke stood in Bluesippan and Om im laughed appreciatively. Then he said, “We were chamfered
by a moron. He gave us each a stack of hard-copy and told us to read it. With some people, that’s not
sufficient.”
He glanced again toward the door, “We’ve tried to talk to Edge-of-Dark, but...” He threw up his
hands and, still in his own tongue, added, “I tell you, Ish shan, with the exception of the old-timers, this
team gets on together about as well as flot and eggri.”
Tocohl grinned: in Bluesippan mythology, the battle between flot and eggri was responsible for the
second destruction of the world. “How long has it been since she’s visited home?”
“A good ten years,” he answered. “Why?”
(Maggy?) Tocohl said privately, raising a finger to hold off Om im’s question. (Look through your
records and pull out some stills of Madly of Ringsilver—pick only those where the background is
blurred—and hold them until I ask for them.)
By the time she had finished speaking to Maggy, Edge-of-Dark had joined their company, but Om
im’s look told Tocohl quite clearly that his question was not forgotten, simply postponed.
With much solemnity and ceremony, Om im presented her. Tocohl took the hand Edge-of-Dark
extended. She kissed it formally, said, “I am indebted to Om im Chadeayne for his kindness in making
you known to me.”
“I too am indebted to Om im Chadeayne,” Edge-of-Dark responded. In GalLing’, she went on, “It is
a pleasure to be in discriminating company once again. Like most of your people, your dress is decidedly
eccentric”—she eyed Tocohl’s moss cloak with jaundice—“but your manners are unfailingly
impeccable.”
Tocohl laid a hand on her breast and inclined her head. GalLing’ suited her just fine for this minor bit
of business. “I imagine this must be a great trial for you,” she said, “I see you have not been back to
Vyrnwy for, oh, five years at least.”
“Almost ten years, now.—How did you know?”
“Come now! Styles do change.” Tocohl laughed, “If you think my dress eccentric, you should seewhat high-born Vyrnwy wear these days!” Tocohl gestured at Edge-of-Dark’s clothing and said, “Not
that I suppose it matters much—this is perfectly suitable for surveying.”
Edge-of-Dark flushed as deep a red as swift-Kalat. “Tell me,” she said, “describe it to me.”
“I’m not much at description. I could show you some pictures, if you’d like.”
“I would,” said Edge-of-Dark eagerly and Tocohl finished, “Tomorrow, then... if Captain Kejesli
grants us the time. (Maggy, we’re going. Bring the arachne.) Today I am here on business and I must
deliver my messages.”
Still flushing, Edge-of-Dark offered her hand again, this time to take hasty but formal leave of Tocohl.
Sparing only the briefest of embarrassed glances for the others, she hurried to the door and out into the
thinning veil of rain.
“Little bugger’s really rude today, even by her standards,” Buntec said. “Wonder what bit her ass?”
Om im stared thoughtfully, first after Edge-of-Dark, then at Tocohl. Touching a finger to his brow, he
gave Tocohl a delighted smile. “Ish shan always was an ass-biter,” he said in his own tongue. “Unless I
miss my guess, Edge-of-Dark will not be seen until she is once again in fashion—and the fashion will
include shoes.”
“Boots,” corrected Tocohl and grinned impishly in response, pleased that she could accomplish that
much at least.
Om im made a deep bow. “You shall have fair payment, Ish shan, that I promise you!”
Maggy’s arachne pricked its way through the crowd just as Tocohl bent to return the bow. Mistaking
her intent, the arachne leapt into the crook of her arm, to settle itself there like a Gaian cat. Tocohl
laughed once as she straightened but, again face-to-face with swift-Kalat, she said soberly, “Now,
swift-Kalat, you and I will have a word or two.”
Swift-Kalat found it hard to withdraw his attention from the behavior of the arachne; the ethologist in
him was fascinated. No adult could have mistaken Tocohl’s bow for an invitation—its controller was
evidently a child.
But Tocohl was correct, the two of them had business, and the glance the Hellspark gave van Zoveel
made it clear that simply speaking in Jenji would not be sufficient privacy.
“Of course,” he said. Reluctantly, he released Alfvaen’s hand, and gestured Tocohl to follow him.
Privacy was difficult to arrange on Lassti or, perhaps, that was only his perception, after three years with
the same forty people in the same small compound. He did not even think of his cabin as private in that
sense, it was too familiar. Too many of those people had been within its door. So he drew aside the
membrane and looked out. It was still raining, but the storm had passed, the danger from lightning with it.
He led her out into the rain, his boots squelching in the mud at every step, taking her only a few feet
around the side of the common room building. Lightning still played above the stand of lightning rods
beyond camp; his ears rang with it. He tapped the wall behind him. “If we speak quietly,” he said, “we
are alone. All of the buildings were heavily soundproofed the second week of our stay.”
Tocohl twisted her head, agreeing to the place. Swift-Kalat breathed a sigh of relief; with the one
gesture, she had somehow become someone he could talk easily to.
“We will discuss your fee,” he said. That was another area where he lacked expertise, never having
dealt financially with a Hellspark.
She lifted a finger no. “Alfvaen and I have done so,” she said. “The fee we agreed upon is 2,000 G,
contingent of course on my being permitted to stay.”
That was singularly low for an open-ended task the like of this, of that much swift-Kalat was sure.
“It was clever of you to send a Siveyn,” she went on before he could protest, “whether the
cleverness was intentional or not. It’s impossible to dicker with someone who takes one’s first price as
fixed. I don’t rob babes.” She snapped her wrist with such authority that he almost heard the weight of
her status on this subject.
“It was not intentional,” he said.
“Never tell a trader that!” She countered with a smile—and again snapped her wrist to give ring to
the command. “In fact, the next time you call a liar”—he jerked at the unexpectedness of the
obscenity—“put him to work: let him deal with the traders.”She phrased it so adroitly that he could object to neither the words nor the suggestion. And in that
moment he would have risked his status on the statement that Alfvaen had found him the one person who
could tell him without fail whether or not the sprookjes had a language. He smiled. “I accept your fee and
your contingency. And I shall consider your suggestion.”
“I see I pass,” she said, smiling back. “To business then: when you sent your message to Alfvaen
requesting the services of a Hellspark glossi, did you tell anyone of your intention?”
That seemed an irrelevancy but, from her manner, it was not. “Yes, I told Oloitokitok. He was
concerned about the sprookjes”—she exposed a bare arm to indicate her unfamiliarity with the
term—“that is the name van Zoveel gave the disputed species. He was concerned about the sprookjes,
as I was, so it was natural to mention what steps I had taken.”
“When you sent your message, were others sent at the same time? If so, do you know by whom they
were sent?”
“Others were sent, yes; by whom, I do not know. Investing in an automated message capsule was
unnecessary, for I made the decision at the time of the last supply ship. It would be little risk to assume
that everyone sent messages at that time.”
She raised a finger. Thoughtfully, she said, “No confirmation, then.”
The words disappointed him. He had hoped for an explanation of the queries. But if she was not
ready to speak about the subject, there was little he could do, except ask again in GalLing’. After van
Zoveel’s misspeaking, he was not about to risk that.
“Are there any Inheritors of God among the survey team?”
“I do not know.”
“How did Oloitokitok die?”
Again, he said, “I do not know. It was reported to me that layli-layli calulan believes he was
electrocuted by a live-wire or a blitzen.” He used the GalLing’ terms for both; they conveyed some sense
of the menace of the creatures.
“Do you accept this?”
He realized, to his own surprise, that he had told her the fact had been reported to him, not that it
was generally accepted as indeed it was among the remainder of the survey team. She waited quietly
while he reconsidered his own thoughts. At last he said, “I think it unlikely: neither of the creatures has
ever ventured into that particular habitat of the flash wood, in my experience. Their prey and their modes
of behavior argue against it.”
“What then killed Oloitokitok?”
“The third possibility is lightning. It is as unlikely as the first two.”
“What special knowledge did Oloitokitok possess? What was his area of expertise? Could he have
known something about the sprookjes that no one else knew? I am asking for conjecture, only: no
conclusions on your part are necessary.”
“He was on record primary engineer, secondary physicist, tertiary botanist. Shortly before he
disappeared, he was excited, it seemed, although I am no authority on Yn. He told me at that time that I
need not worry about the sprookjes. I inquired, but he would speak no further.”
Tocohl Susumo stared at him thoughtfully for a few long moments. At last, she said, “Nor may I, as
yet.” She turned, ready to head back to the others.
“Wait,” he said. “Can you judge the sprookjes’ sentience?”
“I am only one. I will do my best, given the circumstances.”
His query was ambiguous, he realized. She had taken it to mean in her capacity as byworld judge,
and she had graciously reminded him that a judgment of sentience required at least four such without
calling his status to question. It left no doubt in his mind as to hers; were she Jenji she would ring as
loudly as he.
She wiped streaming rain from her face. “Now, let us see what we can do in the small time allotted to
us.”
Swift-Kalat raised a finger in agreement, although it meant returning to the presence of van Zoveel.
He had the sudden thought that he was perhaps ascribing sentience to the sprookjes largely because hewas more comfortable with them than he was with the survey polyglot. Tocohl Susumo could make all
the difference. At least, he might learn to his own satisfaction the actual state of the matter.
Chapter Five
TOCOHL HAD GIVEN considerable thought to the matter while they rejoined the others. That a
Jenji of swift-status had made the assumption that she was a byworld judge surprised her no end. It had
taken considerable verbal maneuvering on her part to avoid calling his reliability into question without an
outright lie of her own. Now the conversation could in retrospect be recalled with no disgrace to either
speaker. She only hoped she could handle the sprookjes as well.
Swift-Kalat had offered his cabin for their further discussions. Typical of survey living, it was still a
cut above standing out in the rain. Small, stamped from a single mold, it had been carefully personalized.
While swift-Kalat searched for an additional chair and found a pillow for van Zoveel, Tocohl set
Maggy’s arachne in the middle of the floor. Maggy promptly unfolded it and began a careful inspection of
the surroundings. Tocohl did the same, with special attention to the holograms (they were originals, and
very fine) and the tyril, a small flutelike instrument of red porcelain.
“Alfvaen,” said swift-Kalat, “I worried that something had happened to you when you were so long
in coming.”
Alfvaen began, “You had cause—”
But before she could finish, Tocohl interrupted. “Your pardon, swift-Kalat, Alfvaen. We have little
time, and a great deal to discuss.” Tocohl had no intention of letting Alfvaen bring up the matter of the
Inheritors of God until she knew more about the members of the survey team. It was also something she
was likely to misspeak about in swift-Kalat’s estimation.
“Yes,” Swift-Kalat said. “Please sit.” The two women followed the invitation, but van Zoveel made
as if to decline.
“Sit,” said Tocohl, firmly. She had not failed to note swift-Kalat’s uneasiness with van Zoveel or its
cause. “It’s one of my cultural taboos,” she added with a smile.
The polyglot stared at her. “I thought the Hellsparks didn’t have any cultural taboos.”
“Anyone who says she has no taboos is a fool.—Please,” she indicated the pillow to her right and
van Zoveel obliged. Swift-Kalat looked relieved.
“Now,” she went on, “tell me about your creatures; or, better still, show me one.”
“I can’t,” said swift-Kalat.
Van Zoveel said, “The sprookjes leave the camp during the thunderstorms. They won’t be back until
the rain lets up, if then.” The big man’s brow furrowed. “I am unable to speak to them,” he said. His
hand slapped his thigh. “I’m not stupid: I’ve puzzled out three nonhuman languages during my career with
MGE—and yet I feel stupid now! I’ve tried every tongue I know, but all the creatures do is parrot!” He
thrust two fingers in swift-Kalat’s direction. “Don’t ask me about the sprookjes, ask swift-Kalat!”
Such had been Tocohl’s intention in the first place and without hesitation, she turned to him. He said
“I’ll show you.” A moment later, he handed her a large orange fruit and a knife.
“That’s a native plant,” he said, “and it’s an artifact.”
“A biological artifact?”
“You’ll see. Cut it open.”
Tocohl sliced the fruit in half, then in quarters, then in eighths—it was pulp all the way through. No
seeds. If it had no seeds, how did it propagate? “Runners?” she asked; but, as she expected, swift-Kalat
said, “It has none.”
Tocohl said, “Then why doesn’t MGE accept this as initial proof that something on the planet, not
necessarily the sprookjes, is capable of creating an artifact?”
“I was not hired as a botanist.”
“That makes you no less knowledgeable,” Tocoh said.
“To Kejesli, it does,” said Alfvaen. “When I worked with him before, he considered a person’sprimary specialty his only specialty.”
“Ah, and the team botanist?” asked Tocohl.
“He considers Flashfever wildlife so unusual that anything is within the realm of possibility—that we
simply haven’t found this plant’s particular mechanism yet.”
Tocohl gazed down at the slices of fruit spread on the surface of the table. “But you say artifact.”
Emphasizing his words with a clash of bracelets swift-Kalat said, “I say artifact.”
“They’re too curious not to be sentient,” said van Zoveel. “They are interested in everything.”
“So’s Maggy,” said Alfvaen abruptly. She pointed: the arachne was opening cupboards.
“Cut that out, Maggy,” said Tocohl. “That’s impolite. You should always ask permission before you
open a closed door.” (—At least, if you’re doing it in public,) she added, sotto voce.
Tocohl gestured. “Come over here.—I apologize, swift-Kalat. When you told us your house was
ours, Maggy interpreted it in Hellspark. That’s the language she knows best.”
(You lie!) said Maggy.
(Polite, social,) Tocohl told her, (but take care not to call anyone a liar aloud in the presence of a
Jenji. Now, come apologize.)
The arachne made a skittering dash for the spot Tocohl had indicated. “I’m sorry,” said Maggy, using
the vocoder in the fat body.
“Don’t apologize to me, apologize to swift-Kalat. You know enough Jenji for that.”
The arachne dipped slightly before swift-Kalat and said, “I apologize if I have given offense. I
intended none.” Tocohl recognized the phrasing as her own, pitched to match Maggy’s voice.
Alfvaen delighted at their surprise. “Magic to a Hershlaing,” she said to Tocohl. Tocohl smiled.
(Hershlaing?) Maggy asked privately. Tocohl said, (Hershlaig is a mythical world so far off the
beaten orbit that the Hershlaing consider any advanced science—even striking a match to light a fire—to
be magic. Introduce yourself, Maggy, and give them an idea what state-of-the-art is.)
Before Maggy could begin, a tall creature pulled the door membrane aside and stepped, its feathers
silvered with rainwater, into the cabin. It was a beautiful thing. Tocohl stared at it in wonder, and it stared
back at her.
At last, she let go her breath. “Sprookjes—fairy tale creatures,” she said. “Now I understand the
name.”
Maggy’s arachne walked slowly around the sprookje for a better look. The sprookje turned to
follow the movement; it showed no hesitation in turning its back to the humans.
Tocohl rose, only the soft rustle of the moss cloak betraying her movement. Van Zoveel caught her
am “They bite,” he said, quietly. “Everyone on the team has been bitten once.”
“Have there been any ill effects?”
“No, but I didn’t want you to be startled.”
Maggy had completed her circle and the sprookje was brought face-to-face with Tocohl. The two of
them stared at each other. The creature’s brown and gold feathers gleamed and whispered as it took a
step closer.
Tocohl held her ground. When the sprook stopped, she slowly and deliberately rolled up her cuff and
lifted her arm to bring her hand a scant two centimeters from the beaklike mouth.
The sprookje accepted the invitation and bit, its head flashing forward with startling suddenness
Tocohl flinched but made no outcry—she was more surprised than hurt, for she hadn’t been snapped at
by the potentially nasty beak. It was exactly like being stabbed with a pin.
She brought her hand slowly back to inspect the wound—yes, a mere pinprick.
“Buntec calls it their sampling tooth,” van Zoveel volunteered. The sprookje now walked around
Tocol in the manner of the arachne’s inspection.
Alfvaen gave a sharp cry of warning. Tocohl turned swiftly to find the sprookje drawing back. “I
thought it meant to bite your shoulder,” Alfvaen explained “I’m sorry I startled you.”
(Maggy? What happened?) asked Tocohl; and Maggy replied, (It bit your cloak; from the trajectory,
that was all it intended to sample.)
“Alfvaen,” Tocohl said aloud, “are you willing to try an experiment?”“Yes, of course.”
Tocohl unclasped her cloak and tossed it into the Siveyn’s arms. “Put that on,” she instructed, “then
come out here and do exactly as I did.”
Alfvaen followed her instructions to the letter, even to letting the sprookje complete the distance, as
she had previously done with swift-Kalat. With the same deliberation Tocohl had used, she lifted her
hand, and, as expected, the sprookje nipped. Tocohl watched the entire procedure as closely as she
could. (Maggy? Did the sprookje’s cheek-feathers puff out, or was it my imagination?)
(No imagination—want to see?)
(No, I want to confirm that they didn’t when the sprookje nipped me.)
(Confirmed,) said Maggy.
The creature circled Alfvaen slowly. Tocohl kept her attention close, curious to see what it would do
about the cloak. After a moment, it seemed to have completed its inspection of her. It had completely
ignored the moss cloak.
Then the sprookje’s beak flashed forward—Alfvaen yelped in surprise. Rubbing her wrist, where the
sprookje had bitten her a second time, she said accusingly to van Zoveel, “You said everyone had been
bitten once! I thought you meant only once!”
“He did!” said swift-Kalat. “You are the first to have been bitten more than once!” He was echoed
word for word by the sprookje.
One could easily develop a stutter from speaking in the presence of one of these creatures, thought
Tocohl; it was like listening to oneself on a two-second delay.
Tocohl was struck by another oddity: the puzzling fact that the sprookje echoed only swift-Kalat.
Swift-Kalat seemed to have learned to ignore it. He came toward Tocohl excitedly, “And the cloak!
Why would it bite your cloak?”
The excitement was too much for the sprookje. Even as it repeated swift-Kalat’s words, it backed
hastily away, its cheek-feathers now unmistakably puffed.
“Quietly,” said Tocohl. “—That’s a moss cloak,” she explained. “Your sprookje can obviously tell
the difference between living and nonliving. It didn’t bite the arachne, after all. And it lost interest in the
moss cloak having bitten it once.”
She glanced at the pinprick on her wrist. “I think ‘sample tooth’ is dead on.—As for Alfvaen,
Alfvaen tastes different than the rest of us!”
“Of course,” said Alfvaen, “I have Cana’s disease!”
“Yes,” said Tocohl. (Maggy,) she added privately (tomorrow morning, if necessary, I will have a
violent attack of an unidentifiable plague, probably from having been bitten by our fine feathered friend
over there. If I have to get this planet quarantined to gain time, I will!)
Sunrise on Flashfever met the omnipresent rainclouds with a rare brilliance. From within swift Kalat’s
cabin came the sweet, silvery sound of the tyril. Tocohl leaned back against the door frame to appreciate
them both before returning her consideration to the compound.
Any creature’s behavior is affected by its environment. Like most survey camps Tocohl had seen,
this was utilitarian. It was standard operating procedure to sterilize an area of ground for base camp.
Here, the result was thick red mud everywhere. Tocohl thought it odd that no walks had been built, either
at ground level or higher. The uniform, nondescript cabins (a small town of them—privacy was a very
real need when some forty people had to spend two to ten years together) stood partly raised from the
mud on stubby stilts.
Only one of these had been personalized on the exterior. It was painted a lavish blue and decorated
with Yn mystic symbols of white and gold. Two pennants hung near the door, drooping heavily with
rainwater. That must have belonged to the dead man, Oloitokitok, she thought.
A sprookje splashed through muddy puddles to stop some distance away. Seemingly attracted by
the sound of swift-Kalat’s tyril, it cocked its head to listen, but made no attempt to mimic the spritely
dance tune. After a while, it knelt, pressed its hands into the mud. She wondered what it might be doing.
The sprookje’s presence reminded Tocohl that she was ill—ill with something unknown but notdebilitating enough to require bed rest. With Maggy’s assistance, she chose a handful of symptoms and
set to work initiating them.
By the time she was done, the sprookje also had finished its task, if indeed it had been at one, and
stood gracefully. It ran long fingers through the feathers on its knees and shook away some of the clinging
mud. Tocohl blinked at it but, for a moment, she could not see clearly.
Still dazed from effort, she was dazzled by the flashwood that ringed the camp, pressing at every
length of fence, as if offended by and yet drawn to the barren space within. Its glitter made the camp
more stark by contrast.
As her vision cleared, she saw that the fence was barbed wire, not the electrified barrier favored by
survey teams. When the dance tune came to an end, she peeled back the membrane and asked
swift-Kalat, “Why barbed wire?”
Swift-Kalat laid his tyril aside and joined her in the doorway. His glittering bracelets and the sun
raising iridescent highlights in his black braid shamed the compound as much as the flashwood.
“So much of Flashfever’s wildlife uses electricity as an energy source that an electric fence only
attracts trouble. Buntec suggested we try that sort. It works quite well.”
“I see,” said Tocohl.
She judged it time to act, and because swift-Kalat was Jenji and had the traditional reputation for
truthfulness, she decided to let him draw his own conclusions. She raised her hand to her forehead and,
looking puzzled, let the blood drain from her face as if she might faint.
“Your hand,” he said, and caught her wrist to examine the pinprick she’d received from the sprookje
the night before.
The area around the puncture was an angry red and slightly puffed—a matter of dilating the local
capillaries. Once done, Tocohl could maintain it indefinitely without strain, despite the effort of
concentration it required to initiate.
It had the desired effect. Swift-Kalat pressed gently but firmly at the edges of the swollen area; his
fingers left whitened marks. Tocohl winced. “The doctor must see this,” said swift-Kalat. Without
releasing her hand, he drew her across the compound to the blue cabin. He struck a chime.
“You may enter,” said a regal voice from within.
The survey team’s doctor sat cross-legged in the center of the room, on a blue mat ornamented with
designs of power. Her mouth was broad and rich with hidden smiles, the fine lines at the corners of her
eyes could only have come from laughter. Her whole face was designed for joy—and yet she did not
smile. Her dark eyes brimmed with anger, although it was not directed at Tocohl or swift-Kalat.
She was plump and deceptively well muscled beneath that plumpness. By swift-Kalat’s standards
she was, no doubt, overweight; but Tocohl, who was already thinking in Yn, took her on her own
culture’s terms and found her beautiful.
In her lap lay the rich glitter of a koli thread with its fantastical tangle of knots. Around her lay a
chalice, three silver knives, and a strawlike pile of jievnal sticks: she was preparing to enter deep
mourning. Tocohl was glad she had decided to act quickly; to interrupt mourning would be risky, even
for her.
Tocohl raised both hands in greeting and, as the woman lifted her head and hands to reply in kind, all
of Tocohl’s hopes for a quarantine vanished. Two long scars slashed across her left cheek and on each
index finger she bore a bluestone ring. The doctor was an Yn shaman.
“I am layli-layli calulan,” she said, in a cool, quiet voice.
Tocohl inclined her head a fraction of an inch and responded, “I am the tocohli susumo.” To give
one’s true name to an Yn was to give that Yn power over one. Accordingly the Hellsparks had, from the
very beginning of their trade relationships with the Yn, convinced them that no Hellspark name was more
than a title, the equivalent of the designations Yn women gave to others. She also took the liberty of
ascribing to herself the sound of power, the tiny phoneme i, which gave her status, though nothing like
that the doubling i gave layli-layli calulan.
“You lie,” said the shaman, in GalLing’.
Swift-Kalat took in his breath with a hiss. His braceleted arm came up as if to ward off a blow, butTocohl caught it and quieted him with the sharp negative tap of a finger.
To layli-layli, she said solemnly, “As do you.”
(I don’t understand,) said the voice in Tocohl’s ear.
(Check a tourist guide to Y and I’ll fill you in later, Maggy.)
Tocohl turned to swift-Kalat. His forehead was beaded with sweat. “It is a ritual greeting,” she said.
“I apologize for the mistranslation.”
Swift-Kalat jerked his head from one to the other. “In my culture,” he said, “it is an insult of the
highest order.”
“I am aware of that. I said, ‘mistranslation,’” Tocohl repeated. “The Yn word means both ‘lie’ and
‘dream’—it only becomes a problem when you try to pick an equivalent in GalLing’. There is no
equivalent in GalLing’, but ‘dream’ is much closer to its emotional meaning.”
She could see him make a visible effort to replace his emotional reaction with an intellectual one.
Then he pointed to Tocohl’s swollen hand. “She was bitten by a sprookje,” he began.
Tocohl interrupted. “Your pardon, swift-Kalat, but I must speak to layli-layli calulan alone.”
Gratefully, swift-Kalat accepted the dismissal.
When he was gone, layli-layli calulan said, “You are not alone with me.”
Tocohl was startled. Either layli-layli calulan was sharp-eyed enough to have seen the muscle
twitch that signaled her subvocal exchange with Maggy, or she was relying on her shaman’s espabilities.
“No,” admitted Tocohl. She tapped the implant. “My partner, the maggy-maggy lynn listens as
well.”
Because she now spoke Yn, Tocohl used the my that signaled personal relationship rather than
property, which in Yn culture included males as well. That was how she thought of Maggy, she realized,
as both her partner and female. She had also translated lord into the Yn doubling, quite unintentionally
giving her equal status with layli-layli herself. She made a mental note not to introduce Maggy as a
demonstration of state-of-the-art after all.
Instead, she said, “I would introduce you properly, but maggy-maggy has no facilities for speech
except through me. If you wish to greet her, please do. She will acknowledge the introduction through the
vocoder in her arachne later.”
Layli-layli calulan made Yn formal greeting to Maggy.
When she was finished, Tocohl crossed her ankles and sat before the Yn shaman. She held out the
“injured” hand. Her ruse was still worth the try, but it was not worth upsetting swift-Kalat if she was
found out. “I was bitten by a sprookje last night. This morning... well, it’s infected, I think, and
swift-Kalat tells me that’s never happened before.”
The shaman lit a jievnal stick and its piney odor filled the small room. She thrust the slender rod into
her hair, took Tocohl’s hand gently in her own. For a moment, her dark eyes looked puzzled, then she
said, “You did this to yourself? To my knowledge, there is no one in the survey team who could have
done this for you.”
“Could have done what?” said Tocohl with puzzled innocence.
Layli-layli calulan’s dark eyes lit suddenly with amusement, and Tocohl dropped her gaze before
that knowing scrutiny. “All right,” she said, “I was trained in the Methven rituals.”
“You are an adept,” said layli-layli calulan.
“Not adept enough.”
Layli-layli released Tocohl’s hand and twisted the bluestone ring from her left finger. The rings, by
Yn tradition, prevented the accidental release of power. In reality, Tocohl suspected that the rings only
worked because the Yns believed they worked—many espabilities needed a channel or focus or, in this
case, a control.
The shaman held out her right hand and Tocohl laid her swollen wrist across the waiting palm. The tip
of layli-layli’s bare finger touched her injury with feathery delicacy.
Just for a moment, for the pure devilment, Tocohl concentrated on maintaining the dilation of the
capillaries. Dark eyes met the Hellspark gold, and a trace of smile touched the corners of layli-layli
calulan’s broad mouth. Then the heat in Tocohl’s wrist cooled, the swelling began to subside.Activated by layli-layli’s espability, Tocohl’s cells found their normal pattern and set about to regain
it. Against the shaman’s gift, Tocohl had no chance of maintaining the artificial illness.
The red faded to its original shade. Soon only the pinprick remained, and that too was healing
rapidly.
Layli-layli calulan replaced her ring and said, “You too believe swift-Kalat. So did Oloitokitok.”
She took up the koli thread from her lap, and as she spoke, her fingers added knot after intricate knot to
its tangled glitter.
“Long before you dreamed your first dream,” layli-layli calulan began, in the manner of a mother
telling a tale to a child, “there was a man named Oloitokitok who was not like other men. He thought and
dreamed like a woman. He dreamed a dream so strong that it took him to a world no woman’s eye had
ever seen...”
Listening to the Tale of Oloitokitok, Tocohl heard much that someone unfamiliar with Yn culture
would have missed. The Yn were so gynocentric that only in the last hundred years had their men been
taught to read. For Oloitokitok to have achieved as much as he had, he must have been very special
indeed.
He had agreed with swift-Kalat’s assessment of the evidence, and he had chosen to gather evidence
of the sprookjes’ sentience on his own. Although layli-layli calulan confirmed swift-Kalat’s
observations about Oloitokitok’s manner on the day of his disapperence, Oloitokitok had told no one,
not even what he planned to do or where he planned to go.
Tocohl wasn’t surprised. To the members of the survey team, Oloitokitok may not have been a
token male but, in his own mind, he may have thought himself so. Given partial evidence in favor of the
sprookjes’ sentience and a belief that no one would credit his opinion, he had quite likely chosen to
gather such overwhelming evidence as to present a fait accompli that would force belief.
Now what evidence he might have had was lost with him.
As if echoing Tocohl’s thoughts, layli-layli calulan said, her voice harsh, hurt, “The dream was lost
with Oloitokitok.” With that, she grasped the free ends of the koli thread and gave a slow, steady pull.
One by one, the glittering knots unraveled, until she held only straight bare line shining coldly between her
outstretched hands. The tale was ended.
Tocohl gave a sharp upward jerk of her chin. “No,” she said, “I keep the dream.” She gestured at
the string. “It’s true a single koli thread leaves no knots, but, alive, Oloitokitok would have knotted his
thread with the beings of this world. Despite his death, it is still possible if you wish it.”
Layli-layli looked hesitant. Tocohl wondered how important Oloitokitok had been to her. Looking
down, she once again saw the chalice, the knives, the jievnal sticks. This time she registered them
properly. Layli-layli calulan was preparing to go into deep mourning—something only done for women,
never for men.
When she looked up again, layli-layli placed her palms together, ring on ring, and said with quiet
defiance, “He was my mate.” She used the my for relationship.
Tocohl held out both her hands, the strongest symbol of understanding and agreement available to
her in the Yn mode, and clasped layli-layli calulan’s wrists in her own supporting grip.
Swift-Kalat was only partially relieved that Tocohl Susumo had sent him away. He needed the time
to put his thoughts in order. The last time he had heard someone call another a liar in GalLing’, the
ensuing fight had resulted in a death, so he was well aware of the potency of the word even in its
unreliable Gal-Ling’ form. To hear it used as a greeting was more than he could handle. He found himself
envying the Hellsparks their ability to deal with such rupturing of their social order. Having at last settled
his thoughts on the matter, intellectually if not emotionally, he now wished he were back inside layli-layli
calulan’s cabin, listening to the conversation between the two.
“Jaef! Jaef!”
Even though the sound was distorted by the shout from across the compound and a peal of far-off
thunder, he knew it had to be Alfvaen. Of all the surveyors, she alone knew and had the right to use his
soft-name. She raced toward him, heedless of the muddy water she splashed with every footfall.Breathless, she drew up beside him—too far away, some small portion of his mind noted—and said
in GalLing’, “Jaef, Kejes-sli’s-s readying an automated message capsule... He’s s-sending the report to
MGE now!”
It was deductively true: beyond her swift-Kalat could see the other surveyors coming from their
cabins to gather before Kejesli’s quarters. The final report was a matter of ritual, requiring the presence
of all those responsible. Except that Oloitokitok would not be present.
Still staring up at him anxiously, Alfvaen swayed suddenly. He shot out a hand to steady her,
remembering as he did so that stress aggravated her condition. “Your medication, Alfvaen,” he said. She
focused with effort on his face, then her eyes widened in an exaggerated manner and she reached for her
pouch. He waited only long enough to assure himself she could stand on her own, then he released her
arm to ring the chime beside the door to layli-layli calulan’s cabin.
He did not wait for an answer. Instead, he thrust his head inside, to find layli-layli calulan and
Tocohl Susumo with their hands clasped.
“Will you help?” Tocohl asked layli-layli calulan. Wanting to hear the answer as much as she,
swift-Kalat held his tongue. Layli-layli calulan said, “By quarantining Lassti? That would give you time,
not necessarily understanding.”
She said no more. Swift-Kalat felt he must make the urgency of the query clear. “Alfvaen tells me
that Captain Kejesli is preparing an automated message capsule for MGE now,” he said.
Tocohl jerked her head back to stare at him. Releasing layli-layli calulan’s hands with a few
murmured words in another language, Tocohl rose smoothly to her feet. Layli-layli calulan remained as
she had been, her stare holding Tocohl in place.
She said, “Should I help creatures that were responsible for Oloitokitok’s death?” Spoken as it was
in GalLing’, the question was directed at him as well, but he had no answer. The question itself was
unreliable.
Again Tocohl dealt with the matter on a level he himself would not have been able to. She spoke one
word only; the word was, “No.”
Catching him by the elbow, she ushered him out, stopped momentarily in her tracks to scan the
compound, said, “Ah: Kejesli’s quarters?” When he confirmed that, she touched her fingers briefly to the
ornate pin at her throat. “One more try,” she said, pausing to give Alfvaen a reassuring smile, then she
squared her shoulders and strode across the compound, her cloak swirling like heavy mist in the light
rain.
Swift-Kalat put his arm around Alfvaen’s shoulder, as much to comfort himself as to support her, and
led her in the same direction. At the edge of the crowd, he heard Tocohl bark rapid-fire some dozen or
so words, each with the sound of a different language to it. Heads turned in succession, and the crowd
parted to let her through.
Without Tocohl’s skill at linguistic manipulation, swift-Kalat and Alfvaen found themselves stayed at
the edge of the crowd. “I must get her to teach me that,” Alfvaen said, giggling despite her overall
anxiety.
“Teach you what?”
“I only recognized the Sheveschkem ‘Cheap tattoos!’ but I’ll bet all the others were the
same—whatever a waiter says to negotiate a crowd with a tray of hot dishes.”
He stared down at her, fondly at first, appreciating the joke as she had found it, then he raised his
eyes to stare into the distance, deep in consideration.
Tocohl had found something in layli-layli calulan’s last phrase that she could answer, and that fact
still concerned him. Could the question be answered in Jenji? Could it even be asked in Jenji?
He tried framing it carefully in his mind: Should I help creatures that were responsible for
Oloitokitok’s death? But death in GalLing’ was ambiguous; it could mean “natural death” or “accidental
death” or even “murder.”
Murder, he thought. He patted Alfvaen’s arm absently and released it, to pace away from the noise
of the crowd to follow the thought. He himself had told Tocohl the causes given for Oloitokitok’s death
were unlikely. “What then killed Oloitokitok?” she had asked.That was a question that indeed could be framed in Jenji... One to which he would very much like an
answer.
Chapter Six
THE CEILING IN the captain’s quarters had been lowered to conform to Sheveschkem spatial
standards—no doubt to the extreme discomfort of most members of the survey team, thought Tocohl.
Generations of sailing had left their mark on Kejesli even here, as a need to keep the ceiling within reach.
Nothing better sustained balance below deck in stormy seas than a flattened palm against a ceiling. Under
the circumstances, Tocohl had to suppress her own impulse to reach for the ceiling. “Captain,” she
repeated, “all I’m asking is a few months’ grace.”
Alone with Kejesli, she automatically followed his lead and “danced” Sheveschkem, despite the fact
that he spoke GalLing’ and she replied in kind. She spoke in GalLing’ because Kejesli refused to speak
Sheveschkem with her. She wished it weren’t so; she might have been more convincing in Sheveschkem.
She continued, “If you send your final report now...”
Kejesli tightened his grip on his desk, as any Sheveschkem captain might grip the bolted furniture for
support. “Hellspark, you can stay as long as you wish. Half the survey has made a point of requesting
your continued presence.” He was clearly not pleased about that. “If you find evidence—beyond
swift-Kalat’s sleight-of-tongue—that the sprookjes are sentient, you can always appeal to the Comity’s
courts.”
Tocohl’s hand swept to one side, a derisive gesture on Sheveschke. “It would take years in
court—and by then irreparable damage may have been done to the sprookjes, to their world. Veschke’s
sparks, man, will you be responsible for genocide?” She shot the word at him, and he flinched.
Just for a moment, Tocohl thought she had struck home; both knew how Veschke would take such
an act.—Then Kejesli stiffened and said, “I don’t know they’re sentient.”
“That should be sufficient reason to allow us more time.”
Kejesli’s knuckles whitened. “I rely upon what my people decide; and, in this case, all their evidence
points to nonsentience.”
“—All?”
“We hire people to do specific jobs in specific areas. They have done them.” His beaded hair swung
to the side, past stiffly set jaw.
No Sheveschkem sea captain could have said that: In an emergency, the cook lowers the mainsail.
Tocohl frowned, and saw Kejesli suddenly for what he was. He was a man trying not to be
Sheveschkem, without conscious knowledge of what being Sheveschkem actually entailed. He spoke
GalLing’ but danced Sheveschkem; he wore worlds’ motley, but lowered his ceiling. Not comfortable
with the cultures surrounding him, he was no longer comfortable with his own, so he substituted the rule
book for culture. If I can give him a way out by the rule book...
Under her scrutiny, Kejesli once more gripped the desk. “I would like to oblige you,” he said, “one
should always be obliging to Hellsparks... But in this case I cannot. The thunderstorms have already left
us behind schedule. Now MGE has pressed me for a quick decision.”
He loosed his grip on the desk and rose. He did not reach for the ceiling; the storm was over as far
as he was concerned. He had reached his decision. Tocoh knew she had lost the battle.
As he showed her to the door, it occurred to her that he had, at least, agreed to let her remain or
Flashfever. Here, it might still be possible to follow Oloitokitok’s lead, and present a fait accompli.
Perhaps because of her silence, perhaps because, for him, the emergency was over, Kejesli’s manner
softened. As he drew the membrane aside and stepped into the wan sunlight, he said, “Come,
Hellspark—for you it is only a theft at Festival. For me, it is a good deal more.”
She did not reply. If her oblique appeal to Veschke’s good opinion had not worked, then the only
way around him was by the rule book. Her quarantine ploy would have worked—could still work. A
glance at her hand showed redness remaining; it would take her only minutes to reestablish her spuriousinfection with layli-layli calulan’s assistance.
Tocohl plunged through the crowd that had gathered outside Kejesli’s quarters in anticipation of the
ritual that marked the end of their job. Alfvaen swift-Kalat, and van Zoveel turned anxiously to her but
she brushed them aside absently. “I can stay,” she said, “but the report goes.”
The news brought a mixed reaction. Swift-Kalai turned abruptly and walked a short distance away,
anger and disappointment stiffening his gait. Tocohi automatically caught Alfvaen’s arm to prevent her
from following him: an angry Jenji is, by definition, unreliable. He would not appreciate her company at
the moment. Still without conscious thought, Tocohl drew Alfvaen along with her.
Watchful, the shaman stood quietly apart, a jievnal stick laced through her hair. Her eyes followed
the sprookje that wandered among the humans. Only layli-layli calulan had the power to grant the
sprookjes a stay of execution, Tocohl thought. Would she?
Thrusting Alfvaen forward—a talisman of serendipity to influence a shaman—Tocohl folded her arms
across her chest and stared long and hard at layli-layli calulan, willing her to speak.
“Tocohl Susumo!” swift-Kalat’s voice and instantaneous sprookje-echo rang with such command
that all, Kejesli included, turned to him.
Caught by his tone, Tocohl responded formally. “Yes?”
Swift-Kalat’s bracelets flashed as he leveled his arm at the sprookje. The sprookje, feathers ruffling,
imitated his gesture with frightening accuracy. And, as swift-Kalat spoke, it echoed word for word: “I
accuse the sprookjes of the deliberate premeditated murder of Oloitokitok. Will you agree to judge?”
At Tocohl’s side, Alfvaen gave a short, sharp gasp. Tocohl caught her shoulder and gave her a look
of silent command. Alfvaen held her tongue.
“Yes,” said Tocohl, “I agree to judge.”
Maggy pinged furiously for attention and, when Tocohl ignored her, said, (The penalty for
impersonating—)
(I know, Maggy, now just shut up.)
“You can’t,” said Rav Kejesli. It came out like a plea. “The sprookjes would have to be sentient in
order to commit murder.”
“Yes,” said Tocohl, “they would.” She could not help but grin. “I will first be obliged to make a
judgment on the sentience or nonsentience of the sprookjes.—Would you be kind enough, Captain
Kejesli, to have your team put their files at my disposal?”
For a long breath, Kejesli said nothing; his face had the look of a man in great pain. Then, slowly and
almost implausibly, he smiled.
“In that case, I will hold my report until you have made your judgment.” His eyes shifted from her
face to the pin of high-change in her cloak. “—In Veschke’s honor!” he finished.
Chapter Seven
FORTY-TWO MEMBERS OF the survey team crowded the common room with excitement,
jostling each other and speaking in whispers. You’d think, and Tocohl did, that surveying an unchartered
planet would be enough excitement for anyone, but obviously it was not so. News of a judgment,
coupled as it was with the accusation of murder, stirred even the oldest and most blase of the team
members.
Tocohl scanned the crowd for the reactions of those she had already met. Om im had been accorded
a position in the front, in deference to his size, and he grinned at her and winked broadly, gesturing across
the room to Edge-of-Dark. To her costume of the night before, the programmer had added a second
victoria ribbon, this one pale green, which crossed her breast at right angles to the first, and tall laced
softboots of Ringsilver fashion. Tocohl flashed a wink at Om im; her pictures of Madly had worked.
Captain Rav Kejesli made a grudging formal introduction and the room became silent but for the
monotony of rain.
“By now,” said Tocohl, “you’ve all heard that the sprookjes have been accused of murder; and mostof you realize we’ve an unusual situation on our hands. In essence, in order to judge the guilt or
innocence of the creatures, I must first know to my own satisfaction whether or not they are sentient.”
The whisper of noise became a surprised chatter of voices, and Tocohl raised a hand. “Wait and hear me
out.”
When the noise quieted once again, she continued, “I know that your primary specialists all seem to
have reached the conclusion that the sprookjes are not sentient, but I would like to keep an open mind
on the question. Some of the secondary specialists are not so convinced, and a secondary specialist is
not mere backup. Survey teams were designed to have as many talents and specialties available as
possible, and I believe that the original intent was to take advantage of the synergistic effects among the
surveyors as well.
“So I’m asking for your cooperation in an experiment. Let us for the moment forget authority. If
anyone has anything to say on the subject of sprookjes, I want to hear it. I don’t care how wild it is, I
don’t care if it’s totally out of your field of expertise—I want to hear it anyway. I’ll even listen to
anecdotes about the sprookjes.” She flashed a grin at Om im. “Story for story,” she finished, to add a bit
of a bribe for their effort.
Once more, she scanned the group—surveyed the surveyors, she thought with a smile. By virtue of
the novelty of the situation, she’d get her cooperation and then some. As for slighting the primary
specialists—each primary specialist had a secondary or tertiary specialty; given the chance she offered,
they’d be delighted to show off.
“One last thing,” she said, “before I send you all off to dig out material for me. Has anyone here fallen
on Pasic?”
There was a titter of amusement—obviously some had.
It was John the Smith who pushed forward to say, “Pasicans are the closest things I’ve ever met to
the Hershlaing in the flesh. They’re as nontechnological as they come, at least within the known human
realm. They don’t even have, oh—matches or flints!”
Within the known human realm, Tocohl observed with satisfaction, you’re thinking already.
“True,” she said aloud. “Now, a Pasican once told me the difference between himself and an
orival—that’s a small native animal. ‘An orival does not know how to put branches on the fire when it is
dying, therefore a Pasican is human and an orival is an animal.’”
It brought a chuckle of superiority from the crowd. Tocohl waited for it to pass, then she said, “I may
not be human to a Pasican.”
That got their full attention. Spreading her cloak for the added drama of the gesture, she went on,
“My 2nd skin provides me with all the warmth I need. My spectacles can push for available light. A fire is
of no particular use to me.—Not knowing I must prove myself human in this one fashion, I may let the
fire go out!”
“See here!” It was John the Smith again, and this time he was angry. “Are you saying that the
sprookjes may be so advanced—.”
Tocohl said, “No. I’m saying that they may be so different that we don’t recognize one of their
artifacts when we get our noses rubbed in it... And I’m saying that even Homo sapiens within historical
time have had difficulty in proving their humanity to other Homo sapiens. I’m asking that you all consider
the circumstances in which you would be hard put to prove your sentience, especially if you were
unaware that you were being tested.”
Enlisting Buntec and a daisy-clipper, Tocohl and Alfvaen made a quick trip out to the Margaret
Lord Lynn. Maggy, for once giving no warnings and predicting no doom, taped Tocohl’s subvocalized
message on the way and waited with open door when they arrived.
Buntec stowed their belongings in the daisy-clipper amid cheerful obscenities and colorful
blasphemies. And Alfvaen said, in Siveyn, “You told me you weren’t a judge.”
Tocohl said quietly, “I lied to you.”
“But why? We’d only just met; you had no reason to lie to me.” The small hand flew lightly outward,
dismayed.“No offense intended,” said Tocohl easily. “I like to keep in practice. In Veschke’s honor.”
Tinling Alfvaen frowned up at her, as if squinting into the sun. Tocohl could almost read the thoughts
as they rippled through the Siveyn’s mind: anger, then suspicion, and, finally, concern.
“All right,” said Alfvaen. “You lied to me. No offense taken.”
Alfvaen would keep her own counsel; but the concern in the Siveyn’s green eyes did not fade.
Buntec bellowed from the hatch, “All stowed! Let’s move—Flashfever looks about to do its act
again!”
As they sped toward base camp, Alfvaen maintained a pensive silence as Buntec cajoled and cursed
the daisy-clipper along its way.
Behind them, the Margaret Lord Lynn rose solemnly into the sky and disappeared. Tocohl watched
the ship go, and answered Buntec’s query with an economical, “Geosynchronous orbit. Better for
communication.”
“Oh,” said Buntec, “if I’d known you had one of those top-line computers, I’d’ve stuck around to
watch. You have an implant too?”
“Yes,” said Tocohl, and Buntec said, “Before you run out on us, give me a guided tour, will you?
Talk about technological toys... !”
Tocohl grinned. Not only did Buntec have a passion for technological toys but, Tocohl suspected
from the way the Jannisetti handled the daisy-clipper, she was a gifted player as well. She hadn’t seen
any research on the subject, but she’d always suspected that there was an espability relating to machinery
that was kith and kin to the more common “green thumb.” A “metal thumb,” perhaps; whatever it was,
Buntec was a prime example. “If you’ll keep Maggy’s abilities to yourself for the duration, Buntec, I
promise you a chance to talk to her yourself.”
“Talk to her? That top-line?” Buntec raised her eyebrows, simultaneously demonstrating her pleasure
by raising the daisy-clipper in a neat arc as well. “A nosy-poke computer?”
Tocohl laughed; she’d never heard the Jannisetti term for a computer of Maggy’s capabilities, but she
was willing to bet that a literal translation. “A nosy-poke computer,” she repeated, “that she most
certainly is.”
(Should I resent that?) Maggy asked. Tocohl couldn’t help but repeat the query for Buntec’s benefit.
“Resent it?” said Buntec. “Shit, no! Wow! And hello there, Maggy! I meant it as a compliment.”
(Tell her thank you for me.)
Tocohl relayed the message.
“You’re on, Tocohl. My mouth is stitched shut. But I do warn you there are a couple-three
smartasses in the crew might spot a nosy-poke faster than me.”
“Just don’t give them any help.”
(Stabilization of orbit in three minutes,) said Maggy, sparing her the details. (I launched the message
capsule, and it should reach Sheveschke in about six days, unless something goes wrong.)
(Fine,) said Tocohl. (Now if Alfvaen asks you whether or not I’m a judge—though I doubt she
will—if she does, you are to tell her that I am.)
(You want me to lie?) Maggy somehow managed to sound outraged.
(That’s it exactly. I want you to lie.)
(I can’t lie.)
(Nonsense. Of course you can. That’s a direct order, so I’ll have no more of your lip.)
(Suppose Captain Kejesli asks his own computer: it won’t lie. I tried to talk with it, and it’s too
stupid to lie.)
(Nicely phrased, Maggy.—And no doubt it does contain a list of byworld judges. In which case it
will contain the name Tocohl Sisumo.)
Maggy made a rude noise, and Tocohl almost choked with laughter. (That’s your father,) said
Maggy. (That won’t help at all.)
(The rude noise,) said Tocohl, (was not quite appropriate, but I’m glad you’ve added it to your
repertoire—at least, I think I’m glad. In any event, if Kejesli sees Tocohl Sisumo, he’ll assume it’s a
lousy transliteration into GalLing’. Stop worrying, Maggy; Kejesli would stand for a higher garble-factorthan that.)
(That’s not what I’m worried about,) Maggy said primly.
(Okay, okay. But keep your worries to yourself,) Tocohl finished, and turned her attention back to
Buntec, who said cheerfully, “Gossip away. Don’t let me interrupt. Move ass, you dopes!”
This last was shouted out the window, as Buntec steered the daisy-clipper into the compound,
spraying all those who hadn’t turned and run with a comprehensive layer of red mud. Directly opposite
swift-Kalat’s door, Buntec grounded the daisy-clipper with feather lightness.
“You’re not interrupting,” Tocohl said. She slid from the craft, caught at the door frame abruptly.
“Watch your step,” she cautioned, “it’s slippery out here.”
“Always is,” said Buntec. “Makes a fine mess of things, doesn’t it?” She landed beside Tocohl with
a splash. “I’ve been thinking,” she said as she snatched luggage from the daisy-clipper, “swift-Kalat says
he’s got a biological artifact—Hitoshi Dan say it’s not an artifact, but he can’t figure out how it
propagates, right?”
“Right,” said Tocohl as she took her parcels from Buntec. “What do you have in mind?”
“Suppose,” said Buntec, hefting the last of the parcels herself and following them up the steps into
swift-Kalat’s quarters, “Suppose we just assume it’s an artifact and go from there. Where does that get
us?”
“Good question: by Comity standards, we’ve got to prove the sprookjes have language, artifacts,
and art or religion. It could be argued that language is an artifact—and has been, in fact. As I recall, both
dolphins and whaffles whistled by on the strength of their poetry. And that,” said Tocohl, dropping her
bundle, “means that art and artifacts overlap as a category.”
“So all we have to do is prove to our mutual satisfaction that the sprookjes are sentient,” Buntec
observed. She glanced about and, failing to find a spot to stow the parcel she carried, raised an
interrogatory hand at Alfvaen. “All!” snorted Alfvaen, misunderstanding the query.
“What I’m getting at,” Buntec said, handing the Siveyn the parcel and turning again to Tocohl, “is that
perhaps we should assume all their artifacts will be biological. We haven’t found anything else, after all.”
Tocohl stopped in the act of stowing to give Buntec her full attention. “You think we should be
looking for other biological artifacts?”
“Why not?”
“Why not, indeed,” Tocohl agreed. “Do you have anything particular in mind?”
“I came straight off the farm.” Buntec grinned and lifted a foot. “That’s not mud you see,
honey.—We had our share of gene-tailored crops and animals. Now that’s a biological artifact right
there, but it’s not one you could spot. But even with all the high-order stuff, we did the basics. Grafting is
about as basic as you can get, aside from the simple switch from hunting-gathering to genuine
agriculture.”
“The sprookjes appear not to have made that switch,” said a new voice.
“Neither did dolphins,” said Tocohl. She looked at Buntec questioningly.
Beckoning in the newcomer, Buntec said, “Timosie Megeve, Tocohl Susumo, and Tinling Alfvaen.”
Timosie Megeve was Maldeneantine, from the severe wind-red of his oversuit to his earpips, held as
they were by a thread about the cap of each ear—Maldeneantine frowned on violation of the body. His
GalLing’ held a slight but distinct accent, as did his hands, held close to his body as he spoke, making his
gestures tight and spare. “Please, go on. I hadn’t meant to interrupt—”
Buntec swung her hands wide, encompassing all three of them in the arc. “You think of cultivation as
nice neat rows and the same sort of plant in each row, but you can get much better results in some cases
by mixing plants. Using a second crop to keep out weeds or pests, or to nitrogenate the soil. And why
bother with nice neat rows?” She turned to Tocohl. “Maybe the sprookjes don’t like nice neat rows.”
“Maybe not,” Tocohl agreed. “I admit that’s a possibility; one I hadn’t thought of.” Choosing a spot
of rug, she crossed her legs and sat, to consider the problem. “Let’s find out what they do like. Do you
think you could spot a graft?”
“Bet your ass I can spot a graft, if I can find one new enough! I plan to start immediately.” Buntec
hauled over a chair and sprawled her chunky body into it, immovable. As if on cue, rain roared againstthe roof in earnest.
Alfvaen, still stowing her belongings, glared up at the sound and said, “Immediately isn’t possible on
Flashfever, is it?” She brought her eyes down to bear on Tocohl, where the glare softened to resignation.
“I wish there were something we could do now.”
“Now,” said Buntec, “Tocohl can tell me all about cosying up to Vyrnwy.” The pronouncement drew
a startled look from Timosie Megeve. Buntec waved an arm at him: “Edge-of-Dark got decent. If there’s
anything I can do to keep her decent, I’m for doing it. Bet your ass it’s worth the trouble to me.”
“... Cameras on!” said Kejesli.
A blurring of motion as the camera swung upward, and a moment before the image focused.
Another voice said, “Don’t make any sudden moves; you’ll scare them.”
Three tall sprookjes filled the center of the frame, taller than the ones in camp by perhaps a foot, if
the stand of tick-ticks was any guide. They craned their smooth flexible necks forward, and their
cheek-feathers ruffled. No sound came from the humans off-screen, only the glasslike tinkling of
frostwillows graced the tape.
Then one of the sprookjes took a step forward, its gold crest and multicolored yoke brilliant in the
patch of sunlight. “Hello,” said van Zoveel’s voice; and the sprookje spread its hands (as van Zoveel had
done) as if to show them empty of weapons.
The sprookje said nothing.
“Hello,” said van Zoveel again. “They have hands, Captain. They may have a language.”
Tocohl had the eerie feeling that the sprookje was speaking, or lip-synching to van Zoveel’s words.
This was the fifth time she’d watched the tape and hearing it through her implant didn’t give the location
of the sounds.
Maggy abruptly cut off the tape, thrusting Tocohl back into a jolting here-and-now as a shattering
clap of thunder reverberated through swift-Kalat’s room. The cup of winter-flame leapt in her hand and
spilled across the table.
“My apologies, Ish shan,” said Om im from the doorway.
“Not your fault,” said Tocohl. “The thunder caught me by surprise, not you. I had hoped I’d grow
accustomed to it after two days of continuous racket.” She grinned. “That’s not to say I don’t like it, but
a week of unending high would wear anyone out.”
Om im Chadeayne bowed, dripping, and came to settle himself in the chair across from her. “I know.
I suspect that’s one reason we’ve had so much trouble with personnel on this survey.”
Tocohl wiped winter-flame from her stack of hard-copy and gave him a sidelong questioning look.
“Ionized air,” he explained. “It evidently has the same effect on Hellsparks as it does on Bluesippans.
I’ve seen a couple of studies that show it to be an activator of sorts: creative people get more creative,
and nuts get nuttier.”
“Have you mentioned this to layli-layli calulan?”
“Yes,” he said, “but she knew about it—there are certain advantages to shamanism. She says there’s
really nothing she can do, short of tranquilizing everybody, and—”
“She wouldn’t advise that either,” Tocohl finished. She leaned forward, folded her arms on the table.
(Maggy, have you got anything on that?)
(Let me look,) Maggy said, much to Tocohl’s amusement, and then there was silence. The pause
was clearly provided for esthetic reasons, leaving Tocohl to wonder how Maggy would time its
duration... by the length of time it would take a human to access the information from her by keyboard,
perhaps?
She focused again on Om im. “Sorry,” she said, for her moment of inattention.
“Don’t be. It’s worth consideration. Be aware that it might lead you into rash action.”
Maggy broke silence, but only to comment, (It already has.)
(Swift-Kalat gave the sprookjes a chance. lonization or no ionization, it would have been worth
taking him up on it. What’s done is done, Maggy; there’s no point in nagging me about it.) Aloud, Tocohl
said to Om im, “You think this ionization effect is responsible for the disturbances among the survey
team?”“Only partially,” he admitted. “As you noticed, we were chamfered by a moron. But a number of us
have worked together before, and I’m seeing edginess I’ve never seen. Take Kejesli: I’ve worked with
him on two previous occasions. He’s not a great captain, but he’s a good one ordinarily. Now nobody
wants to talk to him.”
He drew his knife, considered the blade thoughtfully. There was no menace in the action, it was
simply one of those things a Bluesippan will do when he wants to think. Reflecting in a blade, they termed
it.
“No, I’m wrong,” he said, tapping the flat across his palm, “nobody wants to talk to him unless he
comes to them—or will meet them in the common room. I don’t know why, but there it is.”
“I can answer that one,” Tocohl said. “The lowered ceiling in his quarters makes most of you mildly
claustrophobic.”
“Come now, Hellspark. You’re right that he’s lowered his ceiling—and that’s unusual now that I
think of it—but I’m hardly likely to bump my head...!”
Tocohl chuckled. “That has nothing to do with it. The ceiling in your own cabin is a good three feet
higher than the one in Kejesli’s. It’s a matter of what you’re comfortable with. Am I to understand that
Kejesli’s quarters on previous surveys have had higher ceilings?”
“Now that you mention it, yes. Are you seriously telling me that’s why nobody wants to visit the
captain?”
“Yes, the low ceiling makes you all uncomfortable... even if you aren’t likely to bump your head. The
point is, that low ceiling makes him comfortable, and if what you say is true this is the first survey he’s felt
he needed that. Perhaps that’s his reaction to the ionization stress.”
“Perhaps. But I think the haft of the matter is more likely what happened on Inumaru—or more
properly what happened after Inumaru.”
“Were you there?”
“Yes, for both.” He frowned. “A lot of people were plenty angry when he refused to back Alfvaen,
when MGE canned her.”
“You?”
“No, not really. I agree with him that contracting Cana’s disease hardly seems serendipitous. It was
the rest of us he was trying to protect, after all. But...” Again he gazed into the fine blued blade of his
dagger. “But. Who knows, maybe there was a serendipitous reason that she caught it with everyone
else”—he tilted the blade toward her—“you see my point.”
“I do. I also call your attention to the fact that she and I are both here now.”
“Your presence, Ish shan, is certainly worth the trouble,” he acknowledged. He spread his hands in
offering. “What can I do for you today?”
“Today you can tell me about Oloitokitok, and about the sprookjes,” she said. “I’ve seen the tapes;
now I need some on-the-spot reports.”
Om im tilted his head slightly to the side and said slowly, “Now, the moment before the cameras
went on, one of the sprookjes—the one that gestured at van Zoveel—was tearing up a
thousand-day-blue.”
“That wasn’t in any of the reports,” said Tocohl.
“That’s why I mention it: you said you wanted any information related to the sprookjes.—It probably
wasn’t mentioned because sprookjes don’t eat thousand-day-blues.”
“They just tear them up?”
Om im grinned. “No. That’s what seemed worth mentioning. On that occasion, I found a recently
pulled patch of earth and the shredded remnants of the blue, but since then I’ve seen perhaps a hundred
sprookjes pass by an equal number of thousand-day-blues without paying them the slightest attention.
Which is a little hard to do. The tapes won’t give you an idea of the smell of a thousand-day-blue
either—it’s raunchy.”
“Interesting,” said Tocohl. “Not very enlightening, but filed and noted.” A flash of light crackled
outside the membrane, and Tocohl waited out the thunder before speaking, then said, “Go on.”
“The fact that the sprookjes have hands was what made van Zoveel so excited. You should haveseen him!” Om im Chadeayne’s eyes sparkled. “Perhaps you did: that sprookje was like his reflection.
But, as you saw on the tape, those sprookjes didn’t say a word and when van Zoveel got close, the
sprookje nipped him. Everybody overreacted and the sprookjes got frightened and disappeared into the
flashwood. Nobody followed; we were all too concerned about van Zoveel.”
Shifting forward in his chair, the Bluesippan continued. “Van Zoveel came to no harm, except for the
reaming out Kejesli gave him for ignoring safety rules. “Evidently the sprookje didn’t either, because the
went on nipping everybody they came across.” He smiled. “After a while, the pinprick became
Flashfever’s badge of acceptance.”
“But that came later?” asked Tocohl.
“That came later, when the parrots had moved into camp.—I wasn’t around the second time van
Zovee tried talking to the sprookjes, so you’ll need another eyewitness.”
Tocohl filled in from the tapes she’d seen: van Zoveel had used his vocoder and tried high frequency
thinking perhaps that the sprookjes might be that one-in-a-thousand species that heard only in the upper
ranges. The sprookjes had heard it, all right—heard and run!
“I can tell you,” Om im raised his voice as a gust of wind outside brought a particularly heavy crash
of rain against the north wall, “why they ran. We’d made some tapes—including the high-frequency
range—of general flashwood noises. That’s not as easy as you might think: we had to hang the tapers
from poles or all they’d have picked up was tk-tk, tk-tk, tk-tk.” He made the appropriate scolding face
to accompany the sound of the paired tongue-clicks.
Tocohl grinned. “So that’s why they’re called tick ticks—you named them but no one else on the
team can do the tongue-clicks.”
“Yes,” said Om im, “I should have left well enough alone, but you’ve heard them yourself and you
know they sound like a chiding parent...!”
He grinned back before taking up his tale once more: “We—Buntec and Megeve and I—were
taking advantage of an hour’s sunlight. You’ll find everybody does that here, sits outside and spreads her
feathers for drying. We were studying our tapes but outside in the middle of the compound.
“And all of a sudden, the ugliest thing you ever saw—and believe me, I’ve seen some ugly things in
my life—I’ve fallen on Stuckfish!—swooped out of the sky and ate the taper.
“It sat for a moment—it didn’t turn bilious green because it already was a bilious green—but it gave
two resounding belches and vomited up the taper. Then it flew away, cursing, or so I assumed from its
tone. Timosie cursed just as much over the loss of his taper, but Buntec and I must have howled for
twenty minutes. It was at least that long before we could tell the rest of the team what had happened.”
He leaned forward, his expression turning serious. “But tape-belchers, we later found out, are nothing
to laugh about. Megeve got a nasty slice taken out of his side when he got too close to one’s nest. Even
tape-belchers don’t like tape-belchers: they tear each other up constantly.”
Tocohl had seen hard-copy on that too. Evidently the tape-belchers were territorial and held that
territory beak and claw, especially against other tape-belchers. According to swift-Kalat’s notes, the
taper-eating incident had probably been sparked by a recorded challenge of another tape-belcher that
the live belcher had taken for genuine.
Om im gestured at her cup. “I know where swift-Kalat keeps his supplies. Would you like a refill, as
long as I’m getting myself a cup?”
“Please,” said Tocohl. As Om im crossed the room, she said, (Maggy? Are you getting all this?)
(Of course,) said Maggy. (He’s right about the ionization stress effects. It could be enough to
account for your lack of sense.)
Tocohl breathed a sigh. (But probably not,) she said. (Let’s hear the short version of what you’ve
found.)
It took no more than a minute from Maggy’s choice of quotes and displays for Tocohl to see that the
exhilaration she felt was not merely an emotional reaction to Flashfever’s gaudy displays of lightning but a
genuine physical reaction to the ionization of the air. (I think,) she said, (I can probably tone down the
effect a little with the Methven rituals.)
(Then do,) said Maggy, (or who knows what you claim to be next. And I’m not sure I approve oflying. You did say not to lie to Jenji...)
(I didn’t. Not precisely. I said I’d agree to judge—never said I was one.)
There was something akin to a muffled snort. Tocohl squinted, as if she might see the speaker if she
looked hard enough into her spectacles. (That’s not my snort of disapproval, is it?)
(No, it’s Buntec’s. Does it match the rest of my voice? Did I use it correctly?)
(Yes, and yes again,) Tocohl said. Deciding it was time to change the subject, she added, (What are
you up to?)
(You mean what is the arachne doing?)
(Mm. Yes. Even Hellspark doesn’t have the proper words to cover all possible situations.)
(Exploring the perimeter. Would you like to see?)
(Please,) said Tocohl, and was rewarded by a portion of the arachne eye view of barbed-wire fence,
no doubt the most interesting area in Maggy’s opinion.
Heavy rain lashed a grove of frostwillows into frenzied display of light. Their ordinarily sweet tinkling
sound had become a disturbing one of shattering glass that could be heard even above the rushing
downpour. Something slithered past in the forground, and as it passed through a clump of flashgrass
Tocohl saw that it was a lizardlike creature, as brassy as penny-Jannisetts.
(You might show that tape to swift-Kalat,) said Tocohl. (I don’t recall having seen that particuli
animal in their files.)
(You’re right. They don’t have a picture of that one. Maggy had evidently checked while Tocohl was
speaking. (They should,) she added primly.
Tocohl chuckled. (Perhaps they know enough to come in out of the rain; you and the lizard-thing
don’t.)
(The storm is not yet overhead. The arachne is in no danger.)
(No offense,) said Tocohl.
(None taken,) said the voice in her ear, and Tocohl said, (Perfectly put.—And, Maggy, you’re
making good choices about what needs an immediate response and what can wait.)
To this last, Maggy made no response, but the vision of the camp perimeter vanished. Om im set a
second cup of winter-flame before Tocohl and reseated himself, cradling his own cup for its warmth, an
unconscious response to the dankness of the weather.
“Thunderstorms,” he said, “are a time for talk. There’s not much else to do on this world during one
except drink winter-flame and cavil about the weather.”
“I wish the sprookjes felt that way,” said Tocohl, “about talking during thunderstorms, I mean. It’s
been two days now and I haven’t gotten to talk to a sprookje—or gotten one to talk to me. What do
they do during thunderstorms?”
“Nobody is willing to brave that”—Om im flourished a hand in the direction of the door; a flash of
lightning gave the gesture more emphasis than he had intended and he rubbed his fingertips in delighted
surprise—“in order to find out.”
Before Tocohl could open her mouth to comment, he said firmly, “If you’re going to suggest arachnes
and other robot probes, Ish shan, be assured we thought of that. And we promptly lost five of them to
Flashfever’s wildlife, most of which either gives electric shocks or feeds on them.
“As long as yours stays within the perimeter, you probably won’t lose it, unless it gets hit by lightning,
but I wouldn’t risk it outside if I were you.” Om im paused, then went on, “And as for getting the
sprookjes to talk to you when they’re around, don’t feel neglected. I’ll finish my eyewitness account and
you’ll see what I mean.”
“Do,” said Tocohl, and raised her cup.
“After the episode with the high-frequency sounds, none of us saw much of the sprookjes, except an
occasional glimpse in the distance that might have been one. Then, one day about six months later, a
handful of the brown ones showed up in camp.”
Once again, he drew his dagger. He peered critically at the blade, then drew a whetstone from his
pouch and began to hone it, comfortably matching the rhythm of his words to the motion. “The ones that
came to camp are all brown and all smaller than the crested ones. I never thought about it before, but Isuppose the camp sprookjes are younger, or a different sex?”
“There are speculations to that effect in the hard-copy,” said Tocohl, “though I did notice that no one
did an anatomical study.”
Om im stopped honing, shocked. “When half the survey team thought they were sentient? No
way—”
“I only meant no one had found a dead sprookje to autopsy. You give me an undeserved reputation
for bloodthirst.”
“Sorry,” said Om im, “I intended no offense. The situation makes us all a little edgy one way or
another.”
“And you lean toward defending the sprookjes. Why?”
This time the Bluesippan looked not so much shocked as surprised by her words, “You know,” he
said, “I do think along those lines, but I’m afraid I haven’t any idea why I do.”
With a faintly puzzled air, he went back to his story—as if he were listening for some clue to his own
attitudes. “After the handful, more and more trickled in, and three months later, we had one apiece. Now
I had better be specific...
“For the exact date, I’d have to check my records, but it was late afternoon and I was sitting on a
stool I’d brought outside, ‘drying my feathers,’ as I said before; and there was a sprookje, staring at me
with those great solemn eyes of theirs. So I said hello. And it said hello—”
“Just a minute, Om im. In what language?”
“In GalLing’. It was too tall for a Bluesippan, after all. At any rate, I was stunned and it was stunned,
or gave a good facsimile thereof. Finally I said, ‘I’m pleased to meet you,’ and went on to introduce
myself. I got about halfway through my self-introduction before I realized that the sprookje was parroting
me, word for word, inflection for inflection. I was so surprised I stopped midway through my name, and
a second or two later, that’s precisely where the sprookje stopped.
“By this time, a couple of other people had come over, slowly, of course, so they wouldn’t frighten
the creature. So I tried again. This time, I introduced Buntec. And the blunted sprookje kept pace again,
just a little behind me.
“But when Buntec spoke, also in GalLing’, it was as if she didn’t exist at all. And that, children, is
how your uncle Om im acquired his sprookje.” The Bluesippan’s puzzled look was replaced by an ironic
one; his narrative had failed to give him the clue he’d been seeking.
“If it’s any consolation,” said Tocohl, “I didn’t find anything either.”
Om im lifted his gilded eyebrows and raised his cup to her. “Sharp as Tam shan’s blade! You come
by your reputation honestly, Ish shan.”
“Hah! You established it in your own mind when you chose that nickname for me.” She leaned back,
then said, “I believe you have payment coming. What do I owe so far?”
“I think,” he said slowly, as if in an effort to keep his voice light, “that you have more than repaid me.
You’re right: I believe the sprookjes are sentient. Strongly enough at least to know they must be given a
chance. The chance is yours.”
Tocohl met his eyes with practiced misunderstanding.
He laughed, his eyes merry beneath his gilded brows. “No, Ish shan,” he said, “that won’t help. My
Hellspark may not be the best, but I can tell a hawk from a handsaw when I hear it in your tongue.”
It took Tocohl a moment to understand... In the Bluesippan translation the words were identical but
for a si and a su, the difference between her name and her father’s.
By the time she had grasped his meaning, she knew she had no cause for alarm. His dagger was on
the table; he slid it, hilt-first, across to her. “My blade is at your service, Tocohl Susumo,” he said. “That
is the least I can do for Oloitokitok and for the sprookjes.”
She laid her hand across the hilt, accepting his service and his silence.Chapter Eight
A KISS ON the hand is worth all this? thought Buntec incredulously as she looked down at the
table spread with Vyrnwyn delicacies. She didn’t recognize any of these foods, but the Vyrnwyn
obviously considered the visual side of eating at least as important as the flavor. Spread before her were
a dozen separate plates, each a different size—here a delicate gold paste heaped high in a black bowl,
topped with a sprinkling of something round and rosy; there, on a pale blue plate, semitransparent slices
of something pure white arranged in the shape, yes, in the shape of a frostwillow.
Buntec stared at each wonder in turn... When she found her voice at last it was to say, “They’re
beautiful, Edge-of-Dark, beautiful! Surely you don’t expect me to eat them!” Realizing this might be
misunderstood, she added hastily, “If I eat them, they’ll be gone. Shouldn’t we at least take a picture
or—!” Her arm flung wide, as if of its own accord, to encompass the entire display.
Edge-of-Dark smiled. To Buntec’s surprise, it was not the patronizing smile she’d seen so many
times before but a genuinely warm and open smile that suited her rich features so perfectly that Buntec
was overwhelmed.
“Perfection never lasts,” Edge-of-Dark said. “We eat them because they are beautiful. If they
weren’t, we shouldn’t bother.” Smiling still, she added, “We differ so much, you and I, I was unsure of
your tastes in food. I’m glad to know that I am already partially correct.”
Buntec hesitated, unwilling to disturb that luminous image of frostwillow.
“That one,” said Edge-of-Dark, “is eaten with this”—she indicated one of the three unfamiliar utensils
that lay before Buntec—“and I won’t know if you like the way it tastes unless you taste it. Please.”
Buntec raised the little gold-pronged implement Edge-of-Dark had indicated and, taking a deep
breath, speared a piece of “frostwillow.” In that brief moment, she found the time and the honesty to
admit to herself that if Edge-of-Dark’s wearing boots could mean that she felt so relieved, then perhaps a
little hand-kissing could make all the difference to Edge-of-Dark.
The “frostwillow” was cold and crisp and delicately spicy. She couldn’t tell if it was animal or
vegetable, but she reached for a second piece and found Edge-of-Dark smiling at her again.
Edge-of-Dark, her hand poised over a pile of flamboyant red and purple curls on a striped platter,
said, “These are to be eaten with the fingers. The... uncertainty is part of the appeal.” She demonstrated,
dipping one of the curls into a bowl of gold paste.
By “uncertainty” Edge-of-Dark clearly meant that of getting curl and paste down the gullet instead of
plopped into her lap. Buntec smiled back, intrigued by this new aspect of the Vyrnwyn programmer.
Fashionable clothing seemed terribly important to Edge-of-Dark; to see her risk splattering it was a
double wonder.
Following Edge-of-Dark’s example, Buntec dipped one of the red curls into the gold paste. “I still
don’t see,” she began, but the paste was as uncertain as Edge-of-Dark had implied. Seeing it about to
drip, Buntec tilted the curl first one way, then the other. When that did no good, she hastily caught the
spill with her other hand. “—Oops! I’m sorry, Edge-of-Dark. I don’t know a thing about Vyrnwyn table
manners. Did I just cheat? Will you be offended if I lick my palm or—?”
“Ordinarily one doesn’t begin that until much later, after one has had a good deal to drink.
‘Cat-drunk’ we call it, because the Gaian cat has such fastidious manners even though it cleans itself with
its tongue. You’re a beginner, Buntec, so that’s a different matter altogether. If you’ll have some wine,
we’ll consider it to be in good taste,” Edge-of-Dark said, then added, with the caution the question
deserved, “As far as I know, I don’t have any taboos having to do with dinner. But I’m not sure. I never
realized that I speak with an accent until the first time I stayed in—well, a very different part of my
country on my own world.—Do you have any table manner taboos that I should know about?”
The two women considered each other warily, each afraid of her own provincialism. Then Buntec
grinned and held up a chunky hand. “Do you like dOrnano wine?”
“Yes,” said Edge-of-Dark in a puzzled fashion, and Buntec went on, “Then I have a solution: the firstof us to spot one of her own culture’s food taboos gets treated to a bottle of dOrnano wine—to be
shared with the other, of course.”
“Of course,” Edge-of-Dark said.
Happy with this solution, Buntec took a sip of wine, then licked the paste from her palm. “In good
taste is right,” she said. “This is better than good.” She also caught some taste of the Vyrnwyn game: the
paste was heavily laced with brandy, potent even without the blackwine that complemented it.
“Mm! no way this perfection will outlast my appetite!” Buntec reached for another curl, a purple one
this time. “Well, I was going to say: I still don’t see how you can put so much artistry into something so
perishable.”
Edge-of-Dark poured herself another glass of blackwine. Gesturing left with the decanter, she asked,
“Would you find that too perishable to bother with as well?”
Buntec followed the gesture: on a low rectangular table in the far corner of the cabin sat a flattish
container filled with a variety of local plants. She was momentarily surprised that she had not seen it
before, until the thought occurred to her that perhaps it was intended to be seen only from this point.
The design—for design it was, she realized—caught her eye and held it: three rich red-purple leaves
from a flames-of-Veschke, their spear-head shapes rising from the container, each higher than the last; a
single intricate piece of arabesque vine bound them loosely and wove its way down to a tiny knot of
penny-Jannisetts...
Like light sculpture, but done in plants! Buntec had never seen anything like it. She rose, intending to
take a closer look.
“No,” said Edge-of-Dark, “it is to be seen from a distance; we call that naoise-style.”
“Is it something you invented?”
“No, of course not. Flower art isn’t done on your world? It’s very common on mine. Not everyone
is good at it, but everyone does it.”
Buntec, her attention torn between the flower art and the food, said, “Jannisett’s a farm world. We
grow a lot of plants inside, especially the flowering ones, but nobody ever thought of doing anything like
that! And it sounds to me as if you’re talking about more than one kind of flower art, like different
schools of light sculpture.”
“I wasn’t, but there are. Within each school, there is a viewing-distance factor. For example, joliffe
flower art, no matter what school, would be something we’d place in the center of this table, to be
viewed at this particular distance; joliffe-che would be a composition to be seen from all sides at this
distance.”
Buntec said, “You must be a grand master, or whatever is the appropriate term.”
“Thank you,” said Edge-of-Dark, “but I’m scarcely more than an occasionally inspired amateur. If
you like that, you really should see the works of Shadow-Blue or Spite-the-Devil. They are grand
masters!”
Edge-of-Dark lifted one of the as yet untouched plates of food and offered it; Buntec took some. As
she chewed it slowly, trying to figure out what the aftertaste reminded her of, Edge-of-Dark said,
“Flashfever opens up a whole new world of flower art. If only I could find some way to keep frostwillow
or flashgrass or Christopher-bangs fresh, the sounds and lights would add a completely new dimension to
a piece! But that’s hardly as simple as putting penny-Jannisetts in water...”
“But that should be easy!” said Buntec.
“Easy?”
“Sure, all you’d have to do is—” Buntec, excited by the idea herself, launched into a highly technical
description of how it could be done. Somewhere in the middle of it, she realized that Edge-of-Dark
wasn’t following her. “Sorry,” she said, “it is easy, though. I’ll make you some little things you could kind
of plug the frostwillow or whatever into. Movable, so you can put them in the right place.” She swung her
broad hand, “And for the drunken dabblers, we can build a fountain—recycle the water but move it fast
enough to keep them alive and healthy.”
Grinning, Buntec spread flattened hands. “I think I’d better calm down. I’m not paying the food the
attention it deserves. I’m sorry if I sound like a little kid who’s just discovered outdoors.” She lookedagain at the work of art that graced the corner table and shook her head in amazement. “I never saw
botanical art before and it’s—” The thought struck with the force of a blow. Unable to complete the
sentence, Buntec let her jaw drop and stared...
“Buntec? Is something wrong? You have the oddest look on your face. Is the food all right? Buntec!”
“Wrong!” said Buntec, hard put to keep from bellowing the news: “Everything’s perfect!
Everything’s wonderful! We’re geniuses, you and I!”
“What are you talking about?”
“Edge-of-Dark, think of the damn sprookjes—swift-Kalat has that fruit without seeds. He says it’s a
biological artifact. Well—hell!—if the sprookjes have biological artifacts, what kind of art do you
suppose they’d have?”
Grinning, Buntec waited. Edge-of-Dark did not disappoint her—the Vyrnwyn’s face lit once more in
that beautiful glowing smile—and she whispered back, “Biological art. Botanical art. Flower art.”
Astonishment mingled with delight, Edge-of-Dark half rose, “We’ve been looking for the wrong things!
We must tell everyone... !”
“Down, girl,” said Buntec, “first things first.” Choosing another red curl, she scooped paste and, this
time, brought it to her mouth without incident. “Well,” she said, grinning in triumph, “as you say,
perfection never lasts... but it sure oughtn’t go to waste! First, we eat—then we wise up the yokels!”
Chapter Nine
WITH THE STORM now raging overhead, Maggy judged it time to move the arachne out of
danger. As the discussion she monitored through Alfvaen’s handheld made it clear that swift-Kalat
intended to dissect one of the golden scoffers, her choice of shelter was obvious. She couldn’t watch and
record the dissection through the hand-held, only through the eye of the arachne.
She scratched politely at the entrance to swift-Kalat’s cabin; behind her a flash of lightning made her
acutely aware that human politeness was often at odds with survival. She stored that observation to look
into at some later time, while she attempted to calculate the effect of an unannounced entry.
Alfvaen saved her the trouble by splitting the membrane. “Hello, Maggy,” she said, “come on in.” A
second jagged fork of lightning ripped through the air behind her. Having been invited, Maggy bent the
arachne’s legs and sprang it to safety. There were only two mobiles, after all, and she wouldn’t be able to
watch a thing if something happened to this one—a least, until she could send another down.
“Hello,” Maggy said in return, responding from the hand-held. “Is that correct, Alfvaen? I thought in
GalLing’ you say hello only when you first meet. Or is that a holdover from Siveyn custom?”
Alfvaen glanced down at the unit on her hip and made a noise that Maggy tentatively interpreted as
not understanding. To test the interpretation, she elaborated, “I’ve been listening to your discussion; for
some time now.”
“Oh,” said Alfvaen—in a tone that conveyed sufficient confirmation for Maggy to tag the previous
noise as understood. “You’re right. But you haven’t said anything through the hand-held for hours, so I
thought you’d gone away.” She cocked her head to one side and stared directly at the arachne. “I guess
I’m not used to the fact that you can be in two places at once. The arachne makes your presence more
visible somehow, solider, if you know what I mean.”
“No,” said Maggy, “I’m sorry, Alfvaen, but I don’t understand.”
Swift-Kalat, who up to this point had merely watched the two of them, said, “The human eye is
automatically drawn by the movement of the arachne. Were the arachne still, and silent, we would be
inclined to forget your presence, as Alfvaen did with the hand-held monitor.”
“Oh,” said Maggy, employing the same tone she had heard from Alfvaen only moments before. “The
arachne seems more of a discrete entity?”
“Yes, that’s it,” said Alfvaen. “After all, it’s dripping on the rug.”
Clearly, that was construed as impolite; yes, except when she was very excited, Tocohl toweled off
carefully on entering a shelter. “I’m sorry,” Maggy said, dipping the arachne in the bow of apologyTocohl had taught her, “I hope I haven’t given offense.”
“None given, none taken,” Alfvaen said, “but let me find something to dry you off with.”
Given Alfvaen’s greeting and explanation of it, given also the way Maggy’s voice seemed to startle
Alfvaen whenever it issued from the hand-held, given this new use of “dry you off” that unmistakably
meant the arachne, Maggy concluded that it was convention to think of the mobile as the whole.
Accordingly she said, this time using the vocoder in the arachne, “If I understand you correctly, you
would be more comfortable if I spoke from here?”
“I would,” Alfvaen admitted. “It’s not so much of a surprise that way.—Ah.” She pointed and
swift-Kalat, who had been watching their exchange with evident interest, turned, reached for a towel, and
brought it to Alfvaen.
Kneeling, Alfvaen held it out. Maggy sent the arachne to her as gingerly as its mechanisms would
permit, to avoid further dripping. When it was within reach, Alfvaen said, “May I?”
Tocohl would have categorized that as Dumb Question. “I can’t do it myself,” Maggy said, but
because it was Alfvaen who asked, she simultaneously checked the odd usage. Concluding that Alfvaen
had intended to be polite, she immediately added, “Oops. You meant to be polite, didn’t you? I’m sorry
again, Alfvaen.”
“No offense,” Alfvaen said as she toweled the arachne briskly. “Yes, I meant it to be polite.” Maggy
turned and tilted it to expose its various surfaces. Scraping mud from its legs, Alfvaen said, “Stop being
sorry, though. Your use of ‘oops’ was absolutely perfect.”
She cast a quick glance upward at swift-Kalat. “Maggy learns,” she explained, “so it helps to tell her
when she does something right, not just when she does something wrong. Just like any kid.” She smiled
directly at the arachne’s lens. “In fact, like most you get a rather low-angle view of everything. Why don’t
I put you on the table where you can see something besides feet?”
“Yes, please. If swift-Kalat won’t mind?”
“I don’t mind—” swift-Kalat began. Alfvaen lifted the arachne. “Where would you like to be,
Maggy?” she asked.
“Where I may watch and record swift-Kalat’s dissection of the golden scoffer.”
Alfvaen set the arachne on the table, giving it a clear view of the small furry cadaver. Maggy shifted it
slightly, to avoid obstructing swift-Kalat’s work with its shadow, and settled the arachne with its legs
folded.
“Maggy,” said swift-Kalat, “I would appreciate some information.”
“I have tapes of an animal not in your computer’s memory,” she offered, “I recorded it in the
flashwood at the perimeter of the camp. I will transfer them if you wish.”
“You are in two places at once!” said Alfvaen.
“Four places,” Maggy corrected, to set the record straight.
Swift-Kalat said, “Yes, I would like you to transfer your record, but I meant a request for specific
information from you.”
“I’ll answer as reliably as I can.”
“Please understand that I mean no offense. I do not know what culture you belong to or I would
avoid the known taboos.”
“Hellspark, I think,” Alfvaen said.
“Yes, that’s right,” Maggy confirmed, “Hellspark is the culture I’m most familiar with. I have a good
working knowledge, Tocohl says, of the Jannisetti, the Sheveschkemen, the Holyani, the Dusties—”
Laughing, Alfvaen held up her hands. “Enough, Maggy. You’re definitely Hellspark.” To
swift-Kalat, she added, “Her Siveyn is very good, and her knowledge of Jenji is better than mine—she
has the vocabulary at her command; I don’t.”
“I can look things up faster than you can, Alfvaen, and I don’t have to worry about where to stand.”
“You’re sweet, Maggy.”
“Am I? Tocohl says I’m a pain in the butt.”
“It is possible to be both.”
“Oh,” said Maggy, and filed that for future reference. “What information do you request, swift-Kalat?I apologize for having strayed from the subject.”
He paused. Maggy recognized this only because Tocohl had given both her and Alfvaen training in
the timing of responses in Jenji. This was, for Jenji, too long a pause; Maggy inferred that he was having
trouble framing his question.
At long last, he said, “Maggy, how old are you?”
That was an oops, thought Maggy, that implies he thinks I’m a child, and a Hellspark one too.
Tocohl had told her not to tell people she was an extrapolative computer, but this was a Jenji asking, and
Tocohl had also made a point of telling her not to lie to Jenji. Tocohl was busy or she would have asked
Tocohl what to say. As it was, she balanced odds one way, then another, and decided that swift-Kalat
had hired them both. So she shouldn’t lie to him, even if she had permission to lie to Alfvaen about the
judgeship. Alfvaen was likely to correct his impression, anyway, so Maggy had better find a way to do
so politely. She took a nanosecond more to search through all she knew of the Jenji...
Alfvaen said, “Oh, no! Maggy’s n—”
That would have been impolite, the way Alfvaen was headed. Maggy interrupted, “I was
manufactured eight standard years ago, thirteen and a half Jenji, but Tocohl says I’m only three
standard.”
Alfvaen closed her mouth, peered curiously at the arachne. “Why three standard, did she say?”
Maggy would have replied with Tocohl’s own words had they not been in Hellspark. Instead she
translated: “Because that’s when I started mouthing off.”
Tocohl sat at a large table in the common room, Om im beside her. To another Bluesippan, the blade
offer and acceptance needed no announcement; that he sat at her left hand and thus guarded her
unprotected side would have been enough. The surveyors took it for simple gallantry.
As a handful of others, among them John the Smith and Rav Kejesli, approached the table, Tocohl
said softly in Bluesippan, “Don’t let your instincts run away with you, Om im. I won’t take a parting of
blades for a threat—at least, not yet.” She had asked him about the Inheritors of God but, to his
knowledge, no one on the team was a believer in the faith. Aside from that, she was not yet ready to
jump at shadows.
“I know, Ish shan,” he said, “and John the Smith will be the first. I guarantee that, and I guarantee my
own judicious behavior.”
“Good,” she said, “I can’t stand the sight of blood.—Why John the Smith?” But the question came
too late for Om im’s answer for the party was already in hearing.
John the Smith promptly answered the question in his own way by attempting to draw a chair
between Tocohl and Om im. He was Sobolli—of course!—his status accorded him a place to the left of
the one he addressed. Om im took it well; as promised he did not take the action as a threat, he merely
closed in on Tocohl and doggedly refused to relinquish his position to John the Smith.
“Here, John,” said Kejesli—he was at least partially aware of the problem, Tocohl noted—“Beside
me.” With poor grace, which Kejesli was unaware of, John the Smith once more rounded the table, this
time to “outrank” his captain.
Tocohl mentally wished the team’s chamfer the Death of a Thousand Butts. To put a Sobolli,
guaranteed to approach on the left, on the same team as a Bluesippan, who took a left approach as
threat was to court disaster. Not to mention severe injury to the Sobolli... She found sweat beading her
forehead and wiped it away.
Om im chuckled. In Bluesippan, he muttered, “If I haven’t killed him for blind-siding me yet, Ish
shan, I’m not likely to today. I think I mentioned we were chamfered by a moron...” In GalLing’ he said,
“The Hellspark’s in search of sprookje tales. I promised her each of us had one of her own, certainly to
the acquiring of her own personal sprookje.”
“Do we!”
That was Kejesli reaffirming his status by taking the first word, and telling the first tale. When he had
finished, a half a dozen others told their own in turn, but the end result was no new information about the
camp sprookjes. In every case, the experience had been almost identical to Om im’s: each sprookje hadbegun to mimic one surveyor—no apparent reason for the choice, no apparent understanding of the
words echoed, and no sprookje echoed more than one surveyor.
“And,” contributed Hitoshi Dan, the team’s botanist, when he had finished his own acquisition tale,
“they don’t speak unless they’re spoken to. They never volunteer a word. In fact, you don’t know if the
sprookje’s ‘yours’ until you say something, and then you wish you hadn’t.”
“Wrong,” said Om im genially.
Just as genially, Hitoshi Dan splayed his fingers before his throat. “Never insult a man with a knife,”
he said. “I did forget: my small sharp-edged friend there can tell the sprookjes apart before their echo
gives them away.”
Jabbing both thumbs at his temples, John the Smith made a scoffing sound. Luckily, both Om im and
Hitoshi Dan took notice only of the scoffing sound.
“True, John,” Om im said, “I’ve gotten to the point where I can tell which sprookje is whose.”
“He can. Ask him sometime as they come into camp,” Hitoshi Dan said, but he shifted his gaze from
the Sobolli who so clearly disbelieved to Tocohl, who assured him with a tap to her nose that she
planned to do just that.
To Tocohl, Om im explained, “Each of the sprookjes has its own face and its own personality. I can
tell you which will mimic whom. But Dan’s right that they never speak unless spoken to.”
Then, once again turning his attention to John the Smith, he said, “Homo sap is essentially lazy. He
looks at two cats and he says all cats look alike. Or he’ll go so far as to acknowledge that one cat is
striped and the other isn’t, but he’ll only acknowledge gross differences.”
He rose with an easy arrogance, as if the act of standing proved his point. “People look at me and
say, ‘Ah, a Bluesippan!’—I am forced to say, no, that’s not sufficient. I’m Om im Chadeayne, I am
myself.” He paused, then, “I am different than you are but I am an individual. I am informed by my
culture and my world but I am not defined by it.”
Very true, thought Tocohl, or you would have drawn on John the Smith. But even that is unlikely to
have made an impression on him.
As Om im resumed his post (to all appearances settling himself easily back into his chair), John the
Smith said, “But that’s hardly the question here. We’re talking about sprookjes and they aren’t—” He
broke off suddenly.
“Yes?” said Om im, and the Smith looked embarrassed.
“You were, perhaps,” said Om im, “about to add, ‘and they aren’t human,’ were you not?”
“Yes,” the Smith admitted.
“But that’s a question we have yet to decide,” Om im said. “You see? I’m not faulting you in
particular, John. It’s a language problem in more ways than one. Assuming the sprookjes have a
language, then we’re having trouble with their language and with our own as well.”
Confident now of his ability to retain his audience, Om im paused to sip his winter-flame, then
continued, “You call them all sprookjes, which defines them in a certain way—and limits your ability to
think about them. I’m not much better: I think of them as John’s sprookje, swift-Kalat’s sprookje, and so
on.”
“All right, Om im, but calling them human won’t make them human.”
“True, but calling them nonhuman or inhuman will set limits on our perceptions of them.—Tocohl, you
see what I mean?”
“Perfectly,” said Tocohl. “You took one look at me and made up your mind that I fit your image of
Ish shan—a legendary giant from Bluesippan folklore,” she added for the benefit of the others around the
table, “who was known for her ability to outwit the gods. So, from that moment on, anything I did in your
presence became highly charged: my successes will be more than successes, my failures will be more
than failures. All this through no fault of mine. Being thought more than human has problems all its own.”
Having taken Om im’s audience from him, she paused for a sip of winter-flame before continuing.
When she resumed, it was to say, “Take von Zoveel for an example. He named them sprookjes. To him,
they’re fairy tale creatures, something from a story for kids. Surely that affects his image of them.”
Hitoshi Dan stabbed the air for attention. “Your point is well taken, but what do you suggest we do?Shall we call them ‘native humans’?—Must we, in order to see them as human?”
“I don’t know what human means,” interjected John the Smith, “especially after Tocohl’s story
about the Pasicans. Human is itself a highly charged term!”
“Human means like me,” Tocohl said.
“Human means having art, artifacts, and language,” Rav Kejesli corrected sourly. “That’s the legal
definition.”
“Which,” said Tocohl, smiling, “is nothing more than a complex way of saying like me.”
For a brief moment, Kejesli tried to stare her down, but still smiling, she met his stare with her own
level gaze, and at last he touched the pin of remembrance. “For the moment, I’ll grant it,” he said.
Touching her pin of high-change in acknowledgment, Tocohl went on, “Convince enough
people—and I use that term loosely—that any given species is enough like them and they may even find
a way to circumvent the legal definitions. Witness the dolphins. Perhaps it was just wishful thinking on the
part of Homo sapiens, but dolphin song was judged an artifact. Personally, I think Homo sapiens
wanted so badly to ally itself with such a graceful, gentle, and talented species that cheating was the
obvious answer.” She drew her hands close about her wrists.
“I don’t begrudge the cheating, certainly.” She smiled. “It gained for Homo sapiens a greater
humanity in that other sense of the word.”
Hitoshi Dan smiled wistfully, as if in remembrance. Evidently he’d had some contact with the Gaian
dolphins. “Perhaps we’re using the wrong term,” he said, “if it’s a language problem. Perhaps sentient or
HILF would be less loaded?”
“In GalLing’, yes,” said Tocohl, “but how many of you actually think in GalLing’ rather than
translating automatically into your homeworld’s tongue? The Yn word for sentient translates literally as
‘she who speaks.’” She gestured toward Rav Kejesli: “And the Sheveschkem term means ‘sparked’ or
‘enflamed’—though neither GalLing’ word quite captures the imagery of the original.”
John the Smith said, but with a slight smile, “You have an admirable talent for confusing the issue,
Hellspark.”
“That wasn’t my intention; I meant merely to break down the boundaries to some extent, to show
you how flexible reality is when compared to our means of expressing that reality.”
There was a long moment of silence. Tocohl scrutinized the faces of those around her: Kejesli
scowled, John the Smith looked thoughtful. In fact, to Tocohl’s satisfaction, thoughtful looks
predominated. At last, Hitoshi Dan broke the silence. “All right,” he said and—as if in positive answer to
a question—rose and strode away.
His departure signaled a general turnover of those assembled. Most of the deserters, Tocohl felt
confident, were off to think over what she and Om im had said. Only John the Smith and Kejesli, still
scowling, remained.
“What else can we tell you about Flashfever,” Om im prompted. “What else might be of help?”
“Tell me what’s so valuable about this world,” said Tocohl. “You’re the geologist—is there anything
that might make Flashfever especially interesting to colonists or exploiters?”
Kejesli’s scowl deepened, but this time was not directed at her. When he unwittingly touched his pin
of remembrance, she knew he too was thinking of the sort of exploiters who had burned Veschke.
“Not from my end,” said Om im. “Flashfever has the usual supply that you’d find on any previously
unexploited planet in its category; a colony wouldn’t have any lack of resources, certainly. But ores
aren’t worth the cost of export, and so far I haven’t turned up any unusual gemstones that would be.”
“Not gemstones,” John the Smith snapped suddenly, “forget gemstones, Tocohl. This is one of the
most valuable planets ever surveyed!”
Kejesli jerked to face him, eyes widening in surprise at the Smith’s sudden animation. A quick yank
brought the Smith’s chair an inch closer to Kejesli: the Sobolli meant to convince. When what would
have impressed another Sobolli only made Kejesli kick his own chair the same inch backward, John the
Smith turned his attention back to Tocohl.
His irritation vanished in his general excitement. “Animals that give electric shocks have been known
on other worlds. But for Flashfever, biologists will have to establish whole new categories of plants!Plants!” He threw up his hands. “I see what you mean about the flexibility of reality. I’ll bet Hitoshi Dan
calls them plants because there isn’t a GalLing’ word that covers them. I don’t know of any.
“Take the drunken dabblers, for instance—they grow in the middle of a fast-running stream and
convert the energy of the water into electricity and use that for sugar conversion, cell-building, and so
on! As common a weed as flashgrass has its own biological piezoelectric cells and uses wind power as
an energy source.”
“So that’s why it’s so thick in the vicinity of the hangar,” Tocohl said.
“Right,” said the Smith, “flashgrass gets on with daisy-clippers like... you and Om im.”
Tocohl grinned at the analogy. Om im said, matter-of-factly, “I’m the windy one.”
With a chuckle and a sidelong glance at Kejesli, John the Smith went on. “All due deference to you,
Captain, but I wish we’d put the camp beside a stand of lightning rods, just for the show.”
“The tall black spines,” asked Tocohl, “I’ve only seen them from a distance. I admit the show’s
spectacular, but I side with the captain on this one—I’m not sure I’d want to be that close, either.”
“Ah,” said the Smith, “but they really are lightning rods. That’s the beauty of the thing. They are that
tall purposefully to catch the lightning they use for cell-building, energy, reproduction. They also channel
the lightning’s energy to the shorter, younger shoots in the stand; any excess beyond that, they bleed
harmlessly into the ground.”
In his enthusiasm, he rose from his chair to lean across the table, blind-siding Om im once more. This
time was not as threatening from the Bluesippan’s point of view, for both of the Sobolli’s hands came
down flat on the table. All to emphasize the authority of his conclusion...
“On this planet,” John the Smith said, “the safest place to be during a thunderstorm is in a grove of
lightning rods.”
Tocohl inclined her head to the left to acknowledge. John the Smith eased back into his chair, adding,
“That’s theoretically speaking. Nobody’s tried it, you understand, and I certainly wouldn’t advise you to
experiment.”
Feeling it best to acknowledge that as well, Tocohl once again tilted her head left. “Don’t worry,” she
said, “I don’t plan to.” She reached for her cup of winter-flame, found it empty, set it back down.
“Mine too,” said John the Smith. “Refill?” A rumble of thunder drowned out her response but,
reacting to the tilt of her head, he gathered both cups and headed for the dispenser.
Most of the other surveyors stood in small knots at various corners of the room, some in highly
animated conversation. For the moment, John the Smith was alone at the dispenser. Tocohl said, “Excuse
me, Captain,” and rose. Motioning Om im to stay put, she strode across to join the Smith.
Quite deliberately, she came up on his left side, “outranking” him. In this matter, she was fully
confident she deserved the position. Her spoken Sobolli was not the best, but it was more than adequate
to make her point. “A word in your ear, John,” she said, taking her cup nonchalantly from him. “High
status to a Bluesippan is the reverse of yours. If you continue in this fashion, Om im will consider you a
groveler of the worst sort, completely beneath contempt.”
His upturned face went white. “Are you sure?”
“As sure as I’m standing here,” she said. The reference to her high-status position was quite sufficient
to reinforce her information.
“I didn’t know... !”
“Neither did your team’s chamfer. You’re hardly to be blamed for that. It’s not your area of
expertise.”
“What should I do?”
“Approach him from the other side. Start now; you’ve a lot of damage to your image to repair.”
“Thanks, I will.”
Chapter Ten
TRUTHFULLY, LAYLI-LAYLI,” Timosie Megeve urged, “do you really think the sprookjesmurdered Oloitokitok?” Responding to her gesture, he touched the cadaver with obvious reluctance,
grasped it by the shoulders, and helped her roll it into a prone position. He rubbed his hands against his
thighs. “Isn’t it more likely that he just blundered into an Eilo’s-kiss or was hit by lightning?”
Layli-layli calulan prodded the body gently. If dealing with her late consort in this fashion bothered
her, she was careful not to show it. Her plump hands stroked and probed with professional deftness. She
said, “Blundered? Oloitokitok? Not likely. Think, Megeve: If you were a sprookje with no weaponry,
but with a knowledge of the wildlife of this world, what would you turn against intruders? I admit only
that there is a possibility of murder.”
She looked momentarily away from her work to fix him with a firm stare. “Possibility” she said, “is
not probability.”
Her hands halted suddenly at the base of the neck. Although Oloitokitok had been wearing his 2nd
skin, he had closed neither hood nor gloves; it was his face and hands the golden scoffers had scavenged.
Here, however, the flesh showed only the effects of bacterial decay.
But layli-layli calulan leaned closer, pressed her fingers to the area, frowned slightly.
Megeve followed her gaze but could see nothing to warrant such attention. “What is it?” he asked, at
last.
“I don’t know,” she said, “but I missed it the first time I examined—” Her voice broke off. She
frowned a second time and turned to her comunit to address a spate of technical jargon to swift-Kalat.
Within minutes, swift-Kalat appeared at the door to the infirmary, with Alfvaen and the Hellspark’s
arachne at his heels, all three streaming rivulets of rainwater. Layli-layli gave them no greeting nor any
chance to dry. Drawing swift-Kalat to the body, she indicated the base of the neck.
Wondering what had so excited the doctor, Megeve watched as swift-Kalat pressed gently at the
indicated spot. Megeve turned away, unwilling to observe the ravagement done by the golden scoffers.
“What is it?” demanded Alfvaen impatiently. Trailed by the arachne, she moved forward for a closer
look of her own.
“It appears that Oloitokitok was bitten by a sprookje,” said swift-Kalat, “a second time, as you
were, Alfvaen.”
“So you know of nothing else that would make a similar mark?” layli-layli inquired.
“Nothing,” said swift-Kalat, and layli-layli calulan continued, “The mark was made after his death.”
“After his death,” mused swift-Kalat. “If I theorize from the behavior of the crested sprookje I
observed in proximity to the cadaver, I might deduce that the sprookje wished to investigate the changes
that had occurred in the alien’s metabolism.”
“That doesn’t account for the swelling,” interrupted layli-layli calulan.
“Swelling?” Swift-Kalat touched the indicated spot a second time. Then he stepped back, lifting his
hands. “My fingers aren’t as sensitive as yours, layli-layli. I can’t feel what you refer to.”
“There is something beneath the skin at that point,” the doctor explained, “something living.” A flash
of lightning whitened her face, solarized the scars across her cheek.
Megeve shivered in anger at the sight. Barbarian, he thought, then shook himself to relieve the
sudden chill.
Layli-layli calulan had no such effect on swift-Kalat, for he merely said, “We have two alternatives.
The first, to dissect; the second, to remove the cadaver from stasis to observe the results.”
“Observe the results?” Layli-layli calulan frowned at the cadaver, then at swift-Kalat.
“I have just completed the dissection of two of the golden scoffers found near the cadaver and
placed in stasis two days ago. Of the two, one had no unusual marks of any sort. The second, however,
which I saw bitten twice by my own sprookje, now has garbage plants growing on it.” His lips
compressed. There was a soft chime from his silver bracelets as he reached out to touch layli-layli’s arm
with his fingertips. “I do not intend to cause pain,” he said, “merely to convey information I think
significant.”
Layli-layli ran her palm lightly across his fingers. “I will bear the pain for the sake of the information,”
she said, then returning to the body, she added quietly, “Let us see.”
Megeve, caught between his curiosity and his nausea, realized the layli-layli intended to dissectOloitokitok’s body. The sour feeling in the pit of his stomach became a hard knot. Without a word, he
turned and left.
There had been a second turnover in the group that surrounded Tocohl. For the moment, only Om im
and Rav Kejesli remained. The captain’s gray eyes stared past Tocohl to the message board over her
right shoulder, and Tocohl twisted in her seat for a look. Someone had scrawled: “Maybe they only talk
to plants?” It was signed “Bezymianny.” Beneath that an “anonymous” couplet, in Bluesippan, read:
“I’m sorry I woke ya,
I’m only a sprookje ...”
Tocohl laughed aloud and grinned appreciatively at Om im, who feigned innocence with a complete
lack of success. Kejesli scowled.
“No,” said Tocohl, “it’s nothing untoward. I’ll translate the content, but I could never match the
rhyme in Sheveschkem. It just isn’t possible!”
Even as she translated, Tocohl realized that the few lines in an incomprehensible language could not
have been responsible for Kejesli’s scowl. Nor did it vanish when she’d finished.
Kejesli shrugged, Sheveschkem-style, and said “You haven’t asked about Oloitokitok, Hellspark.
Shouldn’t you be investigating his death as well as the sprookjes?”
“Tell me about Oloitokitok,” Tocohl said, and Kejesli blinked as if caught completely off guard at the
question. Om im raised his head slightly, about to speak, then glanced at his captain and held his tongue.
After a moment, Kejesli murmured, “I don’t know, I—never knew him, not really. He was—” He
stopped speaking abruptly, his hands worried the edge of the table. He looked away, his face darkening.
When he looked back again at Tocohl, he was angry: angry with himself. “He did his job and he
never complained. I can’t tell you anything more than that; you’ll have to ask someone else.”
“Ask Timosie Megeve,” suggested Om im, “he and Oloitokitok seemed close. I’ll give my views, but
you’ll have to bear in mind that”—he jerked his thumb back over his shoulder, a modified point that
included Kejesli—“they’re all crazy!”
Tocohl eyed him solemnly. “You’re working in a madhouse,” she said, then added to Kejesli, whose
scowl had become still more pronounced, “Sometimes the only way to deal with other cultures is to
assume they’re harmless nuts—because they are, by your culture’s standards.”
“And you, Hellspark?” said Kejesli sharply.
Tocohl spread her hands grandly. “I am the maddest of all: I shift from culture to culture.” She
inclined her head slightly in expectation of applause.
“Charlatan,” said Om im. “You Hellsparks are the wardens. All you do is humor the inmates and
keep them from killing each other whenever possible.”
“You,” Tocohl said, “have an exaggerated esteem of Hellsparks that transcends all reason.”
“Hardly that, Ish shan. I once saw a Hellspark drive a person to complete distraction by simply
talking to him. Mind you, I understood the language they were speaking, and the content of the
conversation gave me not a clue as to how the trick was done. But it was deliberate and we all
appreciated it.” To Kejesli, he said, “Havernan, remember?”
“I remember,” Kejesli said grimly, “that unbelievably rude Katawn customs inspector.”
Om im looked surprised. “I didn’t think rude, so much as long-winded and boring.” He turned back
to Tocohl, “In any event, none of us liked him; all of us wished he would go away. It was at that point
that the Hellspark breezed through. She had little patience for customs at best—none for the Katawn,
apparently. She talked to him for some fifteen minutes at me and the next thing we knew he was
screaming at us to get out and never come through his station again.
“I always thought she’d insulted him or blackmailed him in some fashion,” Kejesli said.
“No,” said Om im, “I assure you the conversation was completely innocuous. So what did she do,
Ish shan, or is that a Hellspark state secret?”
Tocohl considered him. It was only a matter of idle curiosity that sparked his interest, but the feel of
Kejesli’s was something much stronger. How would you drive a Katawn to distraction, she thought.
Then she had it.“Think back, Om im,” she said, “try to visualize it. As the Hellspark talked, did she keep walking?”
He obliged by closing his eyes. When he opened them again, he said, “Yes, she did. She picked up
this and examined that and walked here and there...’
“Then I can tell you how she did it: by picking up this and examining that and walking here and
there... Constantly moving as she spoke, right?’’
“Yes,” said Kejesli.
“It’s simple. A Katawn can’t hold a discussion with someone unless he’s facing them, across a table
or across a customs counter, for example. Or turned to face them”—she demonstrated by turning to
address Om im face-to-face—“like this.”
She grinned at the deviousness of that other Hellspark. “As long as that Hellspark kept moving the
Katawn couldn’t address her properly, and he had no sense of what was wrong. What sheer frustration
that must have been for him!”
“You’re right,” said Kejesli, with a small sound akin to a gasp. “He kept trying to stop her, to get in
front of her.”
“Finally,” Om im said, “he burst into tears and—as I said—screamed at us all to get out and never
darken his customs office again.” The small man gave her an almost proprietary look of admiration. “I
had no idea, Ish shan, how easily you can manipulate people with language!”
“You do well enough in your own,” Tocohl observed, and he arched a gilded brow in pleased
acceptance of the compliment. Then he grunted and whipped his arm up sharply. (Watch out,) Maggy
snapped simultaneously.
There was little need for the warning: Om im caught Kejesli’s wrist against his own with a subdued
crack that bespoke considerable force. Glaring at Kejesli, Om im reached for his dagger with his right
hand.
Kejesli, totally stunned by the smaller man’s reflex action, eased back into his chair. He splayed his
hand at his throat. Om im returned the partially drawn dagger to its sheath.
The two continued to eye each other warily.
(Need the arachne?) Maggy asked. It was not as unlikely a query as it seemed; Maggy had already
learned to use the arachne to trip Tocohl’s opponents in a brawl.
(Thanks, no), Tocohl said, although she continued to eye the Sheveschkemen warily. Aloud she said,
“Yes, Captain?”
Kejesli lowered his splayed hand to rub his bruised wrist, glaring at the two of them while he did so.
Then he leaned forward once more, this time very slowly. “What are you really like, Hellspark?”
The intensity in his manner shocked her; she met it with curiosity of equal intensity. “I don’t
understand the question.”
Still glaring, Kejesli said, “You charge into my quarters like Veschke herself; you kiss that hull-ripping
Vyrnwyn’s hand and that hull-ripping Vyrnwyn puts on hull-ripping boots!—Now you’ve got Om im
acting like a maniac!”
Om im, now content to settle back and watch Kejesli with equal interest, said to Tocohl, “I told you,
Ish shan. They’re all crazy!” This time his thumb jabbed at his own chest.
“You change,” said Kejesli. As he delivered it, the observation was an accusation.
“No,” said Tocohl, “I don’t. Not in the way I think you mean. You accuse me of changing my
personality to suit the culture I’m dealing with?” Yes, from his reaction, that was what he was asking.
“Captain,” she said, “what you saw in your quarters was... the real Tocohl Susumo. I don’t change
personalities when I switch languages. Think of it, well, like transposing a melody from key to key. It’s
still the same melody, right?”
He had stopped scowling, but the intensity of his interest remained. “Go on,” he said.
“That’s what I do when I switch from language language: I transpose. That’s all I do. I assure you
Edge-of-Dark thinks me as flamboyant when I speak Vyrnwyn as you think me when I speak
Sheveschkem. Or as Om im thinks me when I speak Bluesippan.
Kejesli looked unconvinced.
“Perhaps,” Tocohl said thoughtfully, “it might be some help to you if I spoke Hellspark?”“Yes,” he said, as if surprised by the suggestion. “It might at that. I’ve never heard Hellspark spoken;
every Hellspark I ever met spoke Sheveschkem to me.” The scowl returned briefly. “Or spoke some
other language to someone else.”
“It is considered the polite thing to do—use the language of the person you’re speaking to, if at all
possible,” Tocohl pointed out.
“I know,” Kejesli said curtly. His sweeping gesture disposed of politeness for the moment. “I’ve
heard Hellspark had a manufactured language, like GalLing’. I’ve never heard it spoken and—yes—I’d
like to very much.”
“Then you will. First, though, I want to correct popular misunderstanding. Yes, both GalLing’ and
Hellspark are artificial languages, but other than that, they bear no resemblance. In fact, they are
diametrically opposed in intent. GalLing’ was originally composed of all the sounds all the known human
languages held in common, so that a speaker of any of the languages at that time could speak GalLing’
without an accent. Oh, inflection gives you a clue, so does intonation, word choice, and so forth.” She
paused to drain her cup of winter-flame and set it aside.
“Hellspark,” Tocohl went on, “took the opposite route. It was originally composed to incorporate
every known possibility of human language, all the sounds of all the various tongues, not to mention such
refinements as inflection, tonal changes, proxemics and kinesics, as well.”
Kejesli blinked, and Tocohl decided to leave well enough alone. She said, “Simply put, someone
who speaks Hellspark can speak any of the known human languages without an accent. Nothing comes
as a surprise. Where GalLing’ was designed to exclude, Hellspark is inclusive.”
Om im said, “Languages change too.”
“They do. And every time a new possibility pops up, somebody very quickly coins a handful of new
words to incorporate it. The words get flung like candy to the youngest kids, who tease us old-timers
with them until we ‘catch on.’” She grinned. “You wouldn’t believe the word games that go on in a
children’s Babel.”
“You’re right,” said Om im, “I wouldn’t.”
Tocohl turned back to Kejesli. “So, Captain Kejesli, what would you like to hear in Hellspark?”
“Veschke’s Refusal,” he said.
She began in almost a whisper, knowing he would hear her over the echoing rumbles of the storm.
Like an incantation, the rhythms of her rising voice drew others from all corners of the common room.
Although they did not understand the language, they knew a performance when they saw one. Intent on
catching at least the flavor of the original, Tocohl saw them come only hazily. Again and again and again,
she bade them “Strike! Strike!” Again and again and again, Veschke refused them coordinates of
Sheveschke, and each time she bade them strike, Kejesli jerked in angry agreement. She paused a
beat—the room grew utterly silent—
Dropping her voice, she delivered the words once more in a whisper: “Steel or fire, strike! Strike!”
A ripping crack of thunder split the air. Tocohl took a deep breath, grinned. Looking up, she said,
“Couldn’t have been too bad a translation. Thanks for the special effects, Veschke.” With a swirl of her
moss cloak, she sat down.
Someone had begun to snap his fingers, another clapped, a third stamped his feet in time. Startled,
Tocohl looked around her—the first thing she thought was, I’m glad Buntec’s not here to see that! The
second...
“Veschke’s sparks!” she said in GalLing’. “If you think that was good, you ought to hear the original
the way Jassin does it!”
Om im said, “That’s the best argument I ever heard for learning Sheveschkem.”
“That’s the best reason I know for learning any other language,” Tocohl said. “Well, Captain, did I
give you some idea?”
“What?”
Pretty potent stuff, Veschke’s Refusal, even in my butchered translation, Tocohl thought. Aloud she
said, “Have I given you some idea of the sound of Hellspark?”
Kejesli shook himself visibly. “Yes,” he said, “yes. Thank you. I think even Jassin would haveapproved.”
“That’s high praise indeed! I cheated a bit here and there, using a word that wasn’t exact but had the
better sound. In translating something like that, accuracy in feel is more important than accuracy of
phrase. I could give you something prosaic if you wish, handy phrases for the tourist, for example.”
He stared at her, as if afraid she might conjure up typhoons with a “Where’s the bathroom.”
“No,” he said, “no thank you. My curiosity is more than satisfied.”
“Fine, then we’re back to the subject of the sprookjes.”
Maggy pinged urgently for her attention. Tocohl thumbed her ear for silence, realized that only
Kejesli would recognize the gesture, and said, “Just a moment.”
(Yes, Maggy?)
Maggy’s only reply was the scene flashed onto Tocohl’s spectacles. Tocohl watched and listened,
then whistled. “Captain? There’s trouble in the infirmary. I think we’d better get there right away.”
Feeling that the captain would wish to deal with the situation with as few complications as possible,
Tocohl had spoken in Sheveschkem. Kejesli rose, this time answering in the same tongue, “Lead, I sail
with your sparks.”
She had him halfway across the compound, splattered in mud and drenched in rain, before she
recognized that too as a line from the Epic of Veschke.
Om im, still at her right hand, reacted to her urgency by foregoing the politeness of a knock or a
chime. He burst through the door to the infirmary, Tocohl and Kejesli at his heels. After that Tocohl had
no time to consider poetry.
“Trouble” was the greatest of understatements: the wash of emotion within was almost physical.
Across the body of what must have once been Oloitokitok, Ruurd van Zoveel, gripping the edge of
the table so fiercely that his veins stood out like rope, bellowed at Alfvaen and layli-layli calulan.
Answering rage filled both their faces. Only swift-Kalat, seeming more concerned than angry, stepped
forward to soothe the Zoveelian.
Alfvaen would not risk him. Taking two steps forward to swift-Kalat’s one, she interposed herself
between the two. “This-s is your choice, then! Look on me, child of fools!” Her arm snapped sharply
across her chest in challenge. The fierce green glare she fixed on van Zoveel brought the enormous man
to bay.
Maggy’s arachne scuttled from beneath the table to position itself for a better view of the two.
Behind Tocohl, Kejesli began, “What’s all this bellowing? What’s the hull-ripping matter?” At that point,
he must have registered Alfvaen’s threatening stance, for his voice dropped to a whisper in Tocohl’s ear,
“Tocohl—”
“Not now,” she snapped. Across the room layli-layli calulan, never taking her dark eyes from van
Zoveel’s face, twisted the bluestone ring from her left index finger and slapped it down beside
Oloitokitok’s body. Tocohl heard a muted exclamation of horror from Om im.
Alfvaen would hold her pose until van Zoveel returned her challenge, but layli-layli calulan reached
for the second ring, began to twist it off.
With a sharp intake of breath, Tocohl charged across the room. Vaulting table and body, she came
down close enough to startle layli-layli calulan into a moment’s pause. It was enough to let her press
on. She grasped the shaman’s wrists, attempting to part the hands with the ring still on her finger and
praying inwardly that Veschke’s blessings covered this. In Yn, she demanded, “Would you have it go
astray?”
The plump woman did not struggle, she merely went back to what she had been doing before
Tocohl’s intervention.
“Layli-layli calulan!” said Tocohl. “You do not know his true name! Will you risk the death of one
of these others, or your own?”
“You lie,” said layli-layli, but her arms stilled in Tocohl’s grasp, their motion uncompleted. Her black
eyes burned into Tocohl’s.
Tocohl held the gaze as she held layli-layli’s wrists, summoning words. “I do not lie. Think! ‘Van
Zoveel’ simply refers to his planet of origin, and ‘Ruurd’ is the commonest of male names on that world.Do you know his true name?”
The burning gaze dropped from Tocohl’s face. Layli-layli calulan reached for the ring she had
removed, and Tocohl released her wrists. The shaman’s smoldering anger appeared to subside as she
slid the bluestone ring back on her finger, but she said, “He’s left his gods behind him, nor could they
protect him if I knew his true name.”
With that, she turned her back, consigning the group to nonexistence.
Relief shivered along Tocohl’s spine. When she rounded the table once more, this time to deal with
the lesser problem of Alfvaen, she moved easily, with a ghost of a smile. Her incomprehensible exchange
with layli-layli calulan had distracted the others enough to lessen their tension as well. Van Zoveel had
stopped his bellowing to treat her to a puzzled look. Using that for a hook, she beckoned him with a
conspiratorial jerk of the head, as if she might explain if he came closer. When he took the step, she
caught him by the arm and tucked him away safely behind her, well beyond dueling range even if he knew
the proper responses.
Om im slid silently into the space left vacant by van Zoveel. He said, “I am the fool, if Ruurd and
Alfvaen will not clasp hands and drink together.” It was in GalLing’, but it was perfectly acceptable by
Alfvaen’s standards.
Maggy’s arachne pricked its way closer and said politely, “May I watch, Alfvaen, Om im? I have
never seen a real duel.”
Still stiffly posed, Alfvaen said over her forearm. “I don’t want to fight you, Om im. I was only trying
to protect swift-Kalat.” It was a break in the ritual, and a welcome one to both Tocohl and Om im.
“Duel?” said van Zoveel, taking a step forward to stare at the two of them in utter amazement. He
spun on Tocohl, his ribbons fluttering nervously, “Are they crazy? Tocohl? I don’t follow this one.”
“You threatened swift-Kalat, Alfvaen challenged you, Om im appointed himself your champion,”
Tocohl said, summing it all up as briefly as possible. In Bluesippan she added, “Fool is right, Om im.”
Om im answered in the same tongue, “I offered my blade; you accepted. Get her to fight on my
terms and we’ll both be fine.”
Instead, Tocohl glanced at van Zoveel. “—Oh, if he’s to save your life, van Zoveel, he would like to
know just what this is all about.”
For the first time, Tocohl saw fear in the Zoveelian’s eyes. “I was angry,” he said; with effort he kept
his voice low and level, “I only spoke words, Tocohl. I have no actions to perform.”
Tocohl raised her voice, “Alfvaen? He says he didn’t mean it: he never intended harm. He’ll clasp
hands and drink with you and swift-Kalat”—she glared at van Zoveel—“right?”
“Yes, of course,” he said, “with both of you, and Om im, and”—he glanced in layli-layli calulan’s
direction, set his features stubbornly—“with Alfvaen and swift-Kalat and Om im,” he finished.
Alfvaen said, “And I with all of you.” She lowered her arm. “No duel,” she said firmly to Om im,
who said, “Glad to hear it,” and lowered his arm as well. Alfvaen smiled wanly and added, “Not even for
Maggy’s education.”
“That’s settled then,” said Tocohl. “Sorry, Maggy.”
“Maybe some other time?” The arachne took a hopeful step forward to tilt toward the two
reconciled opponents. Alfvaen giggled.
Kejesli cleared his throat. “I want an explanation for this kind of behavior. You first, van Zoveel.”
“They,” van Zoveel began, and his gaze slipped away uneasily. He began again, “They wanted to cut
up the body for no—” This time he stopped completely, and with a supplicatory gesture at Tocohl, he
said, “She doesn’t care. She never did.”
Kejesli frowned but relinquished the floor once more to Tocohl who, understanding his anger at last,
said quietly, “How much Yn do you understand, Ruurd?”
If the question surprised him, it also gave him a chance to collect himself. “I understood the words,
but not the content, of what the two of you said before. I speak the female dialect but I know very little
of the male.” His palm brushed his sideburns. “Oloitokitok was teaching me...” The hand came away
abruptly, fell to his side.
Tocohl crossing the room to where layll-layli calulan stood, facing the wall, her koli threadflickering between her hands and her back as expressive of her anger as her threats had been. Placing an
arm gently about the smaller woman’s shoulders, Tocohl urged softly, in Yn, “Tell him: tell him who
Oloitokitok was.” She bent to look into the shaman’s dark eyes. “He cared about Oloitokitok,
too—that’s why he’s so upset. He thinks that you intend to deny Oloitokitok an afterlife.”
The koli thread stopped its flicker, and layli-layli looked up skeptically. Tocohl said, “Yes, by his
culture’s standards: bad enough the golden scoffers damaged the body, but that you would... ! Only
someone who hated Oloitokitok would injure his chances further. Van Zoveel’s gods only accept the
beautiful, the whole, into their paradise.”
“Is such cruelty possible?”
“Some gods are worse than others.”
Layli-layli turned to fix the skeptical gaze on Ruurd van Zoveel. She said, defiantly, “Oloitokitok
was my mate and my friend. How is it that you care, and yet you refuse us the right to learn the knots of
his life?”
As Tocohl had expected, layli-layli calulan had used the pronoun meaning “related-to.”
Van Zoveel not only heard the distinction but understood its significance. Deep sorrow lit his eyes.
He turned his palms up and knelt. “Forgive me layli-layli calulan, for not understanding the depth of
your feelings.
“On my world, to mutilate the remains of someone after death is the height of cruelty. As I listen to
you now, I realize that your dream is strong enough to give meaning to your acts, no matter how they
might have seemed to me in my ignorance.” His manner was pure Zoveelian but his words were Yn; his
contrition was equally comprehensible in either.
Maggy made a querying noise for Tocohl’s ear only. Tocohl said, (He’s so surprised she speaks of
Oloitokitok as a person and not a piece of property he’s willing to believe his gods will accept the
impression Oloitokitok made on layli-layli calulan as evidence of beauty and let him into paradise.)
(I don’t think that helps much,) Maggy said.
(I’ll try to do better later. Meanwhile, just accept that Homo sapiens can get pretty weird.)
(I don’t know what normal is for Homo sapiens,) Maggy said, with just the right touch of emphasis
to make the observation a complaint.
Layli-layli calulan breathed deeply, and the bright pink drained slowly from her scars. Her face
softened. Two steps brought her before van Zoveel who, though kneeling, was now of a height with her.
She touched the heel of her hand to his forehead, then laid her hands gently in his. “We were both
mistaken,” she said in GalLing’. “Will you accept my apology as well?”
“Of course,” Van Zoveel said earnestly, without hesitation, and layli-layli clasped his hands to draw
him to his feet.
“Well,” said Tocohl, “the storm has passed. If you’ll excuse me, I’ll go outside and see if I can find a
sprookje or two.” And before anyone could reply, she crossed the room and stepped into the clear,
sharp air, still crackling with ozone, beneath a sky streaked with sunlight.
The sodden pennants that declared the shaman within slapped at the infirmary wall in the gusting
wind; the sound made Tocohl jump. She shook her head, half to clear it, half to wonder at the
extravagance of emotion Flashfever drew from them all—suddenly aware of an ache across her
shoulders.
With what seemed extraordinary effort, she unclenched her hands. The tansy scent of bruised foliage
filled the air about her as the moss cloak peeled moistly from her palms. Tocohl inhaled deeply, glad of a
familiar scent to soften the sharp tang of Flashfever’s air.
Maggy pinged softly, almost inaudibly.
(I’ll be fine, Maggy,) she said, (give me a moment. Go talk to Alfvaen.)
A walk would ease her muscles of their tension-induced stiffness. She was down the infirmary steps
and five strides into the compound’s courtyard before she realized, from the soft squish of mud beside
her, that Om im was still at her side. His tactful silence was the most likely cause of Maggy’s softened
ping.
Tocohl signed her appreciation and began a Methven calming, matching her stride to the rhythm ofthe ritual. Slowly she felt her panic fade. When the last vestiges of fear were gone, she found herself at
the perimeter of the courtyard, looking out into the brilliant expanse of flashgrass. There she stopped. The
Methven ritual, she noted with interest, seemed also to have eased her storm nerves. Layli-layli calulan
was right about the ionization effect. Certainly she had been feeling it. Nobody in her right mind would
have interfered...
Clicking her tongue at her own behavior, she turned to Om im and grinned wryly. “Still with me?” she
asked.
He made the deepest, most flamboyant bow she had yet seen, came up grinning with relief. “I was
about to ask you that same question,” he said. “You pile risk on risk, Ish shan.”
“Who accepted Alfvaen’s challenge?”
In answer, he cocked his head and clapped a palm proudly to his chest. “That was only to distract
her from poor old Ruurd. I was afraid he might know the ritual too and use it simply because that was the
appropriate response, verbally speaking.”
When she glanced at him in surprise, he gave her the Bluesippan thumbs-up yes. “I’ve seen him in an
analogous situation. A rote inquiry in a little back street bar led him into a rote response that nearly got his
head knocked off.”
Tocohl laughed. She could think of any number of possibilities that could have led to the situation.
Sobering, she said, “It’s not likely that van Zoveel would have taken up Alfvaen’s challenge though,
especially as a rote response. He was so angry he was having trouble remembering his GalLing’—every
third bellow was in Zoveelian and even that wasn’t what I’d call articulate.”
“It was less likely that Alfvaen would take me up on my offer. I’ve worked with her off and on for
nearly twenty years total. I’ve seen her challenge more than once, but I’ve never seen her go through with
it.”
“She needs the proper responses,” Tocohl said, “which you provided.”
“I’m not from her culture: I don’t know any better. She was looking for an excuse to break it off.”
“How do you figure you rate if van Zoveel wouldn’t?”
He chuckled. “Have you taken a good look at me lately, Ish shan? At my height, I’m visibly not
Siveyn.”
Puzzled, Tocohl eyed him askance.
His chuckle turned to a laugh. “No, no, Ish shan. You’re looking with the wrong eye. Don’t look at
me in Bluesippan. Look at me in Siveyn.”
She did her best to comply to his instructions—and saw precisely what he had intended her to see.
Her laugh joined his.
Of course Alfvaen couldn’t have gone through with the duel. Who in her right mind would think some
alien midget an honorable opponent? Not Alfvaen—especially not Alfvaen, given her romantic bent. Van
Zoveel, on the other hand, would not have been so lucky. Om im was right, there. Slipping back into
Bluesippan, she turned up her thumbs to tell him so.
“You wouldn’t have had that protection either, Ish shan,” he added. He turned his face up and closed
his eyes. “Ah, sun!” he said fervently. “You’ll notice the rest of the team is beginning to turn out. You’ll
get your sprookjes soon.”
“About time,” Tocohl said.
He pointed to the perimeter fence at the opposite end of the compound where the flashwood
crowded close, dripping water and reflected light. “They usually come out of the flashwood just about
there.”
Taking the pointed finger for an invitation, Tocohl started back across the compound.
“Risk upon risk,” Om im said again as he fell in beside her. “The least I could do was handle Alfvaen.
My blade was no help against layli-layli calulan. When she started to remove that second ring of
hers...”
“Am I to understand you credit those bar stories about the Yn shaman’s Death Curse?”
He stopped in his tracks, gave her a mocking look. “What, a cosmopolitan fellow like me believe bar
stories? Let’s be reasonable,” he said. His face sobered and so did his tone. “I’ve seen layli-laylicalulan’s ability to speed the healing process with her touch and her rituals. What she can do for good, I
have no doubt she can do for ill as well.”
When she made no response, he went on, “At any rate, I’m in good company. I saw your face when
you attempted to stop her from removing that second ring. I warned you about rash acts, Ish shan, but I
had no idea how rash your acts could get!”
(I’ll be fine, Maggy,) Tocohl said, (give me a moment. Go talk to Alfvaen.)
To judge from the readings Maggy was getting from the 2nd skin, Tocohl needed silence for a
Methven ritual of calm. As Om im did not seem to distract Tocohl, Maggy felt no need to warn him to
silence, so she headed the arachne back to Alfvaen.
She thought she understood most of what had transpired. It had taken a lot of file-searching though.
The material on Yn shamans was purely non-experiential but there was a lot of it, especially what Tocohl
tagged “bar stories.” That would ordinarily have put it in the category of fiction but Maggy was still
unsure of the differences between fiction and fact, despite Tocohl’s attempted explanation. Listed as fact,
she had several scientific papers on the shaman’s ability to help or hinder another creature’s bodily
functions, which seemed to give credibility to seven of the bar stories. She continued to search and
compare, while she watched them and recorded everything. She was not about to miss a clue to the Y
shaman.
Layli-layli calulan was speaking to Kejesli: “Captain, I offer myself for disciplinary action. The fault
was entirely mine. Ruurd van Zoveel has every right to ask for compensation.”
Kejesli gave her what Maggy tentatively interpreted as a bewildered look. Maggy hoped layli-layli
would explain, but van Zoveel gave her no chance. “Oh, no!” he said. “That’s not necessary, layli-layli.
We had a misunderstanding, that’s all!”
If Maggy’s information about the curses was fact, it had been a very dangerous misunderstanding.
“All right,” said layli-layli calulan. (Except to note that she was no longer angry, Maggy was unable
to categorize the shaman’s expression.) She hesitated, then said, in the very gentle tone with which
Maggy had heard parents address injured children, “Then swift-Kalat and I may return to our task?”
Van Zoveel opened his mouth to speak but nothing came out. He tried a second time. Maggy had to
enhance his reply to hear it. “Yes,” he said.
“Then you would be less distressed if you did not stay,” layli-layli calulan told him, again in her very
gentle tone.
“Yes,” said van Zoveel. “I’ll be outside. I’d better explain to Timosie, anyway—he was also very
concerned.”
That sent Maggy on another file search. She knew, from Tocohl’s explanation to layli-layli calulan,
why van Zoveel had been so angry. But Timosie Megeve was Maldeneantine, so she wanted to see if the
same explanation applied.
Alfvaen leaned suddenly back against one of the infirmary beds. She looked very much like she had
when Tocohl had asked Maggy to find a real doctor for her. Maggy wasn’t sure if layli-layli calulan
was to be considered a real doctor in Tocohl’s use of the term but Maggy had enough points of
congruence to assume a similar situation.
She sent the arachne a few steps closer to Alfvaen and said through its vocoder, “Psst, Alfvaen.”
Alfvaen peered down at the arachne. “What is-s it, Maggy?”
Yes, she was having the same trouble focusing, Maggy saw, and the same speech difficulty. That was
sufficient confirmation to act on.
On the theory that speaking of infirmities was to be done quietly, Maggy kept the vocoder low.
“Perhaps you are in need of your medication, Alfvaen. Your speech has slurred and—”
“You’re right. I am.” Alfvaen drew the pill box from her pouch. Once again she had difficulty opening
it, but Maggy couldn’t help her as Tocohl had. The arachne was less adept at fine control than Alfvaen
was at the moment. At last, Alfvaen managed to open the box; she took her pill. “Thank you, Maggy,”
she whispered, “I appreciate it. The worst part about Cana’s disease is that your judgment gets bad just
when you need it the most.”
That seemed to call for a response, so Maggy said “I think I understand.”During their exchange, van Zoveel had left. Now layli-layli bent swiftly to her task.
Alfvaen peered down again, then reached for the arachne and set it on the bed beside her, where the
lens had an unobstructed view of the procedure.
Making a score of delicate incisions at the nape of Oloitokitok’s neck, layli-layli calulan extracted
samples of the tissue. These she inserted into—Maggy angled the arachne slightly—a machine which
Maggy tentatively identified as a diagnostic of some sort.
While she waited, layli-layli calulan silently knotted and reknotted her koli thread. It served a
purpose similar to that of the Methven ritual for calm, and Maggy noted with approval that the readings
from Tocohl’s 2nd skin were dropping to normal.
The machine chimed and issued a neatly racked series of slides. Layli-layli calulan pulled all the
knots from her koli thread, wrapped it several times about her right wrist, then picked up the rack of
slides. These she brought to Alfvaen.
Maggy projected that she intended to give them to Alfvaen for some reason but instead layli-layli
calulan placed the rack carefully on the bed beside the arachne. Then she touched Alfvaen’s chin to
examine her eyes. “I believe you are in need of medication,” she said, confirming Maggy’s private
approach to the suggestion.
“Yes,” said Alfvaen, “I just took care of it. Maggy reminded me.”
Layli-layli calulan turned her gaze on the arachne. “Thank you, maggy-maggy,” she said.
“You’re welcome,” Maggy told her through the arachne’s vocoder since she too seemed more
comfortable addressing a discrete entity, “I am glad to have your confirmation.”
Turning to meet Alfvaen’s eyes once more, layli-layli calulan extended a hand to indicate the rack
of slides. “Choose for me, please,” she said.
From across the room, Kejesli said, “Layli-layli calulan, Alfvaen has Cana’s di—”
The shaman ignored him. “Choose for me,” she said again.
Kejesli made a noise that Maggy stored for later reference, while Alfvaen brushed her hand lightly
across the edges of the slides, her face intent. Maggy focused the arachne’s lens tightly on Alfvaen’s face
and hands, in the hope that she might record Alfvaen’s serendipity at work.
“Try that one,” Alfvaen said, indicating a slide. Maggy reran her tape twice; whatever Alfvaen had
done, it wasn’t on the record.
Swift-Kalat came forward to accept the slide from layli-layli’s hand. He strode to the microscope,
inserted it. Alfvaen picked up the arachne to follow him, holding it carefully before her as if the arachne
were very delicate.
Layli-layli calulan and Kejesli joined them to watch without comment as swift-Kalat keyed the
computer for display. Images, some stored, some current, played acoss the screen as swift-Kalat
examined and fined his examination with the use of various light sources and filters. At last the screen held
a single image.
“Yes,” he said, “if we were to leave the body out of stasis, these would become garbage plants.”
“And what’s the significance of that?” ask Kejesli, as if the question had been drawn from her by
force.
Swift-Kalat sighed. “I don’t know. All I can say that, of all the dead creatures we picked up, only
that”—he gestured at Oloitokitok’s body—“and the golden scoffer I dissected the same day are growing
garbage plants. They were both bitten by sprookjes. If the other golden scoffers have no sprookje bites
and no garbage plants, then I would theorize that the sprookjes were responsible for the garbage plants.”
“And what would be the significance of that?”
It was layli-layli calulan who answered: “If the sprookjes are responsible for the garbage plants,
they may also have been responsible for Oloitokitol’s death.”
Chapter Eleven
YOU’LL SEE SPROOKJES any minute,” Hitoshi Dan assured Tocohl, indicating with awide-flung palm the same area of the perimeter fence Om im had.
His arrival precluded any further discussion of her rash acts but Tocohl knew this was only
temporary; blade service gave Om im Chadeayne not only the right but the duty to reproach her for a
risk such as she’d taken. Maggy would no doubt add a few words on the subject, as well. On her own
part, Maggy minded redundancy not in the slightest. But for the moment, Tocohl had a respite from both.
Respite from Flashfever’s storms brought other surveyors out in number—how was it Om im had
phrased it?—to dry their feathers. Tocohl watched as they formed small celebratory clusters, each a
hodgepodge of style and manner. In all, the survey team was as diverse a group as she’d seen anywhere,
no two from the same culture and most in worlds’ motley.
“Yo, Hellspark!” came Buntec’s bellow from across the compound. Tocohl turned, to see Buntec
grab Edge-of-Dark by the hand and drag her along. The two arrived breathless with excitement, but
before Tocohl could learn the reason behind it, Om im nudged her. “Sprookjes,” he said.
Forgetting Buntec and Edge-of-Dark, Tocohl turned to stare into the brilliance of the flashwood. A
handful of sprookjes, disturbingly dark amid the sparkle, pushed through the chattering, tinkling foliage.
One by one, they squirmed cautiously through the great circles of barbed wire.
As each emerged and paused to preen the mud from its plumage, Om im named them, each in turn:
“Bezymianny’s sprookje, John the Smith’s, Captain Kejesli’s, Edge-of-Dark’s, Hitoshi Dan’s...”
He was quickly proven accurate. Those he had called Hitoshi Dan’s and Edge-of-Dark’s sprookjes
made straight for Hitoshi Dan and Edge-of-Dark. The others passed on as if the little group of humans
were invisible.
In turn, Om im ignored the sprookjes to pay court to Edge-of-Dark with a formal greeting.
Edge-of-Dark flushed pink from her hairline to the tops of her boots. “We’re old friends, Om im!
You don’t have to do that!” The protest was echoed by the sprookje Om im had designated hers.
Casting a frown at the creature, she stiffened and went on, “We’ve got something important...” So
did the sprookje. Edge-of-Dark’s frown turned to an open scowl and she jabbed both forefingers at the
sprookje. “I can’t stand talking when that thing is around! You tell them, Buntec.”
“Botanic art,” said Buntec, and looked furtively around her, to see if the other sprookje would mimic
her. When it didn’t, she went on to explain, in glowing terms, Edge-of-Dark’s flower art.
Tocohl listened with interest but she kept her eye on the sprookjes. They were beautiful indeed, their
sleek feathers beaded with drops of water that they shook away in tiny discrete spatters, rippling first one
set of muscles, then another. Their control was so remarkable that Tocohl wondered if each feather might
be moved separately. Their gold eyes were intent upon the humans, but neither gave any indication of so
much as hearing Buntec.
Hitoshi Dan did, his eyes sparkled with excitement. His first few words proved that Om im had
correctly identified the second sprookje. “If that’s so,” Hitoshi Dan said—and the sprookje’s
pace-keeping echo did nothing to dampen his enthusiasm—“how about landscaping? That’s a botanic art
too! You use whole plants rather than cuttings and your arrangement is... an artistically planned
environment.” His circled arms implied base camp in its entirety. “Better than mud,” he and the sprookje
said.
In concert, the two went on to explain landscaping at length. Buntec and Edge-of-Dark, neither of
whom it seemed had encountered this art form, listened to their combined voices with growing wonder.
More sprookjes made their way through the barbed wire. Two more—Tocohl supposed them to be
Om im’s and Buntec’s—joined the little group, as intent as the first and, for the moment, as silent as
Buntec’s.
Tocohl looked again from Buntec’s sprookje to Hitoshi Dan’s. There were subtle differences in
marking. On the feathers of one the brocade held more gold than brown, on the feathers of the other the
loops-and scallops were deeper, more defined. Om im had a sharper eye than she, to be able to
distinguish them at a glance, and at that distance—or his pattern recognition was better.
“I’m no expert on the subject,” Hitoshi Dan finished, “but somebody on the team must know
something about it—landscaping is an art common to many worlds.”
“Hey!” said Om im sharply, and something behind Tocohl startled her forward by saying “Hey!” justas sharply and just as unexpectedly.
Tocohl spun at the warning. Her cloak swept a wide arc, caught suddenly as if snagged. Behind her,
a startled sprookje let the edge of the moss cloak jerk from its fingers and backed hastily away.
(Sorry,) Maggy said, (your cloak’s in the way. I didn’t see it coming.)
“Speaking of botanic art,” Om im continued, in a lighter tone, one of relief, “doesn’t your cloak
qualify?” He set his face, determined not to let the echo bother him, and waved off his echoing sprookje
as it approached Tocohl from the other side. “That’s the most interest they’ve shown in anything.”
It was true, both sprookjes edged toward her and the other two showed signs of developing the
same tendency. Keeping a wary eye on them, Tocohl held out her hand. “Lend me your knife, Om im.”
There was a gasp of objection from Hitoshi Dan that sounded still more horrified in the beaked
mouth of his sprookje. “Hellspark!” said Buntec, “I thought you, of all people—”
Om im laid the hilt of his knife in Tocohl’s outstretched palm.
Buntec repeated, “You of all people,” this time in quite a different tone. “Mighta known,” she
muttered at her sprookje, who agreed in an identical mutter, its cheek feathers puffing.
Tocohl lifted the edge of her cloak and slipped the knife through it, cutting four pieces. She flipped
the knife first blade up, then blade down, and returned it to Om im with a generous bow.
That done, she held out the first bit of cloak to the sprookje she’d startled, Hitoshi Dan’s. First in
Gal-Ling’ and then Hitoshi Dan’s language, she told the sprookje, “My name is Tocohl Susumo. Please
accept this with my compliments.”
The sprookje said nothing, to her great disappointment, but she continued to hold the bit of moss
cloak extending from between her fingertips, so that the creature might touch it without touching her.
The puffed feathers along the sprookje’s cheeks slowly deflated and the creature stepped cautiously
forward to examine, then snatch, the piece of moss.
Tocohl watched as it stared (happily?) at its prize, then she turned to the second sprookje and
repeated the process in Bluesippan. As before, the sprookje said nothing but accepted the tuft of moss
from her fingers. Tocohl sighed and went doggedly on, in Vyrnwyn, in Jannisetti. Each time she met with
the same result: a silent but accepting sprookje.
“They know it’s a gift!” said Om im, voicing it for the rest. His sprookje echoed the words, giving
Tocohl the impression that it too spoke for its companions. Om im gave the creature a sharp look but
went on, “I don’t know if that feather-puffing is sprookje-surprise or sprookje-pleasure,
but—look!—they’re each keeping the piece you gave them!” Then, with something like satisfaction in his
voice, he added, “If they aren’t intelligent, my blade has no edge!”
“I can vouch for the blade,” said Tocohl. “You tried an exchange of gifts once before...”
“We did,” said Hitoshi Dan (and sprookje), “and they ignored what we offered. Maybe they didn’t
recognize what we gave them.”
“Perhaps they didn’t,” Tocohl agreed. To Buntec and Edge-of-Dark she added, “Now we have
some new avenues to explore.”
Buntec grinned at the Vyrnwyn. “Move ass, Edge-of-Dark,” she said, “Let’s get to it.” Her sprookje
repeated Buntec’s command with the same emphasis; this time Buntec laughed. As she dragged
Edge-of-Dark away, their sprookjes, tufts of moss still clutched in their hands, followed—one still
echoing Buntec’s laughter.
In the hour that followed, Tocohl, accompanied by Om im, van Zoveel, and their sprookjes, met and
introduced herself to each of the sprookjes that came into the camp, with no success in any sense. Ruurd
van Zoveel gave her a demonstration of his own sprookje’s ability to mimic by running through twelve
different languages. The sprookje had, without accent (or rather, with the same accent—that dictated by
the sprookje’s beaklike mouth), repeated every single one accurately. Yet when Tocohl tried the same
thing the sprookje remained silent.
Most of the surveyors envied the silence, she discovered. Although Hitoshi Dan and swift-Kalat
would speak in the presence of a repeating sprookje, the others would not except in necessity: the result
was irritating as well as frustrating. Flashfever’s irritation index was singularly high in more ways than one,
Tocohl thought, as frustrated by the sprookjes’ silence as the others were by their volubility.Yet, as she sat in the misty sunlight and gazed into the flashwood, she felt the world had more than
sufficient beauty to compensate for the trouble it caused.
Ruurd van Zoveel, Om im, and their respective sprookjes kept silent company beside her.
Edge-of-Dark had vanished into the flashwood to pick leaves and flowers for a demonstration for the
sprookjes, and Hitoshi Dan had appointed himself a delegation of one to find a surveyor who knew
something about landscaping, so far without any luck.
She turned her attention back to the sprookjes, who watched their humans with proprietary interest,
as if waiting for an opportunity to speak. Their feathers rippled with the intensity of their concentration.
Why do they never volunteer a word? Tocohl wondered. Maggy volunteers information to the point
of distraction. She smiled to herself at the thought. New definition of sentient: that which gives unsought
advice.
It struck her that Maggy had been uncharacteristically quiet. (What’s swift-Kalat up to, Maggy?) she
subvocalized.
(He is examining the golden scoffers found dead near Oloitokitok’s body for sprookje bites and
garbage plants.)
(What’s a garbage plant?)
Maggy flashed a brief image on Tocohl’s spectacles of a refuse heap. Long, silvery-gray filaments,
like algae or seaweed, grew from it. (This,) she said, (swift-Kalat says to tell you,) and here the voice
shifted to that of swift-Kalat himself, (“The common name is garbage plant because we have only found
them growing on our refuse. Until Oloitokitok’s death, that is. They take several components poisonous
to the indigenous wildlife and break them down, rendering them harmless.”)
There was a brief pause—Maggy was evidently putting together several bits of conversation rather
than relaying one as it happened—then swift-Kalat’s voice continued: (“So, if there are no sprookje
bites, there are no garbage plants. If there are sprookje bites, as on Oloitokitok’s body and on two of
the golden scoffers, there are garbage plants. We may be able to verify that with a simple experiment.
We give the sprookjes access to the remaining golden scoffers and we observe.”)
This last remark was evidently a simultaneous transmission, for Alfvaen’s wave caught Tocohl’s eye,
and Tocohl looked across the compound to see Alfvaen emerge from swift-Kalat’s cabin. She placed a
small table at the foot of the steps, then darted back up to sweep the membrane aside for swift-Kalat.
His hands were full of stacked boxes which he carried down the steps to place on the table. Maggy’s
arachne squatted on the top step, ready and waiting to observe.
Alfvaen crossed to Tocohl who said, by way of greeting, “I know. Maggy’s been keeping me
posted.”
“Now if you could only teach her to scratch backs...” said Alfvaen, grinning.
“Do you want your back scratched?” asked Maggy, from the hand-held on Alfvaen’s belt. “I will
send the arachne if you wish.”
“Oh, no!” said Alfvaen, somewhat embarrassed. “My arm’s not broken, Maggy. I was just making a
joke.” She canted an arm behind her and scratched self-consciously.
“I don’t understand jokes very well,” said Maggy, sounding apologetic. “Should I laugh?”
Tocohl cocked her head to one side and, after consideration, said, “No. Not now—the laughter has
to come at a specific time to have the correct effect.”
“Okay. Sorry, Alfvaen. I hope I will do better next time.”
“No need for an apology, Maggy.” Alfvaen looked around for something to smile at reassuringly, and
settled for Tocohl’s spectacles. Tocohl thought that an admirable choice.
Maggy said in her ear, (Will you explain later, please?)
(Yes,) said Tocohl. She watched inquiringly as Alfvaen reached again, apparently involuntarily, to
scratch at her back.
Seeing the look, Alfvaen said, “It was seeing all those garbage plants, even though the ones on
Oloitokitok hadn’t broken through the skin yet—the whole idea makes me feel crawly all over.”
Tocohl spread her hands in sympathy.
“Garbage plants?” said Om im and sprookje, and Tocohl let Alfvaen explain.“Hey!” said Timosie Megeve. “What are you doing?”
Buntec looked up, unabashed, from the innards of the largest daisy-clipper, a smear of graphite
across her face.
“I thought I’d give the Hellspark our grand tour, as long as I’m going out to look for evidence of
grafting. You know as well as I do we’ve been up to our asses in equipment failures—no, no!—no
reflection on you! I just thought I’d check things out beforehand as a precaution. Why take chances?”
She closed the service panel with a snap, wiped her hands and—in response to Megeve’s pointing
finger—her face as well.
“Why the big one?” the Maldeneantine asked.
“Because I haven’t flown the big one for weeks, and I want to give the flashgrass a thrill. Why not
the big one? That’s a gorgeous machine, you gotta admit.” She patted it affectionately on the prow.
“True,” said Megeve, “but I do wish you’d keep your fingers out of it... after all, that is supposed to
be my job.”
“I’ll try,” said Buntec seriously, “but it’s not gonna be easy.” She gave a wistful glance at the
daisy-clipper and said, “See you later—I’m off to find the Hellspark before Flashfever comes up with
another gully-washer.”
She left him with a grin at his bewildered look and trotted happily back to the compound, where she
found Tocohl trying once again to get Om im’s sprookje to respond to Bluesippan—to no avail.
“How’s about checking out the wild ones, Tocohl? I’m just about to go graft-hunting in some of the
places we’ve seen the high-class sprookjes—you know, with the crests?” Her hand swept above her
head in graphic explanation.
“It’s worth a try,” said Tocohl. “I don’t seem to be having much luck here.”
“Speaking of luck,” said Buntec, “want to come, Alfvaen? You’re a whole lot better at catching wild
geese than I am.”
“Jaef?” Alfvaen used swift-Kalat’s soft-name as a full and complete query.
“Go with them,” swift-Kalat told her, “I’ll find someone else to hold the camera.”
The arachne’s legs telescoped to the fullest extent of their length, much as if Maggy wanted to make
herself more noticeable. “I’ll help you, swift-Kalat, if you like,” said the voice from the vocoder.
He hesitated, considering the spidery mobile.
Maggy said, “I’ll save everything.”
Tocohl laughed. “She means that. She won’t dump any data until you’ve told her to specifically, if
that’s what’s worrying you.” Snapping her wrist to give ring to her words, Tocohl added, “Her ‘eye’ is
better than any camera, not only because it sees the full 360 degrees, but because she makes good
choices as a rule. The fuller the information you give her, the better the choices. Just tell her what you
want taped and why. At your leisure you can sort through what footage you wish to keep and she can
transfer that to the survey computer.”
“All right,” said swift-Kalat to Tocohl. Then he appeared to think better of it and turned to the
arachne, “Yes, Maggy. I’d appreciate the help.”
“That’s settled,” said Buntec to Alfvaen, “you’ll come?”
“Of course. If there’s room.”
“The more the merrier. Timosie’s bugged that I want to use the big daisy-clipper. He’ll feel better if I
bring a crowd. Ruurd?”
“Thank you, but no,” said van Zoveel, “not where Buntec’s piloting.”
“You don’t know good when you see it, Ruurd,” Om im and sprookje said.”Count me in, Tocohl.”
He rose to his feet and the sprookje followed suit.
Tocohl glanced from one to the other. “How do the sprookjes feel about the daisy-clippers?”
Om im and sprookje said, “Timosie shoos them away from ‘his’ equipment. He figures it’s too
dangerous to chance.”
With a broad grin, Buntec said to the sprookje, “I know just how you feel; I got shooed away, too.”
To the humans, she said cheerfully, “Let’s move—Dyxte will keep us posted on the weather, but wealready know there’s going to be another thunderstorm anytime now—there always is.”
Leaving Ruurd and Maggy’s arachne behind to keep swift-Kalat company, the five of them—Tocohl
counted the sprookje—walked to the main gate of the compound. Just beyond it lay the makeshift
hangar with its population of daisy-clippers, large and small. Beneath her feet, the flashgrass, which
anchored itself with something like roots, was springy and glistening. After the mud of the compound,
Tocohl delighted in the sensation. Soon the flashgrass reached their knees and Tocohl lifted her cloak to
prevent its being caught on the flickering wiry strands.
Timosie Megeve came around the largest daisy-clipper, shouted, and clapped his hands at a
sprookje that stood before it. The sprookje moved off a short distance but no farther. Megeve appeared
to find the distance acceptable and came from beneath the hangar to greet them.
A throat mike swung from his hand. “I thought, just as a precaution, Tocohl, you ought to be able to
talk to Buntec when you’re out of earshot.”
At this distance, Megeve’s sprookje didn’t bother to echo him. “It’s no good for long range, mind
you—a mile is about the limit—but it might come in handy.”
“It might at that,” said Tocohl. “Thanks for thinking of it.” She clasped the curve of plastic about her
throat and put the tiny earplug in her ear; the weight was scarcely noticeable. As Buntec had taken an
identical device from her overpocket and put it on, Tocohl said, experimentally, in Jannisetti, “How’re
you holding?”
“Just fine.” Buntec’s reply came as clearly through the earpiece as it did through the sparkling air of
Flashfever. “Just fine,” Tocohl repeated in GalLing’ for Megeve’s benefit. “Can you find us two more?”
Timosie Megeve frowned. “Two more?”
Indicating Om im, then Alfvaen, Buntec said, “Wouldn’t dream of going out to hunt sprookjes
without a serendipitist, or somebody with rocks in his head.” Om im chuckled delightedly, but Megeve
didn’t share his amusement, merely looked stubborn.
“Don’t be chintzy, Timosie,” said Buntec. “It’s not as if we’re going to eat your equipment!”
“Or use it to hang pictures on,” added Om im and his sprookje.
Buntec said, “Please stop talking until we get into the daisy-clipper and on our way—I can’t stand
it!”
“Then you’d better keep your mouth shut, too,” said the small man affably. “Here comes your
sprookje.”
Having finished its echo, Om im’s sprookje turned with the rest of them to see a third sprookje push
its way through the flashgrass. Timosie Megeve muttered in despair and raised his hands to clap.
Tocohl caught his hands in midair. “No!” she said sharply but quietly. “I think it’s got something in its
hand. For Veschke’s sake, don’t do anything to scare it!”
She did not take her eyes from the arriving sprookje, but she felt Megeve’s muscles relax and
released his hands. Beside her, the others all turned cautiously to follow her gaze. Except for the sound of
wind and distant thunder, everything was still as Buntec’s sprookje pushed slowly toward them.
(Maggy,) Tocohl said, (record this.)
The sprookje did indeed have something in its hand.
It stopped a scant two feet from Buntec and stared at her, its feathers ruffling in the wind. It seemed
almost expectant. Then it came to Tocohl and, with a gesture identical to that with which Tocohl had
passed out the bits of moss cloak, it stretched out its arm and offered her a tuft of something red and
velvety-looking.
Praying that the fragile moment wouldn’t break, that no one in the group would shout and frighten the
sprookje away, Tocohl leaned forward, not daring to step closer, and reached out to take the gift. She
turned it over slowly in her hands, allowing Maggy a good steady view through the spectacles, felt its
texture. It was as velvety to the touch as it was to the eye, and it was not merely red, but patterned in
shades of red. It was definitely vegetable matter.
Maggy said, (Nothing on it in the survey files; shall I ask swift-Kalat—)
(Later, Maggy,) said Tocohl. Aloud, she addressed the sprookje in Jannisetti. “Thank you,” and on
impulse, she held out her hand, just barely touching the feathery softness of the sprookje’s wrist. Thesprookje’s cheek-feathers puffed slightly but it did not move away.
Tocohl drew her hand back and stroked the bit of mosslike substance, then, as slowly and carefully
as she had done it the first time, she again reached out and touched the sprookje’s wrist. This time she
made a light stroking motion. The sprookje’s cheek-feathers smoothed.
Grinning with excitement, Tocohl brought her hand back and made a great show, for the sprookje’s
benefit, of placing their gift into the pouch at her hip.
Very softly, she said, “Buntec, I’d like to try an experiment.”
Buntec said in the same tone, “You name it; I’ll do it,” and her sprookje agreed.
“Your knife, please, Om im.” Tocohl brought the knife up under the cloak, thrust it outward just
beneath her chin, and made a long cut down the lay of the cloak’s foliage to its now-ragged hem. A long
narrow strip of moss peeled away into her hand. Its tansy scent spiced the air about her.
Cheek-feathers puffed on every sprookje face. Buntec said, interpreting the look, “You’re a rotten
person, Tocohl, cuttin’ something beautiful like that.” Again her sprookje echoed her words.
“You could be right, Buntec. I hope they’ll forgive a bit of expediency.”
Om im did, for he accepted the return of his knife without the customary ceremony.
Taking the long strip of moss by either end, Tocohl spread her hands and gave it a gentle tug to
demonstrate its strength. Then she held out one end to the sprookje, who took it. Still holding the other
end of the strip in her left hand, she reached out again with her right to stroke the sprookje’s wrist. The
strip looped between them and remained even when she withdrew her right hand.
After a moment, the sprookje’s cheek-feathers relaxed. Tocohl said, in the same soft voice, “Now,
very calmly, one at a time, I want you all to get into the daisy-clipper. Megeve, if you’re not coming with
us, please go back into the hangar, out of sight.”
He stared at her aghast. In the same hushed tone, he said, “But you’ve got an exchange of gifts!
You’ve got to tell the captain—that’s more than we’ve gotten in these three years...” His sprookje too
let its voice trail off as if in astonishment of its own.
“Good,” Tocohl said, “but not enough, not for your captain.”
“What more could you want?” Megeve and his sprookje demanded.
“Art, artifacts”—Tocohl glared at the sprookjes that refused to echo her and let the exasperation
leak into her voice—“language. We’re still going to look for those grafts, with a native guide if possible.”
She indicated the sprookje at the other end of the strip of moss with a twist of her head; her grin
returned.
For a long moment, Megeve regarded her in shocked silence. At last he found his voice. “Crazy like
a Hellspark,” he said, and his sprookje thought so too. “You’ll need those microphones. Don’t worry, I
won’t do anything to frighten off your sprookjes.”
The tableau held as Megeve stepped cautiously away. He was gone from sight for a few moments
only, and his sprookje did not follow him, choosing to remain behind and watch with great golden eyes. It
only backed slightly away when Megeve returned to distribute microphone necklaces to the others.
Om im took his with a very broad grin. “We’ll leave you the pleasure of telling the captain of the Ish
shan’s first success, Timosie.”
“I wouldn’t dream of it,” the Maldeneantine grinned back. “It will keep until you get back. Buntec
would... strip me naked and roll me through the razor-grass—” The sprookje paused on the same
querying note, as Megeve glanced at Buntec, who doubled her fists and jerked her elbows back sharply
to attest to the accuracy of his phrasing. “—If she missed seeing the expression on Kejesli’s face,” he
finished.
“Absolutely,” agreed Buntec and her sprookje. “And the minute we’re gone he’s gonna hightail it
back to camp and shout his lungs out”—but despite her excitement, she kept her imaginary shout to a
whisper—“The Hellspark swapped gifts! The Hellspark swapped gifts!” She exchanged a look with her
sprookje and together they finished, “I know I would.”
(He probably will,) Tocohl observed privately to Maggy. (If the arachne is available, try to record the
expression on Kejesli’s face for Buntec when he does.)
(It’s important?)(To Buntec, it is. Now, hush. I have to pay attention.)
“Please be careful,” Megeve was saying, “I don’t trust that thing around my machinery.” His own
thing voiced the identical sentiments. He glared at it. “If you must take one, take mine.” Then he vanished,
scowling, back into the hangar.
“Okay, Buntec,” said Tocohl, “you first.” And each of the others in turn climbed into the
daisy-clipper, leaving Tocohl alone with the three sprookjes. Flashgrass danced and flickered around
them.
She took a single step toward the daisy-clipper, allowing the strip of moss cloak to go taut, and then
a second step. The sprookje let go.
Patiently, Tocohl picked up the fallen end of the strip and offered it again to the sprookje. The
sprookje accepted it, and Tocohl took another step. This time the sprookje followed, although at the
farthest extreme the length of moss permitted.
The other sprookjes came too, picking their way through the flashgrass warily, as if the strip of moss
gave Buntec’s sprookje a shield that they did not have.
The strip loosened as the sprookje quickened its steps to keep pace a foot or two behind her. When
Tocohl reached the daisy-clipper, she played out another foot of moss and climbed in, sliding well over in
the plush seat to allow sufficient room for the sprookje, and began to reel in the moss strip. The sprookje
did not move; the strip went taut and Tocohl tugged gently. The sprookje let go. Tocohl slid to the door,
picked up the end of the strip, and offered it again to the sprookje. But before Buntec’s sprookje could
accept, the second sprookje, Megeve’s, grabbed for the trailing end and caught it. Startled, Buntec’s
sprookje skittered back a few feet.
Megeve’s sprookje took a single step toward the daisy-clipper.
Om im said quietly, “I think you’ve got a volunteer.”
“We’ll see,” Tocohl said. With a slight frown of concentration, she slid back to her position next to
Om im and, slowly but surely, began to reel in Megeve’s sprookje.
And slowly but surely it followed. Through the door it came, its cheek-feathers fully puffed. Inside, it
looked around slowly, taking everything in as carefully as Maggy’s arachne might.
Its thigh to calf ratio was different from the human, but it sat, turning slightly to the side, to allow room
for its knees. Aside from a few ruffled cheek-feathers, it did not look uncomfortable.
Tocohl fastened her seat belt with great deliberation, so the sprookje could see exactly what she was
doing and how. She had hoped that the sprookje would follow suit, but she was disappointed. When the
sprookje made no similar move, she leaned across it and, very gingerly, fastened its seat belt—snapped it
open to reassure the creature—then fastened it again.
The sprookje watched her gravely, and snapped the seat belt open. “Okay,” said Tocohl, “you’ve
got that down.”
When the sprookje made no further move, Tocohl patiently fastened its seat belt again, leaned farther
across to close the door, and waited to see what the creature would do next. The sprookje looked at the
seat belt, looked at the door, and sighed—as if resigned to its fate—and eased back into the seat.
“All right,” said Tocohl, “now that we’ve got our trusty native guide, let’s get this expedition off the
ground, shall we?”
Buntec said, “I think you’re crazy, Hellspark, but at least you picked one that doesn’t talk back—not
to us. You just made Timosie’s day in more ways than one.” She turned to her instruments, adding over
her shoulder, “Give a holler if your fine-feathered friend gets too hopped up. I’ll take this as slow and
easy as I can.”
Swift-Kalat was pleased. Within minutes after he had removed the wrappings from the corpses of the
golden scoffers, the first sprookje to see them—his own, in fact—had nipped each of them in turn. The
first it had bitten twice, each of the others only once, as if extrapolating from the first.
He placed the bitten corpses into sterile boxes. In a few days, he’d know whether or not the
sprookjes themselves were the source of the garbage plants. Whether or not that datum would make a
difference to the survey team, he had no idea.Layli-layli calulan’s reasoning seemed farfetched but he was not one to ignore any theory without
good cause. If the sprookjes were consciously injecting the garbage plants into all the human debris to
cleanse their world of human-borne poisons, might they not also wish to cleanse their world of the human
intruders as well?
Stacking the boxes, he paused to consider the sprookje and found it watching the arachne. The
arachne, oddly enough, had skittered from one end of the table to the other. There it crouched, then
suddenly shot up to its full height, crouched a second time, and skittered the length of the table once
more. His first thought was that something had gone wrong with it but, no, its movements were too
purposeful...
“Maggy... ?” he began; so did the sprookje.
The arachne sprang once more to full height, startling the sprookje back in mid-query.
“I wanted to see if I could get it to notice the arachne.” The voice from the vocoder sounded
pleased. “It did.”
Swift-Kalat suddenly regretted that he had not called another surveyor to record for him. However
well Maggy recorded the sprookjes, the film would not include her own behavior which, to him, was
equally intriguing. Then he remembered that Tocohl was in constant communication with Maggy through
her implant. He and the sprookje asked, “Did Tocohl suggest you try that?”
“No, she’s busy. She told me to hush. I’ll show her later.” The arachne made another abrupt bob.
This time the sprookje only blinked at it. “I deduce that it considers the arachne harmless,” Maggy said.
“Yes,” he said, the sprookje seconding him. He picked up the boxes to carry them inside the cabin,
pausing on the threshold to scrape some of the mud from his feet. The arachne sprang from the table to
follow him. The sprookje did not; perhaps, like most of the surveyors, it was drying its feathers.
There was an odd sound behind him. He turned to look and found the arachne scraping its legs one
against the other in imitation.
“Shall I enter the material in your files?” Maggy asked.
“Yes, please, Maggy,” he said. “You are very useful to have around.”
The arachne bobbed a bow. “Thank you.” It pricked its way on delicate feet to the console and
stuck an adapter into one of the tiny receipt openings. The rest of its legs straightened to give it the height
to reach the keyboard.
“Do you need help?” asked swift-Kalat, suddenly realizing how much he was taking for granted
about the capabilities of this probe.
Implausibly, a chuckle came from the arachne. “No,” it said, “my arm’s not broken.—Did I say that
correctly?”
Swift-Kalat laughed, as much from astonishment as from amusement. “Yes,” he said, “I think so: you
sounded very like Alfvaen.”
“Good,” said the arachne, setting about its task.
For a long moment, swift-Kalat watched; there was nothing to see. At last he remembered the sterile
boxes he still held and crossed the room to put them in a safe place. When he returned, it was to draw up
a chair and sit, his elbows on his knees, his chin in his hands, and the best possible view of the arachne.
His bracelets clashed down his arm to his elbow.
His expertise, he found, was being challenged by more than the sprookjes. By definition, an
extrapolative computer was not sentient but, by definition neither were the sprookjes. The only difference
seemed to be that Maggy responded to questioning.
When the arachne withdrew its adapter, he said “Maggy, are you sentient?”
There was a long pause. Whether it indicated deep thought—and Maggy’s deep thought would be
faster than human—or was merely supplied for aesthetic reasons, swift-Kalat couldn’t judge.
At last, Maggy said, “I don’t know. From what Tocohl says, none of the definitions of sentience in
my memory is true and sufficient to cover all cases. Legally speaking, however—no, I’m not sentient.
Why do you ask?”
“In reaction to the extent of your curiosity.”
“That’s basic to an extrapolative computer. Curiosity is rather simple to program in: If the informationreceived doesn’t gibe with other information I have stored, I seek additional information; if I don’t have
enough information, I seek additional information. In me that’s called programming. In a human, that’s
called curiosity.”
“Again, a matter of definition,” swift-Kalat pointed out.
“I see what you mean. Yes, a matter of definition.”
Swift-Kalat fell silent. If he were asked, he wondered, would he be able to say, as he had with the
sprookjes, that he deduced sentience in an extrapolative computer? The question brought him full circle
to the legal definition: art, artifacts, language. Language, Maggy certainly had. And given the proper
waldoes, he did not doubt that she could produce an artifact if she chose to.
The computer console chimed an interruption to his thoughts and he rose absently to answer it.
The caller was Kejesli. “Hello, swift-Kalat. Is Tocohl with you?”
“Buntec took her into the flashwood. I don’t know how long they’ll be gone.”
“So she’ll miss what she seems to have started. Well, you come then. It should be of interest to you
as well: there’ll be a brief lecture on landscaping in the common room starting about five minutes from
now.” With that, he signed off.
When swift-Kalat glanced down he saw that the arachne was already on its way. He followed.
“Curious?” he asked.
“Curious,” she agreed as the two of them started across the compound. “Besides,” she added, “I
want to see the look on Kejesli’s face.”
“I don’t understand.”
“Neither do I, but it’s important to Buntec. If I record it for her, maybe she’ll explain why,” Maggy
said, then added, “If she tells me, I’ll tell you. I promise.”
Edge-of-Dark skirted the barbed-wire perimeter to the main gate, her arms laden with stalks of
flowers and leafy branches representing almost all of the local flora of the chlorophyll and rhodopsin
families. She’d picked too many; she always did. Getting the gate open would probably require as much
skill as arranging the flowers—and no less art.
She smiled to herself, thinking it wasn’t often she’d been called upon to put her artistic talent to use
on a survey. The chance gave her satisfaction of a kind she’d never before experienced.
The ground under the flashgrass became uneven. Not having a very good view, she slowed, stepped
cautiously. The intermittent sight of the green boots brought a second smile to her face and shoulders.
Amid the flicker of the flashgrass they were wonderfully aesthetic. Perhaps she had been going about her
clothing incorrectly; perhaps it should be taken as a whole with its surroundings... What the Hellspark
wore was in some peculiar way more fashionable for Flashfever than her own carefully chosen garb.
Guessing that she’d neared her destination, she halted to shift her still-dripping burden enough to look
for the gate. To her relief, Timosie Megeve stood beside it, waiting to hold it open.
“Thank you, Megeve,” she began, then she peered again through the foliage. She was no expert on
Maldeneantine expressions, but he seemed agitated. “Is something wrong? You look like a womble
about to bite someone.”
Even as it left her mouth she realized he probably wouldn’t understand the expression, but before she
could explain he said, “Nothing’s wrong, Edge-of-Dark. At least, I hope not. Hellsparks are crazy, that’s
all.” He swung the gate wide and went on, “They took a sprookje along with them—in the
daisy-clipper!”
“Who else went?” And when Megeve told her, she smiled as broad a smile as was possible with her
arms full and, to reassure him, she added, “I wouldn’t worry. With Buntec piloting and a serendipitist
along, they can hardly get into trouble.”
Megeve started. “I hadn’t thought about the serendipitist. Do you believe in that sort of thing?”
“I believe in anything that works, I suppose.”
A stalk of penny-Jannisett fell from the crook of her arm; Megeve stooped to retrieve it. He held it
out to her, but realizing she had no free hand, he said, “Shall I carry some of that?”
“Just the one you’ve got. If I try to divide it up, I’ll drop it all, I’m afraid. It would be gracious of youto help me into my cabin.”
“Of course.” He swung the gate shut behind her and took a few quick paces to lead the way.
“What’s all that for?”
“You missed our brilliant idea,” she said, “Buntec’s brilliant idea, in truth, although she is gracious
enough to name me her collaborator...” And on the way to her cabin, she explained at length, her
enthusiasm growing still more as she spoke.
“I see,” he said, holding aside the membrane to her cabin to ease her entry. “It is a theory worth
exploring, I suppose.”
He watched as she maneuvered her plant cuttings onto a low table. When her hands were free, he
held out the sprig of penny-Jannisett. “Edge-of-Dark,” he said, “if you don’t mind my asking—why did
you start wearing boots all of a sudden?”
“Fashion,” she said, over her shoulder. “Although I admit I have had some further thoughts on that
subject.” Deciding those would be of little interest to him—the Maldeneantine had no aesthetic sense that
she had ever seen—she said simply, “You’ll find a digital picture on the table by the console. The
Hellspark tells me that’s current fashion.” She went from closet to closet to gather her working materials:
scissors, wire, bowls, and vases.
To her deep regret, she had not been able to bring as large a selection of containers with her as she
had wished. That was always true, but this time, the lack of choice frustrated her. Perhaps, she thought
suddenly, she might ask the rest of the team. Who knows what sort of container Om im or even Kejesli
might have brought, as art or as ritual item...
Megeve said, “I thought you were Vyrnwyn, Edge-of-Dark. Why should you be interested in
Ringsilver fashion?”
It took her a moment to understand the implication of his question. “Ringsilver?” She strode across to
stare at the picture over his shoulder. “Are you sure?”
“Of course I’m sure,” he said. “I was there just a few years ago and they all dressed like that.”
But Edge-of-Dark realized she had no need of his answer. Taking the picture from his hands, she
stared at it. “Ringsilver!” she breathed, and promptly burst into laughter, fully expecting Megeve to
follow suit. She looked up to find a scowl on his face.
Subduing her laughter, she raised her hand to splay fingers at her throat. “Your pardon, Megeve. It’s
not you I’m laughing at, it’s me. Won’t Om im love this! I have been—ever so graciously—tricked by
that Hellspark of his!”
She flourished the picture at him and went on, “What a great deal of trouble to go to, to get me into
boots for the sake of Buntec’s sensibilities! What is it Buntec would say, ‘Crazy like a Hellspark’? It’s
true isn’t it?”
“You mean she hoaxed you? And you’re not angry about it?”
The question sobered her. She gave it the respect it was due and concluded, “No, I’m not angry.
Consider for a moment: when I put on boots, I suddenly became human to Buntec. And Buntec
reciprocated”, she added, as the thought occurred to her, “by learning the Vyrnwyn formal greetings—so
she became human to me.”
Her long nails tapped the picture absently. “Almost like an equation. Edge-of-Dark plus boots equal
human. Buntecreih plus formal greeting equals human. What do you suppose we have to add to the
sprookjes to arrive at a similar result?” Thoughtful, she stared at Megeve without really seeing him.
“Well,” she said, “perhaps it’s flower art. If you’ll help me carry out the table, we’ll find out soon
enough.”
Megeve’s only response was to bend to the task.
Outside, they placed the table in the mud. While Megeve leveled it with small flat stones,
Edge-of-Dark settled herself on the bottom step, ignoring the damage the mud might do her clothes, and
began laying out her tools and her bowls.
A sprookje, perhaps the one that mimicked her, approached, its golden eyes widening at the sight of
the flowers. Although she admitted that might be nothing more than wishful thinking on her part, she
vowed to do her best for this audience of one.She began with the black lacquer bowl and reached for a stalk of tick-tick. As she brought the
cutting upright, it began to chide gently. “For sound,” she said happily, “I must choose them not just for
sight, but for sound!” The sprookje agreed, but caught up in the wonder of her new creation,
Edge-of-Dark scarcely heard the echo.
From the perch on which swift-Kalat had placed the arachne, Maggy had an unobstructed view of
the whole of the common room, including Kejesli’s face.
Hitoshi Dan waited for the small interested group to assemble and quiet, then he thrust Dyxte
ti-Amax forward in a manner Tocohl would have called “showing him off.” That might have been because
his 2nd skin was elaborately patterned, in reds and blues, to resemble the anatomical pattern of human
veins and arteries. Maggy, interested in defining Tocohl’s concept of “beautiful,” made a note to ask
later, when Tocohl was no longer occupied, if she thought this beautiful as she had Geremy Kantyka’s
patterned 2nd skin.
There was a good deal more of interest. Dyxte was ti-Tobian. Maggy had already opened a file on
another member of the team, Vielvoye ha-Somol, a ha-Tobian. Aside from a tourist guide which Tocohl
had told her to tag superficial, Maggy had no information on either culture, and here was a chance to
observe both.
Dyxte ti-Amax was also an expert at landscaping. The only thing Maggy knew about that subject
was that Hitoshi Dan had categorized it as a botanical art form, so she was glad for his explanation, as
brief as it was.
“Remember where we first met the sprookjes,” Dyxte asked, of no one in particular. “That entire
area could easily have been a deliberate artistic effort.” Maggy reviewed the tape she’d drawn from the
survey computer. She didn’t see what he meant, but then “artistic” gave her the same problem as
Tocohl’s “beautiful.”
Dyxte went on, “Of course, their idea of aesthetic may differ entirely from ours—from mine.” He
drew a stubby forefinger along the vein in his arm fron wrist to elbow. “Art comes from the heart, but the
heart is instructed by the culture.”
Now that information was useful, Maggy thought and tagged it accordingly.
“The best way to find out,” he said, “is to landscape an area of the compound.” He laid a hand over
his heart, tensed the muscles of his entire arm as if in reaction to pain. “I confess, I’ve been aching to do
just that since our arrival, so it will be as much to my benefit as to the sprookjes’.” Glancing at Kejesli, he
said, “I see no reason to wait.”
When Kejesli made no reply, Dyxte started for the door. Swift-Kalat paused to lift the arachne from
its perch and place it on the ground beside him so Maggy might observe too. They were not the only
ones to follow Dyxte outside; clearly a number of the surveyors were equally curious about this art form.
They watched in silence as Dyxte paced thoughtfully around the compound, stopping and turning at
several of the cabins, now taking a step back for better vantage point, now ignoring a cabin as if it did not
exist.
At last he paused contemplatively before layli-layli calulan’s cabin, watching the white and gold
pennants flap and spatter in the breeze. “Here,” he said, and strode toward the door chime.
“No,” said Maggy; simultaneously Kejesli—and his sprookje echo—shouted, “Wait, Dyxte!”
Their warning was the louder. Dyxte ti-Amax halted, turned on his heels to give Kejesli his full
attention. “Yes, Captain?” he and his sprookje said at once.
“You should have read your orders,” Kejesli said. The sprookje echoed that, causing Kejesli to cast
a scowl in its direction. “Layli-layli calulan is in mourning for Oloitokitok and can’t be disturbed
for”—he glanced at the display on his cuff—“another twenty hours unless it is a genuine emergency.”
“Oh. Your pardon, sir.” Dyxte turned to look once more at layli-layli calulan’s cabin. “Suppose I
simply go ahead and do it without her permission? Does anyone know enough about Yn to tell me—?”
Maggy sent the arachne to his side. “I might be able to help,” she said. “What would you like to
know about the Yn?”
With no warning, Dyxte dropped to his knee to peer into the arachne’s camera. “Extrapolativecomputer?” he asked; the question was as unexpected as his action had been.
Maggy considered her options. With swift-Kalat present she didn’t think a lie advisable; but Tocohl
had told her not to volunteer this information. Was answering a direct question volunteering? If she didn’t
answer, swift-Kalat would. He would not lie, not “too stupid to lie,” like the survey computer, but
definitely not programmed with the ability. Culture was like programming! She tagged that conclusion
important.
As for Dyxte’s question, Maggy decided to wait for Tocohl’s advice. Meanwhile, her best course
was to leave her options, and Tocohl’s, open...
“Hellspark,” she said, knowing Dyxte would perceive no delay between question and response, “at
your service.” She had the arachne bob a curtsey.
“How do Yn feel about plants?”
“Yn in general or Yn shamans?” she asked.
“Layli-layli calulan,” said Dyxte.
That made the responsibility for discrimination hers. She searched her files on the Yn with particular
attention to their shamans, to conclude, aloud, “There is an eighty percent chance that she will be very
pleased to find her cabin surrounded by plants when she leaves her mourning.”
“Maggy? Why an eighty percent chance?” swift Kalat asked, sprookje-echoed.
“Because layli-layli calulan is an atypical Yn shaman,” Maggy explained, “I can only extrapolate
from the general behavior of Yn shamans.”
“She does like the Flashfever wildlife,” Hitoshi Dan observed.
Kejesli gave a one-handed shrug. “Go ahead, Dyxte,” he and his sprookje said, “if that’s the area
that suits you. For safety’s sake, I’ll make that a order.”
“No need of that,” ti-Amax told him. “Eighty percent is high enough odds for any ti. Just let me get
some scratch paper and start my planning.”
A ruffled sprookje was either frightened or excited, perhaps both, thought Tocohl, as she kept a
careful eye on the passenger beside her. And Megeve’s sprookje was bristling all over, had been since
the daisy-clipper lifted gently up and began its voyage into the depths of the flashwood.
To Tocohl’s relief, however, the sprookje did not panic. It made no attempt to free itself from or to
struggle against the seat belt that restrained it. The creature looked out the door for a moment, then
turned deliberately away: the rushing ground beneath clearly made it more uncomfortable than did its
human companions. It watched Tocohl with huge unblinking golden eyes.
Tocohl touched it gently on the wrist and it looked down at her hand but did not draw away. Tocohl
stroked the ruffled feathers lightly, following the lay of the feathers, and hoping that the gesture might
bring some reassurance. Evidently the hope was fulfilled, for the feathers began to subside slowly—first
on the sprookje’s extremities, then on the chest, and, at last, those on the sprookje’s cheeks. Its fine gold
and brown brocade pattern sparkled in the intermittent sun as Buntec steered the daisy-clipper deftly
along the surface of a fast-moving stream.
Beyond the sprookje, Tocohl saw the surface of Flashfever unroll—a sudden shattering brightness of
frostwillows, a misty blue of monkswoodsmen, a squat stand of spit-outs—then the daisy-clipper dipped
beneath the shadow of a cloud and the woods flashed brilliantly alight with spitfires and whirligigs. Tocohl
blinked in wonder.
“Do you ever get used to it?” she breathed, to no one in particular. After a moment’s pause, Om im
tore his eyes away from his side of the daisy-clipper and said, “Hunh?”
“Never mind,” said Tocohl, “you’ve answered my question.” With a careful eye on the sprookje,
Tocohl reached forward to tap Buntec on the shoulder. “How far out are we going?”
Buntec didn’t take her eyes from the rushing view before her. “About twenty miles from here—take
us about the same number of minutes. Riding the river may be the long way, but the ride is smoother and
the view is better.”
“Thanks,” said Tocohl, “and thanks.” Without reluctance, she centered her attention once more on
the sprookje. “I am Tocohl Susumo,” she began again, first in Maldeneantine, then in GalLing’. “I dubthee Sunchild until such time as thou wilt share thy name with me.”
“Sunchild?” said Alfvaen.
Tocohl explained, “In the Zoveelian fairy tales, Sunchild was the bravest of all the sprookjes.”
“Sounds good to me,” said Buntec. “Fits.”
Om im leaned across Tocohl and saluted the sprookje. “Sunchild,” he acknowledged. “I wonder if
I’d have the courage to climb into a daisy-clipper with four crazies.”
“You did, didn’t you?” Tocohl said with a grin.
The stream had broadened into a roaring expanse of river before them. Sunlight glinted off its
churning waters, and along its torrent-swept banks, waterplants glittered and sparkled with a light that
was their own. Sunchild, as if in affirmation of its name, cautiously turned to the door to look down at its
brilliant world.
Buntec made a snapping motion at the control panel and grunted. Something in her tone raised the
hair at the nape of Tocohl’s neck. Two more snapping motions—then Buntec swung the daisy-clipper’s
joystick to the left. The craft did not respond.
“Won’t slow,” said Buntec sharply. “Won’t turn, either.”
Before them a stand of frostwillows rushed ominously closer. “Hold tight,” commanded Buntec, “this
is going to be rough!”
Alfvaen and Om im responded instantly, tucking their heads between their knees, sheltering in their
hands. Tocohl grasped the sprookje’s head, forced it down, sheltering it with her own shoulders.
The daisy-clipper slanted abruptly downward. For a long moment, Buntec fought it to a smooth
glide, then the craft struck the surface of the river with a thunderclap. It lurched against the current like a
skipping stone.
Tocohl gasped as the seat belt cut sharply into her flesh, and again as her shoulders smashed into the
seat before her. The sprookje shivered in her sheltering arms but made no outcry.
Twice more the craft lurched forward, battering her against the seat back. Then, with a final screech
of metal, it came to a shuddering halt amid a tangle of ripped and sparkling waterweeds.
“Abandon ship!” shouted Buntec. “I don’t know how long this thing’ll stay afloat—no! my side! You
can’t swim that white water, you barefoot fool!”
As Tocohl swiftly unsnapped the sprookje’s safety belt to shove it through the door to safety, Buntec
grabbed Alfvaen by the wrist and bodily drew her onto the bank of the river.
Om im scrambled across the seat. Tocohl, clinging to the frame of the door, fairly threw him onto the
slippery vegetation at the edge of the river. Buntec pulled Tocohl out of waist-deep water before Tocohl
had fully realized she was in it.
The daisy-clipper, its crumpled prow snarled in the glittering weeds at the edge of the river, rocked
lower and lower as water sprayed and pounded off its side. After a moment, it gave one last shudder and
sank. Only the arc of its bubble cabin, spattering reflected light, showed above the dashing waters.
Buntec stamped her foot on the springy flashgrass of the bank. “Foot,” she said in a matching torrent
of Jannisetti curses, “Heel. Sole. Toes, with green toenail polish!”
That last refinement owed much to Edge-of-Dark, Tocohl noted absently, but did much to assure her
that Buntec was unhurt. She looked around her. “Is everybody else okay?”
Alfvaen could not tear her stunned eyes from th daisy-clipper. “Alfvaen,” Tocohl said sharply, and
repeated the question in Siveyn.
“Yes,” came the muted reply.
Om im said, “Fine,” but there was a nasty-looking gash across his cheek. As he raised a hand to
touch his face, his eyes widened and the color drained from his cheeks. He clamped the hand to the gash
to stop the bleeding and sat down heavily.
Before Tocohl could reach his side, he gave her a wan imitation of his old grin and pointed instead to
the sprookje. Like Buntec and Alfvaen, Sunchild was still watching what was left to be seen of the daisy
clipper. It looked dazed.
Worried about possible shock—she had no idea what a sprookje’s metabolism was like—Tocohl
took a stumbling step toward it. She was a little dazed herself. Catching her balance, she began aMethven ritual to right that as she walked unsteadily over to the sprookje.
She stroked its feathers. “I’m sorry, Sunchild,” she said, keeping her voice as low and soothing as
possible, “it’s not supposed to do that.” Alfvaen giggled but Tocohl recognized the sound of relief rather
than hysteria.
Tocohl watched Sunchild closely. On the off chance that warmth would counter shock in sprookje as
in a human, Tocohl draped her cloak gently about the creature’s shoulders, clasping it at the feathered
throat. Sunchild’s feathers ruffled the entire length of its arms. Tocohl gently smoothed the agitated
feathers. After a long while, Sunchild seemed to become aware of her, then of the cloak. Its
cheek-feathers puffed.
Buntec, her cursing finished for the moment, attended to the gash on Om im’s cheek; Alfvaer
scratched nervously as she looked on. “... Along to pick us up in no time,” Buntec was saying, “I did
manage to punch the emergency locator before we hit water.”
“If the water didn’t get into the transmitter, that is,” said Om im.
“Right,” agreed Buntec, with a glance at what remained visible of the daisy-clipper. “Hold still,” she
added sharply.
Alfvaen frowned at the small object in her hand; Tocohl recognized it as the ornate box in which she
kept her medication. “I hope you’re wrong, Om im, because the water did get into my pills.” She held it
where he was able to see the result without twisting away from Buntec’s ministrations. “Nothing left but a
s-single s-soggy lump. I’ll just s-sit here and get drunk.” Her free hand clawed at her back. “It is s-safe
to sit here, isn’t it?”
“As safe a place as you’ll find on Flashfever,” said Om im, “as long as a storm doesn’t come up.”
Not for Alfvaen, Tocohl suddenly realized; without medication, the alcohol level in her bloodstream
could rise high enough to kill her. To Buntec, she said sharply, “How long before they come for us?”
“Twenty minutes, half an hour, depending on how long it takes them to pinpoint our location.”
Tocohl whistled impatiently. Maggy would have pinpointed their location the moment Buntec signaled
trouble with the daisy-clipper, but Alfvaen’s pills were another matter. (Maggy,) she said, (have
layli-layli calulan synthesize the proper medication for Alfvaen and send it with the rescue party.)
There was no response.
(Maggy!) she said again, this time with shocked urgency, (Maggy!) She left the sprookje’s side,
scrambling across the glittering grass to Alfvaen, to grab the hand-held from the Siveyn’s belt. “Maggy!”
she repeated aloud, “Maggy, what’s wrong?”
But there was no response from that link either.
Alfvaen stared at it and then at Tocohl with widened eyes.
Tocohl’s shoulders ached suddenly with remembered pain... the battering she had taken protecting
Sunchild. “I should have realized,” she said, “I took that much too hard!”
“Ish shan, are you all right?” All three of the stared at her with obvious concern.
She drew a deep breath. When she was sure she had her voice under control, she said, “I’m fine;
Maggy isn’t. Something’s happened to her.”
Shielding her eyes from the wan light of Flashfever’s sun—because Maggy did not,—Tocohl threw
her head back to search the sky, hoping for a glimpse, a flicker of light to tell her that Maggy was still
secure in her orbit. All she could see were gathering storm clouds.
For a time, swift-Kalat held the arachne where it might see both Kejesli’s face and Dyxte’s rapid
sketches of the front of layli-layli calulan’s cabin. But Dyxte was losing his audience to Edge-of-Dark.
When Kejesli and Dyxte’s sprookje wandered away Maggy said, “Do you think Tocohl would be
more interested in the sprookjes than in the landscaping?”
“I am,” swift-Kalat said, “I can’t speak for Tocohl.”
“That’s confirmation enough. I want to see what Edge-of-Dark is doing.”
Swift-Kalat set the arachne down and followed it.
Across the compound, at the foot of Edge-of-Dark’s cabin, all of the camp sprookjes jostled and
shuffled each other, not daring to displace the humans—to see the flower art.The intensity of their interest seemed to impress even Captain Kejesli, for he caught swift-Kalat’s
arm as he passed. Drawing him just out of echo-range Kejesli said, “In Veschke’s name, swift-Kalat,
show me an artifact or a language!”
“I showed you an artifact,” said swift-Kalat, with a snap of his forearm that made his bracelets clang
with such authority that a dozen or more sprookjes and surveyors started and turned at the sound of it.
For all the effect swift-Kalat’s status had however, the captain might as well have been deaf. Releasing
swift-Kalat’s arm, he turned and stalked away, his beaded hair chattering at each angry step, to vanish
into his quarters.
The crowd parted slightly to let Timosie Megeve ease his way from within. A wave of his pale hand
encompassed sprookjes and humans alike. “I don’t know what this proves,” he said, “except that they’re
as bored as we are.”
“Only that the sprookjes are interested in Edge-of-Dark’s flower art. It does not prove they have a
flower art of their own. Still, it means that we could be on the right track. It should be interesting to see
what the sprookjes make of Dyxte’s landscaping.” Smiling slightly, swift-Kalat craned past Megeve’s
shoulder to watch as Maggy’s arachne claimed Megeve’s spot on the steps... like a small child at a
parade, he thought.
Megeve snorted. “And who’s looking after the weather forecasting while Dyxte is messing around
with sketches? We’ve got people out in the field.”
“John the Smith. Meteorology is his—”
“Swift-Kalat!”
His head jerked up as he sought the source of the shout. The arachne stood at full stretch, a number
of surveyors peering at it curiously. From Maggy... ?
The arachne dashed down the steps and zigzagged through the crowd to splash to a stop before him.
In a surprisingly sharp voice, it said, “I can’t reach Tocohl. She doesn’t answer—neither does Alfvaen!”
“What?” Megeve gaped at the arachne in astonishment.
Swift-Kalat knelt in the mud. “Repeat that please, Maggy.”
“I can’t reach Tccohl or Alfvaen,” Maggy repeate The sharpness was no illusion. “Something’s
happened to them.”
“What was the last thing you saw?” asked Megeve quite calmly now.
“Nothing dangerous. No crash, if that’s what you mean. They were just flying along the river when
the picture, the sound, everything, cut off.” The anxious note was still in Maggy’s voice, but swift-Kalat
was reassured by her words.
“Swift-Kalat can tell you how much equipment failure we’ve had on this world,” Megeve said. “It’s
probably only something gone awry with your receiver. Would you like me to take a look?”
As he stooped and reached, the arachne took a measured step back. “No!” it said. “Nobody
touches my equipment except Tocohl!”
Megeve splayed a hand at his throat, rose. To swift-Kalat he said, “I wouldn’t worry. They’ve got
the transmitter in the daisy-clipper if it comes that.”
“You’re sure? With both Tocohl’s and Alfvaer communication broken off simultaneously?”
“That’s why I’m sure. Nothing could cut off both the same time, except a defect in that”—he
indicated the arachne with his toe, and it shied back a step farther—“or in the main part of the computer
Megeve finished.
“He’s wrong,” said the arachne. “I’m fine; Tocohl isn’t.”
“If it will reassure you, swift-Kalat,” Megeve said, “we can contact them on the main transmitter.”
“Yes, please,” said swift-Kalat, and Maggy sprookjelike, echoed the request. They trailed Megeve
swiftly across the compound to the common room, where he seated himself at the console and punched
the code for the daisy-clipper. After listening to the earpiece for a moment, Megeve took a flat case of
tools from the pocket of his oversuit and began to dismantle the transceiver.
“Something’s wrong with this one, too,” he said disgustedly—and glanced at the arachne as if it were
responsible. To swift-Kalat, he said, “And the fault is definitely with this equipment, swift-Kalat. Stop
worrying; as soon as I get this back together, you can talk to your precious Siveyn and Hellspark.”“How long will it take?”
“I have no idea, but I can work faster if nobody’s breathing down the back of my neck.”
“I don’t breathe,” said the arachne, unmoving.
“I meant you, too,” said Megeve, and Maggy’s arachne—after a bob and a “Your
pardon!”—followed swift-Kalat outside.
Swift-Kalat had intended to return to the sprookjes still crowded about Edge-of-Dark, but he found
he could not keep his mind on them. The arachne, instead of resuming its place on the steps, stayed at his
side. From that vantage point, it could see nothing but the backs of various legs.
Curious at her sudden apparent loss of interest, swift-Kalat wondered if Megeve might not be right.
Perhaps some failure in the arachne... ? “What are you doing, Maggy?” he asked.
“Thinking.”
“Thinking about what?”
Tocohl’s voice issued abruptly from the arachne’s vocoder. “Once a thing happens twice, you must
think about it three times.” Then Maggy added in her own voice, “This has happened three times, to
three separate communication devices. If Tocohl does not return in twenty minutes, we must search for
her.”
“Why twenty minutes?”
“Contact was broken when the daisy-clipper was twenty minutes from camp. Tocohl will discover
the loss of contact and return here to see if anything is wrong with me. If she does not, she cannot.”
“Are you sure of that?” swift-Kalat said. Not only could the question be asked in GalLing’ without
giving offense but often the question had to be asked in GalLing’.
Maggy answered it by repeating all of her previous statements in Jenji, assigning a degree of reliability
to each. “If she does not, she cannot,” Maggy finished; the arachne raised one spindly appendage and
snapped it down.
The gesture was awkward—the arachne’s joints were unsuited for it—but it gave the ring of
authority to her words. There was nothing wrong with a computer capable of such reliability of speech.
Swift-Kalat stared thoughtfully at the reflection of his own status bracelets in the puddle of water at
his feet. “I see,” he said, and then fell silent. This had happened three times, she had said, assigning the
statement to her own experience. In his experience...
Four times. In his mind’s eye, he watched as Timosie Megeve repaired the transceiver—not now,
not today—but on the day Oloitokitok disappeared. There had been no locator signal, no emergency
signal. Oloitokitok, so convinced of the sprookje’s sentience, had died. What then killed Oloitokitok?
To categorize two separate events as one as he did, perhaps improperly, would lead to deductions
that, if false, were dangerous to speak. Yet he could not close his mind to the implications of the theory.
He had to find a way to test it.
“Come, Maggy. We will not wait the twenty minutes,” he said, and without waiting for a reply he
started toward Kejesli’s cabin.
They found Kejesli playing a somewhat reluctant host to Dyxte, who was saying, “But the
penny-Jannisett that grows in the local flashwood isn’t large enough. Now, if I take one of the
daisy-clippers out into deep flashwood—it’s only about a fifteen-minute trip—I can get the perfect
plants!”
Beaded hair rattled in agitation against the backrest of Kejesli’s chair. “No,” he said, “you’ll wait until
Megeve has fixed the main transceiver. It’s bad enough we’ve got one party out there unable to call in; I
won’t send a second one out until the communications are restored.”
“The deep flashwood you’re referring to, Dyxte,” interjected swift-Kalat, “is that where we first
found the sprookjes?”
“Yes,” said Dyxte and swift-Kalat turned to the captain.
“That’s where Tocohl’s party was headed. Maggy”—he gestured at the arachne—“lost contact with
them, too. She believes something has happened to the party. We must take one of the daisy-clippers
and find them.”
“Absolutely not.” Again, the rattle of beads. Kejesli stubbornly gripped the edge of his desk.“Megeve tells me there’s something wrong with that—thing. You know we have had a great deal of
trouble with our probes, swift-Kalat, and that one has been poking around outside during a
thunderstorm. I’m not surprised it’s behaving oddly. You, and Dyxte, will wait until Megeve has the
transceiver fixed.” His gesture was one swift-Kalat had learned to interpret not only as final, but as a
dismissal as well.
Swift-Kalat tried to frame the thoughts that concerned him into words, but found himself unable to do
so without creating so frightening a truth that—with a shudder, he turned and left.
“Whew,” said Dyxte when they were out of the captain’s earshot. “He wasn’t this bad on Inumaru! I
wonder what’s biting him?” Swift-Kalat darted a look at Dyxte, who added, “An expression, I didn’t
mean it literally.—Is it true that the arachne has lost touch with Buntec and the others?”
“Yes,” said swift-Kalat. He wanted to say more; the possibility of misspeaking prevented him from
doing so. The pennants that hung from layli-layli calulan’s cabin, dry for this brief moment, snapped
gaily, caught his eye, and suggested a possible course of action.
Tocohl and layli-layli calulan had calmly called each other liars. A mistranslation only, Tocohl had
assured him, yet if the Yn term included the Jenjin meaning of the word, then perhaps a shaman had the
ability to deflect the danger of misspoken words.
It was the only course of action open to him. He would try speaking to layli-layli calulan.
A hand caught his elbow, brought him up short. “She’s in mourning, remember?” Dyxte said, having
read his intent correctly. “We aren’t supposed to interrupt her.” Releasing his elbow, Dyxte thrust his
hand straight into the air, as if to protect himself from some overhead threat. “Unless it’s an emergency?”
he said; the sudden concern in his voice made the gesture seem one of emphasis—or fear.
Swift-Kalat stared at him. Even that was beyond his power to voice. To claim an emergency might
be to doom the party. Alfvaen! “I cannot say that.”
“You look it,” Dyxte said.
“Swift-Kalat...” The arachne tapped at his calf delicately. “Will it help Tocohl if you speak to
layli-layli calulan?”
“It might, Maggy. I don’t know what else to try.”
“Then I will arrange it.”
“Wait!” said Dyxte. “You can’t go in there either.”
“Your pardon for the correction, Dyxte ti-Amax,” Maggy said, “but your ignorance is no fault of your
own. You were chamfered by a moron; everybody says so. I’m female. As such, I have sufficient status
to call on layli-layli calulan even in her time of mourning.”
With that, the arachne darted off, sending up a shower of mud all the way across the compound, to
stretch for the chimes at the entrance of layli-layli calulan’s cabin. A moment later, it disappeared
inside.
As she paused on the threshold of layli-layli calulan’s cabin, Maggy checked it all through once
more. She fervently wished that she had someone, Tocohl or even Alfvaen, to discuss it with—especially
after Megeve’s remarks about her sanity—but she could see no flaw in the plan.
She could not speak to swift-Kalat about it, that she knew. Swift-Kalat could not lie, Tocohl had
said, and her memory banks backed up that statement; but she, Maggy, could lie for him. Hadn’t Tocohl
told her to lie to Kejesli? And didn’t layli-layli calulan approve of lying?
Once again, she ran swiftly through all of her stores related to the Yn culture. The plan was eighty
percent good. Given that layli-layli calulan was an atypical Yn shaman, that was the highest rating she
could give it. If it would put her into contact with Tocohl again, it was worth the risk. Maggy had never
before been out of touch with Tocohl for this long.
Maggy struck the chime.
Layli-layli calulan did not answer, but having made the decision, Maggy could not turn back. She
pulled the membrane aside and stepped quietly into the cabin, relieved that she could find nothing in her
data stores to indicate that a shaman’s curse would work on a mechanical device.
As at the time Tocohl had first visited the cabin, the Yn shaman sat cross-legged on her blue mat.
This time, however, a dozen jievnal sticks, set at precise intervals—precise for a human, Maggycorrected—around her burned dully. Layli-layli’s plump hands flashed and wove with the intricacies of
the koli thread. Maggy read her moving lips without difficulty:
“... This for the love of woman to man”—her hands wove another knot—“this for the love of woman
to woman”—still another, the thread shortened and twisted—“and this,” she said, grasping the two tiny
ends of thread that protruded from the glittering tangle, “is for death.” She gave a slow, steady pull and all
the apparent knots in the koli thread came inexorably unraveled, to leave nothing but the straight gold
string to link her hands.
It was the death-song of a woman for a beloved sister whom she named Oloitokitok.
When layli-layli calulan looked up at last, Maggy raised two of the arachne’s appendages and held
them out before her. Despite its limited likeness to the human gesture, she hoped layli-layli would
understand she meant to extend sympathy. Then, drawing the ritual words from a tragic drama of Yn
origin, she said, “I must speak of one whose life is intertwined with mine. Let the dead be dead, and grant
them the peace of tuli-tuli the beast.”
The dark eyes were calm, the broad mouth turned up at the corners despite her mourning. Layli-layli
calulan said, “I will speak to the living. What is the problem, maggy-maggy lynn?”
“Will you, layli-layli calulan, permit my sister to speak with you?”
“Who is your sister?”
“Her name is swift-Kalat twis Jalakat, and I claim her as my sister by right of the Hellspark ritual of
change.” There was Maggy’s lie; no such ritual existed. She waited, observing layli-layli calulan
carefully for change of manner that might suggest anger, a common reaction to an uncovered lie, or
amusement, a less frequent but a possible response. While Maggy had seen both, she was not sure how
either would appear in an Yn; she kept her Yn files active.
There was a pause, one that Tocohl would have categorized as thoughtful. Then layli-layli calulan
said, “And her true name?”
The lie had passed for truth.
“Her true name is hers alone, not mine to give.” That part was not a lie, thought Maggy—perhaps she
should be amused—because she had heard swift-Kalat’s soft-name but doubted that anyone in the
survey knew it besides Alfvaen.
“You share the name of strength, maggy-maggy lynn. I will speak to your sister. Bring her to me.”
Once more, Maggy held out the arachne’s two appendages in sympathy. Then she sent it back to
swift-Kalat at a run—for the second part of her lie.
Dyxte was still with him. Maggy checked through all the examples of lying she had and found Tocohl
saying, “The fewer the witnesses the better...” It had been said with a smile, but Maggy knew Tocohl’s
smile did not always negate information. Besides, she had little enough to go on. “Please leave us,
Dyxte,” she said, “I must speak to swift-Kalat alone.”
The look they gave the arachne she interpreted as puzzlement, but Dyxte said, “I’ll speak to you
later,” and walked away. Swift-Kalat bent to listen.
Now, thought Maggy, something to satisfy swift-Kalat. “Hold out your hands,” she instructed, and
when he did, she placed two of the arachne’s muddy extremities in them. “Please say the following after
me...”
Recalling a tape of a Ringsilver magician who, in Tocohl’s frame of reference, could change a
hard-boiled egg into a live bird, Maggy checked it through. Even knowing it to be illusion, Tocohl had
been enchanted. Maggy hoped Tocohl would be as delighted with this illusion, so she said aloud, “Hey,
presto!”
“Hey, presto!” repeated swift-Kalat. “What did layli-layli calulan say, Maggy, and what’s this all
about?” He wiped his muddy hands on his thighs.
“That was the Hellspark ritual of change,” said Maggy, “that makes you, legally speaking, a woman.
And layli-layli calulan is willing to speak to you now because you are my sister.”
“Is that possible?” swift-Kalat squinted at the arachne.
“I just did it,” said Maggy, in a tone she’d heard Tocohl use many a time for a fait accompli. “Come
talk to layli-layli.”The arachne led the way to layli-layli calulan’s cabin and entered behind him. The shaman looked
up from her magically patterned mat and said, “You dream, swift-Kalat.”
Swift-Kalat obviously recalled the exchange he’d seen between Tocohl and the shaman. He replied,
“As do you.”
“Be seated, sister and sister of my sister,” the shaman said. Maggy folded the arachne’s legs and
placed the body on the floor where she could keep her camera eye on both; swift-Kalat knelt on his
heels. “Now speak,” layli-layli calulan said.
With great care, swift-Kalat chose the words to tell layli-layli calulan what had happened to
Tocohl’s party. To Maggy, it sounded completely reliable, even in GalLing’, but it was not enough to
explain swift-Kalat’s sudden decision to begin the search immediately.
Layli-layli listened without comment until he had finished. After a moment’s pause, she looked at him
closely and said, “That is not all. If that were all, you would have waited the twenty minutes Maggy
specified.” When he didn’t speak, she added, “Tell me what else is happening in the camp, or has
happened, or will happen. You will not judge, but perhaps I will.”
There was a brief jangle of bracelets. Without a downward glance, swift-Kalat jammed them to his
elbow to silence them. “Oloitokitok died while Megeve repaired the transceiver. Megeve repairs the
transceiver now, and Maggy cannot speak to Tocohl or to Alfvaen.”
Maggy added that bit of information to stores and ran extrapolation on it. The results wouldn’t have
appealed to Tocohl, and they did not appeal to Maggy on the same grounds.
“I hear what you will not say,” said layli-layli calulan.
Sweat beaded swift-Kalat’s forehead. “Can you speak it without adding to the risk?”
“I can.” Layli-layli calulan twisted the bluestone ring from her right forefinger. “With this hand, I
will.” She rose from the mat in a single smooth motion, gathered up the jievnal sticks. A rumble of distant
thunder made her face suddenly passionate. “Hurry! We may already be too late!”
She sprang for the door and darted across the compound at so light and quick a pace that even
swift-Kalat found it hard to match.
By the time the arachne caught up with them in Kejesli’s quarters, Maggy found layli-layli calulan in
the midst of an elaborate lie. Like Maggy, layli-layli took advantage of Kejesli’s lack of knowledge of
her culture, a lack Maggy did not share.
Nothing new in technique, Maggy noted, but she recorded it for her growing file on lying, for its style
and for its purpose, which she hoped might become clear.
“The gods Hibok Hibok and Juffure,” layli-layli was saying, “have sworn vengeance against our
enterprise. The jievnal sticks”—she thrust them, smoking, before Kejesli’s face—“tell me that only a
red-haired woman can prevent disaster to us all.”
Y herself was the only Yn god, and the jievnal sticks were not used for divination. Swift-Kalat was
no likelier than Kejesli to be aware of that but for her to speak of disaster...! Didn’t layli-layli calulan
know what effect that would have on a Jenji? Maggy searched the files, looking for a way to mitigate the
damage, as, gasping, swift-Kalat flinched from layli-layli calulan to cradle his braceleted forearm as if
he were in pain. Nothing, Maggy could find no precedent—
Hearing the gasp, layli-layli calulan glanced his way. Still holding the jievnal sticks inches from
Kejesli’s face, she stretched out her bare right hand to clasp it about swift-Kalat’s wrist. “I speak a
dream, swift-Kalat,” she told him, in a tone that commanded. “A dream can be turned.”
Whether her words or her espabilities did the trick, Maggy couldn’t tell, but swift-Kalat took a deep
breath and said, “Do what you must.”
Without releasing swift-Kalat’s wrist, layli-layli calulan fixed her eyes once again on Kejesli. In the
same tone of command, she said, “I invoke taboo.”
Now Maggy understood the purpose of the lie. Only by claiming a taboo situation could layli-layli
calulan force Kejesli to an action he had forbidden.
Kejesli, coughing from the smoke, braced a hand against the ceiling. “What is it you need?”
“A daisy-clipper and permission to take it out despite the broken transceiver,” layli-layli calulan
said, “nothing more.”Kejesli lowered his hand; halfway down, it bobbed once in a Sheveschkem shrug. Relief, Maggy
decided, as Kejesli crossed to his computer console. He tapped a code and, even before a face
appeared, demanded, “John, what are the weather conditions?”
It was Dyxte’s face that sprang into view. “Captain, that storm is going to be nasty—and we can’t
reach Buntec’s party to let them know it’s coming because—”
“The transceiver is out of commission,” Kejesli finished for him. “Could we send a party in person?”
“If we do, they’ll be caught in full storm on their return.” Dyxte scrubbed his forehead as if to erase
the deep lines etched above his brows.”We’ve got about ten minutes before the storm hits camp.”
Kejesli broke contact without a further word and turned, his hand still clamped to the console’s edge.
“I can’t let you go, layli-layli, no matter what your gods say.” Shoulders gone taut, he stared past her.
“They’ll be all right, you know. Buntec will ground the daisy-clipper. As long as they stay inside, they’ll
be fine.”
Layli-layli calulan held up a forefinger bare of its ring. “They had better be,” she said quietly. On the
roof above, rain began to drum.
Chapter Twelve
YOU COULDN’T SEE the ship from here anyway,” Buntec said, “much less tell if anything were
wrong.”
Tocohl brought her hand down, saw that it was shaking. “If it were dark...”
“You said geosynchronous orbit. Even in the dark, you couldn’t get a glimmer. Besides, this is
Flashfever—stupid planet doesn’t believe in dark any more than the Port of Delights.” She thrust out a
hand. “Let me have a look at that thing.”
Tocohl passed her Maggy’s hand-held. Buntec laid it on her knee while she prodded pockets;
eventually she found what she had been looking for, some small implement adequate to open the back of
the device. She glanced up in mid-examination to say, “Don’t worry, Hellspark, I don’t have the faintest
intention of going after your implant. That could have been damaged when we got battered around. Are
you hurtin’ from it?”
“No,” Tocohl said, rubbing the area. There was nothing to feel, neither bruise nor swelling. (Maggy?)
she said; there was still no response. “No,” Tocohl said again, “I lost contact with her before the
crash—I’m sure of it. She’s very protective. If she’d been in contact, I wouldn’t have bruises.”
Buntec snapped the hand-held shut. “Nothing wrong I can see but even locking’s a bitch without the
right tools. Well, even if the problem’s at the source, the ship’s in geosynchronous orbit...”
Meaning, Tocohl thought, the ship will be fine. She bit down hard on her anger, said only, “You
mean a problem at the source can be repaired... yes. But, Buntec—the result might no longer be
Maggy.”
Her urgency was lost on Buntec. Tocohl should have expected as much. Buntec hadn’t had the hour
by hour contact with Maggy she’d had for so many years. Unable to explain, Tocohl lapsed into
silence—and shivered at the depth of that silence.
Out of habit born of precaution, she manually ran spectacles and 2nd skin through their test modes:
warmth, yes; heightened vision, yes; infrared vision, yes.
It was thornproof still and tougher in fact than the 2nd skins the rest of the party wore. The sensors
on its surface made tiny patterns against the skin of her back and shoulders. Buntec was up and pacing,
she realized, and realized as well that she could not have interpreted the patterns without hearing
Buntec’s actions.
(Oh, Maggy,) she said and then, into the silence, in Sheveschkem she added, (Veschke guide thee!)
She took a deep breath. First things first: that meant Alfvaen. It might well be necessary to walk back to
base camp. If only she had some better idea of Alfvaen’s condition.
Alfvaen scratched furiously.
A sharp curse from Om im distracted them both, jerked them about to face the river.A surge of water swelled the river and Buntec jumped back to avoid it. The swell splashed noisily
against the hull of the daisy-clipper. With a hideous squeal of metal, the craft tore loose from the bank
and rushed downstream like some ponderous underwater beast. Buntec howled with rage and stamped
her foot obscenely after it.
Startled, the sprookje jumped to its feet and backed a dozen steps so quickly that the moss cloak
closed, as if protectively, about it. Then its head snapped from the swollen river to the horizon. Its
feathers bristled. Its beak jerked open, revealing a tongue that glowed the ominous red of a warning
telltale.
It turned its head slowly and carefully, as if to display the tongue at its human companions. Like a
deer flagging the white of its tail for danger, Tocohl thought. When Sunchild looked again to the horizon,
she looked too.
“Eight-footed and bare-toed.” That was the first understandable thing Buntec had said. “We’ll have
to follow that eight-footed—follow the daisy-clipper,” Buntec went on, “Old Rattlebrain’ll never find us
unless we’re sittin’ right on top of it, twiddlin’ our toes.”
There was a grunt of firm assent from Om im, a “Yes” from Alfvaen.
But Tocohl did not take her eyes from the lowering line of sooty black clouds that moved toward
them. More than the dramatic loss of the daisy-clipper, the approaching storm frightened the sprookje.
And with good cause.
Now the thunderheads crackled with light. Tocohl shook off the hand Buntec laid on her arm,
ignored Buntec’s query—to count softly to herself.
At the low rumble of thunder, Tocohl turned to the rest and said, “There won’t be a rescue team.
That’s over the camp now, and it’s headed our direction. Rattle-brained Kejesli may be, but not
rattle-brained enough to send anybody out in that.”
Buntec said, “We’re dead then, Hellspark. We might have made it in the daisy-clipper but—” Like
reflected lightning, fear brightened her eyes; her voice was flat.
“John the Smith!” said Om im suddenly. Buntec stamped her foot at his apparent irrelevancy, but he
went on, “Ish shan, John the Smith said to look for a stand of lightning rods. Theoretically we’d be safe in
a stand of lightning rods.”
“Theoretically,” said Buntec; she stamped her foot again.
“Unless you’ve got a better idea,” Om im told her.
A flash of warning red caught Tocohl’s eye, brought the sprookje to her notice. She saw that it had
walked some twenty feet in the direction of the flashwood. Now it stopped—facing them, displaying its
tongue.
Feathers puffed with fear, it retraced its steps until it stood a pace or two from Tocohl. Again it
displayed its tongue. Then it held out the edge of the moss cloak to her.
“Where do they go in thunderstorms?” Tocohl demanded suddenly. “Somewhere safe!”
“There’s your better idea, Buntec,” Om im said.
“Yes-s,” Alfvaen agreed.
Buntec grumbled, “Better than sittin’ in the wide open waitin’ to be fried.”
“Follow the native guide then,” Tocohl said; she took the proffered edge of the cloak. The sprookje
closed its mouth, turned about, and set off across the flashfield at a trot, the rest of the party close
behind.
Within a few yards, Tocohl loosed the end of the moss cloak. Sunchild cast a brief glance backward
to assure itself they were still following, then plunged on. A sudden gust of wind whipped alight the
flashgrass, surrounding them with patterns now made ominous. With it came the first spatter of rain. The
sprookje quickened its pace.
Thunder rumbled ever closer.
Flashfield gave way to flashwood. On its outer fringes stubby chuckling and ticking curiosities
competed unsuccessfully with the sound of thunder. A head-high frostwillow, tossed by ever-stronger
rain-laden gusts, shattered the air with the sound of a thousand crystal glasses breaking simultaneously.
Om im shouted over it, “Heads up, Ish shan. Some of these plants are as deadly as the lightning we hopeto avoid.”
The sprookje glanced back again and, displaying a red tongue, made a wide path around a slender
tree, notable only because it seemed pronged rather than branched.
Om im grasped Tocohl’s wrist. “Eilo’s-kiss,” he supplied, “that’s one of the nasty ones. Remember
what it looks like—even the little ones can stun a human. The big ones can kill. Sprookjes too, it would
seem.” Before releasing her wrist he hopped a step forward to precede her. “Blade right,” he said, then
added without turning, “Buntec, you’d better guide Alfvaen.”
Buntec reached for Alfvaen’s hand. Tocohl acknowledged Om im’s blade right with a raised and
curled hand. She knew he saw neither—his attention was fully on the path the sprookje broke.
They were headed away from the river, but Tocohl knew she’d be able to locate it again. Even
without Maggy’s assistance—Tocohl shivered at the harshness of the thought—Tocohl had a good sense
of direction. Assuming they survived the storm, they could follow the river back to camp.
The thunder was closer now, and the sprookje quickened its step still faster. Om im, whose shorter
legs needed three steps to her two, was forced into a run but did not appear wearied by the pace.
Glancing up, the small man said, “Practice,” and grinned as if he’d read her mind.
The sprookje wove through a thick wall of arabesque vine. Tocohl, following close behind Om im
and the sprookje, did not look up until she had negotiated the fine but wiry barrier. “Lots of zap-mes,”
Om im warned as she freed Alfvaen and herself from the last of the tangle. They spent the next few
moments avoiding a lashing from zap-mes of every conceivable size.
At last, the zap-mes seemed to subside to ankle-height new growth and Tocohl looked ahead. She
drew in her breath involuntarily. Before her was the embodiment of the “blasted forest” of so many
Zoveelian fairy tales.
Gaunt black spikes, trees unrelieved by branch or leaf, jabbed high into the blackened sky. Beneath
them, and for a short distance beyond, nothing grew—but here and there the remains of something that
looked charred. This was the stand of lightning rods. The sprookje stood before them, welcoming.
Well, she thought wryly, it’s a suitable setting for a sprookje, I suppose. I hope it’s as suitable for
humans. The rain had turned earnest.
To Om im, she said aloud, “Looks like the native guide had the same idea you did. We didn’t see
this from the daisy-clipper. Sunchild must know the territory very well.”
The sprookje picked its way warily into the recesses of the lightning rod stand.
“You couldn’t have chosen a better guide,” Om im said. “Now watch where I put my feet. Whatever
energy the lightning rods don’t need, they bleed off into the ground—once in a while there’s a surface
node that can give you a dangerous shock.”
In cautious silence, the party continued its way to the heart of the shelter, where the sprookje sat
waiting. Tocohl and the others followed suit as a gust of wind dashed water in their faces.
“Too bad we haven’t time to build a shelter,” Tocohl said.
“Oh, well, can’t have everything,” said Om im.
“Why not?” said Tocohl, and drew a grin from him.
Buntec settled Alfvaen, then herself. Curling up on her side, she threw an arm over her head and
announced, “Nothing to do but sleep.”
Lightning struck the tallest spikes of their shelter with an ear-splitting crack that brought Buntec bolt
upright, staring wide-eyed and openmouthed. When the sound died away and their numbed ears could
once again hear the shattering of frostwillows in the distance, Buntec said grimly, “Sleep, my foot.”
Tocohl blinked away red Catherine wheels, turned her face into the rain to clear her eyes of the
stinging tears the lightning flash had startled from them. The air seemed too full of rain to breathe, but she
did not draw up the mask of her 2nd skin. It would only serve to remind her how much Maggy would
have enjoyed this experience; few had ever sat amid lightning and lived to tell the tale.
She twisted her hair into a single mass to channel the water down her back, pocketed her spectacles.
Enhanced vision was the last thing she needed at the moment, she thought, squeezing her eyes tight
against a bolt of lightning so intense that even through closed lids it reddened her sight.
Wind tore through the empty spaces between the lightning rods, flinging leaves and bits of branch atthem. Here and there a wet leaf struck one of the nodes Om im had warned her of—struck and struck
sparks.
Thunder deafened and deadened their ears until they could no longer distinguish a silence from the
thunder in their heads.
And through it all the sprookje, wrapped tightly in the moss cloak, left its own afterimage in Tocohl’s
eyes: a ghostly glowing image of regal unconcern.
After an eternity, the storm passed on...
Alfvaen, exhausted, had fallen into a fitful sleep. Wordlessly, for the words might not have been heard
through still-ringing ears, the others agreed to rest; the run to shelter had exhausted their bodies, but the
storm had exhausted their spirits as well. Buntec jerked in violent dreams.
Without knowing she had fallen asleep, Tocohl started awake at a tickling touch—Sunchild stroking
her wrist. “I’m okay,” she told it and was surprised to find that she could hear her own voice.
“That’s good to hear, Ish shan,” Om im said. He pounded the heel of his hand beside his ear, setting
earpips a-jingle, and added, grinning, “In more ways than one.”
“I know exactly what you mean,” Tocohl assured him. Then she looked again at the sprookje. “From
the way Sunchild’s acting, I think it’s safe to leave now. We’ll have to make some decisions.”
“Before we wake the others,” he began—but a jerk of his thumb specified Alfvaen.
“You’re worried about that scratching,” Tocohl said. “So am I. Even if it’s just a reaction to stress, I
want to get her to layli-layli calulan as soon as possible.”
“Yes,” said Om im and woke the others gently. Buntec stretched luxuriously. “I think I’m alive,” she
said, taking obvious pleasure in the sound of each syllable. Alfvaen came to with a gasping sound, held
her head.
“How do you feel?” Tocohl asked.
Alfvaen turned her head from side to side, gingerly testing the result. “Strange,” she said, “so strange.
Giddy, and”—she stood cautiously, as if unsure of the ground beneath her feet—“uncoordinated. I—”
She took a deep breath and stared at Tocohl.
There had been no trace of slur in her speech.
“I can hear it,” Tocohl said. “You should be drunker but you’re not.”
“I’m scared,” said Alfvaen—and her lapse into Siveyn told Tocohl the depth of that fear.
Buntec had been speaking with Om im in low tones, now she said, “We haven’t much choice but to
walk home. Who knows where the daisy-clipper is by now, or if the locator is working. I say we head
for the river and foot it,” she finished, scowling deeply. It was agreed by all, right down to the obscenity.
Placing an arm around Alfvaen, Tocohl steadied her and said, in Siveyn, “My oath: that I will return
you safely to swift-Kalat.” She was rewarded by a lessening of fear in the sea-green eyes and she
squeezed the Siveyn’s arm. “Shall we go?” Tocohl offered her arm formally, and just as formally, Alfvaen
accepted it.
The survey camp crackled with fears and rumors. The repaired transceiver had brought no response
from Buntec’s party and the storm, bringing forced inaction, heightened tension until it was as tangible as
the stench of ozone. Layli-layli calulan kept her own counsel, but Maggy noted that the doctor scarcely
let Timosie Megeve out of her sight. When she did, swift-Kalat followed after the Maldeneantine.
Maggy set the arachne following layli-layli calulan. When at last she reached the privacy of the
empty infirmary, Maggy tilted the arachne, angling the camera up to observe the shaman’s
expression—an important part of any reply, Tocohl had told her—and asked, “Why should you and
swift-Kalat find observing Megeve of such importance? I will help if I know what to look for. What
dream have you?”
Sure she had phrased her query properly in Yn, Maggy found it surprising—yes, that was the term
Tocohl would have used for a response so unexpected—when in response layli-layli calulan twisted
her ring.
“Your pardon,” Maggy said instantly. At the same time, noting shifting priorities, she drew the
arachne back a few steps. Without it, she could know nothing of what had happened to Tocohl. Shedidn’t wish to be impolite but she did want it out of range.
“No,” said layli-layli calulan, reinforcing the word with the sharp upward jerk of the chin that said
the same in Yn. “No dream. A nightmare rather.”
Maggy waited, hoping she would explain. She wasn’t sure it was safe to ask for further information,
not the way layli-layli continued to twist her ring. Maggy moved the arachne a few steps farther away.
“Your pardon, maggy-maggy”—the shaman looked down at her hands—“I didn’t mean to frighten
you.”
“Does that mean a shaman’s curse would work on a mechanical device?”
“I don’t know. I never had occasion to try. But it surely wouldn’t work on you. I don’t know your
name.”
“That’s good,” said Maggy, and she stepped the arachne back to its original position. “Then would
you please explain what you meant about a nightmare? I want to help, but I can’t when I don’t
understand.”
Again layli-layli calulan surprised her, this time with a sudden smile, the first Maggy had ever
recorded in her presence. Maggy suspected Tocohl would have termed it quite beautiful.
“How good is your Yn, maggy-maggy, or do you only understand specific phrases?”
“I have three Yn dictionaries, two grammars, and a library of fiction from which to draw analogy. I
can puzzle out much.”
“Then I have a task, one that only you can perform.”
“Only me?”
“You are able to speak to Tocohl through an implant, just here.” A beringed finger tapped the
analogous spot below layli-layli’s ear.
“But I can’t contact her!”
“Not yet. However, the moment your contact is restored, I want you to deliver the following message
verbatim to Tocohl and to no one else. Do you understand?”
“For Tocohl only,” said the arachne, “so noted and tagged. There will be no unauthorized retrieval of
this information.” Lest that be insufficient, she added, less formally, “Don’t worry, layli-layli, I’m good at
keeping secrets. Tocohl taught me how.”
Layli-layli calulan smiled a second time; this time the smile vanished as quickly as it had come. Face
stern, finger touching ring, she spoke three sentences in soft, rhythmic Yn. Then she added a fourth in
GalLing’: “Remember, maggy-maggy, a textbook translation is not always an accurate translation.”
But Maggy had already begun an exhaustive matching, not only of individual words with her
dictionary stores, but of whole phrases with their context against her stores of Yn writings, fictional and
nonfictional.
The result in Hellspark, the language Maggy knew best, was one sentence long: “Megeve may be
responsible for the equipment failures.”
A mistranslation? Maggy doubted it, so she sought information beyond her language banks for
corroboration.
Item: swift-Kalat telling layli-layli calulan, “Oloitokitok died while Megeve repaired the
transceiver.”
Item: layli-layli responding instantly to this information, wishing to search immediately for Tocohl,
willing to lie to do so.
No, thought Maggy, there was nothing wrong with her translation. But she had insufficient data to
calculate the probability.
“Take me with you when you search for Tocohl.” said the arachne; and layli-layli calulan said, “I
was planning to.”
The two rejoined the knot of surveyors that crowded about Dyxte in the common room, awaiting the
primary meteorologist’s pronouncements anxiously. Matching Megeve’s expressions with what she had
on tape in reference to Maldeneantine, Maggy concluded that he was abnormally nervous. In the light of
the rest, John the Smith, Dyxte, swift-Kalat, Maggy found this inconclusive.
She let part of her system go on with its extrapolations even as swift-Kalat turned to her and said,“Where was the daisy-clipper when you lost communication with Tocohl?”
Maggy touched one of the arachne’s appendages to the map on display. “There. I can be more
specific if you have a more detailed map.”
“No need,” said Dyxte, “the storm has passed well beyond that.” Swift-Kalat had already started for
the door. Dyxte called out, “You’ve got about three hours, swift-Kalat, before that next storm hits. I’ll
keep you posted!”
“All right,” said Kejesli, “start a standard search pattern for that daisy-clipper from that point
outward. You know your positions. Do it now!”
Layli-layli calulan raised her hands, the gesture evocative of that she had used in claiming a taboo
situation. “Shuffle the pattern to cover us,” she said, “swift-Kalat and I intend to let maggy-maggy lead
us. Our pattern, as such, may be erratic.”
Kejesli moved to object, but layli-layli calulan raised her hands a fraction higher and he said
merely, “Take Megeve with you. I’ll want a full report on the causes of the equipment failure.”
The shaman’s smile broadened. “Of course.”
Moments later, a dozen daisy-clippers raced across the landscape of Flashfever toward the spot
Maggy had indicated. “Buntec followed the river,” said the arachne, from its precarious perch on
layli-layli calulan’s knees.
“Yes,” said swift-Kalat, who was at the controls of the daisy-clipper with Megeve at his left, “but this
is quicker. We’ll go straight to the spot you last saw them.”
Megeve twisted about in his seat to peer at layli-layli calulan. “Do you really think that machine’s
trustworthy?” he said, jabbing a finger in the direction of the arachne. “It seems to me—”
Anxious to keep the arachne out of his clutches, Maggy flinched it deep into layli-layli Catalan’s
lap.
The shaman raised a protective arm between them. “It seems to me,” she said, “that this machine is at
least as trustworthy as your equipment.”
“Still,” said Megeve, “I’d feel better if you’d let me check out its circuits.”
“I’d feel better if you’d keep your eye on the landscape,” layli-layli calulan told him, and Maggy
could hear the sharpness in her voice. “We are looking for a lost party. This time we intend to find them
before they come to harm.”
“Yes, of course,” said Megeve. He turned his attention back to the search.
Free from the threat of his reaching fingers, Maggy telescoped the arachne’s legs to give her, once
more, a view of the passing scenery. The craft swerved to bypass a noisy grove of frostwillows—the
wind had not yet died down—and layli-layli calulan grasped the arachne to keep it from tumbling from
her lap.
Without warning, Maggy—high above the world of Flashfever—lost all contact with the arachne. It
was as unexpected and as total as the initial loss of contact with Tocohl had been.
Now she had no way at all of helping to find Tocohl!
True, she had other mobiles, but none was sufficiently adept to handle a skiff. She tried reactivating
the mobile that accompanied layli-layli, swift-Kalat, and Megeve at fifteen-second intervals; and, on the
third try, she succeeded. Reassured, she ran a check, long-distance, on the mobile’s circuitry and found
everything in good working order. She began a read-through of all available literature to find out what
might account for the arachne’s lapse.
“I told you there was something wrong with it,” Megeve was saying. “Now perhaps you’ll believe
me. Flashfever has been ruining all our equipment. Let me have a look.”
Overlapping him, layli-layli calulan asked, “Are you all right? The arachne just suddenly collapsed.”
She held it firmly, well beyond Megeve’s reach.
“I’m all right,” Maggy said, produced a chuckle that seemed to reassure the shaman, and added, “it’s
the arachne I’m worried about. I can’t account for the lapse.”
“We’re here,” said swift-Kalat, and the arachne craned toward the door and looked down.
“Just a few feet upriver is the spot I lost their transmissions,” it said. Within a few moments, all of the
daisy-clippers were poised above the churning stream. A flock of golden scoffers screamed at them all inoutrage.
Swift-Kalat coded to receive emergency transmissions and spoke into his throat mike, but said to the
rest, “Nothing but the search parties answering.” The other five daisy-clippers flashed away in the sunlight
to weave a search pattern along either side of the river.
“You said we would not follow the search pattern, layli-layli” said swift-Kalat. “What would you
have me do?”
“Maggy-maggy?”
Maggy decided that layli-layli calulan meant for her to answer swift-Kalat’s query. “Continue to
follow the river as Buntec did,” she said, all too conscious of Megeve’s muttered objection and of the
arachne’s inexplicable lapse.
“Slowly,” added layli-layli calulan, and the craft glided forward, barely skimming the roaring
waters.
They followed the river for another forty minutes, the silence within the daisy-clipper a sharp contrast
to the clamor of their surroundings, until swift-Kalat said, “This is where Buntec habitually turned to cross
land.”
“When she went to look for wild sprookjes?”
“That’s right, Maggy. Shall I follow her usual route?”
“Yes, please. Perhaps we will at least find evidence that they reached their intended destination.”
“Don’t count on it,” said layli-layli. “The rains will have washed away most of the traces any party
leaves.”
“Not a grounded daisy-clipper,” said swift-Kalat. “We have something large to look for,” and he
guided the daisy-clipper into the blazing cacophony of the flashwood.
The sprookje was barely visible as a dark patch moving through the undergrowth some twenty yards
ahead. Sunchild seemed to have grasped the notion that they were headed back to the river and had
taken the lead. As long as the sprookje was willing (and seemed to be tending in the right direction)
Tocohl accepted the counsel; as Om im pointed out, it was more adept at spotting Flashfever hazards
than the rest of them.
Om im led the human contingent, Tocohl and Alfvaen abreast of each other, with Buntec bringing up
the rear. Om im and Buntec watched for hazards, Tocohl watched Alfvaen. They were traveling all too
slowly for taste, but it couldn’t be helped. The undergrowth was stubborn, dense.
Her eyes teared from the constant dazzle. Her ears she was sure, had not yet stopped ringing from
their assault by thunder. But she knew that the ringing was nothing more than background noise,
Flashfever-standard. She ducked to follow Om im beneath a clamoring frostwillow.
Beyond was a thick stand of tick-ticks entangled in arabesque vines. Om im eased through—there
were certain advantages to his size, Tocohl noted with envy. She leaned her weight against the nearest, to
press them aside for Alfvaen. Sharp pain stung her, hip to ankle, and she jumped forward and spun.
Nothing but a zap-me, she saw, hidden within the tick-ticks. “Just bruises,” Tocohl said. “Serves me right
for not looking. Could have been worse.” She put her back to the task again, before the zap-me could
reset its tendrils.
Alfvaen clicked her tongue, chiding Tocohl in imitation of the plant before she fought her way
through. Her darting glance was bright, too bright to be set in a face as pale as that, thought Tocohl,
following.
On the far side, Alfvaen paused to scrape furiously at her back, tangling the damp blue fringe of her
bodice; then she pushed doggedly on. A swarm of vikries, dislodged from stalks of tick-ticks, followed
briefly along beside her. When she bent to drink water from the up-cupped leaf of a green handplant,
they scattered away.
(That thirst of hers worries me too.) Tocohl had subvocalized the observation. Then, despite the lack
of response, she went on doing it. (Maybe you can hear me, Maggy, and I just can’t hear you... ) Please
let it be so, she thought fervently.
(But I could use your expertise right now. Alfvaen no longer slurs her words. I’d say she wassobering—and the excessive thirst is a symptom of alcohol dehydration—but her behavior isn’t right for
sober either. Om im agrees, and he’s seen her normal behavior. She looks like someone riding an oxygen
high: too exhilarated for sense.
(And she’s exhausted. We all are. But her exhilaration isn’t Flashfever effect. I mean, not the same as
the ionization high.)
The fallen trunk of something like a tree lay across their intended path. The sprookje waited atop it.
“We’re coming,” Om im told the creature. The words had no noticeable effect on Sunchild but the
moment he began to clamber up, the sprookje vanished over the side. Tocohl made a knee for him, then
waited in the same position, expecting Alfvaen to use the same step up.
Instead, Alfvaen glowered at her, took a running jump, and made the top. She overbalanced, toppled
with a crash and an alarming series of squeals. “Alfvaen’s okay,” came Om im’s voice, “she just landed
in a pig thicket.”
Tocohl hauled herself up, balanced on her chest, to see for herself. Bright-eyed and grumbling curses
in Siveyn, Alfvaen stood methodically kicking at the edge of a waist-high clump of silver blue; each time
she did, it let out another series of squeals. (Plant or no, that sounds like la’ista,) Tocohl said privately.
(We must get her to layli-layli calulan! Soon!)
Tocohl reached down a hand to pull Buntec up. Together, the two of them slid over the obstructing
trunk.
To this side was a small clearing, bright with penny-Jannisett and monkswoodsmen. Sunchild waited
impatiently for the rest of them to regroup, and then set out again.
Tocohl held up her hand. “I think we all need a rest,” she said. Om im glanced at Alfvaen, who was
still tormenting the pig thicket, and sat down, pulling Buntec with him. “Alfvaen,” he said, “sit down. You
need the rest as much as I do.”
The sprookje fixed its golden eyes on them, then stared up into the sky. Its cheek-feathers puffed.
Openmouthed, warning tongue displayed, it again held out the edge of the moss cloak to Tocohl, who
said, “No rest for the weary. Get up, everybody, and let’s hope it isn’t far to the next shelter.”
“Someplace without a pig thicket, I hope,” Buntec grumbled. She tapped Alfvaen’s shoulder none
too gently.
Startled, Alfvaen spun to face her. “Your pardon?” she said, for all the world as if returned to normal
behavior. “We’re going on? Why?”
“Because Sunchild gave a very expressive look at the sky,” Om im told her.
“Oh,” she said, forgetting the pig thicket to join him.
Once again the party set off to follow the sprookje to safety, but not before Tocohl and Om im had
taken the opportunity to exchange worried glances.
The route brought them to a swollen stream that, no doubt, channeled into the river they sought. A
happy chirring filled the air, grew louder and louder, until it almost drowned the sound of rushing water.
“Drunken dabblers,” Buntec shouted, over it all, “sound like good times and parties. Nothin’ but plants.”
Alfvaen glared at the two of them over her shoulder. A moment later, she let slip a branch which
snapped and dashed a spate of cold water in Tocohl’s face. The Siveyn had begun to speak softly to
herself in her own tongue, so softly that Tocohl could not make out the words, not even when they had
left the patch of drunken dabblers well behind.
They came to a bend in the stream and the ground rose slowly beneath their feet. Where the water
tumbled swift about a bare and broken jut of rocks, tall weeds spaced themselves at neat intervals,
flicking with the rapids. “Om im?” Tocohl pointed to them,
“Wave power,” he said. “What you see in the water is a runner from these,” he flicked a finger, in
passing, at a black stem topped by a dull red gleam. “Any excess energy is bled off as light. They only
glow like that when the stream is swollen—which means most of the time,” he finished, with a wry smile.
“Wonderful!” said Tocohl, but Alfvaen growled in Siveyn, “Deathlight.” She scratched and scowled,
whether at the itch or the plants, Tocohl couldn’t tell.
The sprookje turned opposite to the stream’s bend to lead them back into deep flashwood. The
ground rose steeply, so steeply in fact that they followed the sprookje’s example and, whenever possible,drew themselves hand-over-hand along the arabesque and leatherstrap vines.
The wind picked up once again and with it the flashwood’s noise. By the time they had clawed their
way to the top of the small escarpment, rain was falling from the darkened sky. The sprookje had a fine
sense of timing, there was no doubt about that. Tocohl, pausing while they caught their breaths from the
climb, counted. The storm would be overhead in minutes. Already the lightning was close enough that she
could scent burning vegetation on the wind after each strike.
To her right, low brush bent in the wind, aflame with its own internal light. Only a hundred yards
beyond rose a second stand of lightning rods. The sprookje had already gone on. (It’s sure we’ll follow
this time,) Tocohl commented, bit her lip at the silence that drowned out all of Flashfever’s noise.
Follow they did, to the edge of the blackened patch that marked the stand’s perimeter. Deep inside,
the sprookje sat, waiting patiently for them to join it in shelter.
“Watch the ground and follow my lead,” Buntec said; she reached to take Alfvaen’s hand.
“No!” Alfvaen thrust her away with such force that Buntec stumbled two steps, three steps back—
Buntec caught her balance just in time to avoid falling onto the hazardous ground beneath the lightning
rods. Cursing, she started for the smaller woman, this time with worried caution.
Alfvaen cut harshly through the Jannisetti curses: “No, Buntec. You must not interfere. I will not let
this so-called Tocohl Susumo harm us further.”
Buntec froze in her tracks, cast a swift glance of bewilderment at Tocohl, then a look of deep
concern at Alfvaen.
“Alfvaen?” said Om im. “What are you talking about?”
“I speak of”—Alfvaen sought the word in GalLing’, spoke it bitterly—“sorcery.” And with fury rising
in her eyes, she swung to face Tocohl.
Tocohl had seen that intensity, that ferocity only once before—on the face of a Siveyn about to issue
a death challenge. Lightning ripped the air, striking the lightning rods, to give livid illumination to Alfvaen’s
anger.
“Don’t touch her!” Tocohl snapped to the others. “Get into the lightning rods and stay there!”
“She’s hallucinating!” Om im shouted over the dying thunder.
“I know what she sees,” Tocohl said sharply. “Now get back.”
Alfvaen stepped toward Tocohl, fringe clinging sullenly as she raised her arm across her chest with
stiff singleness of purpose.
A crawl of sensors along her arm told Tocohl that Om im was moving forward, not back, to blade
right. Without taking her attention from Alfvaen, she shot a single word at him in Bluesippan, ordered him
back by virtue of his blade service. The crawl of sensors told her he had obeyed.
“So, Haining Lefven!” The Siveyn took a second deliberate step forward, her green eyes fixed
unwaveringly on Tocohl, and spoke again in cold, harsh tones, sharpening her native language to a
gleaming point. “You spoke and spoke again, and each time your words were heard by earless folk. You
danced before the sightless and they watched your every move. You drew sweet words from the
speechless. But now there will be an end to magics—I, Tingling Alfvaen, offer you the justice of death.”
Veschke’s sparks, thought Tocohl, wouldn’t Maggy love this! Straight out of the fictions the two of
them were reading en route! And Tocohl chose her reply from the same fictions.
Stretching both arms before her, in the manner of a sleepwalker, Tocohl began, placatingly, “The sun
shines on us both...” She turned her palms up. “The wind blows us both the scent of sea-jeme and sediji.
The earth pulses beneath our feet its rush of life. I, Susumo Tocohl, have no quarrel with another child of
Siveyn.”
She let her hands fall to her sides, slowly, slowly.
But Tingling Alfvaen did not lower her hands. Rigid still with the anger of her own imaginings, she
said, “This is your only choice, child of no one.”
Tocohl said, “Alfvaen, it’s me: Tocohl. You and I have no quarrel—at least none that won’t wait until
the storm passes. I vote the two of us get in out of the lightning before somebody gets fried.”
“And I for life,” said Alfvaen.
Veschke guide me, thought Tocohl, she’s hearing the proper responses! What do I do now?Lightning struck the lightning rods again; in its violent illumination, as she fought for vision, Tocohl saw
Alfvaen blink. She’s hearing the responses, but she saw the lightning flash! Maybe, just maybe, she’s
seeing what’s real!
And with infinite slowness, Tocohl turned her back to Tinling Alfvaen. She blanked Flashfever from
her consciousness, with all of its noises and glittering lights, and focused all of her attention on the sensors
in her 2nd skin, which gave gross but tangible indication of Alfvaen’s position.
To turn your back on a challenge was a strong indication of guiltlessness, but a death-challenge might
not be so easily turned away. The challenger’s desire to duel may outweigh the social structures.
Tocohl’s back crawled. Sweating, she waited out a heartbeat, then stepped aside—Alfvaen went
headlong into the arabesque vines in front of which Tocohl had just been standing. Alfvaen turned swiftly.
Tocohl caught her by the kilt, heaved upward. The move misplaced a kick aimed at her heart; Tocohl
took it stingingly in midchest, gasped, and kneed Alfvaen in the belly.
Not hard enough. Alfvaen, though twisted with pain, jabbed stiffened fingers sharply into Tocohl’s
side.
Still grasping the kilt, Tocohl swung Alfvaen bodily to the right, into a medium-sized zap-me. As the
plant lashed with its several whips, Tocohl dropped the woman and threw a knuckle-blow, hard, at
Alfvaen’s temple. Alfvaen fell unconscious.
Tocohl dropped to her knees beside Alfvaen’s still form. Her breath came in great rasps, aching
through the injury to her side.
Had Alfvaen’s reflexes not been crippled by her illness, or had Tocohl not taken advantage of
Flashfever’s traps—Tocohl shivered and blessed the pain in her side that confirmed her continued
existence. Lightning and roaring thunder brought her to her senses. Rain poured down, drenching her.
Buntec pulled her gently erect, then bent to maneuver Alfvaen onto her broad shoulder. Om im
slashed strips of arabesque vine. “We’ll have to tie her up,” he explained, “we can’t risk that a second
time.”
Trailing vines, he led Tocohl to shelter, walking as slowly as she. She saw his concern, realized that
she clutched at her side. “I think it’s only a bruised rib. We’ll find out when we get back to camp.” She
eased herself down on the blasted heath; letting the rain spill into her face, she began the quieting litany of
Methven ritual against the counterpoint of thunder.
Buntec lay the unconscious Alfvaen beside her. Om im bound the Siveyn hand and foot.
“A waking nightmare,” Tocohl said at last. “She didn’t know what she was doing. Somehow, for
some reason, she has sobered and her brain has a desperate need, waking or sleeping, for dreams.”
“Then, if she sleeps, she’ll be all right?” Buntec asked.
“She should be,” Tocohl said, “once she’s made up the lost dreaming.”
“She may not,” said Om im, a bitter tone to his voice that Tocohl had not previously heard. He
brushed the blue fringe aside and laid a gentle hand on Alfvaen’s back. “Look, Ish shan.”
A flash of lightning showed them what he saw. Beneath the transparent 2nd skin, thin gray veining
patterned Alfvaen’s skin.
“Garbage plants,” said Buntec, and her face paled. “No!”
Rain spattered them, but could not wash away the horror.
The storm had come up faster than expected, and Kejesli had ordered the searchers to return to
base camp. They had worse news to report: the daisy-clipper had been found, some miles downriver.
Save for a school of Flashfever streampuppies, it was empty.
“It crashed in the water and they had to abandon it,” Edge-of-Dark leaned forward in her chair,
poised as if to leap at them.
Kejesli tapped the display: “Here.”
“No bodies?” asked layli-layli calulan.
“No bodies,” said Edge-of-Dark. “They may have been washed away by the storm surge.”
With unexpected force, John the Smith slapped the tabletop and said, “No.” Then, more calmly, he
added, “Suppose they got out safely—assume they did! A storm was coming up. Where would you gofor shelter? Would you stay in the middle of a flashgrass savannah? Of course not, you’d head for the
nearest lightning rod grove.”
Kejesli grasped the straw. “Yes! Veschke light my way, that’s just what I’d do! We’ll check every
grove of lightning rods that can be seen from the abandoned daisy-clipper!”
“We’ll have to start farther back along the river,” said layli-layli. “The daisy-clipper was found here,
but Buntec would have turned across land here”—she indicated the spot—“so the daisy-clipper must
have been washed downstream—and we have no idea how far downstream.”
“Your pardon, layli-layli calulan.”
“Yes, Maggy?”
“May I be permitted to display my tapes? They may be of some service.”
“Do it,” said Kejesli, somewhat abruptly; and Maggy inserted the arachne’s adaptor into the console.
A moment later, images of swift-Kalat’s trip downriver flashed past, then froze, giving Kejesli an almost
sickening jolt. The image expanded to show a detailed portion of the riverbank. The flashgrass was
strewn with uprooted, tattered waterweeds, with chunks of mud so large they had not yet been washed
away by the heavy rains... all thrown onto the bank as if by great force.
Kejesli turned to swift-Kalat, who stared at the display hopefully. “Is there any animal that might
make a mark like that?”
Swift-Kalat said, “Not to my knowledge. Maggy, can you pinpoint that spot on the map?”
The map reappeared instantly, but it was quite obviously Maggy’s and not the display map. It was
neatly marked with a series of bright red arrows, labeled, WRECKAGE OF DAISY-CLIPPER,
BUNTEC’S TURN-OFF, and POINT OF IMPACT?
“Shall I enter it into your computer?” said the arachne.
Kejesli looked to the others for advice. Megeve said, “That machine collapsed twice on us. It is
without doubt faulty, perhaps dangerously so. I strongly advise against relying on it.”
But layli-layli calulan seemed to speak for the rest, and her verdict was “Yes.”
Like every other Hellspark Kejesli had ever met, this one had no respect for authority either. Maggy
did not wait for his permission but went about her task on layli-layli’s word alone.
When layli-layli calulan withdrew to her cabin to wait out the storm, Maggy sent the arachne after
her. Remembering her manners, she paused it on the threshold, tilting to observe layli-layli. “You may
join me, maggy-maggy lynn,” she said, “but I prefer silence for some moments. I must think.”
“Me too,” Maggy assured her. She stopped the arachne just inside the door, squatted it; she did not
want it to drip on the shaman’s ritual mat.
She had run the most extensive diagnostic available to her, first on the arachne, then on her own
hardware. She had found nothing physically wrong with either. Yet she had lost contact with the arachne
twice—the second time, shorter than the first, as they raced the thunderstorm to camp.
Spurred by the message layli-layli calulan had given her for Tocohl only, she moved on to the
possibility of sabotage, despite its low probability.
The search was a long one. She had very little in the way of files on sabotage per se. Making a note
in her active file to stock up as soon as possible, for future reference, she moved on to the only other
source available: fiction.
And there she found references to a number of devices that had the characteristics she sought. Each
would jam not only an implanted transceiver but also the aural-visual transceivers that were critical
components of hand-held, 2nd skin, and arachne.
She settled down to a closer examination of each. The first was the best match but she found she had
appended to the story Tocohl’s comment: “Oh, he lies!!! I don’t believe a word of it, Maggy. He didn’t
do his homework.” So, one could lie in the context of fiction. Well, that also confirmed that one could tell
the truth in the context of fiction. She went on.
The next two were the inventions of cultures that Maggy could find no nonfiction reference to; in fact,
as described, both the jamming devices contravened the laws of physics. Very unlikely, Maggy
concluded, and tagged each with a comment of her own, the rude noise Tocohl had so appreciated.
Moving on to the next, she found a description and sketchy explanation of a device called an Hayashijammer. The explanation was plausible, and the characteristics given were a very good match with those
she sought. Did this author lie?
She cross-matched to nonfiction. There was indeed an Hayashi culture and much of what the fiction
implied about it the nonfiction confirmed. So perhaps the Hayashi jammer exists, she decided. Now
what?
Buntec had assumed that all the sprookjes’ artifacts were biological, so she had suggested they look
for grafts. And she had suggested that the sprookjes might have biological art as well. Tocohl had been
willing to act on the assumption.
Right, Maggy told herself, sounding much like Tocohl to her inner ear, assume the Hayashi jammer
exists. What follows logically?
Where is it? Not in camp, or she wouldn’t have contact with the arachne at all. Not with the
daisy-clipper. They had passed within the stated range of the device on their trip back to camp and the
arachne had not collapsed.
With Tocohl! What if it were with Tocohl?
She hastily called up the map she had displayed for Kejesli. On it she plotted those points at which
she had lost and regained contact with the arachne.
The range fitted. The second lapse had been of shorter duration but tallied nicely with the greater
speed of their return trip. Two points do not make a graph, but she reached a 0.05 probability that
Tocohl herself carried the Hayashi jammer and was moving upstream toward the camp.
Was there any way to confirm such a thing? Had it been done? Could it be done? No one on the
survey team was Hayashi but—the team’s records might show something...
Layli-layli calulan was lighting a jievnal stick. Maggy hesitated to interrupt her thinking, but with
Tocohl and three others in danger, she could not ignore even so low a probability.
She brought the arachne to its feet. “Please, I’m sorry to interrupt, but may I use your console?” It
was not polite, she knew, but she had already started the arachne toward it.
“Yes,” said layli-layli calulan, after what seemed to Maggy a very long pause.
Maggy thrust the arachne’s adaptor into the console and searched for the team’s personnel files. She
met stubborn refusal. “As bad as Kejesli,” she said. The computer was obviously keyed to hold certain
information for authorized personnel only. She could break the coding but it would take time.
“What are you looking for, maggy-maggy?”
The question startled her, too much of her attention had been on the coding to notice that layli-layli
calulan had moved to watch the display. Breaking codes on other computers was technically illegal.
Maggy had no idea what layli-layli would think of it, so instead of volunteering any information, she
answered the question as strictly as it had been phrased:
“Further information about Timosie Megeve.”
Without comment, layli-layli calulan touched the keyboard, then spoke aloud, “Layli-layli calulan
.” The records were obviously voice and fingerprint keyed. “It will oblige you now, Maggy.”
“Thank you.” The arachne touched the keyboard and watched the display, slowing it to human speed
so that she did not offend layli-layli. The details of Megeve’s training and employment inched by and
probability took a jump upward: Megeve had trained in electronics on Hayashi.
“Will you help me to act on a probability of point-oh-six?”
“A hunch?” Layli-layli calulan knelt to look directly into the camera eye. “Yes—if you’ll answer a
question for me. Fair trade, Hellspark?”
“Fair trade,” Maggy told her.
“Is there really a Hellspark ritual of change? The truth, maggy-maggy, in exchange for my help.”
Now Maggy understood why Tocohl considered trading an art. That was a question she had not
anticipated. She couldn’t lie, having declared a fair trade; yet to admit that she had lied... Layli-layli
calulan had lied too—and there Maggy found a possible solution to the dilemma.
“There is now,” Maggy said firmly. “The gods Hibok Hibok and Juffure have so decreed it.”
Layli-layli calulan gave a shout of laughter. When she at last caught her breath, she said, “Now, tell
me about your hunch, maggy-maggy.”Thunder jolted Tocohl to consciousness with a convulsive jerk that sent a searing pain through her
side. She gasped and pressed a hand to the pain’s source, pushed herself to a sitting position with her
free hand. The pain did not ease. The moss cloak whipped about her. Grateful for the distraction, she
tucked its edges firmly beneath her thighs.
Moss cloak? she wondered suddenly, fingering it.
“The sprookje returned it while you were passed out,” Om im shouted over the roar of rain. “Maybe
Sunchild thinks the cloak is for injuries.”
Tocohl glanced at the crumpled Alfvaen, whose face Buntec sheltered from the rain with Om im’s
cape. Why not Alfvaen then? Om im said something further.
“What?”
“I said,” he repeated, “how are you feeling, Ish shan?”
“Blunt, rusted, nicked, and burred,” she said. “Aside from that, I’ve never been better. How long
have I been out?”
“Ten, twenty minutes. You’re not holding us up, the storm is.” He reached toward her, pried her
crimped hand from her side. When she tried to resist, he said, “Ish shan, I’ve done enough fieldwork to
have earned a degree in emergency medicine. It’s blade right, and you know it.”
She did; she let him probe the injury, gasping once or twice despite his gentleness. At last he sat back
on his haunches. She could not see his expression but his tone was anything but relieved: “It could be
broken, Ish shan. You can’t travel: you risk puncturing a lung.”
“Are you volunteering to carry me?” She kept her tone light but it was sufficient to silence any further
warnings.
Lightning flickered, small stuff this time, but enough to light his face and let Tocohl see the depth of his
concern for her. “Talk to me,” she said, “I could use the distraction.”
He gave her the Bluesippan thumbs-up yes and moved closer, just as another massive bolt of
lightning struck not five feet above their heads. A great sheet of light enveloped them; at the periphery of
their blasted heath, zap-mes lashed into action. An acrid smell, like that of burning insulation, assaulted
Tocohl’s nose.
She gave a wan grin at Om im and observed, “Smells like one or two of those zap-mes
overestimated their current requirements.”
Om im batted at his ear. He had clearly not heard her over the still-reverberating thunder.
Not willing to risk repeating that, Tocohl shouted, “I thought you said thunderstorms were a time for
talking.”
Om im shouted back, “Only for sprookjes!”—and pointed.
Twisting to look renewed the stabbing pain. More cautiously, Tocohl moved her whole body to face
in the direction he’d pointed.
Two sprookjes kept each other company; their luminescent feathers, streaming with rain, shone in the
gloom. Eerie light rippled along their bodies as the wind ruffled and twisted their feathers into pattern after
pattern after pattern.
“To your left is Timosie’s—Sunchild,” Om im said next to her ear, “on the right is van Zoveel’s.”
Tocohl scrutinized the two. At last she said, “I don’t know how you do it, Om im. I can’t distinguish
any difference in feather patterns—certainly not in this light!”
He was close enough that she could hear his chuckle over the sizzle of rain. “It’s a gift, Ish shan—and
it has nothing to do with the patterns on their feathers.” He waited out a boom of thunder, then continued,
“The same way I knew you were Hellspark.”
“I assumed Buntec had told you—” Tocohl thought back. Om im had not been among those that
Buntec had notified on their arrival. “How—?”
Om im spread his hands. “I honestly don’t know. I’ve often been accused of an unusual espability
but it can’t be that because I can’t tell the sprookjes apart if I can’t see them.”
“I’d hardly call this seeing,” Tocohl said. A crackle of lightning whited out the gloom, in no way
belying her comment. When she felt she could be heard again, she added in a shout, “Or think seeing thekey component when you’re seeing them from across the compound.”
“There you’re wrong, Ish shan. Across the compound or across the playground—I’ve been able to
do this, whatever this is, since I was a kid.”
He wiped rain from his face and went on, “I knew a pair of identical twins, only they weren’t
identical, not to me. Drove me crazy because everybody thought they were when I could see so clearly
they weren’t—even across the playground. When I objected, strongly, to everyone who called them
identical, and proved that I could tell them apart even at a distance, I got run through the whole bank of
espability tests as a reward.”
Again he chuckled. “I shouldn’t say that. What I got for reward was two of the best friends I’ve ever
had. They were so pleased to have found someone who never once mistook the one for the other...”
“I think I can understand that. Even identical twins are different to themselves. Their voices are
different: one hears the other through air but himself through bone conduction. The difference between
how your voice normally sounds to you and how it sounds from a tape.”
“I think it had more to do with my outrage. By my blade, Ish shan, to this day I still do not
understand how anyone can mistake the two—they move differently. I could see it across a room! Why
can’t anyone else?”
A flash of lightning, this time farther away, lit and broke the scene into eerie segments like the flash of
a strobe. Tocohl saw Megeve’s sprookje shift and twist, saw its larynx bob. Neither sprookje blinked or
winced at the ferocity of the skies.
She saw, in fact, something she should have seen on the tapes, had she not been distracted by the
crests and yokes of the so-called wild sprookjes. Had she not been so stupid!
With a wrench that shot pain the length of her body, she turned to face Om im. Tears sprang to her
eyes, to be washed away by the torrential rain. “Om im!” she said, gasping it past the pain in her chest.
“The wild sprookjes! They don’t have a larynx! She tried to explain but found she had no breath to do
so.
Alarmed, Om im caught her shoulders and firmly eased her to the ground. “Tell me later, Ish shan.
Lie down. Lie down or, by my blade, I’ll tie you down!”
She had no strength to fight and enough sense to obey. She let him stretch her out. Vaguely, she
remembered the pain of a second probing of her injury, then she slipped into a fitful doze...
When she awoke, the storm had passed. She tried to rise, found her way barred by a firm but
comforting arm across her shoulders.
At the sign of movement, Om im withdrew his arm. He brought his face into her line of sight, raised a
gilded brow inquiringly.
“You can’t handle me and Alfvaen,” she said.
“I know. But—carefully, Ish shan.”
She turned up her thumb in agreement. With his assistance, she eased herself to her feet. A trial step
told her she could walk but that too would have to be carefully done or she would know the pain of it.
Alfvaen slept on; Buntec hoisted the Siveyn onto her shoulders as if she weighed little or nothing.
Sunchild—Tocohl saw that van Zoveel’s sprookje had departed—but Megeve’s had remained
behind. To guide them. There was no longer any doubt of that in her mind. Sunchild rose too, ready to
travel.
Following her glance, Om im said, “Now what was all that about sprookjes and larynxes?”
Tocohl smiled wanly and said, “Want to see a trick?” At his frown, she added, “That’s not a non
sequitur, I assure you. Want to see a trick you taught me?”
Without awaiting his answer, she fixed it firmly in her mind that this was Megeve’s sprookje and that,
like Megeve, it too spoke Maldeneantine. Walking gingerly, she approached the sprookje as she would
have any Maldeneantine and, in that language, she said, “My name is Tocohl Susumo, and I greet you
with a full heart.”
Word for word, the sprookje echoed her.
This time, layli-layli calulan’s search party consisted only of Maggy’s arachne and swift-Kalat, afeat layli-layli had achieved by waving her “divination” sticks at Captain Kejesli and invoking no less
than eight fictitious gods. John the Smith had been asked, in confidence, to remain behind and keep a
watch on Megeve. There, layli-layli had simply invoked friendship.
While the other search parties followed up John the Smith’s suggestion, that of checking every
lightning rod stand in the immediate area of the downed daisy-clipper, swift-Kalat guided his upstream.
From the shaman’s capacious lap, the arachne recorded the passing flashwood.
When the arachne’s joints buckled, layli-layli calulan said, “Stop here. If Maggy is right about the
jammer and the range of it, Buntec’s party is within a mile of us.”
Swift-Kalat let the craft hover as he marked the spot on his computer-generated map, then guided
the craft into the woods while layli-layli calulan watched the forest below.
They zigzagged for fifteen minutes, marking the spot each time Maggy’s arachne came to or lapsed
from consciousness, then they headed straight for the rough center of the disturbance.
“There!” Layli-layli calulan stabbed a beringed finger at a shimmering grove of frostwillows.
Swift-Kalat brought the daisy-clipper to an abrupt halt, swung it at right angles to its previous
position, to peer in the direction indicated. “Only a sprookje,” he said, not pausing to wonder at the
disappointment where a day ago there would have been joy.
Then Buntec stepped from the cover of a monkswoodsman. Howling a greeting, she waved
awkwardly from beneath her burden. Joy came and went, as swift-Kalat recognized that her burden was
Alfvaen... bound and unconscious!
Even before he could react and ground the daisy-clipper, layli-layli calulan had laid Maggy’s
arachne aside to transform the seat behind her into its emergency-bed mode. As the daisy-clipper settled
into the groundcover, layli-layli calulan leapt out. Together, she and Buntec eased Alfvaen aboard
where layli-layli began to check her over.
There was nothing swift-Kalat could do to assist layli-layli calulan, so he turned instead to the rest
of the group that straggled from the flashwood. Om im bowed jauntily to Tocohl, took her hand, and laid
it on his shoulder; she straightened—a brief spasm of pain crossed her face—and accepted the support,
to walk the last few steps to the daisy-clipper, begrimed features held high and proud.
Om im said, “Take Alfvaen and Tocohl to camp. We’ll wait here, Buntec and I, for the second lift.”
But Tocohl had found the arachne. “Maggy!” she demanded—and the sprookje echoed her as if for
emphasis—“What’s happened to Maggy? Is she all right?” She lifted the arachne, wincing at the pain the
action caused her, and shook it as if to bring it back to consciousness.
“Maggy’s fine,” said layli-layli from the rear of the daisy-clipper. She was stripping the 2nd skin
from Alfvaen. “She told us how to find you. She thinks an Hayashi jammer’s been planted on you—and
the fact that we found you where she said we would bears that out. Check your throat mikes.”
Layli-layli ran her fingers lightly over Alfvaen’s back; she drew in her breath.
Alarmed, swift-Kalat craned to look but only caught a glimpse of gray before Buntec shoved him
aside to get at the daisy-clipper’s tool kit. With the smallest screwdriver to hand, she attacked her own
throat mike. “Nothing,” she said after a moment’s scrutiny—slammed the throat mike onto the
daisy-clipper’s floor and held out her hand for Tocohl’s.
As Tocohl reached up to remove it, she gasped, causing layli-layli to look sharply in her direction.
Tocohl snapped, “Nothing dangerous. Take care of Alfvaen,” but she let swift-Kalat remove the mike
from her neck and hand it to Buntec.
Om im said, “Possible broken rib, layli-layli,” and earned himself a glare from the Hellspark.
“Never mind that,” said Tocohl with an exasperated look, first at him, then at the sprookje. “I’m
fine,” she repeated with the sprookje still speaking emphasis, “take care of Alfvaen.” Behind the
exasperation, her face was ashen.
“Barefoot!” said Buntec; she reached into the exposed workings of the throat mike with a hemostat
and withdrew a tiny object which she laid in the palm of her hand. “Maggy was right,” she said curtly,
holding it out for the others to inspect.
“Can you disable it?” Tocohl asked.
“Not without the proper tools,” she gestured at what was available to her, “like working on a clipperengine with a sledgehammer.”
Layli-layli calulan slid from the rear of the daisy-clipper. She handed Alfvaen’s throat mike to
Buntec, then climbed into the pilot’s seat and strapped in. “I’ll send someone for the rest of you as soon
as I’m out of range of the jammer. Get in, Hellspark.”
Buntec made a second dive into the interior of the daisy-clipper. “Emergency rations,” she
announced, “I’m famished.” She tore into the box with fervor.
“Get in, Hellspark,” layli-layli calulan repeated. This time Tocohl obeyed. Swift-Kalat helped her
into the remaining empty seat, and seeing she was unable to reach the seat belt without renewed pain, he
drew the strap and handed it across to layli-layli who snapped it shut for her.
“Take good care of Sunchild, Om im,” Tocohl said and smiled wanly at the echoing sprookje.
Om im touched the hilt of his blade in acknowledgment.
Between bites, Buntec said, “Tocohl, layli-layli—don’t mention the Hayashi jammer. Say we lost
the throat mikes! Say anything, or nothing! I’ll find a way to disable it without destroying the evidence.
We’ll see whoever planted them hung!”
“Yes,” said Tocohl, and then she lapsed into silence. Layli-layli calulan turned the daisy-clipper
back toward the river, traveling as swiftly as she dared.
Just as they reached the smoother surface of the river, the arachne stood up. Picking its way carefully
onto the seat, its fragile legs straddling Tocohl, it tilted its camera eye up to scan her face anxiously.
“Tocohl! Are you all right? Tocohl!
“Let them sleep, maggy-maggy. They need the rest.”
But Tocohl stirred and opened her eyes. Her words were barely audible but relief rang in them.
“Maggy,” she said, “you’re all right!” Then she closed her eyes again. Some of the pain had left her face.
The arachne raised an extremity to touch Tocohl’s cheek. Then it canted closer to speak in layli-layli
calulan’s ear, “She may have a broken rib—”
“I know. Have a look at Alfvaen if you can manage it without waking them.”
While layli-layli calulan reached for the comunit to report the location of the rest of the party, the
arachne, balancing precariously, climbed the backrest to inspect Alfvaen.
Buntec brushed crumbs from her breast and said, “I have an idea that might work. Look around for a
small Eilo’s-kiss.” She began to dismantle the two remaining throat mikes to check for additional
jammers. “A good jolt of electricity should fuse the buggers solid. Hah!” She’d found a second jammer,
which she handed delicately to swift-Kalat to hold with the first.
“Toes!” she continued, “what I wouldn’t give for a look at the innards of that daisy-clipper that so
conveniently crashed on us.”
Swift-Kalat said, “You mean it might have been sabotaged?”
“I checked the engine before we left camp, swift-Kalat, and it was in perfect working order then!
After which,” she scowled, “I left it to the mercies of a Maldeneantine and a sprookje.” This time, she
turned the scowl on Sunchild.
“I found one,” Om im announced. “A small Eilo’s-kiss.” He looked at swift-Kalat. “If it hadn’t been
for Buntec, we’d all be dead now: I know of only a dozen people who can handle a daisy-clipper that
well and Buntec’s the only one present on the expedition.” He made a cheerful bow in her direction.
Grinning, Buntec returned the bow, and taking the two Hayashi jammers from swift-Kalat, she
grasped them carefully between the jaws of a rubber-handled pliers from the tool kit. “Lead on,” she
said, and picked her way carefully through the glittering underbrush in Om im’s wake.
She approached the Eilo’s-kiss cautiously and began to stretch in its direction. “Wait, Buntec,” said
Om im.
“Rubber-handled pliers,” said Buntec.
“I’m not worried. Sunchild is.”
Buntec glanced at the sprookje. Its mouth was agape, displaying its red warning tongue. “Yeah, I
know,” she told it, sticking out her own tongue in imitation.
Then, with her free hand, she touched Sunchild on the wrist, stroked lightly as she’d seen Tocohl do.“Relax, friend—I don’t plan to fry.”
She touched the pliers to the Eilo’s-kiss and was rewarded with a snap and a bright spark.
Withdrawing her arm, she looked at the tip of the pliers. “Overzealous little bugger,” she said, releasing
the two jammers into her palm, “Ow!—but that should have fused them solid.”
Buntec thrust the jammers into her pocket. Taking the remains of the throat mikes from swift-Kalat,
she said, “May I have your attention for a moment?” Curious, he gave it—and Buntec flung the mikes
deep into the flashwood, as far as they would go, where they disappeared in arabesque vine and
squealing pig thicket.
“Oh, foot!” said Buntec, “I lost the throat mikes.”
Swift-Kalat smiled; Buntec smiled back, pleased that he found her fiction acceptable by Jenji
standards.
“And just in time.” Om im grinned and pointed: two daisy-clippers emerged from the flashwood.
Moments later, Hitoshi Dan was pounding backs all around, welcoming them like lost children. He was
so happy to see them he gave the same welcome to swift-Kalat, who hadn’t been lost. Buntec didn’t
begrudge it in the slightest.
On the skirt of the daisy-clipper, Om im paused and held out the edge of his cloak to Sunchild. The
sprookje took it, loosed it, then turned and stepped into the flashwood, headed in the direction of base
camp. “Thanks, but I’ll walk,” Om im interpreted with a grin. “After the last ride you had, I’m not at all
surprised.” He bounced aboard and gave Edge-of-Dark a flourish and a wave. Edge-of-Dark took his
meaning; the daisy-clipper dashed for base.
Chapter Thirteen
EVERYTHING REACTIVATED SIMULTANEOUSLY—from the arachne to Tocohl’s 2nd
skin and implant to Alfvaen’s hand-held—and simultaneously Maggy assessed it all. The view from
Tocohl’s spectacles, though Maggy enhanced it, showed only the tuft of moss that the sprookje had
given her; Tocohl had apparently put the spectacles in her pouch. From the arachne, Maggy saw
Tocohl’s still “booted” feet; the programming she’d read into the 2nd skin’s local microprocessors had
held even through the lapse of contact, she noted. (Tocohl!) she said through the implant, (Tocohl!)
There was still no response. Maggy sent the arachne up for a better look while she checked the sensors
in Tocohl’s 2nd skin. Heart—and breath-rate close to normal—normal for unconsciousness, at any
rate—but the swelling beneath Tocohl’s skin at the third rib on the right spoke of injury.
Using the arachne’s vocoder, she tried to rouse Tocohl to consciousness. She was somewhat
surprised to realize that she would have kept trying, despite layli-layli calulan ‘s injunction, had Tocohl
not stirred and spoken. “Maggy, you’re safe!”
(Veschke’s sparks!) Maggy said. (You thought something had happened to me !) But Tocohl had
lapsed back into unconsciousness and Maggy finished the thought to herself, Of course! All our contact
had gone dead. You were as afraid for me as I was for you.
Having assessed the data from the 2nd skin, Maggy concluded a seventy percent chance Tocohl’s
rib was broken. Concluding as well that layli-layli calulan had little interest in probability by percentage,
she said only, “She may have a broken rib—”
Layli-layli calulan, as Maggy had expected, already knew, but asked her to look at Alfvaen.
Assuming Alfvaen still had the hand-held... Maggy sent the arachne over the backrest of Tocohl’s seat.
It was no easy task manipulating the mobile in a moving vehicle, especially without stepping all over
Tocohl. When she had at last landed and steadied it beside Alfvaen, she promptly switched a high
proportion of her attention to the Siveyn.
While layli-layli notified base camp that Om im and Buntec were safe and gave the location where
swift-Kalat waited with them, Maggy probed Alfvaen. “Asleep,” she pronounced when her words would
no longer be an interruption, “at least her heartbeat and respiration are the same as I noted when she
slept normally.”“That’s encouraging. Buntec says she seemed to sober, then began to hallucinate. She challenged
Tocohl to a death duel, that’s why she’s tied. I can’t tell you anything about those growths on her skin.
They might be a symptom of Cana’s disease that hasn’t yet been noted, or they might be—” The
daisy-clipper lurched slightly; layli-layli calulan did not finish the sentence.
She concentrated on piloting for a moment, then said, in a different tone, “Your last hunch was right:
Buntec found an Hayashi jammer in Tocohl’s throat mike. But that’s a secret, you’re not to tell anyone
about that just yet.”
“I will tell Tocohl.”
“Of course. That was understood. But no one else.”
Maggy scanned Alfvaen once more, considering as she did the state of Tocohl’s rib and the fact that
fiction had proved true in the case of the Hayashi jammer. She was silent for the remainder of the trip
upriver as she recalled all the relevant Siveyn literature.
By the time they arrived within sight of base camp, Maggy had reached what she thought a good
conclusion. “That must mean that Alfvaen and Tocohl are best friends,” she said aloud.
“I beg your pardon?”
Layli-layli calulan sounded so puzzled that Maggy momentarily lost her certainty, until she recalled
that layli-layli calulan simply didn’t know the Siveyn as well as she did. “If Alfvaen challenged Tocohl
to a death duel,” Maggy said, “they must be best friends. I didn’t know they were.” She sought the
analogous situation—midway through Alfvaen’s favorite fiction—and found in Tocohl’s gloss the
comment: “Don’t worry, Maggy. It’ll all work out right. It always does in a case like this.” Maggy knew
from Tocohl’s tone that this was intended to be reassuring, so she told layli-layli calulan precisely the
same.
Whether it served to reassure layli-layli or not, Maggy couldn’t tell. They had reached the base
camp and the shaman shot over the barbed-wire perimeter to ground the daisy-clipper, with a jarring
thump, directly in front of the infirmary. She exited shouting commands to those who waited with
stretchers.
Maggy made sure the arachne got in no one’s way, then sent it springing into the infirmary after them
all.
Alfvaen and Tocohl had already been transferred from stretchers to beds. “Out,” layli-layli calulan
commanded, “that means everyone but you”—she pointed a finger stripped of its bluestone ring at
Kejesli. Maggy ducked the arachne under a bed; she was not leaving Tocohl unobserved by any means
at her disposal. “—And you,” layli-layli calulan finished; she thrust her pointing finger under the edge of
the bed in the general direction of the arachne.
Concluding that she meant to let the arachne stay, Maggy poked it hesitantly from concealment.
Layli-layli scooped it and set it beside Alfvaen. “We’ll get to Tocohl in a moment,” she said, stripping
the ring from her other hand, “but notify me if you sense any change in her condition.” Tocohl seemed to
be resting comfortably, so Maggy concentrated her attention on Alfvaen.
Layli-layli snipped through Alfvaen’s bonds and, having shifted her to a more comfortable position,
strapped her firmly to the cot, the catch-releases out of reach. Then she peeled back Alfvaen’s 2nd skin,
giving Maggy’s camera eye a good look at the growths, and attached to her body various medical
sensors. Next she took a sample of the gray filaments from Alfvaen’s skin.
Behind her, Kejesli gasped. “Garbage plants!”
Without hesitation or a need to scan her Sheveschkem files, Maggy interpreted that tone as one of
horror. She corrected him instantly: “No, they are not garbage plants. They bear only a superficial
resemblance to the species I was shown.”
“Good,” said layli-layli calulan, “and thank you, maggy-maggy, that saves me a lot of
time.—Captain, there are sedatives in the cabinet to your right if you need one.” She set the diagnostic
machine to its task of preparing slides of the sample.
Kejesli, as if exhausted, slumped suddenly into a chair, where he watched layli-layli with tired eyes.
If such behavior did not worry layli-layli calulan, Maggy decided, it would not concern her either.
“While we’re waiting for the slides...” Layli-layli held out her arms, inviting the arachne into them.Layli-layli carried it across the room to place beside Tocohl.
Seeing she needed access to Tocohl’s injury, Maggy provided it: the 2nd skin peeled away from
Tocohl’s ribcage in broad strips. Layli-layli first probed the swelling with gentle hands, then confirmed
her shaman’s diagnosis with a sounding scanner. “Yes, the rib’s broken. No complications to that though.
I’m giving her a local anesthetic, maggy-maggy”—she suited action to the words—“then we’ll set the
rib.”
Once again she brought her fingertips to rest on the swelling. The reddening seemed to lessen. “You
have more than average control, maggy-maggy. Is it fine enough to keep Tocohl’s 2nd skin taut in this
area only?”
“Tell me what to do,” Maggy said, “and I’ll do it.”
“Good, she’ll be more comfortable if I don’t tape it.” Layli-layli calulan smoothed the 2nd skin
gently back over the injury where Maggy sealed it. “Be ready, I’m about to set the rib.”
“Ready,” Maggy said, set to record from both the arachne’s lens and from the sensors in Tocohl’s
2nd skin. The job was done in a single swift push... then Maggy was drawing the 2nd skin tight in accord
with layli-layli’s instructions.
“Fine, that’s fine. You’re to keep it that way until I tell you otherwise.”
“Tocohl,” Maggy began.
“Tocohl has no say in a medical matter. If she gives you any trouble, refer her to me. Or tell her it’s
that or taping.” Layli-layli calulan directed a brief smile at the arachne. “And I’ll give you one additional
warning. She’ll be in some pain when the anesthetic I gave her wears off. Do not be tempted to loosen
the 2nd skin—”
Maggy had by this time been through her medical files. “I know,” she said, risking the impoliteness of
an interruption both to save layli-layli calulan the time and to assure her that she would take good care
of Tocohl. “It might make the pain worse—and it could lead to internal damage. I’ll tell her to do a
Methven ritual for the pain instead.”
“Suggest a Methven healing ritual as well. Between the two of us, we’ll have her up and around in no
time.” She walked to the diagnostic machine, where her slides awaited, leaving the arachne one last
instruction: “Tell me when she wakes.”
Maggy sent the arachne bounding after her. At layli-layli’s glance of surprise, she explained, “I can
tell through the 2nd skin when she wakes—and she’ll want to see your results too. I’m recording for
her.”
“I see.” Layli-layli calulan set about her work. When she had examined the slides—and given the
arachne a chance to do so as well—she moved again to Alfvaen’s side, placing the arachne at the head
of the cot.
Again her fingers flickered lightly over the Siveyn’s skin while her eyes scrutinized various monitors.
Alfvaen moaned.
With a suddenness that made Maggy jump the arachne back, Alfvaen flailed against the straps that
held her. Her eyes flashed open, fixed on the empty space between layli-layli and Kejesli. She began to
speak, slowly at first, then building to the fever-pitch rapidity of terror.
“Alfvaen,” Maggy said in Siveyn, trying to cut through the fear, “Alfvaen, there’s nothing there!
You’re safe!”
Alfvaen did not hear her and went on as before. Maggy sent the arachne a cautious step forward to
try again.
Layli-layli calulan laid her hand across Alfvaen’s eyes: the Siveyn’s violent struggles subsided to
steady tension against the straps, her voice sank to a still-fearful whisper. “Can you translate for me,
maggy-maggy?”
“Roughly, she says, ‘They’re coming! They’re coming to get me! Let me go! The Ilistis are coming!’”
That needed clarification. Maggy added, “The only reference I can find to Ilistis describes them as very
ugly, very violent mythical creatures. No other match. She keeps saying it over and over again. I’m
sorry, layli-layli, that’s all I can tell you. I wish I knew more. That can’t be right.”
“I think your reference is probably correct. It fits with her medical condition and with what Buntectold me of Alfvaen’s hallucinations.” Layli-layli calulan brought her hands to either side of Alfvaen’s
temples and murmured softly. A moment later, Alfvaen relaxed back onto the cot, fell silent; a moment
after that, she was asleep—to all appearances, peacefully.
“I don’t understand,” Maggy said, taking care to keep the vocoder low.
Layli-layli calulan answered in quiet calm. “You won’t find it under Cana’s disease, but
elsewhere... Look at the monitors. Those growths are not obstructing her circulation; what’s more, the
blood monitor shows no indication of alcohol in Alfvaen’s system.”
That was so, Maggy had to admit, but... “I still don’t understand.”
“You’ll find the information under delirium tremens. To be healthy, a human being needs to dream.
Alcohol disrupts the ability to do so. Now that the alcohol is gone from her system, Alfvaen’s mind and
body seek instinctively to... catch up on dreaming. Awake or asleep, she dreams—sometimes of duels,
sometimes of Ilistis.”
She strode to a cabinet, brought out a small blue container, strode back to Kejesli. Shaking a pill into
his hand, she commanded, “Take that.”
Kejesli obeyed listlessly, bringing the pill to his mouth, swallowing a number of times. At last, he
looked up at layli-layli calulan. “Can’t you do something to get those things off her?”
Having by this time reviewed all available information on the effect of alcohol on the human body,
Maggy was surprised to hear in Kejesli’s voice the same horror it had held when he had mistaken the
growths for garbage plants. Didn’t he understand...?
Layli-layli calulan explained it for him, simply and firmly: “Those things, Captain Kejesli, are
healing her.”
“Healing her?” swift-Kalat said, when layli-layli calulan repeated her statement for him half an hour
later. He bent beside Alfvaen and stroked her temple gently. “Are you sure?”
“As sure as I can be with an unknown life-form. Check the slides yourself—you’ll see the structure is
similar to, but does not match, the garbage plants.” Turning to draw him with her to the display screen,
layli-layli calulan was forced to an abrupt halt to avoid a collision with Kejesli.
He attempted to move out of her way but his grip on the edge of Alfvaen’s cot prevented him from
backing the necessary distance. Startled, he glanced down at his hands as if he had not seen them
before—or as if he had no control over what they were doing, Maggy thought. With obvious effort he
removed, first one to splay it at his throat, then the other. This time he stepped out of her way.
“No offense given, Captain,” layli-layli calulan said patiently, “but I would prefer that you wait
outside until I have finished checking Buntec and Om im for injury. Your debriefing can wait that long...”
Kejesli splayed his hand a second time at his throat. Without a further word, he walked
unsteadily—as if the infirmary floor heaved beneath him, disturbing his balance—to the door and
vanished through it.
While swift-Kalat pulled a chair to the display screen to do as layli-layli suggested, the shaman
retrieved her sounding scanner. Om im, standing over Tocohl, glanced up at her approach. “That’s not
really necessary, layli-layli,” he began.
“Humor me,” said the shaman. “It gives us the opportunity to speak of things among ourselves that
we might not speak of to Kejesli.”
More secrets, Maggy decided, and realized abruptly that she had not given Tocohl the message she
held from layli-layli. Finding Tocohl alive though injured had drawn from her an unusual response:
without any deliberation, her priorities had shifted. She shifted them back; when Tocohl awoke, she was
to receive layli-layli calulan’s message before Maggy said anything else.
Om im waved aside layli-layli’s invitation to lie down, choosing instead to draw a chair to Tocohl’s
side, blade right. With a look Maggy interpreted as resignation, layli-layli scanned him where he sat,
taking care to approach him from the politic side.
Buntec, who had been silently observing Alfvaen, now turned and strode across to them. For a long
moment, she gave Tocohl the same scrutiny. Then she said, “It’s too bad we haven’t the equipment to
salvage that daisy-clipper”—she punched her palm—“I’d give an arm to see that engine.”“I’d give Megeve’s arm to see that engine,” Om im said.
“You don’t know it was Megeve,” she countered. “Anybody had the chance to plant those jammers.
And we don’t know the clipper was sabotaged.”
Om im eyed Buntec for a long moment, then, tilting the chair back, he drew his blade and began to
hone it. Layli-layli calulan stepped back, gave him a look that Maggy could neither see nor interpret
from the position of the arachne. “Sorry,” he said, sheathing the blade and bringing the chair upright with
a thump, “I didn’t mean to disrupt your examination.”
Layli-layli calulan said nothing, only stepped forward again to draw her fingers lightly down his
body. Maggy angled the arachne for a better look and discovered that she did not actually touch him
except once. “Just a bruise,” Om im said, having noticed the arachne’s interest. “Buntec, show Maggy
those gadgets. You were right, Maggy, about the Hayashi jammers. Buntec’s got them in her pocket.”
Never having actually seen an Hayashi jammer, Maggy sent the arachne skirting Tocohl’s head in
order to record this for future reference. Buntec reached into her overpocket.
(Maggy? Maggy?) Though the words were soft and urgent, Tocohl’s voice rang through their private
channel. (Are you all right, Maggy?)
Instantly, Maggy split her attention. Halting the arachne where it could show her both the palmful of
tiny electronic parts Buntec held out for inspection and Tocohl, she said, in what she judged from
layli-layli’s usage to be a reassuring tone, (I’m fine. You have a broken rib though. Layli-layli calulan
told me to suggest you perform a Methven healing ritual.)
Tocohl’s eyes did not open. (Healing ritual it is.—I’m glad you’re safe, Maggy. It got awfully lonely
without you.) Her voice fell silent but Maggy could tell from the sensors in her 2nd skin that she had
begun the ritual.
Through the arachne, Maggy said softly, “Tocohl is awake now and has begun the Methven ritual
you required.”
Just as softly, layli-layli calulan told the others, “Quiet, please.” She moved to Tocohl’s side;
placing her fingertips on Tocohl’s injured ribs, she too fell silent, as if in ritual of her own.
There went the priorities again, Maggy realized. She did not understand why that was happening.
Forgetting priorities—forgetting to deliver messages one had been told were important—that was
something that happened often in fiction, but it had never before happened to her. She ran a diagnostic.
A moment later, Tocohl opened her eyes and said wanly, “Hi, how’s Alfvaen?”
Layli-layli calulan repeated her diagnosis for the third time. While she spoke, she took Buntec
firmly by the shoulders, sat her on the edge of Tocohl’s cot, and ran the sounding scanner over her.
Laying aside the scanner, she finished her account with the command, “Another moment of quiet,
please.”
Priorities, thought Maggy. She pinged privately for Tocohl’s attention, but before she could speak,
layli-layli calulan said, “That means you too, maggy-maggy.” A finger, bereft of its ring, pointed at the
arachne. Maggy dropped the arachne into a crouch; if the pointed finger was aimed, she could at least
keep the arachne from being a direct target.
Layli-layli calulan smiled. “I only meant, don’t be distracting, maggy-maggy. I had no intention of
quieting anyone permanently.” Tocohl twisted her head to give layli-layli a puzzled look, then twisted
farther to take in the crouched arachne. Understanding lit her tired face and she smiled reassuringly into
the arachne’s camera eye. Maggy kept quiet: she was taking no chances..
When layli-layli calulan had finished treating Buntec’s handful of bruises, Tocohl said thoughtfully,
“The sprookje... suppose it identified the toxin in her system? You did dub that a ‘sample tooth,’
Buntec; maybe the description is more apt than anyone thought. That second bite it gave Alfvaen might
have been what it considered... well, an antidote.”
“As a hypothesis,” said swift-Kalat—he had finished his examination of the slides layli-layli calulan
had prepared and come to stand behind Om im—“that’s safe to say.”
Taking into account what was “safe to say” in Jenji, Maggy knew he was not nearly as sure of that as
Tocohl, but that he wished it so and could speak it without fear.
Buntec shifted on the edge of the cot—with great care—to face Tocohl directly. “Not much we cando for her if it’s wrong,” she pointed out. “Next question: What’ll we tell Old Rattlebrain—about the
jammers, I mean?”
“Unless you know for sure who planted them...” Tocohl began. It was more question than statement:
Buntec punched her palm again, Om im grunted. Tocohl took both for answer and went on, “Let’s not
tell him anything just yet. At the moment, we’re the only ones who know they existed.”
“And the person who planted them,” Om im said, scowling.
“That could work out to our advantage,” Tocohl finished.
“Meaning he doesn’t know we’re on to him.” Buntec spread flattened hands, narrowly missing
layli-layli calulan with the broadness of the gesture. In Jannisetti, it signified clearing the table to deal
afresh with a problem.
Raising her voice, Buntec announced, “Swift-Kalat, I’m speaking in GalLing’ only, and it’s
hypothesis. I’m just gonna tell Tocohl how it seems to me and I don’t wanna worry about the words, so
don’t get all sweated up about reliability, okay? The truth is what we’re trying to get at.” Swift-Kalat
took a deep breath, obviously preparing himself. “Go ahead,” he said, and Buntec launched into a
rapid-fire summary of their suspicions, adding not a few of her own.
“So,” she finished at last, “if—and only if—the daisy-clipper was sabotaged, then Timosie Megeve
did it. He was the only one who had the chance.”
“The sprookjes—” Tocohl began.
“You don’t believe that and neither do I. The sprookjes may have had the chance but they don’t
have the know-how.”
Tocohl shifted to address the arachne. “On the available evidence, Maggy, what’s the probability that
Megeve planted those jammers?”
“Roughly twenty-five percent.”
“Not high enough,” said Tocohl.
“High enough for me,” Om im said; drawing his knife, he began to hone it again.
“Not good enough for Buntec, or she’d have rammed it down Kejesli’s throat the moment she saw
him,” Tocohl pointed out. “And you, swift-Kalat, could you speak of Megeve’s treachery?”
“I could not,” swift-Kalat said. “Someone attempted to isolate the four of you from contact with base
camp, but I do not know who or to what purpose. Megeve had the knowledge and the opportunity, but
every member of the team also had the opportunity. Perhaps others had the knowledge.”
“Maggy?” said Tocohl.
Maggy recognized this as Tocohl’s shorthand way of requesting further information. “According to
the personnel records, three other members of the survey team have also spent time on Hayashi: Hitoshi
Dan, Edge-of-Dark, and Om im.”
The rasping noise from the side of the cot stopped abruptly. “Count me out, Maggy. Suicidal I’m not;
if the daisy-clipper was sabotaged I was meant to go down with the rest of you.”
That made sense and made Maggy reexamine the evidence. “Then if the daisy-clipper was
sabotaged,” she concluded aloud, “there is a thirty-three percent probability that Megeve planted the
jammers. Would that be high enough?”
“Wait up, kid,” said Buntec firmly, “you’re doin’ this wrong.”
“Tocohl?” Maggy said.
“Let Buntec tell you, Maggy. I’m too tired.”
Buntec turned to the arachne. “I’ve been on Hayashi—tearin’ up the local Port of Delights with an
old buddy isn’t the kind of thing Old Rattlebrain or the Older Rattlebrains that tell him when to wipe his
nose would care about. What’s more, I’ve seen those jammers sold on a half-dozen other worlds. You
can’t just go by that. Not enough info.”
“Oh,” said Maggy, revising her tactics. Now it seemed, she had no way of estimating the probability.
Buntec punched her palm, twice, with force. “If only I could eyeball the innards of that
daisy-clipper... Toes, and toes again.”
“Mind your language, Buntec,” commanded layli-layli calulan.
That surprised Maggy who knew perfectly well that toes were not obscene to an Yn. From Buntec’sexpression, it surprised her as well, but grudgingly she said, “Yeah, right. Not in front of the kid.”
“Not in front of my patients. They need calm and rest, all of them.” Layli-layli calulan touched
Buntec gently at the right temple, so Maggy knew that one of the patients referred to was Buntec herself.
The shaman went on, “Using language you consider obscene is far from calming.” She spoke quietly to
Buntec for a few moments longer.
The words she spoke did not seem as important as her soothing tone and her soothing touch, but
when she had finished, Buntec’s face had lost some of its anger. Even her voice was calm. “Thanks,
layli-layli. Lot more shaken up than I realized—lightning blastin’ away no more’n an inch from my face!
You wouldn’t believe... !”
“I believe it will make a good story,” layli-layli calulan said, drawing a rich chuckle from Om im.
A broad grin spread slowly across Buntec’s dark face. “Oh, will it ever!” she agreed.
“Then stick to that,” Tocohl said. “Save the jammers for the sequel. We still have to know more.”
The sensors in the 2nd skin told Maggy how tired she was; drooping eyelids confirmed that.
Tocohl forced her eyes open; with effort she raised a hand to stop Buntec from rising. “Give Old
Rattlebrain a real teaser, Buntec”—a hint of smile touched the corners of her mouth—“tell Old
Rattlebrain I’ll be out to talk to his sprookjes as soon as I’ve had my nap.”
She grinned at the astonished faces looking down at her and added, “And if Alfvaen wakes before I
do, please tell her that her serendipity is unrivaled.”
The room was quiet. Buntec and Om im had gone to make their abridged report to Kejesli;
layli-layli and swift-Kalat bent over Alfvaen, taking further readings and speaking in whispers so soft
Maggy had to enhance to understand them.
Tocohl’s eyes were closed but Maggy knew from her sensors that she was not asleep. Maggy settled
the arachne beside her pillow to keep watch. Then she recalled the priorities... she pinged furiously for
attention, before the odd lapse could occur again.
“What is it, Maggy?”
(Just for you,) Maggy said, (Layli-layli calulan said I was to tell you—only you—the first time we
renewed contact. I’m sorry I didn’t; I don’t understand why; I ran a diagnostic—)
(You’d better get on with it then.)
Maggy relayed layli-layli calulan’s message in the shaman’s own voice, using her taped recording
of it. Then she added, (In Hellspark, that would be, ‘Megeve may be responsible for the equipment
failures,’ right?)
(Right.) There was a pause, a long one by human standards. (So layli-layli thinks Megeve may have
had something to do with Oloitokitok’s death too,) Tocohl said at last. (What led up to that,
Maggy—any idea?)
(What swift-Kalat told her, I think. I’m not sure.)
(Show me.) She opened her eyes.
That meant picture as well. Maggy obliged with the arachne’s eyeview of the exchange between
swift-Kalat and layli-layli calulan that her lie had brought about.
When the tape was finished, Tocohl said, (I see. Megeve “fixed” the transceiver while we were
missing, just as he fixed it while Oloitokitok was missing. Neither swift-Kalat nor layli-layli likes the
coincidence. I see the point, too. An Hayashi jammer doesn’t affect anything out of its range. It wouldn’t
have affected the transceiver at this end and Kejesli would have known something was wrong at our end.
He’d have sent out a search party immediately.)
(He wanted to fix the arachne too,) Maggy said. (I wouldn’t let him. Neither would layli-layli.)
(Good thing, too, under the circumstances.) Again she paused. (But what would he stand to gain by
killing us? Even if we assume he’s an Inheritor of God, that the Inheritors want this world... ) The
sensors in her 2nd skin spiked suddenly. (Maggy! Did Megeve tell Kejesli about the gift the sprookje
gave me?)
(No. Buntec will be pleased: she can still see the look on Kejesli’s face. Should I tell her?)
(Buntec will be far from pleased, Maggy, and I’m an idiot for not thinking of it. With the four of us
dead, Megeve was the only witness to the exchange of gifts—and he told no one!)(A gift is evidence of sentience?)
(Not evidence, no, but strongly suggestive. Even Kejesli would have held his report on the strength of
that.)
(Then, by killing the four of you, Megeve would gain nothing. I recorded the event.)
(Ah, but he didn’t know that, Maggy. I’m not sure he knows that even now. Perhaps he was only
casting doubts on the arachne’s reliability because you were so adamant that something had happened to
us.)
(Or perhaps he was right to cast doubts on my reliability,) Maggy said.
The sensors spiked again. (What’s wrong, Maggy?)
(Nothing that I can find on a diagnostic but—)
(But what?)
Maggy explained about the shifting priorities that had prevented her from carrying out layli-layli
calulan’s instructions.
When she was done, Tocohl said thoughtfully, (The message wasn’t delayed long enough to do any
harm, if that’s what’s worrying you. That your priorities shifted without your being aware of the shift is a
little surprising, Maggy. It’s worth investigating certainly. But... I did the same thing myself—I had to
know first if you were safe. Everything else was secondary.)
(There’s something else.)
(Tell me.)
(I acted on probabilities lower than those I’ve ascribed to Megeve.) Maggy could tell from Tocohl’s
expression that she did not understand, so she explained, (I could accept the high probability that you
were dead and do nothing, or I could accept the low probability that you were being jammed and act.
Layli-layli calulan was also willing to accept the low probability. She called it a “hunch”; is that correct
usage?)
(Absolutely.) A slow smile spread across Tocohl’s face; the sensors in her 2nd skin told Maggy that
she had calmed as if no longer worried. (And, please note, that’s absolutely consistent with your shifted
priorities. You were right about the jammers. Stop worrying.)
(Yes, in this case the low probability was the true one. So—in the future, how should I assign
priorities?)
(Like the rest of us. It’s something you learn by doing. I will say, however, that it always seems to
help to have someone else to talk it over with—even if you don’t take their advice once you’ve done it.)
(With you,) said Maggy. (Layli-layli calulan comes.) Through Tocohl’s spectacles Maggy watched
layli-layli calulan approach. The shaman, her broad mouth stretched broader in the smile Maggy had
recorded only once before, bent over Tocohl. (Beautiful smile,) Tocohl said drowsily; and Maggy was
careful to note that she had correctly anticipated Tocohl’s reaction. “You’re disturbing my patient,
maggy-maggy. She does need the nap she spoke about.”
The arachne bobbed apology. “Your pardon,” Maggy told her through its vocoder.
The shaman’s fingertips brushed Tocohl’s temples; she murmured words incomprehensible to
Maggy. The readings from the 2nd skin began to resemble those of Tocohl when she was drowsy, a
comfortable, easeful drowsiness.
For a brief moment, Tocohl struggled to rouse herself. (With me, Maggy,) she agreed, and then she
let layli-layli calulan’s touch draw her into sleep.
Tocohl woke to the sound of distant thunder and an urgent voice: (Danger, Tocohl! Wake up!
Danger!)
The infirmary was dark, silent. Maggy pushed the spectacles for available light without waiting to be
told; in the brightening, Tocohl saw Timosie Megeve start across the room on silent feet.
She waited, tensed but unmoving: in the darkness he perceived, Megeve could not know she was
awake. (Record this, Maggy.)
(I am,) came the response, audible only to her.
Layli-layli calulan was nowhere in evidence. At Alfvaen’s bedside, swift-Kalat had fallen asleep inhis chair, his head pillowed on his arm, his fingers brushing her face.
Megeve paused as he neared the two.
Under the circumstances, the pause seemed too ominous to chance. “Sssh,” Tocohl said softly into
the darkness, “and don’t turn on the light. Layli-layli calulan says Alfvaen needs as much sleep as she
can get.” She was rewarded by seeing Megeve jump as if she’d struck him.
He recovered quickly, counting on the darkness to keep his reactions from her, and in a quiet,
controlled voice he said, “Sorry to wake you, Tocohl Susumo. I thought I’d drop by and see how you
and the serendipitist were doing.”
(Maggy, I want the arachne in tripping range.) Aloud, covering the faint thump and scuttle of the
arachne, Tocohl said, “Oh, we’re fine. Nothing a little sleep won’t cure, according to layli-layli.” She
raised herself on one elbow, felt the stab of pain, and knew she was at a disadvantage. She let out a hiss
of breath, eased herself back down. He doesn’t know we suspect him, she reminded herself, let’s keep it
that way. “You didn’t wake me, the storm did. Come, tell me what’s been happening. I seem to have
missed a great deal.”
Megeve lifted a chair and brought it silently to the bedside. A soft clicking dogged his heels but he did
not seem to notice. “Nothing much,” he said as he sat beside her. “Is it true you can speak to the
sprookjes?”
(So that’s what this is about!) “Only partially true, I’m afraid,” she said, and saw him relax ever so
slightly. “I can get your sprookje to echo me. It’s a step in the right direction”—there’s an accurate
phrase, she thought—“but hardly enough to satisfy the legal requirements. I’m sorry. I know how much it
means to you, especially because of Oloitokitok.”
“Yes,” he said, his voice rough with emotion. But, at the far end of the tunnel of light her spectacles
provided, she saw his expression: relief.
Tocohl waited a moment, as if allowing him to get a grip on himself, then she said, “Nothing much
happened here?! Layli-layli calulan broke mourning to look for us.” It startled her as she said it, for she
had only now realized that Maggy’s tape had shown the shaman in the midst of deep mourning. How had
swift-Kalat gained access to her? “And that means you had as much trouble as we did, or more. You
weren’t joking about the equipment failures... !”
The cabin’s membrane was flung open, startling them both. Om im and layli-layli calulan spattered
water on the threshold; both had been running hard. Om im, Tocohl saw, carried the hand-held Maggy
had given Alfvaen. (Thanks, Maggy. I didn’t realize you were in touch with them.)
“Hi,” said Tocohl cheerfully. “You needn’t put on the light, layli-layli”—recognition gave nothing
away: layli-layli was recognizable even in the light available to Megeve—“Timosie was just keeping me
from getting bored.”
Layli-layli calulan took the hint. As if nothing at all were amiss, she said, “Will the light disturb you,
Tocohl? I came to change Alfvaen’s IV. It will just be a moment.”
“No, the light won’t disturb me,” Tocohl answered; in fact, as layli-layli calulan raised the light,
Maggy adjusted her spectacles accordingly.
Layli-layli set about changing Alfvaen’s IV—she was alternating glucose with saline
evidently—careful to wake neither Alfvaen nor swift-Kalat. That was a nice touch, Tocohl noted with
appreciation, as was Om im’s pause for a brief glance at Alfvaen before he continued on toward Tocohl.
Tocohl turned to look again at Megeve. “Tell me,” she said, “what was the problem with the
transceiver? Since it almost killed me, I’d like to know the cause.”
“You’re not the only one,” Om im said. He moved smoothly to blade right of her, deftly interposing
himself between Megeve and Tocohl without any hint of rudeness.
The Maldeneantine twisted his hand down his wrist, then leaned back and said, “There’s not much to
tell: it was the usual for Flashfever.—Do you know much about electronics?”
“Not a lot,” Tocohl said, omitting to mention that Maggy’s data stores made her an expert on the
subject, “but tell me anyway. I’m always willing to learn.”
Megeve explained in technical terms and then said, “All of which, in simple terms, means we had
fungi growing on the ’plate. It cooked and shorted out one of the freeloader diodes. Until I found whichone, cleaned up the fungus so it couldn’t happen again, and replaced it, you were all on your own.”
Maggy pinged and pinged again.
“I see, I guess,” Tocohl said, careful to imply that she had no idea what he was talking about, then
she gave a yawn that cracked her jaw. “Sorry, I’m drowsing off again. Thanks for the company,
Timosie,” she added, “but now if you’ll forgive me...?” She yawned again.
“Oh, of course!” Megeve took it as a dismissal, rose so hastily the chair tipped behind him. Om im
caught it, drew it to him; bowing deeply to Tocohl he said, “I’ll keep silent company, I promise, until you
wake. A bored Hellspark is a blot on my honor as a Bluesippan.” The grin he gave her was pure deceit:
he sat blade right.
Megeve glanced frowning from one to the other but could clearly see no cause for alarm. To all
appearances, Om im was merely in one of his gallant moods. Megeve said a few polite good-byes and
started for the door.
Tocohl left him to the wary eyes of Om im and layli-layli. (What is it, Maggy?)
(Probability of sabotage now ninety-nine percent; equal probability that Megeve is responsible.)
That was as high a probability as Maggy would ever commit herself to, and Tocohl said, (What
caused the jump?)
In her spectacles, Tocohl saw Megeve open the transceiver. The image froze abruptly, then
expanded, until she could see an area of the ’plate in microscopic detail. (That,) said Maggy, adding an
indicator arrow to the still, (is the diode he spoke of. There is no evidence of fungus, no evidence of a
short.)
(You’re sure?) said Tocohl, knowing that it was unnecessary to ask.
(I’m sure,) said Maggy, just as unnecessarily. She added, replacing the first image with another, (This
is what a shorted freeloader diode looks like. It’s unmistakable.)
(I’d have said spectacular—but unmistakable it certainly is. I grant you the rise in probability.)
Maggy said aloud, “Om im, you’re going to cut yourself if you’re not careful.” Her inflection made it
as much a question as a statement.
Tocohl turned her head to look—Maggy cleared the taped images from her spectacles—and saw
that Om im pressed the hilt of his blade to his forehead. His hand grasped the blade so tightly that he
was, as Maggy warned, close to letting his own blood.
“Don’t you dare,” Tocohl snapped in GalLing’, then instantly followed it up with the proper phrase in
Bluesippan: “I have need of that hand, undamaged.”
By degrees, Om im loosened his hold on the blade. At last he turned it in his fingertips and sheathed
it. Tocohl breathed a sigh of relief.
“Buntec thought to search the daisy-clipper hangar,” he said. “I stood watch for her. I’m sorry to say
she found nothing of interest.” The knife was out abruptly, once more he touched the hilt to his forehead
in apology; this time Tocohl had no fear that he would cut himself for penance.
“If you want to sleep,” he went on, “I pledge to stand watch myself. We left John the Smith to watch
Megeve—he was the one who warned layli-layli—and swift-Kalat was here...”
“Maggy was here,” Tocohl said, eyeing the handheld at his belt meaningfully. “She waked and
warned me—and called on you for assistance, as you expected her to, or you wouldn’t have taken the
handheld.”
“Yes.”
“Then you can hardly claim to have left me without a watch. Maggy never sleeps—for which I am
very grateful, because she now gives a ninety-nine percent probability that Megeve tried to kill us once
already.”
His gilded brows shot up. “How...?” He glanced at the arachne to address the question in retrospect
to Maggy herself.
Tocohl said, “Let’s get it all over with at once.” Painfully, she eased herself to a sitting position.
“Would you mind lending a shoulder, Om im? Now we have something specific to tell Kejesli.”
From across the room, layli-layli calulan said, “Lie down, Tocohl; you’re still under my care.
Kejesli can perfectly well come to you. He has no broken rib.” It was mildly said but the look whichaccompanied it was a command.
Tocohl inclined her head, acquiescing, but remained erect. In response to Om im’s worried look, she
drew her legs to one side to favor her injured rib. His look did not change—but Tocohl had to be able to
see, to speak face-to-face with Kejesli.
Passing to the comunit, layli-layli paused once, to touch her fingertips to swift-Kalat’s temples. His
eyes came open and he rose to scan the room, as if seeking to learn what had startled him awake.
Finding nothing but layli-layli at her comunit coding a call, he returned his attention to Alfvaen.
“Captain,” said layli-layli calulan, “you wished to speak to Tocohl. You may do so in the infirmary.
She is awake now.” She broke the connection, then made two more calls in swift succession, to ask both
Buntec and John the Smith to join them in the infirmary as well. To Tocohl, she explained, “We may have
need of some we trust.”
Tocohl, as layli-layli drew near, saw her features set in disapproval and set her own in stubborn
opposition. The threatened chiding never reached the shaman’s lips; in its stead, a smile of resignation
turned the corners of the broad mouth wryly up. Without a word, layli-layli calulan laid a gentle hand
against Tocohl’s injured side. Realizing the layli-layli meant to ease her pain, Tocohl closed her eyes and
did what little she could to help herself, a Methven ritual against the sharpness in her chest. After a
moment, breathing seemed less difficult.
She opened her eyes, only to find them caught and held by the dark intensity in layli-layli calulan’s
own.
The shaman’s voice was no less intense: “Will you pronounce judgment on Megeve?”
Tocohl had forgotten that complication. Looking at the faces around her, she realized that swift-Kalat
had joined the group as well. The phrasing reliability demanded was expedient to her own purposes as
well. “No,” she said, “I would not. I have good reason to believe that he disabled the transceiver to keep
you from contacting us, yes—but Alfvaen tried to kill me.”
Before swift-Kalat could begin to object to her phrasing, she went on, “There was a physiological
reason for that. I’d ask no judgment on her, nor offer one. Perhaps Megeve is the same; perhaps the
stress of the ionization...” She fixed her gaze on layli-layli, attempting to match the shaman’s intensity.
“Yes, perhaps. We have, none of us, been behaving normally.” Layli-layli calulan lowered her eyes
to study her hands, still bare of rings. “Even I—”
She did not finish, but Tocohl knew she meant her attempt to curse van Zoveel. “Even you,” Tocohl
agreed quietly. It was not accusation, only a statement of fact.
The Yn shaman was quiet for a long moment, then she turned her dark gaze on the arachne. “
Maggy-maggy... what is the probability that Megeve sabotaged Oloitokitok’s equipment as well?”
Alarmed, Tocohl began, (Ma—) She had no chance to complete her intended warning. “I’m still
gathering data,” Maggy said without hesitation. “The probability is insignificant without sufficient
information.”
Tocohl, who had been about to instruct Maggy to say just that, whatever the truth might be, was
startled. (True, Maggy?) she demanded privately.
(Yes and no,) Maggy said, in the same mode, (did I do right?)
(Yes, oh, yes!)
Kejesli burst dripping into the infirmary, braids a-chatter. He reached for the roof, found none, and
lurched toward them. Behind him followed Buntec and John the Smith. “So what’s up?” said Buntec.
(—Later,) Tocohl added, and silently blessed Maggy’s understanding of verbal shorthand. Aloud she
said, “Give me a hand to the console.—Maggy, bring the arachne. We have something we need your
opinion on, Buntec.”
The arachne scurried past. By the time Buntec and John the Smith had helped Tocohl to the console
and eased her into the chair Om im brought, the arachne was ready to display. To the arachne’s right,
Om im placed a second chair. Tocohl patted it. “Sit, Captain. It’s a long story and you’ll need a surface
to clutch.”
He sat and gripped the edge of the console. “You’ve found a language... ?”
“That I’m still working on. This is somewhat more pressing a problem.”His face darkened. Tightening his grip until Tocohl could almost count his pulse in the risen veins, he
said, “Tell me.” The rest crowded in to watch over their shoulders, John the Smith carefully choosing the
“high status” side of Om im.
“Maggy, we’ll start with Timosie Megeve’s explanation of the transceiver failure.”
Maggy obliged with the tape she’d recorded from Tocohl’s spectacles, adjusting it slightly to avoid
the usual distracting jumpiness caused by minute movements of Tocohl’s head. Megeve’s image was only
halfway through the technical part of his explanation when Buntec grunted and muttered something under
her breath.
(Should I stop?) Maggy asked.
(No, go on.) Tocohl sat, patient against the ache in her side. Having finished that tape, Maggy said,
(Now the tape of the transceiver, right?)
For the benefit of Kejesli and the rest, Tocohl answered aloud: “Now the tape you made when
Megeve tried to contact us because you and swift-Kalat were concerned about our safety.” She watched
again as Megeve tried the transceiver, claimed it did not work, opened the service panel—
Buntec stamped her foot, and pushing between Tocohl and Kejesli to address the screen over the
arachne, she began, “You barefoot—”
Layli-layli’s plump hand caught her shoulder in so firm a grasp that Buntec stopped in mid-bellow.
“Curse all you want, Buntec, but do it quietly. Remember Alfvaen.” Buntec resumed her cursing in a
fierce whisper.
As she had done for Tocohl, Maggy froze the image of the ’plate and enlarged it. “He lied!” said
John the Smith, then immediately repeated it in a harsh whisper.
For a brief moment, everyone muttered and whispered at everyone else simultaneously; then, as one,
they deferred to Buntec for an explanation of what they were seeing.
Buntec gave it, right down to the polished toenails. Then she leaned toward the screen. “Maggy, can
you run the tape back to where Megeve ‘tried’ to contact us?—Yes, there,” she said as Maggy obliged.
“Now give me a close-up of his hands.” Maggy did just as she was instructed. Buntec stamped her
foot—but quietly.
“I don’t understand,” Maggy said.
“It’s a kid’s trick, Maggy. Watch carefully. He’s got a strip of plastic, he edges it under before he
switches on. It looks like the transceiver’s switched on, but the contact isn’t made.”
Kejesli stood so abruptly that his chair struck Buntec’s shin.
Buntec grunted but caught his elbow. “There’s more,” she said, grimly; she reached into her
overpocket with her free hand. Still gripping his elbow, she turned him face on, opened her palm scant
inches from his nose to display the Hayashi jammers. Well past obscenity, she told him only what they
were and how they had been used.
As if to test its reality, Kejesli plucked one of the jammers from Buntec’s palm and squeezed it. His
face turned grim. Dropping the jammer back into her palm, he brought his hand sharply up to the pin of
remembrance he wore on his vest.
Then he spun and punched at the console’s keyboard. Maggy released it to him immediately.
Edge-of-Dark appeared on the display. “Get Dyxte,” Kejesli snapped at her. “The two of you draw
weapons from supplies and report to the infirmary at once. Not a word to anyone else.”
As shocked as she was, Edge-of-Dark snapped back, “We’re on our way,” and was gone.
Kejesli, gripping the console, turned on Tocohl. “Why?” he demanded. “Why would he do it?”
Tocohl laid her pouch across her thigh and drew it open. The piece of moss—still a vivid
red—curled within like some small comfortable animal. “For this, I think,” Tocohl said, taking it gently in
her hand to lay it on the table before him. “Megeve’s sprookje gave me this just before we set out: the
four of us—and Megeve—were the only witnesses.”
“That’s what he thought,” Maggy said.
“Said with just the right emphasis,” Tocohl told her. “Why don’t you show the captain that bit of
tape.”
The display remained blank. “You’ve seen it, Tocohl, or you can watch it on your spectacles—butyou shouldn’t be sitting up. It hurts you to sit up. If you’ll go back to bed, I’ll show Kejesli the tape. Fair
trade?”
Tocohl stared down at the arachne. “No,” she said, “it’s pure, unadulterated blackmail, but you’ve
got yourself a deal.” When she held out an arm for assistance, she found John the Smith ready and
waiting.
“Wait,” said Kejesli. “A gift? This?” He brushed the tuft of moss.
“Looked like more of a trade, actually,” Om im said. “Megeve’s sprookje was hoping for a piece of
Tocohl’s cloak in return.”
“But to keep the sprookjes from being found sentient—”
Tocohl rose to her feet; pain that had gone unnoticed in the excitement returned to drain the blood
from her face. “The Inheritors of God want this world, Captain. I don’t know why. You’ll have to ask
one. And I’d say Megeve’s a likely candidate.”
“Waster!” Kejesli spat the word, so angry he lapsed into Sheveschkem, “So you’d burn Veschke,
would you, waster—”
Tocohl, who had never heard a Sheveschkemen use that strongest of all condemnations, started so
violently that it brought her a stab of pain. She pressed her hand to her side; Maggy stiffened the 2nd skin
against her.
Kejesli’s eyes widened. “My lady—” he began, still in Sheveschkem. Then he shook himself and
splayed his fingers at his throat. In GalLing’, he said, more gently, “Get her to bed; let me see this tape.”
He seated himself once more before the console.
By the time John the Smith had eased Tocohl down onto her cot—Om im keeping close behind to
assure himself the Smith would handle the task correctly—Kejesli was already concentrating on the tape.
A lightning flash illuminated two shapes making toward the infirmary door, distorted beyond
recognition by torrential rain. Buntec stamped across the room to peel back the membrane and peer into
the downpour. She stiffened momentarily. Then, with an air of courtliness quite unlike her, she drew aside
the membrane to allow the two newcomers entry and in a tone of astonishing sweetness she said, “Do
come in, Timosie. We were just speaking about you.”
Maggy withdrew the arachne’s adaptor from the console, stepped it back. Kejesli rose to his feet.
Om im was halfway across the room, aimed like a thrown blade at the door. John the Smith stepped
forward as well, blocking Tocohl’s view. She attempted to rise but layli-layli calulan pressed her firmly
back onto the bed—and pushed John the Smith to one side so Tocohl might see from where she was.
Hitoshi Dan began, “Timosie here tells me—” He got no further. Taking in Kejesli, he took an
instinctive step backward, and then, just as instinctively, turned to look at the object of Kejesli’s fury.
Bewilderment crossed his face when he found only Megeve; he turned back to Kejesli, his expression
clearly seeking explanation.
“One death on this world was not enough for you, dastagh!” The captain took a single step toward
the Maldeneantine. “You needed four more!”
Megeve paled. For a brief moment, Tocohl thought he might turn and flee. But there was nowhere to
run to. As if he had read the thought, Megeve put his hand to his belt—a cocky sort of gesture, but
surprising under the circumstances.
He pushed past Hitoshi Dan and advanced toward her. “What have you done, Hellspark?” he
demanded. “Months of work ruined; sacrifices made in vain.”
Matching his anger, Tocohl said, “What work? What sacrifices? The four of us? The sprookjes?”
That he did not answer. Instead he began to chant in a language Tocohl did not, for all her
experience, recognize. (Maggy, record.)
(I am.)
Megeve continued toward her, his steps timed to the rhythm of the chant, all expression draining from
his face.
As he reached the center of the room, Tocohl stretched out her hand to warn him.
Too late, she saw him blindside Om im. The result was almost too fast for the eye to follow. Om im’s
blade flashed out and up. Megeve screamed in anguish, his hand spattering an arc of blood as he jerkedout of the small man’s reach.
The knife flashed a second time. Tocohl had barely enough time to gasp as it struck Megeve’s side,
ripped upward. Megeve’s belt clattered to the floor where Om im kicked it deftly across to John the
Smith. “Put a foot on that, John,” Om im commanded, “and don’t let it up until I’m done.”
To Megeve, who nursed his injured hand, Om im went on, “Move an inch, Megeve, and you’re
dead.” His glance flicked the length of the Maldeneantine. “I won’t bother with a body blow—your 2nd
skin would deflect it—I’ll go straight for your throat.”
Timosie Megeve froze, good hand gripping injured. The grip was tight enough to whiten his knuckles
but insufficient to stop the steady drip of blood.
Even Kejesli was stunned into inaction. Thus the tableau held—until the arrival of Dyxte and
Edge-of-Dark. Om im, still seeming the only one of them capable of action, said, “Target Megeve. If he
moves so much as a hair, stun him.”
Dyxte, bewildered, glanced at Kejesli. “Do it,” Kejesli said, and the two obeyed. With a sigh of
relief, Om im wiped and sheathed his knife. Then he crossed to John the Smith, where he bent to pick up
Megeve’s belt.
“You look puzzled, Ish shan.” He grinned, a strained expression more of relief than humor. “Could it
be that your education is sadly lacking?”
“It could be,” Tocohl said; attempting to match the lightness of his tone, she succeeded only in
matching its forced quality.
“Watch and learn.” He bowed, added a flourish of cloak. Then he strode a few steps from them,
raised the belt to shoulder height—
There was a sharp spitting sound, followed almost immediately by a rasp of metal on metal. A small
but deadly looking dart embedded itself in the infirmary wall. Buntec whistled.
Maggy said, (You didn’t have your hood up, Tocohl. He could have killed you with that!)
“I’m sorry, layli-layli calulan,” said Om im, staring at the dart. “I didn’t think it was that powerful.”
Kejesli looked from the dart to Megeve. “Strip him down. I want his 2nd skin. We don’t know what
else he might be carrying.”
Under the watchful eyes of Dyxte and Edge-of-Dark, Megeve was stripped and marched across the
courtyard where a cell was hastily improvised from a section of the supply room.
Tocohl could hear Kejesli’s shouted orders even over the rumble of the passing storm. Then she
turned to face the small man who stood patiently beside her bed. “I live,” she said in Bluesippan, “I thank
the sharpness of your blade.”
He grinned; this time the grin was genuine. “Ish shan,” he said, in passable Hellspark, “everybody
needs a friend, some time or other, even a legend.”
He turned the belt over in his hands. “I saw a weapon much like this used when I was on Hayashi,
but it never occurred to me that Megeve—” He met her eyes ruefully. “I didn’t even think of it—until he
made that move.” Flawlessly, he reproduced Megeve’s cocky gesture.
“To arm the weapon then,” Tocohl said. “May I see?” She held out her hand toward the belt, but
Om im drew it from her reach. Mildly puzzled, she said, “How can it hurt me? The one he intended for
me is there.” Her glance flicked to the wall where swift-Kalat held the arachne up for a close look at the
imbedded dart.
Om im called out, “Don’t touch it, swift-Kalat. If the barbs are sprung within the wall, it will be safe;
but it may not have penetrated far enough, in which case you could get your fingers rather nastily sliced.
And Megeve seems capable of having poisoned the dart, too.”
To Tocohl, he explained, “The standard model carries four darts. Let me disarm it for you.” He made
a swift motion, then aimed it again at the wall, well away from swift-Kalat; this time nothing happened.
He handed the object to Tocohl to examine to her and Maggy’s satisfaction.
As Tocohl turned it over wonderingly in her hands, she asked aloud, “Why? Why should he try to kill
me? It’s too late.”
Om im looked at her in surprise. “You of all people, Ish shan, should know that even gods take
vengeance on those who thwart them.”Tocohl slept again; layli-layli calulan had seen to that, and Maggy rather wished she could
understand how the trick was done. She wondered if it weren’t simply a matter of perception, much like
the difference between her perception of the Ringsilver magician and Tocohl’s. She tagged the matter for
further investigation.
Meanwhile, with a camera eye recording those events in which Tocohl would most likely have an
interest, she sent the arachne after Kejesli who was still in the storeroom interrogating Megeve.
“Worryin’ a wound,” Buntec called it, in a tone Maggy interpreted as acid. “Don’t know what he
thinks he’ll hear. He hasn’t liked any of Megeve’s answers so far.” She pounded the side of the
storeroom and bellowed, “Maggy’s comin’ in with me to eyeball the little b.f.f.”
Dyxte opened the door to them. “Just don’t get between me and the target, Buntec.”
“Hah. Where’s he gonna run?”
“He was ready to kill Tocohl,” Kejesli snapped back, “I won’t give him another chance—at any of
us.” Edge-of-Dark’s weapon never left its target and, as Maggy’s arachne scuttled by, Dyxte retrained
his gun in the same direction. “Okay,” said Buntec, “I take your point. Get anything out of the waster
yet?”
“You call me a waster,” Megeve said suddenly, “what do you see here? Nothing. No
communications, no homes, no farms. They waste the resources of an entire world.—Even if they are
sentient, the sprookjes don’t deserve this world. What have they done with it? Nothing!” Despite the
emphasis, he spoke calmly, as if he might convince them of the justice of his actions.
Kejesli’s scowl deepened—Maggy had no trouble interpreting that expression. In a voice that
sounded to Maggy carefully controlled, deliberately even, he said, “It’s a lousy planet. There’s too much
rain, too much lightning. Even in sunshine the air stinks of storm. But if the sprookjes are sentient, it’s
theirs—and I’d burn beside Veschke before I’d let you take it from them.”
“You still don’t understand—” Lifting his hands, Megeve took a step forward. Dyxte and
Edge-of-Dark dropped to firing position.
“Then explain it to me,” Kejesli said.
Warily Megeve lowered his hands, clenching them into fists. “The world’s too valuable for
sprookjes,” he began.
With a snort of disgust, Kejesli turned his back on the Maldeneantine and strode for the door.
Megeve called hastily after him: “Listen to me, Captain—the lightning rods! They’re biological
superconductors!”
Buntec gave a grunt of surprise; interest took her a step closer to Megeve, who immediately turned
to address her. “Buntec,” he said, “you understand, don’t you? Superconductors that don’t require an
entire advanced technology to produce. Superconductors that don’t need to be artificially maintained by
cooling!” Kejesli had turned back to stare at Megeve with widening eyes.
“Yes,” said Megeve. He was smiling now, and Tocohl would have called his tone triumphant.
“Forests full of superconductors! Now you see why Flashfever is too valuable for sprookjes. Give it even
a moment’s thought and you’ll admit I’m right.”
Maggy was thinking indeed; she knew precisely what that would mean, to her and to Tocohl, if what
he claimed about the lightning rods were true. “Cheap memory!” she said to Buntec. “Do you think the
sprookjes would trade for moss cloaks?”
A short, sharp laugh answered her; there was no warmth in the sound. “There, Megeve,” Buntec
said, “even the kid knows better.” Looking down at the arachne, she went on, her voice gaining warmth
as she spoke, “They’ll trade for something, Maggy; and I’d trust a Hellspark to find out what every time.
Don’t you worry—if you can’t figure out a fair trade, Tocohl will.”
“Good,” said Maggy. She split her attention at once, setting a part to work on possible trade goods
for the sprookjes. Given their reaction to Alfvaen’s blood sample, wine would not be high on the list of
probabilities. Biologicals—like Tocohl’s moss cloak—there was the place to start.
The rest she devoted to Megeve. His smile had gone. Again he tried to back away, as if Kejesli’s
look were as dangerous as layli-layli calulan’s curse. Already backed against the wall, he sank instead,sliding his shoulders down until his knees suddenly buckled and he sat on the floor, dropping his head
between them.
It didn’t seem an opportune time to ask for further information, but Maggy wasn’t sure when she’d
have another chance. And Megeve had seemed willing to talk just a moment ago. “I don’t understand,”
she said. “Why would that make Flashfever ‘too valuable for sprookjes’?”
When Megeve made no answer, Maggy said, “Buntec?”
“C’mon, kid,” Buntec said, “let’s get you out of here. He’s what my momma would have called a
bad influence. And Tocohl wouldn’t like you hangin’ around him. Neither of us should be hangin’ around
‘im.” She started for the door, clearly expecting the arachne to follow.
“Bad influence?”
“Somebody you shouldn’t imitate if you want to grow up to be a human being.”
Maggy was not sure that applied to her, but since Buntec seemed to mean it sincerely, she decided to
go along with it, and with Buntec—at least until she had a chance to talk the matter over with Tocohl.
She sent the arachne trotting after Buntec.
Once outside, she found the two of them momentarily alone. Rain still battered the arachne; she did
not, however, expend the energy needed to compensate for the distortion it caused the arachne’s eye.
Instead she sent the arachne at full speed after Buntec. “Buntec, wait!”
Buntec splashed to a halt in mid-puddle. Hands sheltering her eyes and face, she bent to the arachne.
“Buntec, I don’t understand. And it’s secret so I can’t ask anywhere else. I’m not a kid, I’m an
extrapolative computer, and I don’t understand why Tocohl wouldn’t want me ‘hangin’ around ’im.’”
“You may be a computer, kid, but that”—a sharp jerk of her elbow toward the storeroom made
Megeve the referent—“that is a villain, and nobody’s momma wants her kid hangin’ around villains. You
got it now?”
“Yes,” said Maggy, for that one word, villain, explained it all. “I’ve got it now. Thank you!”
“You’re welcome—now let’s get the hell out of the rain before we both get zapped.”
Chapter Fourteen
MAGGY HAD MUCH to think over—so much, in fact, that she spent most of the night swapping
data from active files to inactive and back again, cross-referencing wherever she saw the need. She
regretted that not all of her memory could be active simultaneously. Still, she supposed this to be what
Tocohl called “concentrating on one thing at a time.” If it didn’t hurt Tocohl’s thinking, Maggy saw no
reason it should hurt hers.
Even in the infirmary the sound of thunder could be startlingly loud. Alfvaen had awakened to it
twice; each time, layli-layli called on Maggy to interpret. Maggy did the best she could but Alfvaen still
made little sense. As layli-layli did not seem disturbed by this, Maggy was content to wait and watch
through the arachne at her side. Swift-Kalat did the same, although Maggy would hardly have described
him as content.
Tocohl slept on, stirring only slightly at the sound of thunder. Maggy kept an active watch on the 2nd
skin sensors. The normalcy of the readings reassured her, as did the fact that Buntec and Om im took
turns guarding the infirmary throughout the night.
Morning came but the sky remained a dark patchwork of clouds, stitched with flashes of lightning.
Despite all the questions, cued and waiting to be asked, Maggy did not wake Tocohl at the customary
time. Rest, layli-layli had assured her, was what Tocohl needed most to heal. As long as there were no
sprookjes in the camp, there was little for Tocohl to do but heal.
Still, questions were the next highest priority. Maggy sent the arachne across the room to peer up at
Om im. “Will you tell me what I missed?” she asked, phrasing the question as Tocohl would have. Om im
gave a sidelong glance at the sleeping Tocohl. Maggy said, “If the thunder doesn’t wake her, we’re not
likely to.”
“True,” he said. He reached down and lifted the arachne to place it on the edge of Tocohl’s cot, thecamera eye at a level with his own. “Where shall I start?”
“Where the daisy-clipper went down,” Maggy said and settled the arachne to record it all. The
account was far more interesting than Maggy had expected: not only did it differ in detail from Buntec’s
account of the same circumstances, but it differed in style of delivery as well. Om im’s words were
gentler, his gestures more extravagant—as if to compensate for the softness of his voice.
From time to time Tocohl stirred. Maggy’s readings showed her close to consciousness—then, as if
soothed by the sound of their voices, she would drift back into sleep. Layli-layli calulan woke to find
Om im describing the duel between Alfvaen and Tocohl. To Maggy’s surprise, she did not interrupt.
Without a word, she joined them—to the proper side of Om im—and lightly touched first Tocohl’s rib,
then her temples. What she found seemed to satisfy her, for she smiled and said only, “Good morning,”
before she moved on to check Alfvaen.
Maggy was glad layli-layli had not felt it necessary to interrupt. Maggy was disappointed that she
had not seen the duel between Tocohl and Alfvaen for herself, but Om im’s account was considerably
better than Buntec’s. Trained in a different form of dueling, he was a better observer of both the
movements and formalities involved.
He was in the midst of demonstrating those movements for her when van Zoveel burst into the
infirmary, flinging droplets of water from the end of every ribbon on his tunic. As if it had been a planned
part of the demonstration, Om im crossed to intercept him.
Not, Maggy was sure, that he thought van Zoveel any threat to Tocohl; he simply did not want van
Zoveel to wake her. Enhanced sound confirmed this. Keeping his voice very low, Om im said, “Not yet,
Ruurd. Layli-layli calulan says she needs the sleep.”
Van Zoveel fairly stamped with impatience but he too kept his voice low. “You don’t understand,” he
said, “Captain Kejesli said she had gotten Megeve’s sprookje to echo her. I must know how. There’s a
clue that I’m missing.”
“You’ll learn soon enough,” layli-layli calulan said, joining the two. “Give her a few more hours of
healing.”
“I don’t need her for heavy lifting, layli-layli. I need her for talking.”
“Not now,” said layli-layli calulan.
“I waited all night...”
“Then you’ve had practice. You can wait out the storm. I won’t have her disturbed until there’s a
reason for it, and she can hardly demonstrate without a sprookje to echo her.”
Glaring, van Zoveel started once more toward Tocohl but Om im spread his hands in the Bluesippan
shrug, discreetly cutting off his approach. “We’ll both have to wait for our answers,” said Om im.
“You will wait elsewhere,” layli-layli told van Zoveel. Maggy watched as the two of them glared at
each other for a long moment, then van Zoveel gave way, and stamped back to the door. “If she
wakes...” he began.
“If she wakes, she’ll have breakfast,” layli-layli calulan told him, “and then I’ll notify you.”
Van Zoveel muttered a word under his breath that Maggy had been taught was impolite to say aloud
in Zoveelian society, but he left nonetheless. Neither of the others took offense; clearly they did not
recognize that any had been given. Maggy wondered if she should explain it to them but decided against
it. All but a handful of times that Tocohl had been in an analogous situation, Tocohl had said nothing.
Maggy settled on adding that query to her growing list.
She wanted Tocohl awake as much as van Zoveel did, she found. Experimentally, she rocked the
arachne from side to side, imitating to the best of its ability his impatient stamp, to see if that had any
effect on the matter. It did nothing to help, except obliquely—for Tocohl came awake.
“Your pardon, Tocohl. I did not mean to wake you. I was only experimenting.”
Tocohl blinked puzzled eyes at the arachne. “Experimenting at what?”
Showing her the bit of tape of van Zoveel, Maggy explained. By the time she was done, both Om im
and layli-layli calulan had joined them and were watching the arachne with thoughtful looks. “It doesn’t
do anything,” Maggy concluded, “I guess that means it’s a null gesture?”
With a glance at layli-layli calulan, who smiled in return, Tocohl corrected, “I’d say it worked,Maggy. That’s an unconscious attention-getting device. You used it correctly, and for you, it worked.
You have my attention.”
“But I didn’t mean to wake you. Layli-layli calulan said you needed the rest. That’s why she told
van Zoveel to go away. “And as the possibility had just now occurred to her, she added, “Now she’ll
make me go away too!”
“No,” said layli-layli calulan with another smile—this one directed at the arachne. “You can stay,
Maggy; talking to you comforts Tocohl. Read your sensors: they’re steady.”
What she said was true, Maggy saw, although how layli-layli calulan could tell she did not know.
Again, the ability must have something to do with the shaman’s different perception.
Once more, a figure burst through the door. This time layli-layli calulan’s response was quite
different. “Ah,” she said, “John.” It was as if she used the name as greeting. She turned to Om im and
said, “John the Smith will keep watch for a while. It’s time you had a chance to clean up.” Om im
glanced at Tocohl, but layli-layli calulan went on, “Go. I’ll see she gets breakfast.”
Tocohl said, “Go ahead, Om im. I’ll be fine.”
Maggy considered the stubborn set of Om im’s face and said, “I’ll call you if there’s trouble.”
That seemed to reassure him. He woke Buntec and together they plunged into the courtyard, lost to
sight in the downpour even before the door membrane slapped shut. Thunder rattled the room.
Approaching layli-layli calulan on the side which gave him high status, John the Smith said
something into her ear which Maggy couldn’t make out; enhancing only enhanced the sound of thunder.
The Smith looked serious and seemed surprised when layli-layli calulan answered him only with her
most beautiful smile.
(Tocohl?) Maggy asked.
(I’m sure we’ll find out,) Tocohl said. The sensors in her 2nd skin showed that she was not
disturbed. Maggy folded the arachne’s legs and settled it beside Tocohl as layli-layli calulan
approached the cot once more.
For a long moment, layli-layli calulan looked down at Tocohl and at the arachne. At last she said,
“Hitoshi Dan believes that maggy-maggy is an extrapolative computer.”
“He’s right,” Tocohl said. She glanced across the room, pointed a polite little finger. “Swift-Kalat
would tell you the same.”
Layli-layli calulan smiled beautifully once again. “I do not think so, tocohli, not in Jenji.”
That sent spikes through all the sensors in Tocohl’s 2nd skin. “What do you mean?”
The smile went from layli-layli calularfs face. “Have you thought what the Hellspark ritual of change
might mean on some worlds?”
Sensors spiked again, this time higher than the first, but no change showed on Tocohl’s face. Tilting
her head to the side, she said, “No, I can’t say that I have.”
“I thought as much. You and maggy-maggy have much to discuss. I’ll see to Alfvaen and then to
your breakfast. Think over what I’ve said.” One last time, she gave Tocohl her most beautiful smile, then
she turned and walked away.
(Maggy? I hope you know something about this “Hellspark ritual of change” business?)
(I lied to swift-Kalat. I would have asked you, Tocohl, but I couldn’t.)
(Don’t get excited. Just tell me what happened.) She shifted in the cot—getting comfortable, as she
called it. Something she did before someone began a long report.
Maggy took this to mean she wanted the full story. She began at the point where she had lost contact
with Tocohl, and because she needed Tocohl’s advice, she explained her own actions as she went along.
At the “Hey presto!” Tocohl laughed aloud and said, (Oh, Maggy, that was perfect!)
Tocohl’s laugh alleviated any further worry: the reasoning that had led Maggy to use the “Hey,
presto!” as she had was sound.
(So that’s how swift-Kalat got to see layli-layli when she was in deep mourning,) Tocohl said, (I’d
been meaning to ask you about that.)
Still smiling, Tocohl closed her eyes, and Maggy could tell from the sensors that she was getting tired
again. Layli-layli calulan was right: healing required rest.(You should sleep again,) Maggy said, and seeing layli-layli approach with a tray, she repeated it
aloud, “She should sleep again, layli-layli calulan.”
“She should eat first.”
Maggy checked the sensors again. “She’s very tired.”
“Maggy,” said Tocohl firmly, “that’s nothing compared to how hungry I am. Which doesn’t show on
your sensors, so you know absolutely nothing about it. Greed you’ve a fair grasp of; hunger, no. Let me
eat.” Tocohl eased herself cautiously into a sitting position—Maggy was pleased to note that the action
no longer made her sensors spike as emphatically as it had the previous day—and accepted the tray from
layli-layli calulan.
“Thank you,” said Tocohl. She gave a sidelong glance at the arachne which seemed to imply that she
thanked the shaman for something other than the meal. With a similar glance at the arachne, layli-layli
calulan said, “You are quite welcome.”
Their manner gave Maggy cause for concern. When layli-layli calulan returned to Alfvaen’s side to
wake swift-Kalat, Maggy said, (Did I do right? You told me I could lie.)
(You did right, Maggy,) Tocohl said, around a mouthful of food. (I hadn’t intended to grant you
blanket permission, but you seem to use discretion—and since it saved me a long, painful walk through
the flashwood out there, I can scarcely complain.)
(I won’t lie to you.)
(I won’t lie to you, either. Between friends, it’s not good policy.)
That brought a sudden sense of conflict. (Is swift-Kalat my friend?)
Tocohl stopped eating. (I think he’d be able to say so.)
(But then I shouldn’t have lied to him. Lying causes him distress, even when a stranger only mentions
it.)
(Maggy, I don’t think he’ll hold it against you. Even Jenji permits the establishing—the creating—of a
useful ritual. If I put it to him that you did not lie, but rather created what was needed to suit the
circumstances, he won’t be distressed. Any more than he’d be distressed if someone in the camp had
built a machine that overrode the Hayashi jammers and allowed him to find us safely. Do you see the
distinction I’m making?)
(I think so. I’m not sure.)
(Well, let me assure you the ritual you created is useful, and will be used in the future. Layli-layli
calulan has plans for it already. And I know a dozen traders who’ll be very happy to turn it to their
advantage: all of them male and all of them, up to now, unable to trade successfully with the Yn.)
Maggy skimmed her files on the Yn one more time. (You mean, if you make Geremy your sister, he
could deal with the Yn female to female?)
(Compared to some of the other male traders, Geremy has done pretty well. He’s always had the
advantage of his sharpness, and of his name. If we make him our sister, he’ll—)
Sensors in the 2nd skin spiked; this time pain was not the cause. Maggy said, (What is it?)
Instead of answering directly, Tocohl frowned at the tray of food. Maggy judged that she was
thinking something over and waited politely, not wishing to distract her. She was pleased that Tocohl ate
while thinking—that, at least, meant she could rest as soon as she was done. Meanwhile she took the
opportunity to update her files on lying.
She found herself at once faced with another dilemma. While layli-layli calulan approved of lying,
and did it well herself, she also fell into the category of friend. Maggy knew that friend overrode a great
many other priorities, both behavioral and cultural. Tocohl had approved of her action at the time, but
Maggy needed to know why, in order to know whether the same approval was still operative. The
question was imperative, worth distracting Tocohl to ask. (Tocohl?)
(Um?)
(Should I tell layli-layli calulan the truth?)
Tocohl laid the tray aside. (The truth about what, Maggy?) she asked as she eased herself down.
Maggy replayed the bit of tape. Once again, layli-layli calulan asked what was the probability that
Megeve had killed Oloitokitok. Once again, Maggy answered that her information was insufficient.(Stick to your story, Maggy. I’d have asked you to lie in that case anyway. I’d like to hear your
reasoning, though.)
(Layli-layli calulan intended to kill van Zoveel because of Oloitokitok’s body. The probability—do
you wish the figures?—is extremely high that she would kill a man she thought responsible for
Oloitokitok’s death. While the odds that Megeve was responsible are low, perhaps due to insufficient
information, layli-layli calulan acted upon lower odds when she followed the search technique I
suggested to look for you.)
(Maggy, I’m proud of you. Your reasoning is impeccable.—Now, add this to information about
Megeve, if it isn’t already in your stores. In Yn, the sound ee has strong meaning. Do you understand
that, in some cultures, specific words are thought to have power beyond their simple communicative
use?)
(Sympathetic magic,) Maggy said. (When you feed a code word into a computer, it brings an entire
program into being. Is that the derivation of the idea in human context?)
Tocohl grinned. (I rather doubt it: there were humans and sympathetic magic long before there were
computers, but that’s a good analogy.)
She went on, (All right. Y is the name of layli-layli’s world, that world being the source of all life
and, hence, the greatest, most potent magic of all.
(Please remember, I’m describing a cultural attitude, not a fact.—And the title laylee-laylee calulan
also indicates a power, the doubling of the term expressing her espabilities.)
Maggy saw what she was getting at and interrupted to save her further explanation: (So Geremy and
Timosie and maggy-maggy are all names of power!)
(That’s it! Not as potent, perhaps, as layli-layli calulan, especially now that she knows you’re an
extrapolative computer, but your name might be sufficient to give your words more weight with layli-layli
than anyone else’s.)
She twisted to address the arachne directly. Clearly, she used the arachne as a focus sometimes, too;
Maggy moved it to a position that did not require her to turn.
(Thanks,) said Tocohl, (I see I’m falling into that little habit, too. You shouldn’t have bounced it on
the bed; it did get my attention, in more ways than one.) She was silent for a long moment, then she went
on, (I’m thinking that the ee in Timosie Megeve might have been very important in all this...)
(I don’t understand.)
(I’m thinking that Timosie’s very name might have given his words more weight to Oloitokitok.
Suppose Megeve suggested to Oloitokitok that no one would believe, for instance, that the two of them
had seen the sprookjes behaving in a sentient fashion.) She focused her eyes at some point beyond the
arachne. (Or suppose... Maggy, Sunchild was Megeve’s sprookje! She was willing to chance a ride in a
daisy-clipper! The equipment failures... Megeve’s acting as if the sprookjes would mess with his
equipment! Suppose the sprookjes did mess with the equipment. Suppose...)
(Too many supposes,) Maggy interrupted. (I can’t give an accurate probability on any one of them.
I’m not sure I even follow your line of thinking.)
(Why would Megeve want Oloitokitok dead?)
(Well,) said Maggy, knowing that esthetically such an important question required a pause before the
answer, (he wanted you dead, if I understand this correctly, because he thought the four of you were the
only witnesses to the sprookje’s gift.)
(Yes. Suppose there had been an earlier gift, one only Megeve and Oloitokitok witnessed.)
(Why wouldn’t Oloitokitok tell everyone about the gift?)
(That’s where the name Timosie comes in. If someone named John the Smith had said, “They won’t
believe us. Let’s wait until we can get some real proof,” Oloitokitok would have said, “There are two of
us. We’ve both seen it. Let’s tell everyone and they’ll help us look for real proof.” But if Timosie said the
very same thing, Oloitokitok would have said, “All right, let’s wait until we can get some real proof.”)
(Tocohl, that’s silly.)
Tocohl laughed. (I never said it wasn’t. But I’ve seen it happen. Geremy—because of the ee in his
name—does a rousing business trading with Yn males. They think he’s special and important.)Only one response seemed appropriate. Maggy made the rude noise.
Tocohl laughed again. (Agreed,) she said, (but that doesn’t change the possibility. I never said human
beings were logical, or reasonable, or even sane.)
(I know,) said Maggy. (But they are very confusing.)
(Admit it: we keep you from being bored.) Tocohl flashed a smile at the arachne that Maggy judged
every bit as beautiful as layli-layli calulan’s.
(Yes,) Maggy said, (you keep me from being bored.)
(Good. Now think about this. Megeve never took you into account as a possible witness. There was
someone else he never took into account as well...)
(The sprookjes.)
(That’s right. If only we can find the words to ask, Sunchild may be able to tell us what happened to
Oloitokitok. In the meantime, I agree with you: it’s safer not to give layli-layli any odds at all that might
make her do something rash.)
Through the arachne’s eye, Maggy saw layli-layli calulan approach long before Tocohl reacted to
her footsteps and turned. Tocohl began to rise, but layli-layli said, “This is only a lull between storms,
Dyxte tells me. There’ll be no sprookjes for several hours, assuming the next is the day’s last.”
“So Tocohl should sleep,” Maggy said aloud.
“Yes.” Layli-layli calulan stripped her rings from her fingers and laid them beside the arachne, giving
Maggy an excellent chance to observe them closely. To Maggy’s disappointment, they seemed to be
ordinary bluestone, so she recorded the movements of layli-layli’s hands instead, first as they touched
the injured rib. Maggy could tell from the sensors that Tocohl was reinforcing her healing ritual
simultaneously. Then, as layli-layli’s fingertips brushed Tocohl’s temples, the same sensors began to
indicate drowsiness.
Tocohl sighed and sank farther into the cot. Her eyelids parted ever so slightly. (Maggy,) she said,
glancing sleepily up at the infirmary roof, (where are you?)
It took Maggy only a split second to weigh the pros and cons. Then she formed the image of the
Flashfever starfield Tocohl would have seen from her position had the roof and light pollution of
Flashfever not intervened. She hesitated a moment—Flashfever had no constellations she knew of so
there were no established groupings of stars. That made the task more difficult, but at long last she
decided upon an aesthetic place to put the glittering point that would represent herself. She added an
indicatory arrow and projected the resulting image onto Tocohl’s spectacles, all before Tocohl had
drawn another breath.
And when she drew it, it was a sudden, sharp intake... Maggy knew a sound of delight when she
heard it from Tocohl, and the slow, drowsy smile that followed merely confirmed Maggy’s assessment.
(I missed you, Maggy,) Tocohl said, very softly.
(I missed you too,) Maggy said; then she was silent, letting Tocohl drift into sleep.
All in all, Maggy concluded, she had done right. Tocohl knew that she could not see the ship from
here, so Maggy had not lied to a friend. She had told a pleasing story, and she was very proud of her
new ability.
The storms continued throughout the day and into the night, but morning at last brought to the skies a
clear pale-blue stillness. A fresh wind swept the last tang of ozone from the camp. Buntec took a deep
contented breath of it, scrambled into her boots, and skipped down the steps of her quarters into the first
pale rays of sunlight.
She was the first. If there had been a gong to ring to wake the other members of the team, she’d
have rung it. But there wasn’t—and for the moment, she could not bring herself to venture into shadow
long enough to knock at various doors.
The stillness was loud enough to wake others. One by one, the surveyors stumbled out, blinking up
into the sky, and smiled. Buntec waved at Edge-of-Dark who waved back, glanced down at her own
bare feet, reddened, and darted back inside.
It took Buntec a second or two to realize just what sequence of events had sent Edge-of-Dark backinto shadow. Once she had, she was spurred into action without any further thought.
She splashed across the compound, raced up the steps to Edge-of-Dark’s quarters, and stuck her
head in, uninvited. “Don’t,” she said, “don’t, Edge-of-Dark. I can stand your f-feet”—though the word
was hard to get out when it wasn’t an obscenity, when she didn’t mean it as an obscenity, she managed
to say it and go on to the important part of her objection: “Don’t miss the sun just because of me!!!”
Edge-of-Dark paused in the midst of pulling on her second boot. Her jaw dropped, then closed
abruptly to draw her mouth into a brilliant smile. “Buntec,” she said, “you are one of the nicest people
I’ve ever met. That makes up for any sunshine I missed for these!” She pulled the boot to, tapped it with
a long green nail.
Embarrassed, Buntec ducked her head. “I wish,” she said, “there weren’t so many traps between us.
I like you too, Edge-of-Dark. I like you a lot. I don’t know how to get from here to there”—she
gestured at the expanse of floor that separated them and found she knew the perfect
expression—“without, as the Trethowan say, putting my f-foot in it.” It was minimally easier to speak the
word the second time—and Edge-of-Dark’s peal of delighted laughter made it worth every bit of the
effort.
Still laughing, Edge-of-Dark stood and straightened. “I will close the distance, too. If we warn each
other, look out for each other, we will make it.”
“Yes,” said Buntec, lifting her head and grinning. For a long moment, the two of them simply stood
there grinning at each other across the small separating distance, then Buntec said, “Sunshine!”
“Yes,” agreed Edge-of-Dark. “Would you give me a hand with my table and bowls, Buntec? The
sprookjes need help across a distance too.”
The two of them carried the small table down the steps to set it in the sunlight. Edge-of-Dark made a
quick trip back for bowls and scissors and the rest of her odd paraphernalia.
Buntec leaned back, stretched her legs. When she sat up again, she caught movement at the edge of
camp. “Your sprookjes are coming,” she called up to Edge-of-Dark. There was no response. Well, it
required none, thought Buntec, and settled in to watch the sprookjes appear in the flashwood and start to
work their way through the fence.
She blinked suddenly and rose to her full height, shading her eyes and squinting. Surely she was
imagining it—but she hoped she wasn’t.
Those surveyors closest to the sprookjes turned, gave excited exclamations, tapped others. No, not
imagining things. “Edge-of-Dark, get out here!”
Beside her, Edge-of-Dark responded only to her urgency of tone. “I’m here,” she said, then
absently, “I’ll need to pick flowers. Would you like to come with me?”
Buntec dragged her eyes away from the sprookjes to glance down. Edge-of-Dark was
contemplating her paraphernalia. Buntec caught her by the shoulders, turned her to face the sprookjes.
“Tell me,” Buntec demanded, “tell me if you see what I see! Look at the sprookjes and then tell me if you
need to pick flowers!”
And in Edge-of-Dark’s widening eyes, Buntec found all the confirmation she needed. She turned
again to the perimeter fence.
A dozen sprookjes were cautiously assisting each other through the barbed-wire barrier—and each
carried an armload of brilliant blooms and leaves of all sizes and shapes.
Edge-of-Dark started forward, as if drawn by all that color and noise, but Buntec caught her
shoulder. “Stay here. Stay here. They know where to find you.” She stared again at the approaching
sprookjes, sure beyond dispute that all would come straight to Edge-of-Dark. She sprang from the steps.
“I’m gonna get Maggy. She should be taping this.”
Dazed, still caught up in the sight, Edge-of-Dark sat abruptly. “Yes,” she said, “I should stay.” She
managed to tear her eyes away to look momentarily at Buntec. “They’re coming! They’ve brought
flowers!”
Buntec could feel her own fierce grin. “ Your art they recognize,” she said, for the pleasure of putting
it into words. Then, still grinning, she started for the infirmary.“I don’t care,” Maggy said, “I’m going to wake her. She’ll want to see this.” The emphasis was so
startling that Tocohl at first thought herself in the midst of a particularly vivid dream, then waking and
simultaneously catching the sense of the distantly heard words she realized she was hearing Maggy’s
voice relayed through her implant.
Buntec’s voice, seeming equally astonished by the emphasis Maggy had put on the I, said,
“Whatever you want, kid. Make your own decisions, take your own lumps.”
“Lumps?”
There was a pause as Buntec sought a way to explain. “Take the consequences, good or bad.
Tocohl might not take kindly to being kicked awake, not even for this.”
(Maggy?) Tocohl asked. (What’s going on?)
An image of Om im looking up flashed briefly onto her spectacles; from the angle, someone tall must
have been holding the arachne above his head. “She’s awake,” Maggy told him, and rather smugly. I bet
I sound like that when I deliver a fait accompli, Tocohl thought.
(Look,) Maggy said and showed her a crowd of surveyors and sprookjes, all jostling about in front
of Edge-of-Dark’s cabin. (The sprookjes brought flowers for Edge-of-Dark.)
Despite the irrationality of the act, Tocohl sat up for a better view, noting with relief that at least sitting
was no longer painful.
(Was I wrong to wake you?) Maggy asked, the smugness gone from her voice.
(No, Maggy. You did just right,) Tocohl told her and couldn’t resist adding, (No lumps.)
The smugness returned. “No lumps,” Maggy informed Buntec.
In spectacles, a sprookje, cheek-feathers puffing, seemed to battle with itself, torn between careful
handling of its bouquet and direct contact with Edge-of-Dark.
Edge-of-Dark watched it struggle for a moment only, then, with a sweep of her arm, cleared a space
on the surface of her table and took a step away, giving enough ground to tell the sprookje she would not
attack it. The sprookje’s cheek-feathers settled a little. Settled enough, Tocohl saw, for it laid its bouquet
carefully onto the table and stepped hastily back.
The second sprookje did likewise. The third looked at the table, looked at Edge-of-Dark, and,
fluffing all its feathers to twice normal size, it stretched out its arms to offer its bouquet directly to
Edge-of-Dark.
Edge-of-Dark inched forward to take the gift, holding the chattering sprays as delicately as the
sprookje had.
“Good for you,” said Buntec—it was obviously she who held Maggy’s arachne—and the sprookje
echoed her. “How about that,” Buntec went on, sounding twice as pleased, “you may be a pain in the
butt, but at least you’re not chicken-shit.”
Buntec’s words and echo seemed to reassure the others as well: each of the remaining sprookjes
delivered its burden directly into Edge-of-Dark’s arms, as if the act were a matter of pure course.
Edge-of-Dark, dazed and grinning from behind her armload of sprigs and vines and stalks, began to look
like an artistic composition of her own design.
A hand touched Tocohl’s shoulder, her spectacles cleared, and she smiled back at Om im.
“Megeve couldn’t have stopped it,” he said with enormous satisfaction. “He couldn’t have killed
enough of us to stop it.”
“So I see.” Tocohl tapped the frame of her spectacles with a fingertip. “But I’d like a closer look.”
The spectacles instantly provided a close-up of Edge-of-Dark. “Thanks but no, Maggy, I mean I’m
coming out.”
Om im offered his shoulder for support. As Tocohl got to her feet, layli-layli calulan said, “I
suppose there’s no point in arguing with you?”
“None at all,” Tocohl assured her, “but”—she reacted to the twinge in her side as she
straightened—“I will take it easy.”
“I think she means it,” said Om im, lifting a brow at Tocohl in surprise. “That’s less of an argument
than we got from Maggy on the subject of waking you.”
“Maggy doesn’t have a pain in her side.” But Tocohl released his shoulder and walked slowly to thedoorway on her own. The pain was there but no longer so bad she would be unable to function.
Om im thrust aside the membrane and bowed her into the sunlight, where she stood, dazzled by the
confusion.
The sprookjes had been granted front row center at two separate shows. Edge-of-Dark made art of
the plants they’d brought her, and beside layli-layli calulan’s cabin, Dyxte was up to his elbows in the
red mud, planting a stand of tick-ticks. Sprookjes gathered around both, paying such rapt attention that
their echoing was only haphazard and intermittent.
Around each crowd of sprookjes, small knots of surveyors watched and recorded, trying for all their
excitement not to startle or to distract the sprookjes.
In the hush their gestures and their movements shrieked cacophony. Buntec, now holding the arachne
at waist-height, grinned from ear to ear, while Hitoshi Dan’s grin began at the tips of his toes, shot his
eyebrows up, and ran out his extended arms to spread the fingers of both hands wide. Kejesli shrugged
one-handed. Van Zoveel first turned out his thumbs in puzzlement, then shrugged back at Kejesli with a
down-turned palm. John the Smith jockeyed for position with Tryn Ilan of Dusty Sunday—who was only
trying to find a better camera angle, not assert authority.
Tocohl closed her eyes, made momentarily giddy from the sudden full impact of it all. Her hand
reached out, found Om im’s shoulder beneath it.
“Ish shan? Are you all right?”
She opened her eyes. “For a fool, I’m fine.” Grinning down at him, she added, “I know your secret.
And I’ll bet you can’t tell the sprookjes apart right now.”
He obliged her by looking, first at one group, then the second, then up again at her, perplexed.
“You’re right. They all look alike.”
“They’re too interested in Edge-of-Dark and Dyxte to worry about their toes.”
“That’s not much of an explanation.”
“I know. But I’ve got to find Megeve’s sprookje before I can give you a better one.”
He looked again. “I can’t help you.”
Having finished planting his tick-ticks, Dyxte rose and came toward them, trailing his collection of
sprookjes. “Good,” he said to Tocohl, warming the perfunctory GalLing’ with a thump of his fist to his
heart, “you’re awake. Would you be willing to sacrifice that cloak of yours in a good cause?” His
sprookje echoed his request.
“For art’s sake?” Tocohl said. She sighed. “There’s not much left of it, but you’re welcome to the
remnants.—Inside.”
He thanked her with a spread palm and slipped past Om im, who turned out two fingers and said, in
surprise, “That’s Megeve’s sprookje, Ish shan, but it didn’t echo you!”
“It doesn’t recognize me in all this noise,” she said, adding to herself, at least, I hope that’s the
explanation. This would take conscious effort, she saw, and again she released Om im’s shoulder. Taking
a step toward the sprookje the Bluesippan had indicated, she told herself, You’re talking to a
Maldeneantine: be polite. “Sunchild?” she asked.
“Sunchild?” said the sprookje, its voice overlapping hers.
“Thank Veschke,” said Tocohl, and the sprookje, ruffling, echoed all the feeling she’d put into the
phrase. Without considering the action, she reached out a hand to smooth down the risen feathers.
The sprookje’s head dipped suddenly, beak flashing sharply down toward her hand. Tocohl felt the
prick of its “sample tooth.” When it raised its head again, its feathers had already begun to subside, laying
back smoothly against the body in long rippling waves. “You just wanted to make sure I was all right,”
she said and was echoed.
(It talked to you!) said Maggy.
(Not yet. So far that’s just echo. Bring the arachne over if you will: I’d like to have as much tape on
this as possible.)
“Now how did you do that?” Om im said. Then, in a tone of admiration, “Never mind. Don’t tell me.
It’d be like asking a magician where the doves came from. Go on,” he urged, “I’ll just stand here and
appreciate the results.”“Don’t go overboard,” Tocohl and Sunchild said, “I haven’t got anything yet—except a damn echo
I could well do without.”
Om im laughed. “You’re never satisfied. First you’re unhappy that it won’t talk, now you’re unhappy
because it will.”
She eyed him wickedly. “Let’s try a second experiment, shall we?” Sunchild agreed vocally. “Let’s
see what happens when I go from Maldeneantine”—and here she shifted stance and position—“to
Bluesippan... and keep on talking.”
For the first time, she heard a catch of hesitation in the sprookje’s echo—just at the moment she
shifted from accommodating a speaker of Maldeneantine to a speaker of Bluesippan. “Now,” she and
sprookje said together, “go ahead. Say something, Om im, I dare you.”
“Dare me...” he began, completely puzzled. Then his mouth snapped shut as he realized Sunchild
had echoed him as well. “By my blade,” the two of them said together, “what have you done to me?”
“Not me,” Tocohl assured him, reassured to find the sprookje still echoing her as well, “Sunchild.”
She grinned at the sprookje, feeling as ruffled in her excitement as Sunchild so obviously was. “You catch
on quick,” they said, as if to each other.
Dyxte, trailing the moss cloak, paused on the threshold to look down at them. “It talks to you!”
Sunchild did not echo him. Tocohl frowned at the sprookje, waited until Dyxte had reached the
bottom step, and shifted into ti-Tobian. Sunchild’s eyes widened. “Yes,” said Tocohl as Sunchild and a
second sprookje—Dyxte’s—both echoed her, “but you see how complicated this echo business can
get.”
“You’ve got two echoing you now!” Dyxte said, then threw a protective arm across his face at the
realization that the same two had echoed him as well. “Oh, no!”
Om im glanced up at her, clearly wondering what would happen if he spoke. With the touch of a
finger to the tip of her nose, she urged him to try. “Testing,” he said cautiously, “one, two...” The same
two sprookjes echoed him. “Now you’ve really done it, Ish shan,” all three said accusingly.
“I’m afraid so,” Tocohl admitted. Behind her a chorus of sprookjes sounded the same regret. “Let’s
see if we can get Sunchild inside”—she winced at the amount of echo as another sprookje joined the
chorus—“where I’ll have only one to deal with.”
Dyxte, with a wild look at her, bundled the cloak under his arm and made for layli-layli calulan’s
cabin. Two of the sprookjes hesitated only a moment before following him. Sunchild remained, still
staring solemnly.
It turned at the arrival of Buntec, Maggy’s arachne, and van Zoveel and its eyes widened.
“Veschke’s sparks,” said Tocohl, seeing the look—attention drawn back, the sprookje echoed her—“I
wish you wouldn’t!”
“It’s true!” said van Zoveel. “It echoes you!”
To her relief, it didn’t echo van Zoveel. “Inside,” she said and Sunchild seconded that.
Buntec set the arachne on the infirmary’s bottom step, where it immediately skittered up to the door
and rocked impatiently. “Lumps,” said Buntec, clenching a fist in its direction, “you’re about to learn the
literal meaning of lumps.”
The arachne stopped its rocking instantly and bobbed deferentially. “Your pardon, Buntec,” Maggy
began.
“Don’t worry about it. Just don’t do it. It drives me up a wall.”
Maggy lowered the arachne a fraction of an inch, just enough to appear greatly interested, and said,
“Really?”
Buntec rolled her eyes at Tocohl, sighed, and said, “Kids.” She climbed the steps and held the door,
toeing the arachne inside. Tocohl made the various shooing motions that gestured first Bluesippan, then
Zoveelian in.
Then she took a deep breath and, hoping Sunchild would remember, held out to the sprookje an
imaginary length of moss cloak. The sprookje came forward, took the imaginary end, and followed her to
the threshold.
There it stood, feathers ruffling. Om im said, “This isn’t a daisy-clipper, Sunchild. By my blade, Iswear it won’t crash.” Echoing his words, Sunchild entered the infirmary.
“It echoed you too!” van Zoveel said. “How... ! How ... ?” He rocked impatiently at her side.
“Lumps?” said Maggy, directing her query to Buntec.
“Probably,” said Buntec, with a glare at van Zoveel, “in about three minutes if he doesn’t cut it out.
I’ve warned him about it...”
As if fascinated, the arachne trotted a yard or so away, the better to get a full-figure view of van
Zoveel and Buntec. The movement drew a surprised glance from van Zoveel. “She’s waiting for me to
deck you,” Buntec explained. “She wants to record it for her files.”
Van Zoveel stopped his rocking abruptly. With visible effort, he held his body still, but the ribbons on
his tunic still fluttered his excitement. “Tocohl,” he began.
“No lumps?” Maggy sounded disappointed, and knowing how unusual it was for her to interrupt,
Tocohl decided that was an indication of how disappointed.
“Some other time,” Buntec said, “I’m sure.”
That was enough to satisfy Maggy. To satisfy van Zoveel would not be so easy. She stood quietly for
a long moment, then she turned and greeted him, with Sunchild repeating each word, in Zoveelian: “May
the sun warm you in the cold wind.”
Ruurd van Zoveel responded automatically: “May the wind cool you in the hot sun.” This time the
sprookje echoed him.
With a whoop of delight, Buntec clapped van Zoveel on the shoulder. “Now you’ve got two,
Ruurd!” In her burst of enthusiasm, she added, “Make it do me, Tocohl.”
Tocohl grinned. “Are you sure?” The sprookje echoed her. “It’s more trouble than it’s worth.”
But Buntec was caught by the enthusiasm of the moment. “I don’t mind! Do it!” she said, clapping
her hands together in her excitement, so Tocohl grinned again and, shifting her stance, mimicked the
gesture. “Now talk,” Tocohl and the sprookje invited.
“Hiya, Sunchild!” Buntec said, and when Sunchild repeated her words, she clapped van Zoveel on
the shoulder a second time, crowing her delight.
Perhaps, thought Tocohl, overkill is the way to go. A moment later she had the sprookje echoing
both layli-layli calulan and swift-Kalat as well.
Maggy rocked the arachne. “Me, too. Make it do me, too!” but before she had completed the
sentence, Sunchild had joined in as well. Maggy stopped the rocking, as if in surprise. “What did I do?”
she and her echo asked.
“You just proved,” Tocohl said, with Sunchild picking her up, “that you’re definitely Hellspark.”
In the momentary clamor of voices that followed, Sunchild tried valiantly to echo them all, even as
they interrupted and overlapped each other’s words. But it could manage only a phrase here and a
phrase there, and its feathers began to fluff in its distress.
“That echo,” Tocohl announced—so loudly that it was her voice the sprookje followed—“has got to
go.”
Silence ensued as Tocohl considered the sprookje. “All right,” they said together, “everyone keep
quiet for a moment.” Tocohl stroked the feathers at Sunchild’s wrist until they subsided. Then, keeping a
careful eye on them for any renewed sign of alarm, she slowly raised both hands to the level of its head.
Curling the fingers of her right hand very loosely, she circled Sunchild’s beak. Her left hand she brought
to the level of its larynx, an inch away from the feathery surface length of its throat.
“You must not,” she began—as it tried to echo her the opening beak touched her encircling
fingertips; simultaneously, she pressed the feathers over its larynx—“speak aloud.” The sprookje,
surprised at the contact, snapped its beak shut, leaving her sentence unfinished.
As slowly as she had raised them, she drew her hands away. “Don’t speak,” she repeated. It started
to but she raised her hands as if to circle and press again and the sprookje closed its beak with an
audible snap. “Got that?” This time it made no attempt to mimic her words and she smiled in satisfaction.
“Yes, I see you have.”
Turning to Om im, she said, “I wonder if it’s all or nothing.—Say something, Om im.”
He grinned up at her. “I’m humbled before you, Ish shan,” he said, his words belied by the cock ofhis head and the tilt of his brow.
The sprookje remained silent.
“Good,” said Tocohl. She turned back to face the sprookje. “Now all I have to do is convince you
you’ve got to shout to make me understand.”
“Shout?” said van Zoveel.
“Figuratively speaking,” said Tocohl, still absorbed in the task. “Om im, I need your help.”
“Name it.”
“If Sunchild follows me, I want you to come along too.” At his thumbs-out agreement, she said again
to the sprookje, “Let’s try this. You understood this before.” Once again she held out an imaginary length
of cloak. Once again the sprookje reached for the nonexistent other end.
When Tocohl walked the few steps toward the cot where Alfvaen lay sleeping, Sunchild followed
with Om im at its heels. Tocohl stopped, turned her thumbs out.
Om im turned his own out in jubilant approval.
The sprookje stared, first at one, then at the other. Then, slowly, as if questioningly, it too turned its
thumbs out. Tocohl jabbed hers out a second time, hardly able to constrain her excitement. “Yes! Om
im, tell it yes!” Om im laughed and jabbed his thumbs out a second time as well.
The sprookje mimicked the gesture, this time with ruffling feathers and the same flamboyance Om im
brought to it.
Now Tocohl “extended the cloak” and both followed her until she stood over Alfvaen. Remembering
the bright warning tongue the sprookje had used to indicate danger, she touched Alfvaen’s shoulder and
stuck out her tongue. Layli-layli calulan moved closer, to watch the sprookje’s reaction.
But there was none that Tocohl could see. She tried again: this time lifting one hand to her face to
mimic the puffing of cheek-feathers as she touched Alfvaen’s shoulder. The sprookje’s cheek-feathers
fluffed. It leaned forward, glancing from Alfvaen to Tocohl, and nipped Alfvaen’s hand.
When it straightened, its cheek-feathers subsided. Reaching across to Tocohl, it stroked wrist
reassuringly until Tocohl drew her hand from her face. “It’s not worried about Alfvaen, layli-layli,”
Tocohl said, “at least, I think not.”
“Ask about the filaments, if you can.”
“I can try.” Tocohl tucked her fingers gently beneath the long gray threads that covered Alfvaen’s
shoulder and raised them slightly. Once again she mimed sprookje-alarm.
The sprookje turned out its thumbs with authority.
Startled, Tocohl commented, “Oh, I guess it means ‘I understand.’”
The sprookje bent again to Alfvaen. Using both hands, it began to stroke Alfvaen’s shoulders briskly.
The gray filaments crumbled beneath its fingers, and the sprookje held out a handful of fragments to
Tocohl.
She cupped her hands to receive and the sprookje tipped the fragments onto her palms. In turn, she
offered them to layli-layli calulan. “Now you know as much as I do,” she said.
Layli-layli calulan smiled. “More. The plants are dying, Tocohl. It means that when the alcohol is
gone from Alfvaen’s system, the plants die. They intended that to restore Alfvaen to what they
considered human-normal.” The shaman looked across at the sprookje and, very deliberately, turned her
thumbs out. “I understand,” she said, “thank you.”
The sprookje stared at layli-layli calulan as if seeing her for the first time, then it turned its thumbs
out—at layli-layli, at Om im, at Tocohl. Om im returned the gesture just as vigorously; layli-layli,
smiling, did the same. Tocohl brushed the crumbled fragments from her palms and tipped her own
thumbs out jauntily.
“Yes,” Tocohl said, reinforcing each word with a jab, “yes, we understand. You understand.”
“I don’t understand.” Van Zoveel, silent all this time, pushed toward the little group. “It mimicked our
gestures before, but never as if it understood them!”
“Because it didn’t understand them. Right now, it understands less than a handful: ‘follow’—” She
demonstrated by leading the sprookje away from Alfvaen with her imaginary cloak. “I wonder if that
includes ‘follow suit’?” Pulling up a chair, she sat, grateful for the moment’s respite, and repeated thegesture. It worked: the sprookje followed suit—pulling up a chair and sitting, albeit somewhat
uncomfortably, to face her. Tocohl thumbed approval. “You too, van Zoveel,” she said, making the same
gesture at him, “follow suit.”
He hesitated, and Tocohl said patiently, “Thumbs out to show you understand, then follow suit.” This
time he obeyed.
The sprookje enthusiastically jabbed its thumbs for him as well.
“How...?” said van Zoveel.
“The answer’s been staring us in the face all this time. Om im was the only one who saw it.”
Om im, standing blade right of her chair, jerked his head to stare at her with surprise. “Me?” he
demanded.
“You. How did you know I was Hellspark?”
He shrugged, shoulders high, hands drawn back in fists. “I don’t know. As I said, you looked
Hellspark.”
“And each sprookje looked individual to you—like the team member it mimicked.” She settled back,
to find a position which lessened the ache in her side. Across from her, the sprookje shifted as well,
seeking its own comfort in a chair unsuited to its physique. Tocohl thumbed approval at it and it thumbed
back.
“To tell it from the beginning,” Tocohl went on, “I should have seen it in your tapes of the so-called
‘wild sprookjes.’ They haven’t any larynxes—but our feathered friend here has a very visible one.”
Van Zoveel’s ribbons fluttered as he leaned toward the sprookje, as if to check. The sprookje’s
larynx obligingly bobbed. “The wild sprookjes simply have no visible larynxes, Tocohl. Yours is scarcely
apparent, after all. I do—I did see the difference at the time the brown sprookjes moved into camp—but
it’s as insignificant as the crests and colored yokes.”
“I’m sorry, but you’re wrong. It’s not a matter of visible or hidden. The wild sprookjes have no
larynxes.”
“Oh, but they must!” van Zoveel said, glaring sidelong at the sprookje. “How else could they talk?”
“They don’t. Not audibly.” Tocohl grinned. “You try to talk in a stand of lightning rods with a storm
directly over your head. That’s where the sprookjes go during a storm, swift-Kalat—”
“Yes, so I would deduce from what Buntec and Om im told me of your experiences,” swift-Kalat
said.
“Sunchild didn’t even react to the thunder. She’s got ears but I’ll bet she shuts them down for the
length of the storm. Why waste a perfectly good social occasion simply because you can’t hear one
another?”
Swift-Kalat moved closer to the sprookje, bending down to examine the side of its head. The
sprookje turned to watch him, making the examination impossible. Straightening, swift-Kalat said, “The
only simple test I can devise at the moment requires loud noises. I’d rather not frighten it, unless...”
“I’d rather you didn’t either,” Tocohl said.
Swift-Kalat stepped back, letting the sprookje settle again, and Tocohl went on, “Now, about two
years after you got bitten by the first wild sprookje, van Zoveel, the camp sprookjes showed up. I
theorize that it took them that long to analyze your gene pattern, to compromise between it and the
sprookje’s pattern, and to give the sprookjes what they felt they lacked—a larynx, for example.”
“Get serious, Tocohl,” Buntec interrupted. “That’s too much credit! You’re sayin’ they can muck
around with their genes just for the hell of it?”
“For swift-Kalat’s sake, I posit it as a theory. But—look at what we’ve seen them do, Buntec.
Swift-Kalat’s sprookje nipped Alfvaen and, in the space of a few moments, analyzed the sample, judged
it abnormal, and prepared a living antidote, which it then injected.”
Buntec whistled. The sprookje turned its head to stare at her. “Yeah, Sunchild, I’m impressed,”
Buntec told it. To Tocohl she added, “We’re talking smart cookies here.”
“Smart enough,” Tocohl said, “to have decided that you were sentient long before you so much as
suspected they were.”
“None of which explains the parroting,” van Zoveel said.“I’m coming to that. The compromise sprookjes arrive in camp—bear in mind that they may never
have heard audible speech before!—and they look over the survey team, and what do they find? Every
member of the survey team speaks a different language!”
“We all speak GalLing’,” van Zoveel said.
“Audibly, yes,” Tocohl said, “visually, no. Your pardon, van Zoveel, but no matter what language
you speak audibly, your body speaks Zoveelian—every gesture, every stance you take, even the way
you position yourself to speak is Zoveelian.” Swift-Kalat made a sharp questioning noise and Tocohl
cocked her head at him. “Yes, that’s why you feel uncomfortable around van Zoveel. He may speak
Jenji without an accent, but his movements are wrong. Wrong only in the sense that they are not Jenjin,
which is quite sufficient to disconcert you unless he’s sitting down.”
The arachne straightened to full height. “Proxemics and kinesics!” Maggy said.
Tocohl grinned at it. “Right you are. The moment I danced Maldeneantine at Sunchild, Sunchild saw
me. From then on, I could get her to notice anyone else simply by switching from the proxemics and
kinesics of one language to another in mid-sentence.”
“So if Alfvaen learned her lessons right, swift-Kalat’s sprookje will echo her!” The arachne stepped
closer to the sprookje. “But why did Sunchild echo me, Tocohl?”
Tocohl grinned. “For the same reason Buntec threatened to give you lumps... I’d gotten Sunchild to
see van Zoveel, and you imitated him by rocking the arachne. That was apparently sufficient evidence of
your sentience for Sunchild to follow up.”
“Oh.” The arachne pricked delicately forward, its every step watched closely by the sprookje. When
it was directly in front of Sunchild, Maggy had the arachne extrude both adaptors, twisting them outward
in awkward imitation of the thumbs-out Tocohl had used.
It was good enough for Sunchild, who thumbed back at the arachne enthusiastically. A crow of
delight—obviously adapted from Buntec’s—came from the arachne’s vocoder. Then Maggy added, “I
think it’s going to be tough to talk to her, Tocohl, for me anyhow.”
“Not just for you,” Tocohl said, “for now I’ll settle for a pidgin. I think she’s finally getting the idea
that she has to shout—make broad gestures—to make me understand. I suspect her language is all in the
position of the feathers. I’m no more equipped for that than you are.”
“Proxemics and kinesics,” van Zoveel said slowly. “The schools I studied in never gave more than a
theoretical course in either.” He glanced sharply at swift-Kalat. “Is that really why you always seem so
uncomfortable when we speak Jenji? Because I move wrong?” He slapped his hands despairingly
together. “I am a dangerous fool, Tocohl—”
“Don’t castigate yourself, van Zoveel. I fell into the same trap. If I hadn’t been automatically
compromising my movements and stance to accommodate a mixed group of languages, to avoid
offending anyone, one of those sprookjes would have parroted me the first time I opened my mouth to
speak Jenji or Siveyn or Bluesippan.”
Van Zoveel frowned. “What about layli-layli calulan and—your pardon, layli-layli—Oloitokitok?
Oh!” His shoulders relaxed and he went on to answer his own question, “Then the Yn must have different
proxemics and kinesics for male and female, just as they have different spoken dialects for male and
female.”
“Exactly,” Tocohl said, “likewise your two Sheveschkemen: one from the south, one from the north.
Two different languages in all aspects.” She glanced down at the arachne. “As Maggy said though,
swift-Kalat’s sprookje would have echoed Alfvaen the first time she got her Jenji right in front of it. She
would have too, and fairly soon. That makes me feel a little better: I had her serendipity for backup if I
blew it.”
She shifted again, made more uncomfortable by the thought than by her injury, and finished, “I’m
looking for it, van Zoveel, and all I’ve caught so far is the cheek-puffing business! Om im’s been seeing it
all along without knowing what it was he saw, so I think we’d both better apply to him for assistance.”
“You keep saying that, Ish shan, but I haven’t the vaguest idea what you’re talking about.”
“You said I looked like a Hellspark. I’m betting you watched me come into the common room and
greet three people in three different languages.”“Yes, but I couldn’t hear you over that crowd.”
“You said ‘looked like’—when I greet someone for the first time, I do stick to that culture’s kinesics
and proxemics. It makes a better first impression. You saw the shift, just as you can see the different
stance each sprookje takes to accommodate the language of its respective human. I’ll bet your twin
friends held themselves differently, moved differently, Om im. You never saw it consciously but you saw
it.”
Om im cocked his head to one side. “I’ll take your word for that, Ish shan. Maybe if I look, I’ll see
from now on. How can I help?”
“As I said, the best we can do for the moment is a pidgin. If you’ll pick up and use the same broad
gestures I use, I think Sunchild will understand that we have a language in common.”
“I am in your service,” he said, touching the hilt of his blade to remind her that this was literally true.
The sprookje did likewise.
“Yes,” Tocohl agreed, “but try to avoid extraneous gestures like that one. We’ve just confused
Sunchild... Don’t worry: This is as good a time as any to establish a no.”
It took her some few moments of silent gesturing but she accomplished it to her satisfaction: for no,
fingers scraped emphatically against the thumb as if to rid the hand of something at once noxious and
sticky. Sunchild mimicked her with the same enthusiasm it had given the thumbs-out yes.
“Why so broad a gesture?” swift-Kalat asked. “Sunchild can apparently distinguish very subtle
movements.”
“I can’t,” said Tocohl, “I need movement to attract my eye and remind me to look instead of listen.
And a no or a yes I want to be able to see at a distance.” She rose. “Now let’s see if we can get the
other sprookjes to understand as much as Sunchild.” Signing for Sunchild to follow, she headed for the
door.
Instantly Sunchild rose. So did Om im, causing the sprookje to thumb yes so vigorously it nearly
jabbed Om im in its excitement. “Outside is a good idea,” Om im observed aloud. “We’re definitely
running out of room for all this enthusiastic communication!”
Tocohl laughed and led the entire troupe, thumbs out triumphant, back into the sunlight.
Once outside, however, the sprookje’s triumph turned abruptly to distress. Its gold eyes darted from
group to group of the surveyors and its feathers bristled.
Tocohl, having had the same experience only a short time before, knew the source of the trouble: the
utter confusion of languages it saw danced. Turning Sunchild gently but inexorably to face her, she
stroked the sprookje’s wrist feathers. “Don’t panic,” she said, “watch me,” and she brought both hands,
flat and crossed, to her chest. “Follow”—again she extended the imaginary cloak—“me”—again she
brought her hands to her chest.
Yes, thumbed Sunchild, its feathers subsiding.
Very deliberately, inviting her to do the same, it turned its attention on Dyxte who, having just finished
draping streamers of Tocohl’s moss cloak from a dozen places on layli-layli calulan’s cabin, now stood
back to admire his work. Obviously pleased, he turned and sought others to admire his work as well.
The sprookjes surrounding him, as excited as they might be, did not serve; nor did the handful of
surveyors—all of whom were too intent on observing the sprookjes’ behavior to notice Dyxte’s.
Tocohl drew her party to his side. “Nicely done,” she said and a second sprookje, Dyxte’s, echoed
her approving words.
“Thank you,” said Dyxte, echoed himself. Then he crooked a finger to indicate his sprookje. “It’s still
parroting us both,” the two of them went on. “I’m not sure—”
But as he spoke, Tocohl raised her hands to his sprookje’s beak and larynx, as she’d done earlier to
quiet Sunchild. Dyxte’s sprookje stopped speaking, leaving Dyxte to finish his sentence alone. “—You
accomplished much—”
Dyxte stopped to stare, first at Tocohl, then at his sprookje, in astonishment. “It stopped echoing
me,” he said. “You did it!”
“I don’t think so.” Tocohl eyed Sunchild suspiciously. “I think Sunchild translated for me.” (Maggy,
let me see what Sunchild did while I signed at Dyxte’s sprookje.)The requested image flashed on Tocohl’s spectacles. Watching carefully, she caught the movement
she had seen only peripherally as she had gestured to Dyxte’s sprookje. (Again, Maggy, more slowly.)
This time she saw clearly the ripple of feathers along Sunchild’s thigh, the minute shift of stance.
“This,” said Tocohl aloud, “is not going to be easy. But I’ll be burned if I’ll settle for a pidgin.
Obviously, I need feathers. No—stripes! Maggy, stripe my 2nd skin—make it brown, dark brown, and
gold.”
The stripes began at the tops of her “boots” and raced upward to vanish into the folds of her collar.
Sunchild watched their progress with startled interest. Sure of the sprookje’s attention, Tocohl said,
(Maggy, I want you to imitate that feather ripple by distorting the stripes. Now!); and Maggy obliged.
Sunchild’s eyes widened still farther. It rippled feathers identically, then thumbed a vigorous yes.
“Got it,” said Tocohl triumphantly, and she and the sprookje thumbed happy yeses at each other.
“Now wait here,” she went on, “I’ve got to see if it works on the rest of your people as well.” She signed
follow and flicked no.
Leaving all but Maggy behind, Tocohl crossed the courtyard and plunged into the excited crowd at
the steps of Edge-of-Dark’s cabin to find Kejesli. Her presence sparked a babble of greeting and
echoed greeting.
“Veschke’s sparks, Tocohl,” Kejesli—and his sprookje—shouted to make themselves heard, “your
rib!” With three preemptory shoves he gave her breathing space in the crowd. “Why are you out of
bed... ?” he said, and his sprookje echoed his concern.
She grinned and shouted back, “Because I’ve got a word to say to your sprookje!”
(Again, Maggy. Ripple for quiet,) Tocohl said, and the stripes on her 2nd skin flashed into motion.
“You can talk to them? You can really talk to them?” Kejesli said, and this time he spoke without
accompaniment. “They told me you’d gotten one to echo you but—” He broke off suddenly, shutting his
mouth with almost as audible a snap as the sprookje’s. “It stopped echoing me! What did you do?” he
said, staring at the silent creature beside him as if it were about to bite.
“I told it to shut up,” said Tocohl. “That’s the only phrase I know in sprookje so far—but that’s not a
bad start for learning a language without words.”
Chapter Fifteen
BY THE TIME the sprookjes had begun once again to look at darkening skies and to flash their
tongues in warning, Tocohl had established some fifty gestures, all broad, in pidgin, and with Maggy’s
help she could recognize and imitate five—tentatively—in the more subtle and infinitely more difficult
“language” of the sprookjes themselves. Quite enough for one day, she decided, as she watched them
vanish into the flashwood at the edge of camp.
(You should be resting now. I’ve got sensors spiking all over the place.)
Tocohl stared thoughtfully down at the arachne. (I see you’ve been spending a lot of your time with
Buntec.)
The arachne tilted upward, much as if startled. (Yeah. How did you know? Did I do wrong?)
(No, no, not wrong. And I can tell because you’re picking up her phrasing. Just take care to use it
appropriately. Bear in mind that Buntec is considered coarse and vulgar by about half the surveyors, even
though she’s very refined by her own standards.)
(Okay,) said Maggy, then through the arachne’s vocoder, she said, “Om im, Tocohl’s going to rest
now.”
“Good idea.” He patted his shoulder, offering it for her support. “Ish shan?” She accepted his aid and
found Buntec supporting her from the other side, gently urging her toward the infirmary as the first
spatters of rain began to strike. Maggy trotted the arachne along beside them.
“One moment.”
It was the first time Tocohl had heard true command in Kejesli’s voice. Buntec jerked to a halt, her
surprise confirming . Om im raised a gilded brow and turned as well.Hands on hips, Kejesli waited until he had the full attention of his troop of surveyors. “We’ve wasted
enough time on this world already. I want your revised reports tonight: the message capsule goes
tomorrow morning...” A cheer rippled through the crowd, forcing him to pause momentarily. When it
subsided, Kejesli’s manner softened. Smiling at Tocohl, he finished, “At that time, Flashfever passes from
our jurisdiction to that of the Hellsparks who, I’m sure, are more than ably represented already.”
His hand sought the pin of remembrance at his lapel. In Sheveschkem, he added, “Veschke guide
you, Tocohl Susumo.”
Tocohl responded in the same language, “Veschke got me here... and she’s not one to strand a
trader.”
As she touched her pin of high-change, Maggy said, (Yeah, but how is she on people who
impersonate byworld judges?)
(I don’t even want to think about it,) Tocohl said as she let Om im and Buntec lead her back to the
infirmary.
(I do,) said Maggy, and her I had become a thing of wonder.
Too exhausted to do more than note the fact, Tocohl said only, (Then do it quietly,) and, to her relief,
Maggy obliged. Blessing the respite, Tocohl settled into her cot and slept. For the next three hours not
even the thunder was able to wake her.
When she did awake, it was not to an ear-splitting crash of thunder, but rather to the smallest of
protests—an outraged, bewildered, “Hey!” in a voice that was unmistakably Alfvaen’s and it brought
Tocohl fully awake by the time Alfvaen had finished her complaint, “I’m all strapped down!”
Tocohl sat upright. Beside her, Om im chuckled and called out, “We thought we’d give our
Hellspark a chance to recuperate before you broke her other ribs, Alfvaen.”
“Don’t confuse her,” said layli-layli calulan firmly. Just as firmly, she pushed swift-Kalat aside—he
was attempting to hug Alfvaen where she lay—to loosen the straps and free her. She sat up into
swift-Kalat’s embrace, and for a moment, there was appreciative silence on all sides. Then the two shyly
released each other.
Alfvaen stretched and scratched and said, with a sigh, “That’s better.” As she addressed the remark
to swift-Kalat, Tocohl suspected she referred to the embrace as well.
A sound very like a crow issued from the arachne’s vocoder. “The books were right! I was right!”
The arachne sprang for the edge of the cot, clung precariously by three of its forward appendages until
swift-Kalat boosted it onto the bed beside Alfvaen, where it tilted to peer at her. “But you were
supposed to win the duel, weren’t you? How are you?” she demanded, rocking the arachne from side to
side.
Alfvaen, still bemused, said, “You’re all right, Maggy. I’m so glad. Tocohl was worried!”
“You,” Maggy repeated, doubling the frequency of her rocking, “how are you?”
Alfvaen looked at the arachne and then all about her in wonder. “I feel... fine?” Puzzling over her
own sensations, she frowned down at her hands, as if checking their condition might tell her the state of
the rest of her body. “The last thing I remember, we were following a sprookje and I was... I was
sobering—without my medication. I feel... sober, Maggy, for the first time in years!”
She turned a wide-eyed stare on layli-layli who said, “Yes, although I’d prefer to make a few
confirming tests.”
“Oh, of course! The others who have Cana’s disease! Can they be helped too, layli-layli?”
“I believe so. If the sprookjes are willing.” She glanced across the room at Tocohl. “I agree with
Tocohl’s assessment: your serendipity is beyond question.” Then she busied herself with her instruments,
checking monitors, and interrupting only to draw another blood sample from the Siveyn while Om im told
Alfvaen what had happened since her last clear memory, that of the vanishing daisy-clipper.
Waiting until layli-layli calulan had finished her tests, Tocohl rose and crossed the room to perch on
the edge of Alfvaen’s cot. Hand, palm up in the crook of her elbow, she greeted Alfvaen formally.
“From what Om im tells me,” Alfvaen said, “I slept through all the excitement.”
Om im laughed, startling Alfvaen. Tocohl, smiling, said, “Hardly that. In fact, you were a major
portion of it yourself.”“What do you mean?”
With a gesture, Tocohl deferred to Om im: “His version is the most colorful one that retains some
accuracy.” Buntec’s version had grown to epic proportion in the retelling, to the point that it embarrassed
Tocohl.
Om im touched the hilt of his knife and recounted the duel between the two. He had only begun when
a look of horror came into the Siveyn’s green eyes. Surprisingly, her first action was to jerk her head at
the arachne. “Oh, Maggy. That’s what you meant! I thought—I thought I dreamed it!”
Splaying her fingers at her throat, she turned again to Tocohl. “I dreamed so many terrible things! I
thought that was only one more! Your pardon, Tocohl!”
“You did dream it,” Tocohl said easily. “And like a dream, it’s forgotten on waking. Just don’t do it
again—I only lived through it by trickery.”
“But why? Why would I challenge you, and to death? That doesn’t make sense. What grievance did
I claim?” She laid her hand on wrist. “Please,” she urged, “tell me. I only recall something nightmarish,
something about you that terrified me...”
“It’s not important,” Tocohl began. In retrospect, it would only serve to embarrass Alfvaen further:
her “grievance” had been downright silly.
But layli-layli calulan said, “Tell her, tocohli. It is part of the healing.” So Tocohl described the
incident, repeating Alfvaen’s challenge and her own responses verbatim.
Alfvaen gasped, but Tocohl said, “Now I’ve forgotten it.”
“I haven’t,” said Maggy, “I remember the words but I don’t understand them.”
Alfvaen looked at the arachne unhappily. “I accused her of glamour, of influencing someone’s
emotions”—her eyes glanced at swift-Kalat, slipped away in embarrassment—“by unnatural means: psi
powers, love philtres. It’s a terrible crime on my world. Fortunately it doesn’t happen very often.”
“But it does,” Maggy said. “In the books, it does.”
“But those are only fiction.—And Tocohl replied that she was as natural as I, and took a terrible
chance turning her back on me! You know how dangerous that was, don’t you, Tocohl?”
“Believe me, I know it. I can still feel the hair on the back of my neck rising. But I could feel your
position on my 2nd skin, so I knew when you rushed me. That and the zap-me probably saved my life.”
“I didn’t get to watch,” said Maggy, something almost petulant in her voice. “When you duel again,
may I watch?”
Tocohl smiled again and took Alfvaen’s hand in her own. “As I said, just don’t do it again. If Maggy
were there, you wouldn’t stand a chance against us.”
“I could stay out of it,” suggested Maggy hopefully. “After all, two against one isn’t polite. And
Alfvaen was supposed to win.”
Swift-Kalat said, “I don’t understand what this is about.”
“My low taste in reading matter,” Alfvaen said. “Maggy expects us to do certain things because of
the books she and I read.”
Patiently, Tocohl said, “Maggy, fiction and reality often reinforce one another, but fiction can’t be
counted on to give you a pattern for reality. Alfvaen’s nightmares took the form they did because of what
she reads. Yes, she’s my friend, and yes, we fought a duel over swift-Kalat, but that’s where it stops.
That’s as far as we’ll take the pattern.”
“Oh,” said Maggy, again injecting a note of disappointment into the vocoder’s phrasing, “you mean
she doesn’t love swift-Kalat?”
Om im dropped forward, hiding his face in his hands, while his shoulders shook in suppressed
laughter. Tocohl closed her eyes and sighed, then she opened them again and looked at the reddening
Alfvaen. “Do you want to answer that one, Alfvaen? I don’t see how we can make it any worse than it
already is.”
Scarlet-faced, Alfvaen tilted her head up to face swift-Kalat. In perfect Jenji, she told him not only
that she loved him, but precisely how sure she was of her truth, and that was very sure indeed.
Swift-Kalat replied in kind, stroking her cheek to seal the bargain.
“She says,” Maggy began to Om im, “she—”“Not necessary, Maggy,” said Om im, still shaking with laughter. “For some things no translation is
needed.”
“But you see,” said Maggy, “that’s right, too. And the fiction told me how to find Tocohl, so why
aren’t you going to fight a duel properly.”
“Because I don’t want to fight a duel with Alfvaen,” Tocohl said firmly.
Alfvaen drew her glance away from swift-Kalat, took in the glare with which Tocohl favored the
arachne, and said, matching Tocohl’s firmness, “And I don’t want to fight a duel with Tocohl either,
Maggy.”
“Oh,” said Maggy, this time in quite a different tone, “you don’t want to. Why didn’t you say so in
the first place?”
This set Om im laughing again, and Tocohl nearly bit her lip in two trying not to join him.
“Alfvaen,” Maggy went on, “do you wish me to forget the duel you and Tocohl already had?”
Tocohl gasped out, “Alfvaen, she means that literally. She’ll wipe her records of it if you ask her to.”
“I see.” Alfvaen looked thoughtfully at the arachne. “No, Maggy, don’t forget it. You need it to
remind you that fiction doesn’t tell you the whole story.”
“Thank you.” The arachne bobbed slightly.
Alfvaen continued to watch it for a moment, then she said, “If you like, Maggy, when layli-layli
calulan says I’m healthy enough to release from custody, I could demonstrate the standard dueling
techniques for you...”
“Oh, yes. I’d like that very much!” The arachne suddenly contracted.
“Maggy?” said Tocohl, worried by the abruptness of the movement.
The arachne unfolded and pricked gingerly across Alfvaen to stare upward at Tocohl.
“Tocohl?”—the voice held puzzled surprise—“I know what ‘like’ means!”
Tocohl could feel the smile spread from her toes to her scalp. In deep satisfaction, she said, “And
about time, too. Good for you, Maggy! I like you very much.”
Maggy made no reply. In fact, for the next several hours, she was remarkably quiet. To Tocohl, it
was clear that Maggy needed some time to herself, to think things out. So Tocohl took the opportunity to
catch the rest she still needed.
At long last, she was awakened by Maggy’s urgent pinging, and by the more urgent rocking of the
arachne at her side.
(What is it, Maggy?)
(They’re here.)
(They made good time. That’s a lot sooner than I’d expected.) Tocohl sighed. Rubbing her hands
over her face, she tried to compose herself. When simply waking didn’t achieve that end, she began a
Methven ritual.
Maggy, uncharacteristically, interrupted. (Tocohl,) she said, in what would have been exasperation in
anyone else’s voice, (we could skip. I can have the skiff in and out before they can get Kejesli’s
permission to land.)
A flash of lightning lit the door membrane to an eerie glow. Tocohl pointed an elbow. (In that?)
(I’ll risk it.)
(There’s no need. I told you before, Maggy: I pay my debts.)
(Yes, but—)
(No buts.) She looked fondly at the arachne. (But I appreciate the offer.)
Again Maggy made no reply. Then she said, (I like you too, Tocohl—very much.)
In response, Tocohl laid her hand on the fat body of the arachne, caressing it lightly.
(Why did you do that?)
Tocohl glanced at her hand, drew it away. (Hellspark gestural reflex,) she said, (affectionate feelings
expressed in touch.)
(Like you hug Geremy when you see him?)
(Just so.)
(Put up your hood.)Tocohl cocked her head to look inquiringly at the arachne. (Why?)
(It’s a surprise,) said Maggy. (Put up your hood.)
Puzzled, Tocohl did so. A second or so later, when the hood had molded to her face and sealed
itself, Tocohl felt a phantom weight on her lap. She glanced down, aware that the sensation was Maggy
manipulating the 2nd skin but wondering at the purpose of it. The phantom, very gingerly, leaned against
her.
Laughing, Tocohl closed her eyes and concentrated on the feel of it: a small form had perched itself
on her lap and leaned fondly against her. Small arms encircled her waist, careful to avoid the injured rib, a
head leaned against her cheek—a head complete to tickling curls. The phantom gave her a shy, childlike
hug.
(Oh, Maggy,) she said. Even in subvocalization, it came out a husky whisper. Then in reflex, her arms
closed around the phantom—to find, to the doubling of her surprise, that Maggy had thought of that too.
Her 2nd skin limited her movement to where the child’s body would have been. Her arms found small
sharp child shoulders to hug in return. The illusion was broken only by the lack of sensation in her bare
hands. That, she ignored; concentrating on the presence, she gave a second hug.
Then the weight was abruptly gone. Tocohl opened her eyes to find them stinging with the start of
tears. Her lap was, as she had known all along, empty.
The soft voice in her implant said, (Did I do right?)
(Yes,) said Tocohl, unable to say more.
(The sensors said so but—Are you going to cry?)
Tocohl grinned. (Actually, I’m not sure. But it’s nothing to worry about if I do. It’s a normal reaction
to strong emotion, even strong positive emotion. No, in fact, I have this horrible feeling I’m going to start
giggling.)
(That would be better.)
And that did it: Tocohl did indeed start giggling. (Maggy, why in the world did you opt for a
child-sized impression?)
(You said I was three, and Buntec calls me “kid.”)
(Ah, that makes perfect sense then.) Tocohl grinned foolishly at the arachne. (For a kid, you’re
something special.)
(Thank you,) said Maggy, and her tone retained little of her previous primness.
A shout for the door startled them both. Tocohl’s hands dropped to her lap, the arachne hopped to
the side where it could see beyond her. Layli-layli calulan gave Kejesli a fierce look of remonstrance,
cutting off a second shout.
He charged across the room, clearly agitated, and skidded to a halt beside her. “Tocohl. This place
has suddenly become like festival. I have six Hellsparks waiting in orbit for permission to land, one of
them a byworld judge by the name of Nevelen Darragh, who says you sent for them.”
Despite the fact that he had kept his voice low, Alfvaen had awakened and heard it all. She sat up in
her cot, wide-eyed and openmouthed, using both hands to fend off layli-layli’s attentions.
Tocohl met her eyes, glanced away. To Kejesli, she said, “I did.”
“All right then,” said Kejesli, “I’ll grant them permission to land.”
Alfvaen slid off her cot to intercept him before he could reach the comunit. “Wait, Captain. I want to
know—” She did not complete the thought. Eyes narrowing, she moved to Tocohl’s side with quick,
cautious steps; Kejesli followed, drawn by her manner. “Tocohl,” she began.
“Yes,” Tocohl said, “it’s me they’ve come to judge.”
“Then you’re not a judge after all.”
Tocohl glanced at swift-Kalat just in time to watch a look of horror spread across his face. To him,
she said, “If you’ll recall our conversations, neither you nor I ever said I was a byworld judge,
swift-Kalat.” The slight emphasis she placed on his status made him jerk reflexively. “You accused the
sprookjes of the murder of Oloitokitok; you asked me to judge the matter. In my judgment, the
sprookjes are innocent of blame. Your reliability is not in question.”
He gave the matter careful thought, clearly turning over their several conversations on the subject inhis mind. At last he said, “Neither is yours.”
That drew a laugh from Tocohl. “My reliability in Jenji may be fine, but in Hellspark I’m in serious
trouble.”
Kejesli, recovering at last from his gape, said, “What I choose to believe in Veschke’s honor,
Tocohl, is no reflection on your integrity.” He glared about him as if expecting dissent, looked relieved
when he received none, and went on, “You came at swift-Kalat’s request to learn the sprookjes’
language. Nothing more need be said on the subject.”
“That’s also true,” Tocohl said. “Those judges are here at my request. They already know what I
did; I told them.”
“You told them?” Kejesli gaped again. “But why?”
Drawing the arachne up to its full height, Maggy answered for her. “We pay our debts.”
It fell to swift-Kalat and Buntec to ferry the newcomers into camp. “Byworld judges?” she
demanded as she strode toward the hangar. “You expect me to believe Tocohl needed help? Why’d she
send for more byworld judges?”
Swift-Kalat didn’t answer. Phrasing a reliable response to that was more than he cared to risk; he
had no intention of causing Tocohl Susumo more trouble than she had caused herself. Two steps later, he
ran full-tilt into Buntec. He took a step backward, excused himself for having been so absorbed as to
blunder into her.
Hands on hips, she said, “Swift-Kalat.” In tone, it implied some sort of warning, as did her glare. But
before swift-Kalat could repeat his apology, the glare turned thoughtful. “Okay,” said Buntec, “let me see
if I can phrase this right. Swift-Kalat, is there something going on here that I should know about?”
That he could answer. “Yes.”
She made an odd gesture with her fingertips, perhaps coaxing, perhaps only an expression of
impatience. “Give me a hint.”
“Tocohl never said she was a judge.” That had to be said first, for the sake of reliability, but having
said it, he had no idea how to continue.
Buntec’s eyes narrowed, then widened. Without warning, she let out an ear-splitting whoop,
simultaneously slapping him on the shoulder. Startled, he drew back. “Buntec...”
But she was laughing. Wiping her eyes with her fist, she took a deep breath, sobered. “Ringsilver
boots,” she said, “I mighta known.” Once more she planted her fists on her hips, glared back in the
direction of camp. “So Old Rattlebrain found out and turned her in, did he?”
It took swift-Kalat a moment to decipher this. “No,” he said, “Tocohl sent for the judges.”
“She turned herself in?” In the distance, a single trader put down in the flashfield. Watching it land,
Buntec said, absently, “That’d account for their timing. If she’d sent a message capsule just after
she—arrived.” Once again she slapped him amiably on the shoulder. “Well,” she said, “let me see what I
can do.” Without explaining, she started for the hangar at a trot.
Swift-Kalat hurried after her. As she threw open the hangar doors, she said, “D’ya know why we
need byworld judges?” Not giving him a chance to consider this, she answered her own question:
“Context. And we’re bloody well gonna see that they get all the context they can handle, and then
some!”
Her daisy-clipper was first out of the hangar. As she passed, swift-Kalat could see that she was
speaking into the comunit. He hoped whatever she intended was clearer to her current listener. He
hoped, as well, that whatever she intended would be of some help.
He guided the daisy-clipper toward the trader to ground it just behind Buntec’s craft—he hadn’t the
skill to hover at the hatch the way she did—and slid from it to help his passengers with their gear. From
the amount of it, they intended something of a stay. For some reason he could not identify, this gave him a
sense of relief—as if this implied deep consideration rather than hasty judgment.
Their introductions doubled this sense of relief. Nevelen Darragh had the white hair and lines of great
age—something one seldom accrued without accruing experience to match—and piercing blue eyes that
would miss nothing.
Geremy Kantyka looked mournful, as if he would have preferred to be elsewhere, although thedesign on his 2nd skin seemed to have been chosen to suit Flashfever. “I’m here as an onlooker,” he
explained in Jenji, “I’m an old friend of Tocohl’s.” Which, thought swift-Kalat, went a long way in
explaining his mournful expression.
The third was a puzzle: there was something familiar about her but he could place neither her face nor
her name, Bayd. The familiar was her stare of wonder at her surroundings. Geremy Kantyka had to
nudge her twice before she took formal notice of swift-Kalat. “Bayd,” said Kantyka once again.
“Sorry,” she said, but her gaze was abruptly caught by the flashgrass. “Is it always like this?”
“It’s more impressive during a storm,” swift-Kalat said. “This is a lull—for safety’s sake, we should
be going.”
“Yes.” Again the words were abstract in her wonder. Geremy punched her this time, causing
Nevelen Darragh to laugh and say, “The woman who forgets her manners is Bayd Shandon, swift-Kalat.
Not a byworld judge.” For some reason, this drew a laugh from Bayd Shandon.
“No,” she agreed. “Not a judge. I’m here as a glossi”—swift-Kalat frowned at the unfamiliar
term—“an expert in languages.” Her forearm shot sharply down, proving the reliability of her statement.
She hefted her gear into the daisy-clipper and followed it, sliding to the far window to continue her
gaping. The other two seated themselves in the back, and swift-Kalat climbed in beside Bayd Shandon
with a renewed sense of relief. All three were Hellspark; all three spoke his language as if they had been
born to it. That meant he could explain what had happened in Jenji. In Jenji, they could find no fault with
Tocohl’s actions or words.
The daisy-clipper rose from the flashgrass, drawing a wordless exclamation of delight from the
woman beside him. As he aimed the craft back to base camp, he glanced briefly at her. Her hand shot up
to point: in the distance, lightning crackled into a stand of lightning rods—most likely, the one in which the
sprookjes waited out the storm.
“Tocohl Susumo has made a start at establishing a pidgin to enable us to communicate with the
sprookjes.”
“A pidgin?” Bayd Shandon sounded astonished, as if he had somehow called into question Tocohl’s
reliability.
Swift-Kalat realized the implication. “The sprookjes,” he explained, “communicate by ruffling their
feathers. Tocohl and Maggy, between them, have developed a way to respond in kind, but the rest of us
will have to make do.”
“Ah,” said Bayd, “that’s better.”
“Ruffles her feathers,” said Geremy from behind him. His tone made it sound dire. “The talent runs in
the family, Bayd.”
“Which one, Geremy?”
When Geremy only grunted in reply, Bayd laughed again. And this time swift-Kalat took his eyes
from the terrain long enough to have a closer look at her: the same red hair, the same gold eyes, the same
chiseled features—though in Bayd they were sharpened as if an abstraction of Tocohl’s.
“You are a relative of Tocohl Susumo?”
Bayd grinned at him, leaving no doubt. “Her mother,” she said and in response to another grunt from
Geremy, she added, “Don’t let Geremy disconcert you. He would sound the same if he were being
awarded his fifteenth status bracelet.” She snapped her wrist down, and her laughter passed for the ring
of authority.
Maneuvering the daisy-clipper into its hangar took his full attention for the moment, but once it was
grounded and stilled, he turned to Bayd Shandon. “The presence of byworld judges may call into
question your daughter’s reliability. I assure you I have no such doubts.” He brought his wrist down,
letting his bracelets speak for him. In the confinement of the daisy-clipper, the sound was shattering.
When the last of it had died away, Nevelen Darragh said, “This gets more interesting by the moment.
I look forward to hearing your account, swift-Kalat.”
For once, swift-Kalat wished she had spoken in GalLing’. Unlike his own language, GalLing’ would
have made a clear distinction between an informal telling and the testimony of a trial. Not that he would
have spoken differently in either case but in GalLing’ her choice of word would have given him anindication of her intentions. To ask her to repeat herself in GalLing’ might imply that her Jenji was
inadequate and he had no wish to impugn her reliability. Regretfully he let the matter go and led the three
through the gusting rain and into base camp.
He paused for a moment at the perimeter fence, wondering where to take them. He decided against
the infirmary. Then, seeing Buntec urge her party into the common room, he followed, hastening his steps
as the rain quickened.
He ushered them in and found them towels.
“... That’s right,” Buntec was saying to her charges, “I’m not giving formal evidence so you’re not
listening, but that’s not going to stop me from saying it anyhow.” She glared at Kejesli, set her fists at her
hips, and raising her voice so that it carried to Darragh and Kantyka and Shandon as well, she went on,
“If you find Tocohl guilty of impersonating a byworld judge, when she risked her ass to give the
sprookjes a fightin’ chance, then you don’t deserve the title yourselves.”
“Buntec,” snapped Kejesli, half rising from the table at which he sat, his knuckles blotched from the
effort of gripping its edge, “that’s enough.”
Buntec glared back. “That’s Kejesli,” she said, half introduction, half insult. Under the heat of her
glare, something softened in Kejesli’s face, although his hands remained tense. “For now,” he added.
“Then I’ll save the story of Edge-of-Dark’s boots for later,” Buntec said, her own tension gone as
quickly as it had come. Turning back to the newcomers, she invited inquiry with a tap to the top of her
boot and a broad grin. An answering grin from one of the newcomers told swift-Kalat she’d found a
listener. He was curious himself, although he knew Buntec’s accounts were more fiction than truth,
however careful she was in his presence.
“Swift-Kalat,” said Buntec—again her manner of delivery made it something more than an
introduction—“who will tell you true whether you hear it or not.” It was some form of challenge she
leveled at the newcomers. “He was the one who told us the sprookjes were sentient. Not his fault we
were too stupid to listen and too bone-lazy to check it out.”
She swung her hand to indicate the others. “Yannick Windhoek. Harle Jad-Ing. Mirrrit.”
Yannick Windhoek was a sour-faced man. He scowled at Buntec, scowled at swift-Kalat, then
greeted swift-Kalat in lightly accented Jenji. Zoveelian, like Ruurd, thought swift-Kalat, but, unlike Ruurd,
this man was trained in what Tocohl called “the dance.” His movements caused no discomfort; it was
only his grim demeanor that worried swift-Kalat.
The other two were more reassuring. They held hands like a couple of courting ten-year-olds.
Hellspark both, they greeted him in perfect Jenji. Mirrrit, the woman, was tall and slim and elegant, with
penetrating brown eyes. Harle Jad-Ing—he was Buntec’s listener-to-be—was small, bright-faced, eager.
Still, such impressions gave swift-Kalat nothing he could speak of reliably. He laid them aside,
awaiting further information, to introduce the three who had come with him. And then was forced to
repeat himself as John the Smith, Hitoshi Dan, and Vielvoye—a glance at Buntec’s welcoming grin led
him to believe she had been the one to notify them—entered and gathered, still dripping, to examine the
newcomers.
For a moment, the crowd held a festive air, as if it were nothing more than the excitement of new
faces after three years of the same. Then Kejesli pushed himself forward. “Tocohl Susumo is
recuperating in our infirmary,” he said, taking Darragh for senior, possibly because of her apparent age,
and addressing his edict to her. “You will see her when layli-layli calulan, our team’s physician, so
permits.”
Geremy Kantyka’s morose expression took a sudden turn for the worse. Bayd Shandon frowned,
made as if to speak, but was preempted by Nevelen Darragh, who spread her hands and said, “As you
wish, Captain, although it was she who called us here.”
“What’s more,” said Tocohl’s voice from somewhere at the rear of the crowd, “now that they’re
here they won’t mind a few weeks waiting. It’s the trip that’s costly, not the time spent on Flashfever.”
Hitoshi Dan and John the Smith parted, then pushed farther to each side, to allow passage to Tocohl,
with Om im at her right. Tocohl’s face brightened. “Hi, Mom! What did they catch you at?”
“Curiosity.” Bayd grinned back, mirroring her daughter’s manner. “Geremy told me. I thought I’dcome along and see just what sort of trouble you’ve made this time.” She looked thoughtfully at Om im,
seeing something that swift-Kalat could not. “Is that necessary, Om im?”
It was Tocohl who answered: “It was, for one cut.” And Bayd frowned sidelong at Kejesli. “I heard
you were recuperating, but I assumed Captain Kejesli...”
“Captain Kejesli wasn’t entirely.” Tocohl touched her side. “Broken rib. Maggy’s holding me
together with bailing wire.”
From behind her, Maggy corrected, “I tightened the 2nd skin where layli-layli calulan told me to
tighten it. She should be lying down.” Nudging its way past John the Smith, the arachne stepped warily to
the fore, as if to defend Tocohl, then said, “Geremy!” and darted forward to stop at the woeful man’s
feet. “Tell Tocohl to sit down, at least, then introduce me to Judge Darragh before Tocohl forgets again.
Hi, Bayd! Long time no see!”
“Veschke’s sparks, Tocohl—sit down before you fall down—what have you been feeding her?”
Geremy picked up the arachne to set it on a table, drew a chair for Tocohl, looking hangdog at first one,
then the other. Tocohl sat, Om im still at her right hand.
Maggy said, “I don’t eat.”
“Ha!” said Buntec. “You scarf up everything in sight, kid. You eat info the way a Jannisetti hog eats
hogchow.”
“I don’t get it.”
“We feed ’em by the shovelful,” Buntec said, “they suck it up the same way.”
Bayd said to Geremy, “I think you just had your question answered: a diet of pure Jannisetti. Long
time no see to you too, Maggy—and this is Judge Darragh.” This time Bayd Shandon made the
introductions all around.
When she had finished, the arachne settled in the circle of Tocohl’s arms, tilted upward, and said,
“Are they real judges, Tocohl?”
Buntec guffawed, along with two or three others, notably Bayd and Om im. The rest, swift-Kalat
included, stiffened, not appreciating the implications of the question. But Tocohl laughed too, long and
hard, until she had to bring up a hand to press against her side.
“Was that funny?” Maggy demanded.
“The emphasis was,” Tocohl said, wiping tears from her eyes. “And how would I know? You’re the
one with a list of byworld judges.”
“Could be their fathers.”
To this Tocohl seemed to have no reply. It was Nevelen Darragh who leaned forward and said,
“Would your list have voice signatures, Maggy?”
A rude noise issued from the arachne’s vocoder and to it Maggy added, “I can match any voice
signature, without half trying.”
Tocohl eyed Darragh with a look that was clearly sympathy. “Nice try,” she said.
“Only one way to tell, Maggy,” Buntec said. “There’s an old Jannisetti proverb—” She fixed Darragh
with a gimlet eye. “If it looks like a judge and it acts like a judge, then it is a judge.”
“Oh,” said Maggy, “but what does a judge look like?”
Buntec spread her hand. “Take a good long look at Tocohl,” she said. “Now you know as much as I
do.”
The arachne tilted up at Tocohl once more, as if to indicate that Maggy was doing precisely as
instructed. “I rather think,” Tocohl said, “it’s not that simple.” Laying a hand on the fat body of the
arachne, Tocohl raised her eyes to meet Darragh’s. “Might as well finish what you started,” she said, then
tensing, “I come for judgment—”
Yannick Windhoek snorted. “Damned overeager kids,” he said, scowling fiercely, and Tocohl turned
to look at him, startled. He went on, “I haven’t even had my lunch yet, and she wants a judgment. Never
give a judgment on an empty stomach, child. It’s the surest way to make mistakes.”
Nevelen Darragh glanced sidelong at Windhoek—from his vantage point, swift-Kalat thought he saw
the corner of an amused smile but couldn’t be sure—and then she turned to face Tocohl again. “As you
so rightly pointed out,” she said, “the trip is costly. Once here, however, we are hardly pressed for time.Give us a few weeks to acquaint ourselves with this world before you make demands of us.”
“Yes, of course,” said Tocohl, seeming chastened but no less tense for the temporary reprieve.
“Oh, good,” said Maggy, “that means you can go back to bed and heal some more. Make her go
back to bed, Bayd.”
“What makes you think I have any more influence than you do, Maggy?”
“Geremy then,” Maggy said, “he can check her rib.”
“Don’t tell me the doctor here is a quack!” Geremy said.
“Layli-layli calulan is an Yn shaman,” Maggy corrected, reverting momentarily to her previous prim
tone. “Honestly, Tocohl, I don’t know where he gets these words.”
Geremy Kantyka stared at the arachne, his eyes wide with astonishment. Tocohl burst into laughter
and swift-Kalat could almost see some of the tension drain from her frame.
“Oh, good,” said Maggy. “It was a joke. I thought so.”
When Tocohl had at last caught her breath, it was to say, “I’m proud of you, Maggy.”
The arachne tilted upward. “I’m proud of me, too.” Unfolding the arachne, Maggy stepped it to the
edge of the table. “C’mon, Geremy, make yourself useful. If you check her rib, at least she’s gotta lie
down that long.”
“Go on,” said Buntec, elbowing Geremy in amiable fashion, “make the kid happy.”
Geremy, looking ever more woebegone for the elbowing, said nothing but moved to help Tocohl to
her feet. Om im rose as she did, and Bayd Shandon followed. Maggy settled the arachne once more on
the table. “Maggy?” Tocohl questioned over her shoulder.
“I wanta watch here too,” Maggy said. “Somebody’s gotta make sure they don’t steal the
silverware.”
Darragh eyed the arachne. “Pretty cocky for somebody who’s already tried to rifle my computer,”
she said, surprising swift-Kalat with her use of Jannisetti phrasing.
“Maggy!” This time Tocohl’s voice mingled surprise and disapproval.
The arachne hunched down. “Did I do wrong?”
“What did I tell you about going through swift-Kalat’s cupboards?”
“It’s impolite—at least in public.” The arachne sank lower. “I’m sorry, Tocohl. I didn’t think it was in
public.”
Tocohl splayed her hand at her throat. “My apologies, Judge Darragh. The fault is mine, not
Maggy’s. I set her a bad example.”
“No offense taken. I had expected you to try.”
“But Tocohl didn’t—” Maggy began.
Still looking at Tocohl, Darragh finished, “I had not expected Maggy to try.”
“You don’t know me very well,” said Maggy.
“So I see.” But it was Tocohl that Darragh continued to watch. Tocohl flushed under her scrutiny
and, at last, reached for the arachne. “Let her stay,” said Darragh.
“All right,” Tocohl said reluctantly. “Maggy, mind your manners. If you give them any trouble, they
have my permission to kick the arachne out.”
“What about mine?” said Maggy, imitating to perfection the tone of challenge that swift-Kalat had
heard only once or twice from Tocohl.
Tocohl sighed. Pressing a hand to her injured rib, she said, “Maggy, do you want me to go back to
bed or not?”
“Pure blackmail,” said Maggy, “I set you a bad example too. Okay, if they say get out, I get the
arachne out, I promise. Now go back to bed.”
Swift-Kalat was still observing the arachne as Tocohl left, taking with her Om im, Bayd, and
Geremy. The light touch of a hand on his arm was sufficient to startle him. He jerked to look, first at the
hand, then into Darragh’s seamed face.
“Come,” she invited in Jenji, “sit with us while we eat. Your words on Tocohl Susumo’s behalf have
intrigued me. I wish to hear your account of her actions with no further delay.”
This time swift-Kalat was not sorry she had spoken in Jenji: the word GalLing’ translated as“intrigued” was, in his language, the indication of a thirst for knowledge so strong that by her use of it he
knew her to be a seeker after truth. And if the truth were spoken of Tocohl, she had nothing to fear.
“Yes,” he said, “I too wish you to hear my account.”
But before he could follow her to the gathering of judges, she paused to narrow her eyes at the
arachne that Maggy now bounced from side to side, dangerously risking it at the very edge of the table
on which it stood.
What was the term Tocohl had used? “Kinesics,” he explained to Darragh. “She uses van Zoveel’s
kinesics to express impatience or to obtain attention.” He addressed the arachne: “What is it, Maggy?”
“I too wish to hear your account.”
“I will speak in Jenji,” he told her, by way of warning her of the difficulties she might face in
understanding.
“Good, then I can practice my Jenji. What I do not understand, I will ask you to explain later.” When
neither of them spoke, she added, “I am polite,” bringing a smile to Darragh’s face.
Although Maggy had spoken in Jenji, swift-Kalat knew that, like any youngster, her reliability was
not high in that language, either in the speaking or in the hearing. To Darragh, he said, “She is three years
old, but as she says, she is polite. I will explain to her later.”
Again Darragh smiled; this time swift-Kalat saw the thoughtful look that accompanied it. “Then she
may accompany us,” she said, and swift-Kalat held out his arms to offer Maggy transport for the
arachne.
When he set it down in front of the other judges, Yannick Windhoek scowled at it once, then
resumed eating. Harle Jad-Ing and Mirrrit smiled at each other. “A spy in our midst!” said Mirrrit,
peering with exaggerated concern at the arachne. As she spoke in GalLing’ and her manner so strongly
suggested Buntec’s, swift-Kalat did not bridle as he might have once.
The arachne did. Standing it to full height, Maggy said, “I’m not a spy. I can keep secrets. Ask
layli-layli calulan if I can’t. She’ll tell you.” Then the body tilted abruptly upward at Darragh. “I won’t
keep secrets from Tocohl, if that’s what she means. Do you want to kick me out?”
Mirrrit looked startled. Even Yannick Windhoek glanced up again. “No,” said Darragh. “Maggy, we
have no intention of keeping this a secret from Tocohl. We simply want to know what happened here.
Mirrrit was making a joke.”
“Yes,” said Mirrrit, splaying a hand at her throat, “it was intended as a joke.”
Maggy reversed the tilt on the arachne. “You’re not much better at jokes than I am,” she observed,
and Harle Jad-Ing, laughing, said, “Perhaps, but she needs the practice or she’ll never be good at it.”
Mirrrit punched him amiably in the shoulder.
“Oh,” Maggy said. “We could practice on each other, Mirrrit, if you like.”
Once again swift-Kalat saw the sudden sharpening of interest, this time Mirrrit’s, as she glanced from
Jad-Ing to the arachne. “Yes,” she said, “I’d like that, Maggy.”
“Fine,” said Darragh, “that’s settled then.” Swift-Kalat was momentarily distressed to hear her
speaking in GalLing’, but she raised her voice to take in the rest of the surveyors who made no attempt
to hide their interest: “We beg your indulgence. We have need to speak and to listen in Jenji for the sake
of clarity.”
The announcement drew a number of hostile glares from various members of the survey team, many
of them directed at swift-Kalat, but Buntec grinned. “Tell ’em straight, swift-Kalat,” she charged him; to
the rest she said, “Maggy’ll tell us all later—right, kid?”
“Bet your ass, I will,” said Maggy, flawlessly matching Buntec’s brash good humor, then adding,
“Did I say that right?”
“Bet your ass, you did,” Buntec assured her. And in that brief exchange, swift-Kalat saw the hostility
of the others fade as quickly as it had come, leaving in its absence only an intense curiosity.
He turned again to Darragh, and without waiting to be asked a second time, he began his account by
quoting his own note. If they could hear and understand its import in Jenji, then they would be capable of
hearing and understanding what he had to say about Tocohl.
Behind him, Vielvoye hovered closer, as if proximity might make the words comprehensible; when itdid not, he moved to the opposite side of the table where he might watch swift-Kalat’s face, and
squinted with effort. But the judges—all four of them—could hear what swift-Kalat had said. With a
clash of bracelets to emphasize his reliability, he settled in to tell it all.
When he came to account how Maggy had invoked the Hellspark ritual of change to make him her
sister so that he might speak to layli-layli calulan, Windhoek stopped eating to stare at Darragh. It was
so sharp a change in manner—from all of them—that swift-Kalat immediately splayed his hand before his
throat. “I intended no offense,” he said. “If I have, in ignorance, broken a taboo by speaking of this
matter...” The arachne rose and he ended his appeal directly to Maggy. “You said nothing of taboo.”
“If you will permit me to speak, Judge Darragh,” Maggy said, “I believe I may speak most reliably on
this matter.”
“Please do.”
“Swift-Kalat, before that day, the Hellspark ritual of change did not exist. I invented it to enable you
to speak to layli-layli calulan. It was dream, not lie, and layli-layli calulan will tell you that the
Hellspark ritual of change will serve the Hellsparks well on her world.” The arachne mimed a wrist-snap
of authority, that was all the more compelling for its awkwardness.
Mirrrit said, “Veschke’s sparks, but that’s brilliant! Think of it, Harle, think what you could do as my
sister!”
Harle grinned at her and said, “That’s positively insidious. From second-class citizen to first in one
‘Hey, presto!’”
“It won’t work that well in practice,” Windhoek pointed out.
“Oh, I know that,” Harle said, his grin only fractionally lessened by Windhoek’s scowl, “but think of
the wonderful little seeds of doubt that plants.”
Swift-Kalat might, at some other time, have been fascinated by this exchange, but the arachne’s
silence told him that it was his assessment Maggy awaited. He said what he would have said to any child:
“Think carefully. Have you spoken reliably? Such an invention would seem more characteristic of
Tocohl—”
“Oh!” said Maggy, stepping the arachne closer. “I did not mean to imply that Tocohl was not
capable of inventing the Hellspark ritual of change. Tocohl’s very inventive.” There was a grunt of assent
from someone, but swift-Kalat did not take his eyes from the arachne. Maggy went on, “I’m sure she
would have invented it, or something that would do, if she had been in my situation. But I couldn’t even
talk to her, swift-Kalat. I had to do something! Did I do wrong?”
“No, Maggy,” he said firmly, “you did the right thing. I do not question your reliability.” With that, he
brought down his wrist for emphasis. In the clash of bracelets he thought: She has language and she
created what is surely an artifact—she fulfills two of the three prerequisites for the legal definition of
sentience.
When the sound of his bracelets had died away, Maggy said, “I’m glad. I’m sorry for the
interruption, Judge Darragh.” Once again she folded the arachne’s legs and settled it to watch. “Go
ahead with your account, swift-Kalat; I will not distract you any further.”
Swift-Kalat took up his account where he had left it, without comment. Had he commented, he
would have been forced to call the reliability of her final statement into doubt. Her very presence was a
distraction, as absorbing a distraction as the sprookjes. Even as he told the four judges of the first hint of
Megeve’s treachery, a portion of his mind was planning to speak to Tocohl about Maggy as soon as he
had finished here. If Tocohl could speak of art, then the opportunity to bring Maggy to the attention of
four byworld judges should not be passed by. There would be no survey team to make such a judgment,
rightly or wrongly, in the case of an extrapolative computer.
As reluctant as she might be to admit it with any more than a small sigh of content, Tocohl was
grateful to be off her feet again. Layli-layli calulan said, “You will someday outgrow that streak of
stubbornness and settle into comfort.”
The thought startled Tocohl briefly upright, shock spreading across her face. “Veschke’s sparks, I
hope not!”“Then learn to apply it where it will do good instead of damage.” Layli-layli pushed her gently down
and touched fingertips to her ribcage. The rest kept silent while the Yn shaman applied her own good.
When she had finished, Tocohl introduced to her the bayd shandoni and “my sister, by the
Hellspark ritual of change, the geremy kantyka.”
Having been introduced as her sister, Geremy greeted the Yn shaman in Yn-female without a single
misstep; nor did he flinch, as many would have, under the scrutiny that resulted. “Yes,” said layli-layli at
last, “it will help. Not as much as I had hoped but... it will help.”
“Maggy will be pleased to hear that,” Tocohl said, knowing that the mention of her name was
sufficient to call Maggy’s attention to the tribute. “And before you ask too: She’s not a byworld judge,
layli-layli.”
Layli-layli calulan’s small hand closed tightly around her bluestone rings, her eyes shifted to Bayd.
“Nor is Bayd,” Tocohl said.
Frowning, layli-layli calulan said, “Then I shall seek someone who is. There is, still, the matter of
Oloitokitok to be judged.”
Tocohl reached up to clasp her wrists in sympathy. “Seek Nevelen Darragh,” she said. “She will
satisfy you. She dreams and her dreams have strength.”
“She’s in the common room with the others,” Om im put in. “Put your rings back on, layli-layli.
They’re discussing Tocohl’s fate in there and tempers could be running a bit high by now.”
Layli-layli calulan glanced down at her clenched fist. “Yes,” she said softly. It took visible effort to
open her hand; the rings had left small distorted ovals in her flesh. With great deliberation, she replaced
the rings on her fingers, then she turned toward the door. Abruptly, she turned back, concern mingled
with the suppressed anger in her face. “Om im,” she said.
His concern mirroring layli-layli’s, Om im touched the hilt of his knife and glanced inquiringly at
Tocohl who said, “My knife is yours, layli-layli—but he serves only as a reminder. He will not intervene
to his own risk.”
“I need only the reminder, tocohli. That I swear.”
Satisfied, Tocohl turned her thumb up to Om im. He rose to join layli-layli calulan, and in a moment
the two of them had vanished into a smear of gray rain.
“That’s a useful man to have around,” Tocohl said, settling her head on her good arm.
“That he is,” said Bayd, grinning. “I’ll tell you my story if you’ll tell me yours.”
“Done,” said Tocohl, with the snap of her fingers that sealed a bargain between Hellspark traders.
Then she sighed her relief and said, “Veschke’s sparks but I’m glad to see you, Bayd. I was scared to
death I’d have to leave Maggy in the hands of a novice, however useful he may be under other
circumstances. I’m happier to have her looked after by someone who’s had considerable experience in
such matters.”
Geremy glanced in the direction Om im had gone. “Not a pilot, I take it?”
In spite of herself, Tocohl grinned. “Not a parent,” she corrected. “Bayd, on the other hand, has not
only raised a passle of children, but I can vouch for how well she’s done it. I trust her to do as well by
another one.”
Bayd was silent for a long moment. At last she said, “Then you were right. It’s happened.”
“I think so, yes.” To Geremy, who gaped, she added, “I told you Garbo wouldn’t be so dumb if
you’d put a little money into memory for her.” She turned her attention back to Bayd. Bayd’s own ship
was older, no less treasured than Tocohl’s as a craft—but its computer was not an extrapolative
computer and could not hold the same promise as Geremy’s Garbo.
From the regret on Bayd’s face, Tocohl knew she was thinking along the same line. “I’ll look after
Maggy, Tocohl, if for any reason you can’t...”
“You know the reason, Bayd. The penalty for impersonating a byworld judge is restriction to one
world for life. Maggy’d go crazy restricted like that; she’s not meant for it.” Frowning, she added,
“Neither am I, if I spoke in Jenji, but I knew that. As long as you’ll look after Maggy, I’ll have one less
worry.”
Bayd snapped her fingers; the sound was loud in the stillness between thunderclaps.“Thanks, Mom.”
With effort, Tocohl produced a wry smile. “You’ll have to look after the sprookjes, too. You’ll need
Maggy’s help for that. Did they tell you anything about the sprookjes’ language?”
“Only that they ruffle their feathers.”
“That they do.” This time Tocohl’s smile was genuine. “You’ll love it. Do you know I almost missed
it? Let me tell you—Wait, let me show you as well.” (Maggy?)
(You’re supposed to be resting.)
(I am resting. I can rest and talk at the same time.)
(Oh. Would you like to watch?)
(No, I want to brief Bayd on the sprookjes’ language. Can you link up with her spectacles too? And
Geremy’s.)
(Bet your ass. Better I show her than you get up and demonstrate.)
(Cheeky,) said Tocohl.
(Yeah,) Maggy agreed, and she sounded very pleased with herself.
Tocohl smiled and got on with the business at hand. With the help of Maggy’s tapes, she’d need only
a few hours to teach Bayd everything she’d learned of the sprookjes’ language and the pidgin they’d
been creating as well.
They were well into it—Bayd had spotted two sprookje “expressions” from the tapes alone—when
Maggy cleared Tocohl’s spectacles. She must have simultaneously done the same for Bayd and Gererny
as well, for they both raised their heads to query Tocohl.
“Swift-Kalat wants to talk to you, Tocohl,” said Maggy, and this time the voice issued from the
arachne’s vocoder. The three Hellsparks turned: swift-Kalat stood at the doorway, toweling off the
dripping arachne. “He said I should bring the arachne because it’s easier for him to talk to me this way.”
“All right,” said Tocohl. “What is this about, swift-Kalat?”
“It’s about Maggy.” He strode toward them, wiping rain from his arms as he came. The arachne
scuttled along at his heels.
“About me?” Maggy asked, putting just the proper amount of surprise into her phrasing.
Draping the towel around his neck, swift-Kalat watched the arachne for a moment. Maggy, not
receiving an answer to her question, trotted the arachne around to tug at Geremy’s ankle. Geremy bent
to lift it onto the bed, where Maggy sent it—delicately—the length of Tocohl’s body to stare up at
swift-Kalat. “About me?” she repeated and began to rock the arachne.
Tocohl pointed a long finger, knowing Maggy could see equally well from the rear of the arachne,
and said, “Stop that.” The arachne stopped instantly. “What about Maggy, swift-Kalat?”
He frowned, first at her and then at the arachne. Bayd said, in Jenji, “Would you find it more
appropriate to speak your own language?”
His frown deepened. “No,” he said at last, “Maggy must understand what I say.”
“I speak Jenji,” Maggy said, sounding offended.
“He means no offense, Maggy,” Tocohl said. “He means to make things easier for you. He’s being
polite. Your Jenji really isn’t good enough to handle complex ideas.”
“If you say so,” Maggy said. “No offense taken, swift-Kalat.”
“I can translate for you, if you like,” Tocohl said.
Again he hesitated. Then he looked down at the arachne. “Maggy, are you sentient?”
Maggy gave the question her due-consideration pause, then she said, “Not legally.”
“Yes, not legally,” swift-Kalat said, “but without reference to legalities, Maggy—are you sentient?”
Tocohl, who had been hoping the question wouldn’t be raised publicly, frowned at the arachne. “Just
a moment, Maggy. Geremy, watch the door; warn us if anyone comes.” When Geremy was in position,
Tocohl said, “All right, Maggy. I’d like to hear your answer to that too.”
“Yes,” said Maggy, then, “Yes! Yes!” and the arachne bounced to her excitement. This time Tocohl
did nothing to inhibit the arachne’s display. In a moment, it stopped rocking of its own accord, ran the
length of Tocohl’s body—not nearly as delicately as it had the first time—to peer at her from her pillow.
“You’re not surprised,” Maggy said, accusingly. “Not a single sensor’s worth!” There was a pause, as ifshe were double-checking. “You’re worried.” The arachne hunched, reflecting that worry.
Tocohl laid a hand on its side. “A gesture meant to reassure,” she explained. “You haven’t done
anything wrong, Maggy, but yes, I’m worried.” She cocked her head to look at swift-Kalat, considering
him carefully. “I should have expected swift-Kalat to notice.”
“You mean you’re not surprised because you knew?”
“Exactly. Ask Bayd and Geremy if you don’t believe me.”
“She’s been expecting it for some time,” Bayd said; she too kept her eyes on swift-Kalat.
“Then why are you worried? I don’t understand.”
“Because swift-Kalat expects me to take your case to the byworld judges—there are four of them.
That was your thought, wasn’t it, swift-Kalat?”
“Yes, of course.” He looked puzzled. “If you can prove the three legal requisites, Tocohl. No one
could deny her ability to use language, and she invented the Hellspark ritual of change.”
Tocohl raised an eyebrow. “Perhaps I told her to say that,” she suggested.
(You lie!) said Maggy, but she kept her outrage private.
(Please wait, Maggy. I promise I’ll explain.)
(You’d better,) Maggy replied, a touch of warning in her voice.
Swift-Kalat frowned at Tocohl. “You were not in contact with her at the time.”
“True, but perhaps I left her with the idea in case of an emergency.”
“And told her to claim responsibility as well?”
“Why not?”
Swift-Kalat jammed his bracelets against his elbow. “I see: That is the argument the judges will take.”
Abruptly, he switched to Jenji, “Did you invent the Hellspark ritual of change?”
“No,” answered Tocohl in the same language, “I did not. Maggy did.”
“Tocohl, the judges all speak and understand Jenji reliably.” He snapped his forearm and his
bracelets dislodged to clash authoritatively. In GalLing’, he went on, “There remains only art to prove her
legal sentience.”
In spite of herself, Tocohl chuckled. “Oh, swift-Kalat—forgive me for saying so, but you did not hear
the full import of Maggy’s ritual of change.”
“What have I missed?”
“She might, very simply, have said: You’re now my sister, swift-Kalat. She didn’t. She added, ‘Hey,
presto!’—and that, in no uncertain terms, is art.”
“Yes! Yes, of course!” Swift-Kalat smiled broadly. “You can prove her sentience to the judges!”
“No.”
The word came out more sharply than she’d intended, and swift-Kalat stiffened, his smile shifting and
setting into a frown of distaste that verged on anger.
“Wait, swift-Kalat. Hear me out. I have good reason not to bring Maggy before a panel of judges.”
He folded his arms across his chest. He would wait, the gesture signified, and he would wait
patiently.
Tocohl, without turning, reached out to the arachne once again. “Maggy, please bear with me. I will
explain.”
“Okay,” said Maggy and settled the arachne in the curve of her arm. Its metal shell seemed to warm
to her touch as Maggy adjusted the temperature of her 2nd skin to compensate for its coolness.
“How old is Maggy, swift-Kalat?”
“Three years old, by her own count. Although I believe that was your estimate.”
“Yes. Now, if you were to assume a human being were directing all of her actions, what age would
you assume that human being to be? I’m asking a rough estimate only.”
“A very bright seven-year-old, I’d say.”
Tocohl glanced at the arachne. “Your reliability in Jenji is higher than I thought. Sorry, Maggy, I’ll
keep that in mind in the future.” Turning back to swift-Kalat, she said, “Is a seven-year-old sentient?”
“Tocohl—” he began, astonished.
Tocohl spread her hands. “Without reference to legalities and in Jenji, swift-Kalat, Maggy is aseven-year-old sentient?” She was cheating ever so slightly by forcing him into Jenji and she was well
aware of it, but he would understand her point quite clearly as a result.
In Jenji, there was only one possible answer to her question and, reluctantly, he gave it: “No. A
seven-year-old is not sentient.”
“Not?” said Maggy, lifting the arachne to its full height to stare at him.
“In Jenji,” Tocohl explained, “the term for sentience carries a connotation it lacks in GalLing’—only
an adult who is sound of mind can be sentient. A child lacks the responsibility.” She gathered the arachne
into her arm and said, gently, “You know a great deal, Maggy, but you haven’t had enough experience to
go with it. You still have trouble sorting fact and fiction, for example.”
“I’m learning.”
“I know you are. The point is, you don’t know enough yet. Let me put it this way: Would you like to
go off on your own now?”
The arachne hunched. Tocohl saw herself reflected in the ebony eye of its lens. “Do you mean
without you?”
“Yes, that’s exactly what I mean.”
“I wouldn’t like that at all. Who’d explain things to me? Who would I talk to?” The arachne began to
rock. “You wouldn’t make me go away without you, would you, Tocohl? Say you wouldn’t. Say it in
Jenji.”
Tocohl pressed a hand to her side. The ache was not physical. “That’s something I can’t say in Jenji,
Maggy. You know the penalty for impersonating a byworld judge as well as I do. But Bayd will look
after you—she’ll be there to talk to and she’ll explain things to you and, believe me, she’s very good at it.
She looked after me until I was old enough to look after myself.”
“But she’s not you!”
“I know—but she likes you and you like her. That’s the important thing.”
“I told you we should have skipped out.”
“Yes, and I told you—”
“We pay our debts,” said Maggy. She made a rude noise, drawing it out longer than a human would
have been capable of, and settled the arachne once more in the crook of Tocohl’s arm.
Tocohl looked up at swift-Kalat. “I trust I’ve made my point?”
“With one minor exception: byworld judges can declare a species sentient. That does not interfere
with the raising of that species’ young.”
“To my knowledge, Maggy is the only one of her kind. Geremy’s Garbo is exactly the same model of
extrapolative computer but Garbo isn’t sentient, at least not yet. The judges would be forced to decide
on Maggy and on Maggy alone. I won’t risk that—I’d rather trust her to Bayd until she’s grown up
enough to take care of herself.”
“I understand,” said swift-Kalat. “I will not raise the question before the judges. But, in five years’
time, I will come to Hellspark to inquire into Maggy’s status.”
“In five years’ time, go to the Festival of Ste. Veschke on Sheveschke. Bayd and Maggy will both be
there, along with sufficient judges to satisfy anybody. And make sure Geremy’s treating Garbo right,
too.”
The arachne tilted slightly. “Garbo’s dumb,” said Maggy, “not as dumb as Kejesli’s computer, but
dumb enough not to care how Geremy treats her.”
From across the room, Geremy said defensively, “Garbo’s a baby. And I didn’t know she was a
baby, Maggy. If she can learn the way you have, I’ll see that she has the chance.”
“Maybe you can advise him, Maggy,” Bayd put in. “Your experience might help him teach Garbo.”
“Really?”
“Really. Older children are often a great help with younger children—they remember what it’s like.”
“All right, I’ll help.” And from the vocoder came the sound of snapped fingers. “Get her lots of
memory, Geremy.”
“There goes the new artwork for my 2nd skin,” said Geremy mournfully. “Ah, well. I think I’d rather
have the company.”“Geremy, if it makes you feel any better, Maggy offered to duplicate your Ribeiro for me, and she
can do it too.”
“You said he wouldn’t like knowing that!” Maggy exclaimed. “Why are you telling him now?”
“Because now he will like knowing that. Situations change, and when they do, a human’s reactions
will too.”
“Do you really like knowing that now, Geremy?”
Geremy laughed. “Yes, I guess I do.”
“And before? Would you have liked knowing that at the Festival of Ste. Veschke?”
“I wouldn’t have liked it a bit at the Festival of Ste. Veschke,” he assured her.
“I don’t get it,” said Maggy.
Bayd laughed. “Then make him explain it, Maggy. It will be good practice for him—it’ll help him
when Garbo starts asking questions like that.”
“Geremy? Will you explain—”
“Later, Maggy,” said Geremy. He left the door and strolled toward them. “Om im and layli-layli are
back.”
Maggy, for once, was not to be put off. Thrusting the arachne to its full height, she sent it stalking
toward him. “Later!” she said. “That’s all anybody ever tells me. Why does everybody say ‘Later,
Maggy’”—she used a clip of Geremy’s voice and then followed it with a clip of Tocohl’s—“‘Later,
Maggy.’ And then they don’t even remember when it’s later. Later, my ass.”
This last was delivered in Buntec’s phrasing but it was Maggy’s voice. And the stalking tirade had
brought the arachne to the foot of the bed at the same moment as layli-layli and Om im reached it.
Only Tocohl saw that the expression on layli-layli calulan’s face was not the result of Maggy’s
speech. Maggy bobbed the arachne hastily. “Your pardon, layli-layli,” she said. “Om im, you were
supposed to remind her about the rings.”
Layli-layli calulan, who had been twisting her right ring angrily, dropped her hands to her sides and
forced a smile. “I was only twisting, Maggy. I had no intention of taking it off. Certainly not as a result of
anything you said.” She scowled, and it was as startling as her smile could be. “I agree with your
sentiments. But bear in mind that Tocohl was told ‘Later’ by the judges—and so was I.”
Tocohl raised a brow in surprise. “So,” she said, “we all get to stew in our own juices.” The
realization brought all the tension back to her frame and Maggy, reacting no doubt to the sensors,
brought the arachne back to her arm as if to comfort her by its physical presence. Well, yes, Tocohl
thought, glancing aside at it; I just taught her that and it does work. She bent her arm protectively about
the arachne.
(Tocohl,) said Maggy privately, (I have to say something now, not later.) Tocohl gave her full
attention. (I don’t like this,) Maggy went on. (This feels the way it felt when I couldn’t contact you.)
(I know, Maggy. I’m scared too—but at least we can talk to each other this time. That makes it
easier to live with.)
There was a pause of consideration. (Yes,) said Maggy at last. (At least until the judges decide about
you. And I’m very scared about that.)
For all that happened in the days that followed, they seemed to Tocohl to pass with painful slowness.
Maggy’s behavior had taken an uncharacteristic turn—not surprising, Tocohl supposed, in light of the
attitudes the survey team brought to bear on the panel of judges—but Tocohl took it as evidence that
Maggy was as frightened as she professed to be.
When Nevelen Darragh requested Tocohl’s tapes, Maggy stalked the arachne away. Her voice
sulky, imitative of the tone Kejesli had taken to using with the judges, she said, “Get stuffed. You
wouldn’t let me see your files.”
Shocked, Tocohl dropped to one knee beside the arachne. “Your pardon,” she said to Darragh
hastily. “Maggy,” she began, not quite knowing what to say beyond that.
Darragh’s eyes crinkled. Smiling reassurance, she too dropped to her knee to face the camera eye
directly. “Then I propose a trade, Maggy.”
“Forget it,” said Maggy, “I don’t do business with—”“Maggy.” Tocohl had no idea how that sentence would have ended and didn’t want to know. “In the
first place, showing Nevelen what happened in context certainly won’t hurt. In the second place”—she
touched the arachne lightly—“I trust you’ll make a canny trade and get something useful to both of us
from it.”
“Like what?” This time Maggy sounded interested.
Tocohl spread her hands. “I leave that to you. It will be good practice.”
“You want me to? You’re sure?”
“Context always matters, Maggy. In Darragh’s position I’d want it very badly.”
“Okay, if you say so. But if she wants it badly, I’m gonna deal high.”
“Good for you.” And Tocohl found herself exchanging a smile with Nevelen Darragh as Maggy
stepped the arachne forward to indicate her willingness to accompany the judge.
Tocohl did not kibbitz Maggy’s negotiations; she would have to learn sometime. To her surprise,
Maggy did not volunteer any information about them beyond the observation, “Nevelen Darragh is a
mean trader.”
“Then I hope you learned a few things from the experience.”
“Yes,” said Maggy. But from then on she kept the arachne close by Tocohl’s side, dogging her heels
even when there was no need for a separate presence.
A long series of thunderstorms kept meetings with the sprookjes brief and intermittent, but with
Tocohl and Bayd working in silent concert, knowledge of the language had progressed to primitive
sentences in both Tocohl’s pidgin and the sprookjes’ own native language. Darragh, it turned out, was as
good as Tocohl or Bayd, once she dealt with an established language. Learning that reassured Tocohl:
Darragh would be more than competent to handle any need for judgment that might arise between the
sprookjes and their newfound neighbors.
“Better a judge that speaks the language than one who relies on a translator, even if the translator is
Bayd,” Tocohl observed to Om im as they waited out yet another storm in his cabin.
The judges had taken up residence in the common room, and by unspoken agreement, the surveyors
socialized elsewhere—generally in the infirmary, with layli-layli calulan’s blessing. Rib healed, Tocohl
had long since decided she’d get more sleep on the cot Om im offered, even though her feet hung off the
end of it.
Alfvaen thrust her head in and said, without preamble, “Maggy, Bayd arranged it for us to spend the
storm in the lightning rods with LightningStruck.” That, they had learned, was the name of the sprookje
Tocohl had dubbed Sunchild; it carried more a sense of “reckless” than “brave” and suited her
admirably. “If you want to join us, come now.”
“No,” said Maggy, without hesitation, “but thank you.”
Alfvaen frowned briefly at Tocohl and then, after a quick glance over her shoulder at swift-Kalat’s
eagerness, she shrugged after her own fashion and vanished.
Tocohl said, “I’m surprised at you, Maggy. You won’t miss a thing here if you send out the
arachne...”
“I want to see you. I can’t see you through your spectacles.”
Om im leaned to one side to consider the arachne. “Stubborn,” he said, “I’ll bet I know where she
gets it, Ish shan.” To Maggy, he added, “It seems to me you might be interested in the activities of the
judges.”
“Buntec says they’re doing exactly what Tocohl did when we first arrived: reading the files, watching
the tapes, asking questions. She’s waiting for them to con Edge-of-Dark, she says.”
Om im laughed. “Maybe they have and Edge-of-Dark hasn’t caught on yet. John the Smith still
hasn’t touched his blade to how Tocohl trained him to stand on my safe side.” At Tocohl’s look of
inquiry, he added, “I asked Bayd; she asked Maggy.”
“You didn’t tell John the Smith.”
“Of course not,” said Maggy primly, then giving credit where credit was due, she added, “Bayd and
Om im didn’t think that would be a good idea.”
“I agree. Better he thinks it a matter of prestige than one of hazard. We wouldn’t want Om im to getan undeserved reputation for violence.”
“By all means,” Om im said, “let’s keep it to a deserved reputation for violence.”
Maggy stepped the arachne closer. “Was that a joke?”
“A small joke, but what else would you expect from someone my size? And you both need a little
cheering.” He brought his gaze level to meet Tocohl’s. “Stewing in your own juices is one thing, but with
four judges stirring the pot and Buntec throwing in spice by the handful—”
“Buntec? Have I missed something?”
“Maggy may have missed something: she can’t tape visuals for you from a hand-held like this one.”
He gestured just enough to remind her that Maggy’s hand-held was still at his belt. “Buntec told
Windhoek—the one who looks like he’s sucking a lemon?—Buntec told him that to charge you he’d
have to charge the entire survey team for contributory negligence and creating a public hazard.”
“Oh, Veschke’s sparks,” said Tocohl and laughed in spite of herself. “She didn’t really?”
“She did. And Edge-of-Dark backed her up. I wish you could have seen Windhoek’s expression; it
went well past the sucking a lemon stage.”
“You want to see?” Maggy said. “I didn’t miss it. I was watching with Geremy and Garbo. I didn’t
know it was funny though.”
“By all means, let me see.” And when Maggy had played through the sequence for her, she laughed
loud and long. Windhoek’s expression was all that Om im had promised. “Buntec chooses her targets
well,” she said, when at last she had caught her breath.
“That was funnier the second time,” Maggy said. “Most things aren’t. Why is that one?”
Tocohl gave this the consideration it deserved. “Part of it was actually seeing Windhoek’s
expression. Part of it was relief—Maggy, I was afraid you’d stopped socializing altogether. I was
worried about you.”
“Oh,” said Maggy, “because I keep the arachne here?”
“It hasn’t left my side for five days. Have you been talking with Geremy and Garbo all this time?”
“Since yesterday.”
“At least to members of the survey team and to Bayd and Geremy,” Om im confirmed. “As for the
judges”—he grinned—“well, she used a spate of Sheveschkem on Windhoek that turned Captain Kejesli
a remarkable shade of green. He’s been muttering the same words under his breath for a week. Maggy
may not have been talking, but she’s certainly been listening.”
“It was her not talking that concerned me. Even insults are something of a relief.” She eyed the
arachne sternly and added, in warning, “As long as you don’t make a habit of it.”
Chapter Sixteen
THE STORM RAGED through the night but it was not the storm that kept Tocohl awake. Bayd
had accompanied the party and she put the time to good use, conferring with Tocohl through Maggy.
Tocohl had the easier time of it, for Maggy screened out most of the blinding light and the deafening
thunder to convey only the sprookjes and Bayd’s commentary.
“I’ll take the next storm watch,” Tocohl said. “I wish I’d thought of that sooner; it would have saved
us a lot of time.”
There was a pause as Bayd waited out the thunder. “It’s hardly a matter of neglect. This is not
something I’d volunteer for more than once. If there is a next storm watch, you’ve surely got it. Most of
us are only here because we wanted to see if it lived up to Buntec’s lurid description.”
“And?” Tocohl prompted, amused.
“Buntec didn’t tell the half of it.”
Again there was a momentary silence from Tocohl’s vantage point; again Tocohl knew from the
sharp reactions of those nearest Bayd—Nevelen Darragh and swift-Kalat—that Maggy had blotted out
another thunderclap. The sprookjes sat content, excited only by Bayd’s questions and answers. They
had already learned that thunder and lightning distracted her, although neither Bayd nor Tocohl could tellif they understood why.
Given the ruffling of their feathers, Tocohl thought there was a good chance they were speculating on
the subject among themselves. She could only make out a phrase here and there, and the one that
recurred most often was “strange sprookje.”
When the sound faded back in, Alfvaen—Tocohl saw her at Bayd’s glance in her direction—said,
“Bayd, I’m just curious, but do they have any trouble telling you from the other Hellsparks? When they
talk about me, Om im says, he can always tell because they look like me for just a moment.”
“No, they can tell us apart better than we can them. They’ve had to give us names, though, which
they continue to use. Random syllables don’t translate well into sprookje, and Tocohl and I decided it
was safer for us to learn sprookje before we confuse the issue again by trying to teach them a purely
verbal language like GalLing’.”
“What sort of names?” The voice was swift-Kalat’s,
Bayd turned to give Tocohl a view of him through dimming rain and said, “Remember that they
weren’t aware each of you was from a different culture. When they discussed you among themselves
they referred to you with a proxemic and kinesic overlay that defined each of you unmistakably; in
practice, you were ‘the Jenji,’ Kejesli is ‘the northern Sheveschke,’ Dyxte ‘the ti-Tobian,’ and so forth.”
“Oh, is that all,” said Alfvaen, sounding disappointed.
For Bayd’s ear, Tocohl said, “She was hoping for something more romantic.”
Bayd took the cue and said, “You got an actual name, Alfvaen. You, they call
‘One-Who-Was-Poisoned.’ It took us three days to puzzle that one out. We kept being distracted by
the Siveyn overlay they used and didn’t realize they were being more specific than that.”
Alfvaen looked from Bayd to LightningStruck, suddenly embarrassed. “Oh, Bayd! Can you tell
LightningStruck that I didn’t mean to hurt Tocohl, so she won’t be afraid of me?”
“She’s not afraid of you,” Nevelen Darragh said, in such a way that Alfvaen was fully reassured by
the sound of the statement alone, and once again Tocohl too was somehow reassured by the judge’s
perspicacity.
“How about you?” Alfvaen asked. “If I understand this correctly, Tocohl simply would have been
‘the Hellspark.’ But so are you and Bayd and—”
Bayd laughed. “What they use to signify Hellspark is any behavior that compromises between two or
more
cultures.
Tocohl
is
now
officially
known
as
Strange-Sprookje-Hellspark-With-A-Crest-Like-The-Sun-On-Penny-Jannisett, and I got dubbed all of
that plus ‘Newly-Arrived.’”
Alfvaen turned widening eyes on Darragh who smiled and, making the Siveyn gesture of formal
self-introduction, said, “Strange-Sprookje-With-A-Crest-Like-Frostwillow, at your service.”
Alfvaen began a smile—but it froze and faded. Crossing upturned arms at her wrists, she said only, “I
understand.” The gesture said in no uncertain terms that the two of them were barely on speaking terms
but that Alfvaen would be civil.
Bayd turned swiftly, granting Tocohl a view of Darragh’s reaction: a swift upcurling of both hands
that said, Give me time to prove myself.
“The best she could do, under the circumstances,” Tocohl commented, for Bayd and Maggy only. It
was not sufficient to soothe Alfvaen; Tocohl could see that rigid control set in muscle by muscle. “Bayd,”
she said in warning, “remind her that I called in the byworld judges myself.”
That had the effect Tocohl expected. Alfvaen frowned but her limbs loosened, her shoulders sagged.
“She didn’t have to do it, Bayd. Why did she?”
“But she did have to do it!” Darragh said in surprise. “I thought you understood that.”
“I don’t,” snapped Alfvaen, and swift-Kalat said, “To speak reliably in Hellspark, you mean.”
Darragh looked from one to the other in astonishment. Bayd said, “I think you’d better explain it to
them, Nevelen. They apparently haven’t thought it through.”
“She had to do it for the sprookjes’ sakes,” Darragh said. “The moment she decided they were
worth the risk, she doubled their chances of safety. Your accusation of murder would have held up
Kejesli’s report for a time, swift-Kalat, but for how long? Suppose Tocohl hadn’t found the language.What then?”
He snapped his wrist, startling even Tocohl with the sound. “Then,” he said, “I’d have made an
official request for a panel of byworld judges—” In mid-sentence, he stopped and stared at Nevelen
Darragh.
“Which would have taken months to clear through channels,” she said. “That’s what happens when
you make an official request through a bureaucracy. And meanwhile, the chances are good that MGE
would have sold the planet, the Inheritors of God would have taken possession, and the sprookjes would
have been in very great danger, if what happened to Alfvaen more than once is any indication.”
She shifted to take in Alfvaen and went on, “It takes precisely the same number of byworld judges to
try someone for posing as an official of the Comity—or as a byworld judge. And it gets an instant
response if it goes to the right recipient.”
“Which it did,” said Alfvaen.
“Which it did.”
Maggy’s soundproofing went briefly into operation. Tocohl saw the others flinch but Alfvaen,
thoughtful now, kept her eyes on Darragh. When the sound returned, it was only the sound of rain.
Without a word, Alfvaen turned her left hand palm-up, curling the fingers as if to enclose something very
fragile. It was fragile, indeed, for it was the beginning of renewed trust she offered to Nevelen Darragh.
Beyond her, LightningStruck curled her hand in imitation of the gesture. Bayd said, “Veschke’s
sparks, Alfvaen. That’s going to take me a month to explain!” and started in.
Tocohl and Bayd worked through the night. When the sky cleared briefly as the sun rose,
LightningStruck escorted Bayd and the others back to base camp. After a few clear signs that they all
needed sleep, the sprookje disappeared once more into the flashwood. No others came.
“Do you suppose they have some way of communicating with each other by long distance?” Tocohl
said, stifling a yawn. 
“On this world,” said Om im, “it’s probably by grapevine.”
With so little sleep, this reduced Tocohl to a fit of giggles. “Definitely a botanical artifact,” she agreed,
explaining the joke for Maggy’s benefit. “Go away,” she added to both of them, “let me sleep.” But as
she dozed off, she was well aware that neither Maggy nor Om im obeyed, and she slept more soundly
for that.
When she awoke, it was to Om im’s light hand on her face. “Ish shan, we’ve a full day of sun ahead
of us—and the sprookjes have brought you a royal visitor.” She blinked at him. “You slept through the
night,” he explained, “and there’s a crested sprookje in camp.”
That brought her fully awake and to her feet. She bounded down the cabin steps, Maggy at her
heels, and followed Om im to the little garden Dyxte had planted in front of layli-layli calulan’s cabin.
Dyxte’s plants luxuriated in the pale sunlight and, in the midst of them, stood a brilliantly crested
sprookje.
A sharp smell assaulted her nostrils. Under her breath, she said, “Veschke’s sparks—is it injured?”
She could see nothing apparent wrong with it but the smell was that of infected flesh.
“No, no, Ish shan!” Om im was laughing but trying as well not to breathe in; it gave his laugh a
curious quality. “The mystery of the torn-up thousand-day-blue is solved. That’s what you’re smelling.”
He pointed to a small plant that swelled purplish-blue through the compound’s red mud.
The crested sprookje ruffled at the small group of brown sprookjes. No, thought Tocohl, watching
more carefully—the crested sprookje bristled. The brown sprookjes picked through Dyxte’s garden,
pulling out the thousand-day-blues and tossing them aside into a pile.
A knot of surveyors watched this all, cameras taping furiously. Tocohl stopped beside Dyxte, who
gave her a full-body smile and said, “Graffiti. One of the camp sprookjes planted thousand-day-blues in
my landscape.”
“Watch,” said Bayd, “the brown ones think it’s funny.”
Bayd was right, to judge from the feather rufflings. Despite the smell, the brown sprookjes cheerfully
went about ripping out the thousand-day-blues. When they had found them all, one of the brownsprookjes gathered them into a bundle and walked toward the flashwood, holding them at arm’s length
all the while.
This sent the rest of the brown sprookjes into ripplings of delight.
“Children!” said Tocohl. “The brown sprookjes are youngsters!”
“I think so,” said Bayd.
As the smell dissipated, the crested sprookje stood off to examine Dyxte’s work with what seemed
to Tocohl a practiced eye. Then it stepped in for a closer look at layli-layli calulan’s pennants and the
tattered festoons of Tocohl’s moss cloak. Judging from its stance, it was very pleased with the effect.
Tocohl stepped forward, Maggy rippling the stripes in her 2nd skin in the most formal greeting they
knew in sprookje. The crested sprookje ruffled its feathers in the same pattern; simultaneously, its crest
rose. (Veschke’s sparks, Maggy. How are we going to answer that one?)
(We’re not,) said Maggy.
(All right, but let’s tell His Nibs we’re not physically capable.)
This they managed with some effort. The crested sprookje came closer, examining Tocohl as
carefully as it had the moss cloak, even to running a gentle finger along her arm—and puffing in surprise
to learn she was not feathered. It drew her hand upward to scrutinize. Tocohl winced in anticipation of a
nip but it did no such thing: instead it drew her hand gently along its own feathers, spreading them to
display the skin underneath.
“Did you get that, Maggy? Bayd? I think we just got words for ‘skin’ and ‘feathers.’”
The crested sprookje let her hand drop. From its own vibrantly colored yoke, it tugged a feather and
gave it to her. Feathers are good, it told her silently, try them.
Tocohl bit her lip to keep from laughing and translated this, adding, “Get swift-Kalat.” Swift-Kalat
pushed through the crowd to join her.
“Feathers are good for sprookjes,” she told it, translating aloud in GalLing’ as she went along. “Skin
is good for strange-sprookjes. I give feather to the Jenji to examine.” She was forced to lapse into her
created pidgin—as yet they hadn’t the sprookje word for “examine.” In pidgin, it was the nipping motion
with which everyone in camp had been examined.
“Keep your eyes open, Bayd. He—”
“She,” corrected swift-Kalat.
“—She wants LightningStruck to translate that.”
“Got it.”
(Got it,) agreed Maggy, and together they repeated Tocohl’s phrase, this time ending it with the
sprookje’s own term.
The crested sprookje turned his attention on swift-Kalat. “You examine?” Tocohl translated for him.
Swift-Kalat turned his thumbs up. The crested sprookje looked first at LightningStruck and then at
Tocohl for confirmation. “Yes,” they both told her.
“Give feather,” the crested sprookje agreed. “Feathers are good.” Then she stepped back to indicate
the garden.
“She wants to know if you made that,” Tocohl said.
Swift-Kalat flicked his fingers no. At the same time, Tocohl expressed the sprookje negative.
Reaching into the crowd, she brought Dyxte forward. “The ti-Tobian made that.” And she translated the
crested sprookje’s response for Dyxte: “Her Nibs says it’s very good. Different and strange, but very
good. It’s what she came to see, if I got that right.”
“Thank her for me, Tocohl. Ask if she does landscaping, if you can.” That wasn’t easy, but Tocohl
managed it.
“Yes,” came the answer, “I will show you—” Tocohl broke off her translation. “Did you get that last,
Bayd?”
“I think it’s a time referent. See if she’ll explain it. We need time referents desperately—I can’t even
sort out their tenses, if they’ve got them.”
There was a flurry of activity and a flutter of feathers involving all five sprookjes, three Hellsparks,
and van Zoveel. At the end of it, they were forced to agree that both sides would wait for understanding.And that Dyxte would wait to see the crested sprookje’s work.
By then, most of the surveyors had trickled away to let the glossis get on with their work. Swift-Kalat
had gone to examine the feather. Only van Zoveel and Alfvaen remained. It was Alfvaen who next
attracted the crested sprookje’s attention. “I examine One-Who-Was-Poisoned,” Tocohl translated,
adding, “She means to nip you again, I think. So be forewarned, Alfvaen.”
“Yes,” said Alfvaen, and she held out her hand, flinching only slightly when the expected nip came.
Having taken her sample, the crested sprookje turned to Leaper, the brown sprookje that had been
swift-Kalat’s shadow, the first Tocohl and Alfvaen had seen. The crested sprookje ruffled its feathers
and raised its crest. “Good work,” translated Tocohl for Alfvaen’s benefit, “with a raised-crest fillip.”
“Perhaps it’s a superlative,” Nevelen Darragh suggested. Tocohl raised a brow. Darragh smiled and
went on, “Perhaps youngsters don’t rate a use of the superlative.”
“Anything’s possible.”
“Let’s find out if they are youngsters,” Bayd said. “Good timing,” she said as Om im brought tarps
and spread them on the muddy ground. Bayd sat, inviting the crested sprookje to join her.
Tocohl watched the two, but she found herself increasingly distracted by some elusive thought she
could not quite touch a blade to. Her glance kept returning to Alfvaen: the Siveyn shared a tarp with
LightningStruck.
“Yes,” Bayd confirmed, “the brown sprookjes are youngsters. It was a matter of the larynxes. I can’t
quite make it out. And the fact that youngsters are more flexible in a new situation. FineGarden—that’s
the best I can do on Her Nibs’s name—FineGarden wants to know if we are too. I told her no. We
think a strange land is too dangerous for youngsters.”
Tocohl waited for the rippled reply. FineGarden seemed to say that strange sprookjes could be
dangerous too. Tocohl’s eyes widened. There it was: the strange sprookje that could be dangerous was
Maldeneantine—Timosie Megeve!
She looked again at LightningStruck, completely at ease beside Alfvaen even though she had seen
Alfvaen at her most violent. And she remembered seeing the sprookjes back away from Megeve.
“They’re afraid of Megeve but not of Alfvaen!” she said aloud—and it was to Byworld Judge
Nevelen Darragh that she spoke. “Perhaps one of them saw something!”
For answer, Darragh stood to tap the chimes at the entrance to layli-layli calulan’s cabin. When
layli-layli appeared at the entrance, Nevelen Darragh turned again to Tocohl. “Ask them,” she said, “ask
them about Oloitokitok.”
“I’ll try,” said Tocohl. To layli-layli, she added carefully, “I can’t promise anything.” Frowning in
thought, she rose to her feet and shifted her body as if she were about to speak in the Yn male dialect.
LightningStruck looked startled, then rose. Shifting to match her kinesics to Tocohl’s, she riffled her
feathers in alarm and opened her mouth to display a tongue warning. The feathers settled as quickly as
they had risen, to indicate to Tocohl that she must wait. All this Tocohl translated for layli-layli calulan,
while LightningStruck held a hurried consultation with FineGarden.
Tocohl was unable to follow this, except for LightningStruck’s quick shift into Yn-male (again
signifying Oloitokitok?) and back to sprookje. FineGarden replied just as rapidly and just as
incomprehensibly—then addressed herself to Tocohl.
“We wait,” Tocohl interpreted, “LightningStruck will ask—or possibly will get—Vikry. Vikry?”
FineGarden shifted to Yn-male. “Oh, yes. I understand,” Tocohl said. “Vikry is Oloitokitok’s sprookje.”
“Do you think Vikry can tell us what happened to Oloitokitok?” Alfvaen said as LightningStruck
hurried off into the flashwood. Tocohl gave her Kejesli’s one-hand shrug. FineGarden, who seemed
enchanted by the gesture, drew her into a lengthy discussion that passed the time until LightningStruck
returned.
With her was a sprookje Tocohl had not seen before.
(This could be tricky, Maggy,) Tocohl said privately. (We’ve got to get this just right. Tell me if the
arachne spots anything I miss.)
(Right,) said Maggy, moving the arachne to one side for a clearer view of the new arrival. (Vikry is
carrying what appears to be a short length of cable.)That was curious. Tocohl craned for a look, but the object was obscured by the sprookje’s feathers.
(What kind of cable?)
(I can’t tell from this angle. I’ll let you know in a moment.) The arachne moved slowly, angling closer
to the sprookje.
Knowing how capable Maggy was of splitting her attention, Tocohl went on to greet Vikry and to
introduce herself. Her 2nd skin rippled stripes in several different areas.
“I’m asking what they know of Oloitokitok,” she added, “I’m telling Vikry that you were very close
to Oloitokitok, layli-layli, and we want Vikry to tell you about him.” Brown and gold stripes rippled at
Tocohl’s wrists. “And I will speak for you to understand.”
For a long moment, Vikry turned his enormous gold eyes on layli-layli calulan, his feathers a jumble
of activity.
Tocohl would have reached to smooth the feathers but she did not know if Vikry was familiar with
the pidgin and she had no sprookje for reassurance. LightningStruck did it for her, adding the pidgin
gesture as well, with a glance at Tocohl to see if she understood. Tocohl turned her thumbs up and
simultaneously had Maggy ripple a yes.
Hesitantly, Vikry moved toward layli-layli calulan. “Oloitokitok good,” Tocohl interpreted,
“Oloitokitok very good.”
“Yes,” said layli-layli calulan, turning her thumbs up in agreement with the sprookje.
“Oloitokitok gave Vikry this,” Tocohl went on, still translating, as Vikry held out the short length of
cable to layli-layli.
(It’s a piece of superconducting cable,) Maggy put in privately. (Expensive gift!)
(Thanks,) said Tocohl, and she continued aloud, “He wants to know if you want it back, layli-layli.”
“Tell him if Oloitokitok wanted him to have it, then I want him to have it,” said layli-layli calulan,
motioning in pidgin that the length of cable was his. Tocohl did the best she could in sprookje, but she
was glad to see Vikry had already gotten the idea.
Vikry went on, in both pidgin and sprookje simultaneously. “Oloitokitok good,” Tocohl translated
again, “Vikry thanks you. Vikry gave—I wish we had some idea of tenses—yes, gave Oloitokitok
something—something like cable? I didn’t get that. Did you, Bayd?”
“No, I didn’t.” With Maggy’s assistance, Bayd rippled green and gold stripes asking Vikry to repeat
himself.
Superconducting cable, Tocohl thought as she watched. They gave me moss for moss—
“I still don’t understand,” Bayd said, signing it as well.
“LightningStruck,” said Tocohl, in GalLing’ and sprookje simultaneously, “you and I spent two
storms in good/safe plants. Tall plants. What do you call them?”
LightningStruck made the same riffling of feathers that Vikry had made. “You’ve got it!” Bayd said.
With Maggy repeating the riffle, Tocohl asked Vikry, “Oloitokitok gave you the cable, and you gave
Oloitokitok lightning rod?”
Thumbs up and another riffling, this time across the chest. “Lightning rod and cable—the same!”
Tocohl translated triumphantly.
Thumbs went up all around, everyone happy to have gotten that straight, but Tocohl felt a chill run up
her spine. (Odds on Megeve jumped again,) Maggy reported.
(I know. Now we ask a few nasty questions.)
She addressed herself to Vikry again, translating as she went. “Oloitokitok good. Oloitokitok give
you cable, you give Oloitokitok lightning rod.” Thumbs up on each. Inexorably, Tocohl went on,
“Megeve see you give Oloitokitok lightning rod?”
Vikry again turned thumbs up. Om im growled in Bluesippan, touching the hilt of his knife. Layli-layli
calulan’s face turned grim. A handful of the other surveyors, catching on to the implications of Vikry’s
report, stirred restlessly.
“Go on,” Tocohl said to the sprookje, “then what happened?”
She’d gotten the idea of continuation across, for Vikry picked up the story from there. “All excited,”
Tocohl translated, “Megeve and Oloitokitok make... beak flaps with... no, I don’t...”Seeing her confusion, LightningStruck stepped in and demonstrated, by parroting Tocohl’s last few
words. Then she repeated the feather rufflings that were, unmistakably, the sprookje for “verbal speech.”
“Beak flaps with safe thunder,” Bayd said. “Ah! Distant enough thunder that you needn’t worry
about lightning and needn’t shut down your ears!”
(What is it?) Maggy asked. (You just spiked on every sensor. Are you all right?)
(Help me out, Maggy. I’m going to make another guess.) Tocohl touched Vikry gently on the wrist to
make sure of her attention. Speaking aloud as she went along, Tocohl began, “Vikry. Oloitokitok gave
you cable. You gave Oloitokitok lightning rod. Megeve saw. Megeve and Oloitokitok very excited. You
show me Megeve and Oloitokitok.”
LightningStruck riffled her feathers to sign that she did not understand.
“You give Oloitokitok lightning rod.” Tocohl shifted to a stance that mimicked Vikry’s and made her
an imaginary gift. “What do Oloitokitok and Megeve do? How do they move? Vikry show me
Oloitokitok. LightningStruck show me Megeve.”
The two younger sprookjes riffled at FineGarden. Tocohl couldn’t tell if they were asking
FineGarden’s permission or if they were asking her to explain Tocohl’s request. Whichever it was, Vikry
at least turned back and turned her thumbs up.
“Good,” signed Tocohl. “You show us Megeve and Oloitokitok.” (Tape this, Maggy.)
(You think I’m as dumb as Garbo?) Maggy had already moved the arachne to a position that
afforded her an unobstructed view of the two sprookjes.
(Sorry, Maggy,) Tocohl said. (If this works, we can’t afford to miss the chance to record the result.)
The two sprookjes made a fine show of smoothing their feathers and readying themselves, then once
more Vikry turned her thumbs up.
(It might not work,) Maggy began—but the two sprookjes had already changed manner. Vikry took
on the proxemics and kinesics of an Yn male with an accuracy that would have astonished even a native
dancer of the language. From Layli-layli calulan’s whitening face, Tocohl knew that the sprookje had
caught much of Oloitokitok’s individual manner as well.
LightningStruck—too slender, too small—was nonetheless the image of Megeve in every movement.
The sprookjes clacked their beaks, apparently in imitation of the two humans speaking to each other.
No sound came out—Tocohl had expected none—but she could read the sequence of events in their
movements.
Oloitokitok waved something triumphantly in his hand. He started for... yes, he must have started for
base camp, urging Megeve to follow quickly.
Megeve—angry and fearful—caught him by the tips of his feathers.
His excitement barely controlled, Oloitokitok turned to face Megeve. Megeve made beak flaps.
Oloitokitok watched him, his great gold eyes widening.
Megeve made more beak flaps. Oloitokitok quieted, deflated, then sagged—into a posture that
shrieked humiliation. As Megeve made yet more beak flaps, Oloitokitok resigned himself to failure.
To Tocohl, they might just as well have spoken the words: Megeve had convinced Oloitokitok that
their evidence would not be accepted.
She was not the only one who understood. Beside her, Om im spat out a curse in Bluesippan and
grasped the hilt of his knife. Tocohl gripped his shoulder and he quieted, but the hand on his hilt did not
loosen.
In silent anger, they watched the remainder of the sprookjes’ dumb show, fighting to comprehend the
sense beneath the movement.
At last, Megeve made beak flaps at Oloitokitok that buoyed his spirits. Together the two of them set
off for base camp: Megeve still angry but no longer so fearful, Oloitokitok in anticipation.
“That was how he seemed,” said layli-layli calulan, “the day before he d-disappeared.” Her scars
of office stood out against the pallor of her cheek. “Ask them what happened next.”
“What then?” Tocohl signed, but both sprookjes had already returned to their own individual stances.
In sprookje, Vikry explained that they had seen nothing more that day. A storm had forced them to
take shelter for the evening. The next day, when the weather was safe, they returned to the camp.LightningStruck followed Megeve out to the hangar but Megeve had—here LightningStruck ran out of
understandable signs and showed them—Megeve had raised something heavy to threaten her with it.
“Megeve not safe,” she signed again, showing the red warning of a thrust-out tongue for emphasis.
Vikry agreed. When he had followed Oloitokitok out to the hangar shortly thereafter, Megeve had
frightened him away too.
The sprookjes could show little more. From a distance, they had seen Megeve give something to
Oloitokitok. Then two daisy-clippers had left together. That was all.
“No more beak flaps from Oloitokitok,” Vikry finished.
“No,” said Tocohl, her hand still clenched on Om im’s shoulder. “That was the last anyone heard
from him.”
There was a long grim silence that was broken at last by Nevelen Darragh. “You have your
witnesses, layli-layli calulan.”
Layli-layli calulan, with the calm of an empty suit of iron armor, said only, “Yes.”
“Tocohl, may I borrow your blade?”
It took Tocohl a moment to realize that Darragh was referring to Om im. “Of course,” she said,
relinquishing her grip on the Bluesippan’s shoulder. Her hand ached. “Your pardon, Om im,” she
muttered hastily. She got a brief glancing smile in return as he stepped forward to bow to Darragh.
Darragh smiled at him. “You know the drill, Om im. Call court in the common room in”—she
consulted her own computer briefly—“one hour. Any cases dealing with the world known as Flashfever
may be presented at that time.”
“And presiding?” Om im asked.
“Byworld Judge Tocohl Susumo.”
Tocohl opened her mouth but nothing came out. Across the way, layli-layli calulan met her eyes,
and gave a crisp, satisfied jerk of her head.
“S-susumo?” Tocohl said, her voice harsh with the effort.
“Susumo, it is. I heard it, too, Ish shan—and you’ll remember that I can hear the difference.” Om im
made her a sweeping bow, looking up from the very bottom of it to raise a gilded brow at her impishly.
When he straightened, he called out in ringing tones: “Court called. Court called. One hour from now in
the common room. All cases dealing with the world known as Flashfever may at that time be presented.
Byworld Judge Tocohl Susumo presiding.” With a final flip of the eyebrow, he strode away jauntily to
deliver his message to the rest of the camp.
Stunned, Tocohl could say nothing to the queries of those around her. She noted only in an absent
fashion that Alfvaen, fierce but proud, turned them all away. Standing frozen in an eddy of movement,
Tocohl said to Maggy, (But I’m not—I never wanted to be a byworld judge.)
(I don’t understand,) said Maggy, (you were willing to let them think you were.)
And at last Tocohl understood. It was at that moment that Nevelen Darragh stepped to face her and
Tocohl said, “Yes, I do understand. You mean to make me pay the debt.”
“You cannot be a byworld judge in name only. One way or the other, you must take the
responsibility as well.”
Tocohl’s glance followed Darragh’s to rest on the arachne, Maggy’s only visible presence. “The
choice is yours,” Darragh said.
There was no choice. If four byworld judges found her an acceptable colleague, then only her own
lack of willingness stood in the way. And to turn down that responsibility was to deny her responsibility
to Maggy, which she could not do.
“I pay my debts,” said Tocohl.
“I’m glad to hear it,” Darragh said with a smile, “though I expected no less.” She laid a comforting
hand on Tocohl’s arm. “We do give advice to our younger judges, you know. If you need any assistance
in this matter, any of us will be glad to help.”
“Thank you,” said Tocohl, as relieved as Darragh had intended her to be. “You can give me a hand
with the sprookjes, then. It’s not going to be easy, explaining to them what we’re doing.”
“Tocohl?” It was Alfvaen. Still puzzled by it all, she frowned slightly and said, “I don’t understand.Are you a judge or aren’t you?”
“I am now.”
“Oh!” Maggy spoke up at last through the arachne at her feet. “I understand! You invented Byworld
Judge Tocohl Susumo!”
“It would seem so.” Tocohl smiled at Darragh, then knelt to face the arachne. “Do you understand
what that means?”
“I’m not sure...”
“It means you and I stay together,” Tocohl said, and from the arachne Maggy let out an ear-splitting
whoop of joy.
Chapter Seventeen
THE COMMON ROOM was once more filled to capacity when Tocohl arrived at the appointed
hour. This time the tables had been pushed aside in favor of loose ranks of chairs, all facing a single table
at the end, behind which was one last chair. Beside that stood Om im—dressed, she realized, in the finest
of his finery. Bracketing him, but at a discreet distance, were four other empty chairs.
As she followed Maggy’s arachne in, Om im touched the hilt of his knife to her and said, loudly
enough to make himself heard over the general tumult, “Court called on Flashfever. Byworld Judge
Tocohl Susumo presiding.”
Tocohl felt her face grow hot as the surveyors turned en masse to stare at her. Behind her Harl
Jad-Ing said softly in Hellspark, “Courage, Tocohl. The hardest walk is the length of the room.”
“Besides,” Mirrrit added, “you brought it off the last time. It ought to be easier now that it’s official.”
The crowd, murmuring noisily, parted in waves to let them through—Maggy’s arachne, Tocohl, and
the four judges behind her, legitimizing her by their presence. “See here, Tocohl,” Kejesli began as she
reached a point almost to the fore, “I demand to know what this is all about. As captain of the survey, I
have a right to—”
Yannick Windhoek, as sour-faced as ever, said, “Tocohl Susumo, fourteen years an apprentice, has
risen to judgment on Flashfever. May she fly with Veschke’s sparks.” He touched the pin of Veschke at
his breast and, mouth agape, Kejesli touched his own in response and fell back silent.
Without thinking, Tocohl too touched the pin of high-change in the folds of her hood, only then
realizing its significance. (That’ll teach me to take risks with religions,) she said to Maggy.
(We took the risk,) Maggy reminded her, using the Hellspark tight-we for emphasis.
(Yes, and look what happened to you!)
She had reached the fore. Om im swept her a low bow and drew the chair for her; when she and the
other four were seated, he once again said, in ringing tones, “Court called on Flashfever. Byworld Judge
Tocohl Susumo presiding.”
Under cover of the sound of some forty people jostling to settle, Tocohl said in protest, “You needn’t
overdo it.”
“I like the sound of it,” he said, grinning. “I think I’ll do it again.”
“No,” she said, and he touched his hilt, suppressing his grin to a small quirk at one corner of his
mouth... Layli-layli calulan stood and the last few mutters of the crowd died instantly away. The
shaman’s face was once more serene. “I come for judgment,” she said. “I accuse Timosie Megeve of
Maldeneant of the premeditated murder of Oloitokitok of Y, and of the attempted genocide of the
species known as the sprookjes of Flashfever. Will you judge?”
Tocohl took a deep breath, let it out slowly. “Yes,” she said, “I will judge.” And with that, Megeve’s
trial began.
The trial was swift. The sprookjes had little to add beyond identifying the item (“same/like”) that
Megeve had given to Oloitokitok—before the two of them had taken daisy-clippers into the
flashwood—as a locator. To that, Buntec could only say that it would have been possible to rig a locator
to deliver the shock that killed Oloitokitok. She could not prove it had been done, although there was nodoubt in her mind that it had.
Confronting Megeve brought only a repeated indictment of the sprookjes, for “wasting” the world of
Flashfever.
Tocohl heard them all out. At last, she said, “Before I begin my deliberations, does anyone else have
anything further to add?”
Yannick Windhoek stood. “Yes. The doctor layli-layli calulan, who is also an Yn shaman, has
explained to me that constant exposure to heavily ionized air has caused many members of the survey
team to behave in an abnormal fashion.”
“I’m aware of the effect,” Tocohl said.
“Then I ask that you take it into account when you judge Timosie Megeve’s actions.”
“I intend to,” Tocohl said. She scanned the room, awaiting further comments or suggestions. There
were none. “That’s it?” she asked, her glance resting on Nevelen Darragh. Darragh merely directed her
own glance at Om im, so Tocohl turned to him and said, “That’s it. Get them out of here, Om im. I need
time to think.”
Om im did; in a matter of minutes he had, by voice alone, cleared the room of all except Bayd and
the other judges.
Bayd laid her hands on Tocohl’s shoulders and gave her a comforting squeeze. “Megeve has plenty
of judges to appeal to, you know.”
“I know. That makes it no less my responsibility.”
“Do what’s right—for all of us.” Bayd gave her a smile and a second squeeze, then turned to leave
with Windhoek and Jad-Ing and Mirrrit.
Maggy’s arachne trotted along behind her. “Maggy,” Tocohl said, “you don’t have to go.”
“I know,” said Maggy happily, “I’m staying with you.” The arachne followed Bayd out into the
courtyard.
A cheerful laugh beside her reminded Tocohl that Nevelen Darragh was still present. She turned,
suddenly afraid, and Darragh said, “Shall I stay?”
When Tocohl hesitated, Darragh said, “I assure you Om im is quite as good as sounding board or as
silent support.”
Tocohl blinked down at Om im, who raised a brow at her. “She’d feel more secure if you stayed,
Nevelen. After all, you’re an old hand at this.”
“Stay,” said Tocohl to Darragh. She spoke in panic, but once the word was out, she found herself
oddly calm and accepting.
“What did I tell you?” Om im said. After a bow to each, he strode the length of the room, pausing at
the door to call out, “Now you’ll see what it’s like from the other side, Nevelen. I’ll be right outside,
Tocohl.” Then he was gone from sight.
“He will be too,” Nevelen Darragh observed as she gestured Tocohl into a chair and drew a second
up beside it for herself, “even if it takes you three days to reach a decision.”
To reach a decision, Tocohl thought, and once again heard the echo of Bayd’s words: “Do what’s
right.”
For Tocohl, that meant to begin with the Methven ritual for calm, and then to turn and examine the
evidence against Megeve in her mind one final time, setting it deep in the context of Flashfever. When she
was done she thought, with bitter amusement, So being a judge means that your choices are restricted...
There was only one verdict she could give; and as she looked up at last into Darragh’s eyes, she met
sympathetic understanding—and agreement. “Tell Om im I’m ready,” she said.
“Yes,” said Darragh, rising, “you are.”
As Darragh walked the length of the hall, Tocohl herself rose, discovering only then that her muscles
ached with stiffness. (Maggy? How long—?) (About two hours.) Tension then, not length of time. She
stretched to work out such of the ache as she could.
The common room filled in minutes. None of the surveyors had gone far, that much was clear. Again,
Timosie Megeve was brought; again, Om im called court for her. This time the quiet was instant and
absolute.At the front of the room, Tocohl perched on the edge of the table. Cribbing formal words from a
handful of byworld trials she witnessed, Tocohl said, “In the matter of Timosie Megeve of Maldeneant:
“I find the evidence connecting him to the death of Oloitokitok of Y to be insufficient and
circumstantial.” She met layli-layli calulan’s eyes; the effort of doing so chilled her. “We could not
prove it,” she said with emphasis. Layli-layli calulan dropped her eyes under the scrutiny, an admission
that even she could not deny the truth of that.
“However,” Tocohl went on, “the charge of attempted genocide is quite another matter. There we
can prove that Timosie Megeve consistently, and with forethought, concealed information that would
have enabled this survey team to make a clear evaluation of the sprookjes’ intelligence. We can see the
same pattern in his attempts to disrupt the survey team itself, which also lessened the team’s ability to
make such an evaluation.”
Turning to face Yannick Windhoek, she said, “As for the effects of the ionization, we must—as you
say—take into account the abnormal behavior of other members of the survey team.
“Kejesli acted hastily in the matter of the sprookjes, yes, but he was willing to give them a last chance
by taking swift-Kalat’s charge of murder against them seriously.” She smiled briefly at Kejesli; “Seriously
enough, at any rate,” she pointed out.
“In like manner, layli-layli calulan, although prepared to curse Ruurd van Zoveel, allowed herself to
be stopped by a ruse.” At that, Windhoek’s eyes widened; he glanced at layli-layli who gave him silent
thumbs-up confirmation. Tocohl went on, “And she never bothered to check the survey computer to
learn van Zoveel’s true name. Even Edge-of-Dark was glad of a chance to do right, rather than angry
that she’d been conned... All any of the rest needed was a little push in the right direction.
“Yet Timosie Megeve remained unmovable. Worse, he was pushing in the wrong direction. He
admits that he thought the sprookjes sentient; yet everything he did was an attempt to convince others
they were not. He did not actually commit genocide, yet his actions might, in the end, have resulted in
genocide.”
Tocohl slid from her perch and turned to face Timosie Megeve. “You found the sprookjes so unlike
you in spirit that you judged them unworthy of human rights; in like manner, I judge you. Timosie Megeve
of Maldeneant, I find you guilty of attempted genocide. How do you choose, Megeve: death or
restriction?”
“I appeal.”
“That is your right,” said Tocohl. She stepped back. “Address your appeal to another judge.”
Timosie Megeve raised his hand. In a defiant voice, he said, “Yannick Windhoek, will you judge an
appeal?”
“I will judge,” said Windhoek, his voice as cold with finality as his face: “No appeal. The judgment
stands.”
Timosie Megeve whitened, and Tocohl had no choice but to repeat her query, as if she were caught
in its relentless rhythm: “How do you choose: death or restriction?”
“I accept my role,” he said, “I choose death.”
The option was always given; it was seldom taken. Silence fell heavily. Tocohl stiffened. “Very well,”
she said as, one by one, the members of the survey team raised their hands to shoulder height—each
signifying his or her unwillingness to perform the deed.
Layli-layli calulan rose, twisting the bluestone rings from her fingers as she stepped forward.
“Pattern demands that I fulfill it,” she said, holding out her rings to Tocohl.
Tocohl raised her palm and the rings dropped into it. “Death,” she said softly, “at the hands of
layli-layli calulan of Y. Let it be so.” And to layli-layli calulan, she said, “You have his true name.”
“Death,” said Timosie Megeve scornfully, “at the hands of this barbarian. Do you expect me to
believe in your death curse?” He began to laugh. Layli-layli calulan raised her hands, touched him ever
so gently, and spoke a few grim words against the harsh, rasping sound of his disbelief.
Two days later, Timosie Megeve died, laughing no longer. At the end, he had no choice but to
believe.Windhoek, who was heading in the direction of MGE’s main center, might have carried the final
survey report, but Kejesli preferred to waste MGE’s money. A message capsule went instead, and with
all due ceremony, as subdued and formal as it was under the circumstances.
Tocohl felt odd. There was no triumph, only a sense of relief that it was over at last. She felt
drained—worse, she had no sense of expectation.
Mirrrit and Jad-Ing, with Maggy’s assistance, had put together a program that enabled anyone
whose 2nd skin had graphics display to reproduce the sprookjes’ feather-rufflings; and Bayd had, quite
pointedly, taken over the job of learning the sprookjes’ language.
Alfvaen would return with the survey team. Kejesli meant to restore her reputation with MGE but
Alfvaen would have gone anywhere to be with swift-Kalat, much to Maggy’s embarrassingly outspoken
satisfaction.
But all this only served to leave Tocohl at loose ends. Spatters of rain began to fall, sending the last
few onlookers scurrying for shelter, but Tocohl felt no need to hurry.
“Tocohl?” A hand caught her elbow. It was Nevelen Darragh. “I’d like a word with you in private.”
And the hand at her elbow swept her along. “Om im volunteered his quarters.”
Tocohl was inside almost before she realized what was happening. When she did, it was with great
surprise to see Windhoek, Harl Jad-Ing, and Mirrrit. Maggy’s arachne squatted on the table, chatting
happily with Om im and Mirrrit.
“Sit down,” Darragh said, and Tocohl obeyed, curious at last. Darragh went on, “You have a panel
of four judges at your service, quite enough for a judgment of sentience.”
Tocohl turned to stare at her. It was Maggy’s sentience they meant—they knew, all of them. She
opened her mouth to protest.
“Just a moment,” Darragh said. “I do understand your reasons for not wishing such a judgment
made—but we would be prepared to grant you Maggy’s guardianship at the same time. Kids need
looking after.”
“I—Nevelen, as far as I know, Maggy’s the only one of her kind as yet. I won’t have her... growing
up under intense scrutiny. Being treated as a freak or a curiosity wouldn’t be good for her.”
“Then the judgment will be held closed, reported only to other byworld judges.” Darragh watched
her carefully. “Give it thought before you answer, Tocohl. Such a decision would set a precedent that
would be of great advantage to others like Maggy as they arise, and it would alert the rest of the byworld
judges to look for them as well.”
Darragh was right, Tocohl knew, it would help the others, but Maggy was her first concern. “Maggy,
what do you think?”
“Would we stay together?”
“We’d stay together.”
“Then whatever you decide is fine with me.”
Tocohl took a deep breath. “All right, Nevelen, as long as I’m her guardian and the judgment is
closed. I won’t have her treated as a freak,” she said again.
Moments later, it was official: the extrapolative computer known as Margaret Lord Lynn of
Hellspark had been declared a sentient child, Tocohl her guardian, and the proceedings had been
declared For Judges’ Ears Only.
Tocohl shivered. To Om im, she said wryly, “So you told them about her. I didn’t think you knew.”
But his sudden look of surprise told her she was completely wrong. She turned to Darragh, hoping for an
explanation.
The one she got was not the one she expected. Darragh said, “Om im was our backup. If you hadn’t
asked for a sentience judgment on Maggy, he’d have asked for one on the sprookjes.”
“But that was already decided...”
“Not as far as the government knows. And since we all traveled here for a sentience hearing”—a
wave of her hand ran the range, from smiling Mirrrit to scowling Windhoek—“the government pays our
travel expenses, not you.”
“Oh,” said Maggy, “that’s good! That means we have lots of money left over for more memory!”Darragh laughed and Tocohl laughed along with her. “As for who told me,” Darragh said, “it was
Maggy herself. It was clear from her behavior that she’d gone well beyond what we normally think of as
standard behavior, even for an extrapolative computer. I asked her for the details.”
“I told you she was a mean trader,” Maggy volunteered.
Tocohl eyed the arachne curiously. “And what did you get in return?”
“Nothing,” said Maggy, hunching the arachne, “unless you count the experience.”
“Taking advantage of children, Nevelen?” said Om im. “I’m surprised at you.”
“So am I,” said Tocohl. “Maggy, what exactly was the deal?”
“I wanted to know what had happened to other people who had claimed to be byworld judges. She
offered to trade a complete file on judges and judgments for information about me—but she said I
couldn’t open it until after the four of them had judged you. And they didn’t, so I can’t and I got nothing,
and I promise you, Tocohl, I’ll know better next time.”
Darragh wiped a hand across her face, her shoulders shaking.
“Maggy,” said Tocohl, “they did judge me. You have every right to open that file.”
“I do?”
Tocohl looked pointedly at Darragh who said, “Yes, Maggy, you do.”
“Oh, good,” said Maggy, “then I’m not such a bad trader after all!” She fell silent, probably to
examine the information Darragh had given her.
“I have one last message, Tocohl,” Darragh said. “Your father invites you to join him for a little ‘on
the job training.’ I suggest you take him up on the offer. The talent may run in the family, but experience
always fines it. I’m headed that way myself. Perhaps you and Maggy would like to tag along?”
“Yes,” said Tocohl, “I think that’s a fine idea.”
“Me too,” said Maggy, emerging from her studies momentarily. “We didn’t see Tocohl Sisumo at the
Festival of Ste. Veschke.” And, mirroring Tocohl’s sentiments exactly, she added, “I miss him.”
Tocohl and Maggy made their good-byes. Alfvaen returned them in perfect Jenji. Bayd grinned and
promised to keep up her lessons, at least until the team’s pickup arrived, and sent Tocohl off with a pile
of tapes for Sisumo.
Somehow Tocohl found it was hardest to take leave of layli-layli calulan, who had lost so much on
Flashfever. But the shaman smiled her brilliant smile. “May the threads of our lives twist together again
and again,” she said, “and may the two of you always dream as well as you did here.”
Then Tocohl gestured Darragh and Geremy into Maggy’s skiff. “We have, as usual, a storm to run,”
she reminded them.
Om im stopped her with a gesture. “There’s nothing more I can do here, Ish shan,” he said. “Do you
have an opening for a seasoned judge’s aide?”
The question surprised her but not greatly. At last, she said, “Om im, I’d like that very much, but not
just yet. Maggy and I have a great deal to work out between us and I think—I know—you’d be
something of a distraction.”
He laughed and bowed. “I understand. And I heard the ‘not just yet.’ I’ll see you at the Festival of
Ste. Veschke in a few years, Ish shan, and I’ll ask you again.”
“Done.” Tocohl snapped her fingers and climbed in, glancing back for one last sight of the merry eyes
that glittered beneath gilded brows.. She found herself still chuckling after she had delivered Geremy and
Darragh to their respective ships.
Then all three were on their way and there was nothing much to do except to consider all that had
happened. Once again, Om im’s impish cheer sprang to her mind.
A seasoned judge’s aide! she thought suddenly.
“Maggy,” she said, “I think I’ve been had.”
“You?” said Maggy. Tocohl looked down: the arachne sat at her heels, still activated.
“Me,” said Tocohl. “I think I may just have been thrown in that situation deliberately—to see what I
would do. I don’t know the extent of the setup involved, but I’m going to find out.”
“I can’t tell from Judge Darragh’s files but maybe you could,” Maggy offered.
“Good idea, Maggy. What would I do without you?”“You’d be bored,” Maggy said authoritatively, as she presented an index on the spectacles for
Tocohl to examine.
Tocohl laughed. “Much better that I be in trouble?” she suggested.
Maggy gave a thoughtful pause. “Well, as long as we’re both in trouble, I suppose that’s all right.”
“Yes,” Tocohl said, “that’s very much all right.”



\end{document}

	